\documentclass[a4paper, 12pt]{report}
\usepackage{graphicx}
\usepackage[utf8]{inputenc}
\usepackage[french]{babel}
\usepackage{mathptmx} %times aves le mode math
\usepackage[T1]{fontenc}
\usepackage{fancyhdr}
\usepackage{amsmath,amsfonts,amssymb, empheq}
\usepackage{eurosym}
\usepackage{booktabs}
\usepackage{wrapfig}
\pagestyle{fancy}
\fancyhead[R]{Université Paris-Est Créteil}
\fancyhead[L]{Économie et politique du logement}
\usepackage{array,multirow,makecell}
\usepackage{hyperref}
\setcellgapes{1pt}
\makegapedcells
\newcolumntype{R}[1]{>{\raggedleft\arraybackslash }b{#1}}
\newcolumntype{L}[1]{>{\raggedright\arraybackslash }b{#1}}
\newcolumntype{C}[1]{>{\centering\arraybackslash }b{#1}} 
%\renewcommand{\thechapter}{\Roman{chapter}}
%\setcounter{chapter}{1} % pour numéroter le chapitre 
\begin{document}
	\chapter*{Introduction}
	
\section{Logement et localisation}

Le logement est un bien de consommation aux multiples dimensions, dont le choix est avant tout déterminé par la localisation. Cette dernière influence fortement l'agrément du voisinage immédiat, l'accès aux écoles, aux emplois et aux services.

En France, une tendance marquée vers l’éloignement des centres urbains se dessine, avec une préférence croissante pour la maison individuelle par rapport à l’appartement, phénomène que l'on appelle la périurbanisation. Au milieu des années 1980, la construction de maisons individuelles était deux fois plus fréquente que celle des appartements.

Cependant, depuis 2008, on observe un repli de la construction de maisons individuelles au profit des logements collectifs, en raison de divers facteurs tels que les coûts environnementaux liés au transport automobile et le vieillissement de la population. Entre 2016 et 2021, le nombre de logements individuels a augmenté en moyenne de 0,6 \% par an, contre 1,3 \% pour les logements collectifs, montrant une tendance similaire à celle observée entre 2011 et 2016 (respectivement 0,9 \% et 1,2 \%).

Cette évolution contribue à une augmentation des écarts géographiques de prix et de loyers, révélant ainsi les défis croissants du marché immobilier.

Il existe au moins deux types d’inadéquations spatiales qui impactent le marché du travail.

La première concerne la situation où les emplois sont trop éloignés des logements, ce qui peut être dû à un manque de transports adéquats ou aux coûts de mobilité qui empêchent les individus de déménager vers les zones d'emploi. Cette distance géographique crée des difficultés pour les travailleurs qui souhaitent accéder à des opportunités professionnelles. 

La deuxième inadéquation réside dans le fait que l’offre de travail ne correspond pas à la demande en termes de qualifications. Ainsi, même si des emplois sont disponibles, les candidats peuvent ne pas posséder les compétences requises, ce qui entraîne un déséquilibre sur le marché de l'emploi et limite les possibilités d'embauche pour les entreprises. 

Ces deux problématiques soulignent l'importance d'une planification urbaine et d'une formation professionnelle adaptées pour favoriser une meilleure adéquation entre le logement et l'emploi.

Ce type d’inadéquation est renforcé par la ségrégation spatiale, qui crée des barrières supplémentaires à l'accès à l'emploi et à la formation. Elle peut également être à l’origine d’effets de pairs négatifs, où les caractéristiques du voisinage influencent défavorablement l'accumulation de capital humain, comme l'ont souligné Goux et Maurin en 2005. De plus, la ségrégation spatiale peut engendrer une discrimination territoriale, où le lieu de résidence envoie un signal négatif à un employeur potentiel, comme l'ont observé Petit et al. en 2011.

Ces dynamiques contribuent à perpétuer les inégalités sociales et économiques, rendant encore plus difficile la mobilité professionnelle et l'accès à des opportunités de développement personnel.

La mobilité résidentielle a diminué depuis le début du XXIe siècle, principalement sous l’effet du vieillissement de la population. Cela est particulièrement visible dans le secteur locatif social, où les locataires en place vieillissent et sont moins enclins à déménager. Dans les années 1980, le logement social a joué un rôle de tremplin vers la propriété occupante, comme l'ont noté Goffette-Nagot et Sidibé en 2016. Cependant, ce rôle s’atténue progressivement, limitant ainsi les possibilités de transition vers la propriété et contribuant à une stagnation dans les dynamiques de mobilité résidentielle.

La mobilité a aussi davantage diminué pour les personnes plus âgées, dont les logements sont mieux adaptés et pour qui les déménagements contraints par les revenus ou la santé sont moins fréquents, comme l'indique Laferrère en 2007. De plus, les mobilités de longue distance ont également diminué en raison de la crise économique et des difficultés qu’elle a engendrées sur le marché du travail.

\section{Le parc de logements}

De 1968 à 2013, la population de France métropolitaine a augmenté de 14 millions d’habitants, soit une hausse de 28 \%. Le nombre de résidences principales a quant à lui augmenté de 12 millions, représentant une augmentation de 76 \%. Par ailleurs, le parc total de logements, incluant les résidences secondaires, occasionnelles ou vacantes, a connu une augmentation de 80 \%.

De 2000 à 2016, entre 300 000 et 500 000 logements ont été construits chaque année, totalisant environ 6 millions. Pendant cette même période, la population a augmenté de 352 000 habitants par an en métropole.

La croissance du parc de logements est donc dynamique, avec un accroissement moyen de 1,1 \% par an en France depuis le début des années 1980. Le nombre de logements par habitant, qui était de 519 pour mille en 2012, est proche de celui de l’Allemagne et de l’Autriche, mais reste en dessous de celui de l’Espagne.

Depuis 1982, le nombre de résidences principales s’est accru de 52 \%, une hausse portée par la croissance démographique, qui a augmenté de 20 \% sur cette même période, ainsi que par la baisse de la taille des ménages, due à des facteurs tels que l'arrivée de couples plus tardifs, les ruptures d’union et le vieillissement démographique.

Conclusion : le taux d’occupation des résidences principales baisse de façon continue, illustrée par la diminution du nombre de personnes par logement, qui est passé de 3 personnes en moyenne en 1968 à 2,3 en 2016. Par ailleurs, le nombre de mètres carrés par personne a augmenté, passant de 31 à 40 entre 1984 et 2006, et reste stable depuis.

Ces moyennes masquent des inégalités entre des ménages plus âgés, qui occupent, en tant que propriétaires ou locataires dans le parc social, des logements rendus spacieux par le départ des enfants, et les plus jeunes, qui peinent à accéder au parc social et sont souvent locataires de logements plus petits.

Depuis le milieu des années 2000, le nombre de résidences principales croît à un rythme un peu moins soutenu que l’ensemble du parc, alors qu'il avait augmenté un peu plus vite entre 1990 et le milieu des années 2000.

Conséquence : la part des résidences principales diminue légèrement, passant de 82,6 \% en 1982 à 81,8 \% en 2021.

Depuis le début des années 2010, le nombre de résidences secondaires et de logements occasionnels augmente plus vite que l’ensemble du parc, alors qu'il avait progressé de façon moins soutenue que ce dernier à partir de 1990.

Le territoire n’est pas homogène, avec un taux de vacance élevé dans certaines zones, tandis que d'autres, comme la région Île-de-France ou PACA, connaissent un marché du logement tendu. La vacance désigne la proportion de logements inoccupés, souvent en attente de locataires ou de propriétaires.

Taux de vacance : 8 \% du stock de logements. Les logements vacants augmentent nettement depuis 2006, à un rythme supérieur à celui de l’ensemble du parc.

Taux de vacance : 8 \% du stock de logements. Les logements vacants augmentent nettement depuis 2006, à un rythme supérieur à celui de l’ensemble du parc. La vacance est forte surtout dans les villes moyennes et les unités urbaines de moins de 100 000 habitants. Elle peut provenir d’une inadéquation entre l’offre et la demande en matière de taille, de localisation, de prix, ou correspondre à des logements en mauvais état.

\section{Les conditions de logement}

Les conditions de logement montrent une ancienneté des logements variée : 30 \% d'entre eux datent encore d’avant 1949, tandis que 24 \% du parc actuel (31 \% du parc collectif) a été bâti entre 1949 et 1974. Environ 40 \% des logements (43 \% des maisons) ont été construits après 1975. Aujourd’hui, le confort de base est généralisé à la quasi-totalité des logements. En 2013, seulement 1 \% des logements manquaient du confort sanitaire de base, et 3 \% présentaient plus d’un défaut grave d’isolation thermique, d’étanchéité ou d’installation électrique.

La taille des résidences principales a augmenté, mais peut rester inadaptée au nombre d’occupants. Le surpeuplement, qui a été divisé par deux entre 1984 et 2006 et s'est stabilisé à 8 \%, concerne 21 \% des ménages dans l’unité urbaine de Paris, 16 \% des personnes de moins de 40 ans et 18 \% des ménages les plus modestes.

L’amélioration des conditions de logement n’occulte pas la question de la privation de domicile. En 2020, 3,6 millions de personnes étaient soit privées de domicile personnel (895 000), soit vivaient dans des conditions très difficiles (privation de confort ou surpeuplement) (2 880 000), soit étaient en situation d’occupation précaire (hôtel, caravanes, etc.), selon la Fondation Abbé-Pierre.

Les difficultés de sortir des situations de précarité sont particulièrement marquées pour ceux qui ont connu l’absence de logement personnel. Dans les établissements pour personnes en difficulté sociale, 52 \% des résidents sont des actifs, mais seulement 20 \% d'entre eux sont en emploi ou en stage, tandis que 32 \% se trouvent au chômage. De plus, il est préoccupant de constater que seulement 13 \% des individus avaient un logement personnel auparavant ; la majorité était hébergée par la famille, un tiers ou dans un établissement. Cette situation souligne les obstacles persistants auxquels font face les personnes en précarité pour retrouver une stabilité résidentielle.

\section{Le logement comme placement}

Le logement est également considéré comme un actif, en concurrence avec d'autres formes d’épargne, mais il présente la particularité d’être moins liquide. Les coûts de mobilité, qu'ils soient directs (droits de mutation, frais) ou indirects, sont élevés, ce qui complique les transactions. De plus, un logement nécessite un temps significatif pour être construit, ce qui limite la flexibilité des propriétaires et des investisseurs sur le marché immobilier.

Le taux de propriétaires atteint environ 75 \% à l’âge de la retraite et baisse peu par la suite. Le logement est perçu comme un placement qui protège contre le risque de hausses de loyers plus rapides que l’évolution des pensions de retraite, offrant ainsi une certaine sécurité financière aux retraités.

Le logement peut prendre la forme d'une maison individuelle ou d'un appartement. Environ 80 \% des propriétaires vivent dans une maison, tandis que 75 \% des locataires résident dans un appartement. Cette répartition souligne les préférences distinctes en matière de logement selon le statut de propriété.

Dans le secteur locatif, les mono-propriétaires d’immeubles ont pratiquement disparu. Les mono-propriétaires privés ont été affectés par des facteurs tels que les partages successoraux, les taxes et l'accès à la propriété. Parallèlement, les investisseurs institutionnels ont réorienté leurs actifs dans les années 2000, comme l'indiquent Bessière et Laferrère (2002), en vendant leurs immeubles à la découpe, ce qui a également contribué à cette évolution du marché locatif.

Les particuliers bailleurs sont généralement de petits propriétaires d’appartements. En 2013, 93,5 \% des propriétaires du parc locatif libre étaient des particuliers, contre 83,2 \% en 1996. Cet émiettement de l’investissement privé a des conséquences sur la mise en œuvre de travaux de réhabilitation énergétique. En effet, seulement 14 \% des résidences principales affichent une étiquette énergie A, B ou C, selon François (2014), ce qui souligne les défis en matière d'efficacité énergétique dans le parc locatif.

\section{Le statut d'occupation du logement}

En 2021, 58 \% des ménages étaient propriétaires de leur résidence principale, un chiffre stable depuis 2010 après une augmentation continue depuis 1982, où il était de 50 \%. La part des propriétaires sans charges de remboursement a crû jusqu’en 2010, atteignant 38 \% contre 27 \% en 1982, avant de se stabiliser à 38 \% en 2021 en raison du vieillissement de la population. Par ailleurs, la part des propriétaires accédants est restée stable à 20 \% depuis 15 ans, après avoir diminué de façon continue entre 1990 et le milieu des années 2000, passant de 26 \% à 20 \%. La part des ménages locataires de leur résidence s'est maintenue autour de 40 \% depuis 1990, un niveau légèrement inférieur à celui de 1982 (41 \%). Enfin, les logements détenus par des bailleurs publics représentent 17 \% du parc, tandis que ceux détenus par des bailleurs privés en représentent 23 \%.

Le choix du statut d’occupation est influencé par plusieurs facteurs. Parmi eux, on trouve les contraintes de crédit, telles que le taux d’apport personnel, le taux d’intérêt et la durée des emprunts. Le revenu joue également un rôle crucial, tout comme les anticipations concernant les prix et les loyers. De plus, la position dans le cycle de vie et les anticipations de mobilité sont des éléments déterminants qui impactent ce choix.

L’évolution de l’âge auquel la moitié d’une génération a accédé à la propriété offre un aperçu des tendances à long terme. Ainsi, l'âge était de 47 ans pour la génération née en 1924, de 36 ans pour celle née en 1944, de 33,5 ans pour celle née en 1952, et de 38 ans pour celle née en 1964. Pour les générations suivantes, cet âge est resté stable : les jeunes des générations nées à partir du milieu des années 1970 accédaient à la propriété entre 35 et 39 ans.

Entre 1996 et 2016, plusieurs évolutions économiques significatives ont été observées. Les prix à la consommation ont augmenté de 31 \%, tandis que le revenu disponible brut par ménage a connu une hausse de 40 \%. Parallèlement, le prix des logements anciens a été multiplié par 2,52, avec une augmentation encore plus marquée en Île-de-France, atteignant 2,67.

Après une baisse liée à la crise de 2008, la reprise du marché immobilier a été soutenue, notamment en Île-de-France. Les prix du logement ont légèrement diminué à partir de 2012, mais ils connaissent une reprise depuis 2016. Pour les primo-accédants, la hausse des prix freine l’achat, bien que cela soit en partie compensé par l’allongement de la durée des emprunts, qui est passée à 19,6 ans entre 2009 et 2013, contre 14,6 ans entre 1997 et 2001. De plus, la baisse des taux d’intérêt, passant de 5,2 \% entre 1997 et 2001 à 3,5 \% entre 2009 et 2013, a également contribué à atténuer cette pression.

Au fil du temps, les ménages non propriétaires tendent à être de moins en moins aisés. En effet, le pourcentage de ménages propriétaires dans le premier quartile de revenu a diminué de 8,9 points depuis 1984. Aujourd'hui, le revenu est un facteur encore plus déterminant pour accéder à la propriété qu'il ne l'était il y a 30 ans. Plusieurs conditions favorisent cet accès, notamment la présence de deux apporteurs de ressources au sein du couple, la stabilité d'un emploi en CDI, ainsi que l'aide financière des parents, devenue plus courante et plus importante. En 2013, un quart des accédants âgés de 25 à 44 ans ont reçu un don familial lors de l'achat, et l'apport personnel représentait environ un tiers du montant total de l'acquisition, un chiffre stable par rapport à 2001.

\section{Les dépenses de logement}

En 2018, 26,6 \% de la dépense de consommation finale des ménages était consacrée aux services liés au logement. Cela en fait le premier poste de dépense, bien loin devant les parts allouées à l'alimentation (17,1 \%) et aux transports (14,3 \%). Depuis les années 1990, la part des dépenses dédiées au logement a augmenté de manière significative, étant inférieure de 6,5 points par rapport à son niveau actuel.

\begin{wrapfigure}{r}{0.6\textwidth}
	\centering
\includegraphics[scale=0.5]{../../../Downloads/Screenshot 2024-09-17 at 18-23-33 Microsoft PowerPoint - CM_EL_intro_2024_2025 - CM_EL_intro_2024_2025.pdf}
\end{wrapfigure}

En 2013, 4,5 \% des locataires étaient en situation d'impayé de loyer ou de charges, un pourcentage similaire à celui de 2006. Par ailleurs, 11,5 \% des accédants rencontraient des difficultés de remboursement d'emprunt, contre 8,9 \% en 2006. Depuis le milieu des années 1980, les dépenses de logement des locataires ont considérablement augmenté. Les indices des loyers, à qualité constante, ont progressé plus rapidement que les prix à la consommation et le revenu disponible brut par ménage. Après déduction des aides au logement, un locataire du secteur privé dépensait en moyenne 50 \% de plus au m² pour son logement qu’un locataire du secteur social, alors qu’en 1984, cet écart était plus faible (43 \%). Cette hausse a été particulièrement marquée pour les ménages du premier quartile, en partie en raison d'un rattrapage de leurs conditions de confort par rapport aux autres ménages, comme l’a souligné Briant en 2010.

Depuis le début des années 2000, le niveau de revenu des locataires a diminué par rapport à celui des propriétaires. Dans le même temps, le taux d’effort, c’est-à-dire la part du revenu consacrée aux dépenses de logement, a augmenté. Cette hausse a été plus marquée pour les locataires du secteur privé, passant de 23,6 \% en 2001 à 28,4 \% en 2013, que pour ceux du secteur social, où il est passé de 20,2 \% à 24,1 \%. Cet accroissement de l'effort a particulièrement pesé sur les ménages les plus modestes, malgré le soutien des aides personnelles au logement, qui se sont concentrées sur ces foyers. Pour ces derniers, les aides au logement représentent plus de 30 \% du montant total de leurs dépenses de logement.
\newpage
\begin{wrapfigure}{r}{0.6\textwidth}
	\centering
\includegraphics[scale=0.5]{../../../Downloads/Screenshot 2024-09-17 at 18-30-05 Microsoft PowerPoint - CM_EL_intro_2024_2025 - CM_EL_intro_2024_2025.pdf}
\end{wrapfigure}

Depuis 1990, la part des dépenses de logement des locataires couverte par les prestations sociales s'est accrue. Cependant, en 2018, cette part a diminué, atteignant 14,6 \%. Cette baisse s'explique par deux facteurs principaux : l'abaissement de 5 € des différentes prestations sociales liées au logement et la mise en place du dispositif de réduction de loyer de solidarité (RLS) dans le secteur social.

\section{Politique du logement}

L'État joue un rôle important en régulant la construction, en définissant le droit de propriété, en encadrant les contrats de location et les loyers, et en subventionnant l'investissement ou la consommation. Historiquement, l'intervention publique s'est concentrée sur plusieurs aspects : la construction, notamment avec le développement du logement social, l'aide à l'accession à la propriété via le crédit et les taux aidés, ainsi que les aides monétaires directes au revenu des locataires, comme les APL. Dans les années 1970, la faillite des grands ensembles de logements sociaux a révélé une concentration des problèmes économiques et sociaux dans certaines zones. En réponse, les aides directes ont été étendues à de nouvelles catégories de bénéficiaires, tels que les étudiants.

Depuis 1984, pour contrer la baisse de l’offre locative privée, des incitations fiscales à l’investissement locatif ont été instaurées, allant du dispositif Quilès-Méhaignerie en 1984 au dispositif Pinel en 2014. Plus récemment, d’autres lois, telles que la loi DALO et la loi SRU, ont été mises en place pour réguler les baux, encadrer l’évolution et le niveau des loyers, taxer les logements vacants, et encourager la mixité sociale.

Les politiques du logement ont des effets complexes, rendant leur évaluation difficile. L’encouragement à la construction, qui vise à la fois à développer l’offre de logements et à soutenir l’emploi dans le secteur de la construction, comporte le risque de surinvestissement dans des zones où la demande est faible. Par ailleurs, les politiques de réglementation des loyers doivent trouver un équilibre entre une liberté totale, qui favorise la mobilité mais peut entraîner des loyers excessifs si le bailleur est en position de monopole, et un contrôle trop strict, qui risque de créer une pénurie de logements et de freiner la mobilité nécessaire.

Les aides « personnelles » à la consommation de logement s'élèvent en moyenne à 266 euros par mois. Elles bénéficient à 40 \% des locataires, dont la moitié dans le secteur social, un pourcentage qui a augmenté après 1984 avant de se stabiliser depuis 1996. En revanche, seulement 6 \% des accédants en bénéficient. Ces aides sont en baisse en raison de leur recentrage sur les ménages les plus modestes depuis la fin des années 1980, ainsi que de la diminution de l'accès à la propriété pour ces ménages.

Des études montrent que les aides au logement ont été partiellement absorbées par des hausses de loyer [Laferrère et Le Blanc, 2002 ; Fack, 2005]. L’encouragement à la construction locative privée n’est pas exempt d’effets pervers. En plus d'une mauvaise répartition géographique de l'offre, un impact sur les prix a également été observé dans certaines zones [Bono et Trannoy, 2013].

\chapter{État du logement en France et enjeux des politiques de logement}

\section{État du logement en France}

\subsection{Le logement en France depuis 30 ans}

Depuis quarante ans, l’enquête Logement de l’INSEE constitue le pivot central du dispositif statistique sur le logement en France. Elle permet d'analyser et d'évaluer l'évolution des comportements en matière de logement à travers les générations. Le choix du logement repose principalement sur des considérations de localisation, avec une proximité aux aménités essentielles. 

Au sein des agglomérations, le centre-ville concentre les emplois, ce qui pousse les ménages à faire des arbitrages entre le prix du logement et le coût du transport. Cependant, une tendance générale s'est dessinée vers l'éloignement des centres urbains, favorisant la maison individuelle au détriment de l'appartement, phénomène connu sous le nom de périurbanisation. 

Dans les années 60, la construction massive de logements sociaux a conduit à une concentration excessive des populations modestes, qui se sont ensuite tournées vers l'accession à la propriété de maisons situées plus loin des centres. 

Depuis 2008, on observe un retour à une préférence pour les appartements en immeubles collectifs. Les coûts environnementaux liés au transport automobile, le vieillissement de la population et la concentration des activités ont également favorisé la densification de certaines métropoles, entraînant une augmentation des écarts géographiques de prix et de loyers.

Par exemple, dans l'agglomération parisienne, le prix des logements à caractéristiques comparables est supérieur de près de 25 \% par rapport à d'autres régions. Ces différences spatiales peuvent s'expliquer par au moins deux types d’inadéquation. 

La première se manifeste lorsque les emplois sont trop éloignés des logements, rendant le coût de mobilité élevé pour les habitants. 

La seconde inadéquation concerne les niveaux de qualification et de formation qui ne répondent pas aux besoins du marché du travail. 

Ce phénomène est exacerbé par la ségrégation spatiale, où la qualité de l'éducation dans le lieu de résidence est souvent inférieure, et où les caractéristiques du voisinage peuvent freiner l'accumulation de capital humain. Ainsi, ces facteurs contribuent à creuser les disparités économiques et sociales au sein des territoires.

Globalement, la mobilité résidentielle a diminué depuis le début du XXIe siècle, principalement en raison du vieillissement de la population. Ce phénomène est particulièrement marqué dans le secteur locatif social, où l'âge des locataires en place augmente. 

De plus, la mobilité a également diminué chez les personnes plus âgées, probablement parce que leurs logements sont désormais mieux adaptés à leurs besoins. Cette stagnation dans les déplacements résidentiels reflète une tendance vers une sédentarité accrue, influencée par des facteurs démographiques et des conditions de vie plus satisfaisantes pour les seniors.

\subsubsection{Le confort de base s’est généralisé}

L’état des logements français en matière d’hygiène et de salubrité était dramatique dans les années qui ont suivi l’après-guerre. À cette époque, seuls un peu plus d’un quart des logements disposaient de WC intérieurs, et seulement 10 \% étaient équipés d’une baignoire ou d’une douche. Cependant, des progrès spectaculaires ont été réalisés, notamment avec la construction de nouveaux grands ensembles dans les années 70, qui ont su répondre aux attentes de leurs habitants. En 2020, parmi les 28,7 millions de ménages résidant en France métropolitaine, 78,7 \% estiment que leurs conditions de logement sont satisfaisantes ou très satisfaisantes, une proportion qui a augmenté de 2,1 \% par rapport à 2013. Cela témoigne d'une amélioration significative des conditions de vie au fil des décennies.
\newpage
\begin{wrapfigure}{r}{0.6\textwidth}
	\centering
\includegraphics[scale=0.4]{../../../Pictures/Screenshots/Capture d'écran 2024-09-22 075101}
\caption{Part des ménages estimant leurs conditions de logement satisfaisantes ou très satisfaisantes en 2020, selon le statut d’occupation et le type de logement (et évolution par rapport à 2013)}
\end{wrapfigure}

Si, en moyenne, la taille des résidences principales a augmenté, elle reste néanmoins inadaptée au nombre d’occupants. La superficie des maisons construites avant 2010 a gagné entre 1 et 3 m² selon la période de construction, tout comme celle des appartements bâtis avant 1949. Le surpeuplement a été divisé par deux entre 1984 et 2006, et depuis, il est stabilisé à 8 \%. Toutefois, entre 2013 et 2020, le taux de surpeuplement a légèrement progressé, passant de 8,4 \% à 8,7 \%. Ce phénomène touche encore 21 \% des ménages dans l’unité urbaine de Paris, ainsi que 16 \% à 17 \% des moins de 40 ans, et atteint 18 \% chez les ménages les plus modestes. Bien que l'amélioration des conditions de logement soit notable, elle ne doit pas occulter la question cruciale de la privation de domicile. Début 2012, 82 000 adultes étaient sans domicile, dont 8 000 dormaient dans des lieux non prévus pour l’habitation, tandis que les autres se trouvaient en centres d’hébergement souvent provisoires.

\subsubsection{Le logement comme placement}

Le logement est un actif en concurrence avec d'autres formes d’épargne, mais il présente la particularité d'être moins liquide. Il est souvent perçu comme un placement sûr, surtout pour les vieux jours. En effet, le taux de propriétaires atteint environ 75 \% à l’âge de la retraite et ne baisse que peu par la suite. De plus, le logement est un bien indivisible. Bien qu'il serait parfois optimal d'en acheter une partie et de louer l'autre, cela est généralement impossible, ce qui entraîne des niveaux de consommation et d’investissement non optimaux. Certaines formes plus souples de propriété existent, telles que la coopérative, la propriété partielle (\textit{shared property}) et la propriété sociale, mais elles restent peu développées en France.

Le logement peut être une maison individuelle ou un appartement. Actuellement, 80 \% des propriétaires vivent dans une maison, tandis que 75 \% des locataires résident en appartement. Dans le secteur du logement locatif privé, les mono-propriétaires d’immeubles ont pratiquement disparu. 

D'abord, les mono-propriétaires privés ont été affectés par les partages successoraux, les taxes et l’accès à la propriété. 

Ensuite, les investisseurs institutionnels ont réorienté leurs actifs dans les années 2000, vendant leurs immeubles à la découpe. 

Aujourd'hui, la quasi-totalité des mono-propriétaires d’immeubles collectifs sont désormais des bailleurs sociaux. Par ailleurs, la copropriété s’est généralisée, touchant 94 \% des logements collectifs du parc privé, dont la moitié sont occupés par leurs propriétaires.

\subsubsection{Acheter ou louer}

Le choix du statut d’occupation est influencé par plusieurs facteurs, tels que les contraintes de crédit (taux d’apport personnel, taux d’intérêt, durée des emprunts), le revenu, les anticipations de prix et de loyer, ainsi que la position dans le cycle de vie et les prévisions de mobilité. Devenir propriétaire représente un choix d’investissement, impliquant un arbitrage entre rendement et risque. Les jeunes, souvent mobiles professionnellement, préfèrent d’abord louer un petit appartement en centre-ville avant de devenir propriétaires lorsque leur situation professionnelle est plus stable. Entre 1996 et 2016, les prix à la consommation ont augmenté de 31 \%, le revenu disponible brut par ménage de 40 \%, tandis que le prix des logements anciens a été multiplié par 2,52 (et 2,67 en Île-de-France).

Après une baisse consécutive à la crise de 2008, la reprise a été soutenue, particulièrement en Île-de-France. Les prix ont ensuite légèrement diminué à partir de 2012, mais ont connu une reprise en 2016. La hausse des prix freine l’accession à la propriété pour les primo-accédants, tandis que cela n'affecte pas les ménages déjà propriétaires. Avec la hausse des taux d’intérêt en 2022-2023, les prix des biens immobiliers diminuent, mais cette augmentation des taux constitue un frein à l’accession à la propriété. Les ménages non propriétaires forment une catégorie de plus en plus défavorisée au fil du temps. Les anciennes générations de propriétaires pauvres vivant dans des logements de mauvaise qualité ont disparu, rendant la propriété occupante moins fréquente parmi les plus modestes qu'auparavant. De plus, le revenu est devenu un déterminant plus important de l’accès à la propriété qu’il y a trente ans.

Les éléments qui facilitent l’achat d’un logement incluent plusieurs facteurs clés. Tout d’abord, le fait d’être en couple avec deux revenus et la stabilité du revenu jouent un rôle crucial. De plus, l’aide des parents est significative : en 2013, un quart des accédants âgés de 25 à 44 ans a reçu un don, contre un cinquième en 2001. Cela souligne l’accroissement des inégalités intra-générationnelles entre ceux que les parents peuvent aider et les autres. Par ailleurs, en Europe, le taux de propriétaires varie considérablement d’un pays à l’autre. Il est particulièrement élevé en Europe du Sud, qui est restée longtemps plus rurale, avec 78 \% en Espagne, tandis qu'il est faible dans les régions du Centre et du Nord, plus industrielles et urbaines, où la location a été encouragée, atteignant seulement 50 \% en Autriche.

Depuis la crise de 2008, l’évolution du marché immobilier est marquée par des contrastes. La crise a freiné l’accession à la propriété des jeunes, qui se tournent davantage vers la location. Cependant, certains analystes soulignent que cette tendance est également liée à une mobilité professionnelle et sentimentale accrue. En France, l’attrait pour la propriété immobilière semble demeurer fort : 43 \% des ménages souhaitant changer de logement expriment le désir de devenir propriétaires, comme c’était le cas par le passé.

\subsection{Parc du logement : évolution, répartition géographique et caractéristiques}

\subsubsection{Evolution récente du logement en France}

\begin{wrapfigure}{r}{0.7\textwidth}
	\centering
	\includegraphics[scale=0.5]{../../../Pictures/Screenshots/Capture d'écran 2024-09-22 080959}
	%\caption{Part des ménages estimant leurs conditions de logement satisfaisantes ou très satisfaisantes en 2020, selon le statut d’occupation et le type de logement (et évolution par rapport à 2013)}
\end{wrapfigure}

En 2021, la France métropolitaine comptait 36,2 millions de logements, soit 12,6 millions de plus qu’en 1982. Cette évolution est principalement attribuée à la construction de logements neufs ainsi qu’à la transformation de locaux non résidentiels en logements, un processus connu sous le nom de réaffectations. Toutefois, les démolitions et les changements d’usage de locaux d’habitation, appelés désaffectations, contribuent à réduire le parc immobilier. De plus, le taux de croissance du parc a diminué, passant de 1,1 \% à 0,7 \% par an ces dernières années, en grande partie en raison de la chute de la construction neuve liée à la crise sanitaire.

\subsubsection{Évolution annuelle moyenne du nombre de logements par catégorie}
\newpage
\begin{wrapfigure}{r}{0.7\textwidth}
	\centering
\includegraphics[scale=0.4]{../../../Pictures/Screenshots/Capture d'écran 2024-09-22 081402}
	%\caption{Part des ménages estimant leurs conditions de logement satisfaisantes ou très satisfaisantes en 2020, selon le statut d’occupation et le type de logement (et évolution par rapport à 2013)}
\end{wrapfigure}

Depuis 1982, le nombre de résidences principales en France a augmenté de 52 \%. Cette hausse est attribuée à la croissance démographique de 20 \%, mais également à la baisse de la taille des ménages, due à des facteurs tels que les mises en couple plus tardives, les ruptures d’union et le vieillissement démographique. Cependant, ces dernières années, la part des résidences principales a légèrement diminué, passant de 82,6 \% en 1982 à 81,8 \% en 2021. 

Parallèlement, depuis le début des années 2010, le nombre de résidences secondaires et de logements occasionnels augmente plus rapidement que l’ensemble du parc immobilier. De plus, les logements vacants ont augmenté depuis 2006 à un rythme supérieur à celui de l’ensemble du parc, bien que cette hausse ait tendance à s’atténuer ces dernières années. La vacance s’explique par une inadéquation entre l’offre et la demande, résultant de facteurs tels que la taille, la localisation, le prix ou le mauvais état des logements.

\subsubsection{Évolution annuelle moyenne du nombre de logements par type d'habitat}

\begin{wrapfigure}{r}{0.7\textwidth}
	\centering
\includegraphics[scale=0.4]{../../../Pictures/Screenshots/Capture d'écran 2024-09-22 081804}
	%\caption{Part des ménages estimant leurs conditions de logement satisfaisantes ou très satisfaisantes en 2020, selon le statut d’occupation et le type de logement (et évolution par rapport à 2013)}
\end{wrapfigure}

En 2021, l’habitat individuel représentait 55 \% des logements, un chiffre qui reste stable par rapport à 1982. Cependant, depuis 2008, sa part a légèrement reculé, car le nombre de logements collectifs augmente deux fois plus vite que celui des logements individuels. Depuis 2013, les logements achevés chaque année sont en effet plus souvent collectifs qu’individuels, reflétant une tendance vers une densification urbaine et une préférence croissante pour les logements en collectivité.

\subsubsection{Répartition de l'habitat individuel et collectif selon la taille de l’unité urbaine en 2021}

\begin{wrapfigure}{r}{0.7\textwidth}
	\centering
\includegraphics[scale=0.5]{../../../Pictures/Screenshots/Capture d'écran 2024-09-22 082041}
	%\caption{Part des ménages estimant leurs conditions de logement satisfaisantes ou très satisfaisantes en 2020, selon le statut d’occupation et le type de logement (et évolution par rapport à 2013)}
\end{wrapfigure}

Depuis le début des années 1980, la répartition du parc de logements selon la taille de l’unité urbaine a évolué sous l’effet de deux phénomènes majeurs : l’hétérogénéité du territoire et l’extension urbaine. 

En 2021, 16 \% des résidences principales se situaient dans l’unité urbaine de Paris, tandis que 20 \% se trouvaient dans une commune hors unité urbaine. Les résidences secondaires et les logements occasionnels sont bien plus souvent localisés dans une commune hors unité urbaine ou dans une unité urbaine de moins de 100 000 habitants, avec 77 \% contre seulement 52 \% des résidences principales. 

Depuis 2008, le nombre de résidences secondaires et de logements occasionnels a crû un peu plus vite dans les unités urbaines de 100 000 habitants ou plus, y compris celle de Paris, que dans les autres unités urbaines. 

En outre, la majorité des logements vacants (60 \%) se trouvent dans une commune hors unité urbaine ou dans une unité urbaine de moins de 100 000 habitants. Ainsi, la part des logements vacants dans les unités urbaines de moins de 100 000 habitants a augmenté, passant de 29 \% en 1982 à 35 \% en 2021.
\subsubsection{Répartition des résidences principales selon le statut d'occupation}

\begin{center}
	\includegraphics[scale=0.5]{../../../Pictures/Screenshots/Capture d'écran 2024-09-22 082405}
\end{center}

En 2021, 58 \% des ménages étaient propriétaires de leur résidence principale. Cette part est stable depuis 2010, après avoir augmenté de façon continue depuis 1982, où elle n'était que de 50 \%. 

La proportion de propriétaires sans charges de remboursement a sensiblement crû jusqu’en 2010, atteignant 38 \% contre 27 \% en 1982, en partie en raison du vieillissement de la population.
 
La part des propriétaires accédants s’est stabilisée à 20 \% depuis environ quinze ans, en raison de l’allongement des durées d’emprunt. Parallèlement, la part des ménages locataires de leur résidence principale se maintient aux alentours de 40 \% depuis 1990, contre 41 \% en 1982.

\subsubsection{Evolution du logement en DOM}

Dans les départements d’outre-mer (DOM) hors Mayotte, le parc total des logements augmente plus rapidement qu’en métropole, en raison de la forte croissance de la population. En 2021, dans les DOM, 82 \% des logements étaient des résidences principales, 6 \% des résidences secondaires ou logements occasionnels, et 12 \% des logements vacants. L’habitat individuel représente les deux tiers de l’ensemble des logements ultramarins, une proportion nettement plus élevée qu’en métropole.

\subsubsection{Quelques statistiques récentes}

\begin{center}
	\includegraphics[scale=0.4]{../../../Pictures/Screenshots/Capture d'écran 2024-09-22 082916}
	
	\includegraphics[scale=0.4]{../../../Pictures/Screenshots/Capture d'écran 2024-09-22 083044}
	
	\includegraphics[scale=0.4]{../../../Pictures/Screenshots/Capture d'écran 2024-09-22 083144}
	
	\includegraphics[scale=0.4]{../../../Pictures/Screenshots/Capture d'écran 2024-09-22 083254}
	
	\includegraphics[scale=0.4]{../../../Pictures/Screenshots/Capture d'écran 2024-09-22 083345}
	
\end{center}

\section{Enjeux des politiques du logement}

\subsection{Enjeux et acteurs des politiques du logement}



Le logement est un bien nécessaire, qui présente de nombreuses externalités. De plus, le marché du logement souffre de multiples imperfections, telles que l'asymétrie et l'imperfection des informations concernant le logement, ainsi que sur les acheteurs, les vendeurs et les locataires. Le logement est considéré comme un bien tutélaire, ce qui confère à l'État la légitimité d'intervenir. L'État réglemente la construction, détermine le droit de propriété, régule les contrats de location et les loyers, et subventionne l’investissement ou la consommation. En 2014, les aides publiques au logement s'élevaient à 39,9 milliards d’euros, soit 1,9 \% du PIB. Par ailleurs, le logement est également soumis à des taxes, dont le montant total s’élève à 64,2 milliards d’euros, ou 36,2 milliards d’euros en excluant la TVA et la CSG.

\subsubsection{Politique du logement}

Historiquement, l’intervention publique s’est centrée sur la construction, notamment à travers le logement social, l’aide à l’accession via le développement du crédit et des taux aidés, ainsi que les aides monétaires directes au revenu des locataires (APL). Depuis 1984, pour lutter contre la baisse de l’offre locative libre, de nombreux encouragements fiscaux à l’investissement locatif privé ont été mis en place, allant du dispositif Quilès-Méhaignerie en 1984 au dispositif Pinel en 2014. D’autres lois, telles que le DALO et la SRU, ont eu pour objectif de réglementer les baux, l’évolution, voire le niveau des loyers, de taxer les logements vacants, ou d’encourager la mixité sociale.

D’autres politiques du logement consistent à encourager la construction. Cet encouragement peut parfois viser deux objectifs : le développement de l’offre de logements et le soutien à l’emploi dans le secteur de la construction. Cependant, cela comporte des risques de surinvestissement dans des zones où la demande est faible. Les politiques de réglementation des loyers doivent trouver un équilibre délicat entre une liberté totale, qui facilite la mobilité, et un contrôle trop strict, qui peut générer une pénurie et entraver la mobilité. En moyenne, les aides à la consommation de logement s’élèvent à 266 euros par mois. Elles bénéficient à 40 \% des locataires, dont la moitié réside dans le secteur social. Ces aides, qualifiées de « personnelles », concernent également environ 6 \% des accédants.

Dans leur livre paru en 1991, intitulé « \textit{Politique du logement. 50 ans pour un échec }», Lefebvre, Mouillart et Occhipinti insistent sur l’incapacité française à sortir d’une situation de crise. En 2014, le rapport de la fondation Abbé Pierre rappelle que le mal persiste. Cette situation résulte d’une hausse croissante des prix de l’immobilier et de la stagnation des revenus, ce qui accroît la fragilité de certains ménages. Les politiques du logement répondent à une vaste gamme d’enjeux, parfois contradictoires, qui impliquent des moyens et des responsabilités très différenciés.

Parmi les enjeux, on peut noter le poids des enjeux économiques. Le secteur du bâtiment représente un domaine d’activité stratégique, jouant un rôle clé pour les politiques de l’emploi et la collecte des cotisations sociales, en raison du faible risque de délocalisation. Le secteur du logement constitue également une assiette fiscale de première importance, essentielle pour l’équilibre des comptes nationaux. De plus, le logement représente le premier poste de dépense des ménages, en faisant un facteur clé de toute politique axée sur le pouvoir d’achat et visant à influencer les arbitrages entre épargne et consommation, notamment dans la perspective de la retraite. Ainsi, en période de crise, il est tentant d'utiliser le logement pour relancer l’activité, dégager des recettes fiscales ou maîtriser la dépense publique.

\subsubsection{La multiplicité des enjeux sociaux des politiques du logement}

Le logement est un élément fondamental de la cohésion sociale. C'est à la fois un besoin et un droit, inscrit dans la loi française du 6 juillet 1989. Plusieurs politiques se mettent en œuvre pour assurer la planification spatiale, lutter contre le mal-logement et répondre aux aspirations à la propriété et à la maison individuelle. Les départements jouent un rôle prépondérant dans la lutte contre le mal-logement et les mécanismes d’exclusion. Parallèlement, le secteur associatif intervient principalement auprès des personnes les plus en difficulté pour accéder au logement de droit commun.

\subsection{Les grandes questions : Construire un nombre déterminé de logements par an ? l’accession à la propriété ? la régulation des marchés locatifs, les logements sociaux aux plus pauvres ?}

\subsubsection{Faut-il construire 500 000 logements par an ?}

Plusieurs études montrent qu’il existerait en France un déficit compris entre 800 000 et 1 million de logements. Pour combler ce déficit dans les meilleurs délais, il faudrait s’engager dans un ambitieux programme de 500 000 logements mis en chantier chaque année pendant cinq ans. Toutefois, cet objectif semble difficile à atteindre, car au cours des 30 dernières années, la production de logements a été d'environ 349 000 unités par an. L’Île-de-France est, de loin, la région où la construction neuve a été la plus faible au cours des quinze dernières années, avec une moyenne annuelle de 3,4 logements neufs pour 1 000 habitants.

\subsubsection{Faut–il favoriser une France de propriétaires ?}

La nécessité de développer la propriété repose sur plusieurs arguments : d'une part, elle constitue une aspiration forte des ménages, et d'autre part, la France serait « en retard » par rapport à la plupart de ses voisins européens. Le premier argument est confirmé par toutes les enquêtes, tandis que le second doit être pris avec du recul. Avec 58 \% de ménages propriétaires de leur résidence principale, la France se situe en position médiane, comparable à l’Autriche et aux Pays-Bas. Les taux les plus élevés en Europe se situent entre 65 \% et 69 \%, pour des pays comme le Royaume-Uni, la Belgique, l’Italie, la Finlande ou la Pologne. En revanche, les taux les plus bas varient entre 38 \% et 46 \%, pour l’Allemagne, le Danemark, la Suède et la Suisse. Des pays comme l’Espagne, la Lituanie, la Slovaquie et la Roumanie affichent des taux exceptionnellement élevés, entre 80 et 90 \%.

Il n’y a pas de relation claire entre le niveau de développement économique et le taux de propriétaires. La plupart des pays européens ont développé des politiques favorisant le statut de propriétaire occupant. Cependant, la crise américaine des subprimes a freiné les politiques de facilitation d’accession à la propriété. Cette crise a révélé l’ampleur des risques liés à une accession à la propriété forcée pour les ménages à bas revenus. En France, depuis 1985, tous les systèmes d’accession à la propriété, qu'ils soient aidés ou non, sont assortis d’un haut niveau de garantie et de protection de l’emprunteur. Par ailleurs, la France met en œuvre des moyens fiscaux efficaces pour développer l’investissement privé dans l’offre locative.

\subsubsection{Faut-il réguler les marchés locatifs ?}

L’attention portée à l’existence d’une offre de logements diversifiée est une caractéristique importante des politiques du logement en France. Le parc locatif privé est relativement abondant, visant à faciliter l’accès au logement des jeunes et des personnes les plus mobiles. L’État favorise l’investissement locatif par le biais de mesures fiscales depuis le milieu des années 1980. En 2013, la loi ALUR pour l’accès au logement et un urbanisme rénové est adoptée, débouchant notamment sur de nouvelles modalités d’encadrement des loyers dans le secteur privé. Un débat existe entre la faible réglementation et la responsabilité sociale du bailleur privé, notamment en ce qui concerne la fixation des loyers.

Cependant, les chiffres internationaux plaident en faveur de la responsabilité sociale du bailleur et des contraintes qui l'accompagnent. Par exemple, bien que l’Angleterre soit l’un des pays les plus libéraux, l’offre locative privée ne représente que 11 \% du stock des résidences principales, contre 22 \% en France. En revanche, en Allemagne, le secteur est beaucoup plus régulé qu’en France, mais elle demeure l’un des pays où le secteur locatif privé est le plus abondant, avec près de 50 \% du parc total. C’est également le cas en Suisse, en Suède et au Danemark. Il est donc difficile de soutenir l’hypothèse selon laquelle la réglementation nuirait à l’offre.

Le modèle allemand va amener à remettre en cause l’équilibre acquis en France depuis 1989, notamment avec la loi ALUR. Cette loi introduit une mesure d’encadrement des loyers dans les villes à marchés tendus, s’appuyant sur un réseau d’observatoires créés à cet effet. Cela permettra aux préfets de fixer des « loyers de référence » qui feront autorité dans une fourchette allant de -30 \% à +20 \%.

\subsubsection{Faut-il Réserver le logement social aux plus pauvres ?}

Les marchés du logement dans les grandes villes françaises génèrent de l’exclusion. Ainsi, l’offre réglementée de l’habitat social reste l’outil principal pour garantir des conditions de logement dignes à un prix abordable. Cependant, des questions se posent quant à l’efficacité du modèle français du logement social. À l’origine, le logement social était réservé aux classes ouvrières, puis aux salariés à revenu modeste. Il a ensuite été utilisé comme un instrument permettant à la classe moyenne d’accéder à la propriété. La hausse de la pauvreté au cours des années 80 a conduit à la mise en place de politiques de droit au logement, visant tant les populations exclues de l’emploi stable que celles écartées du marché du logement décent.

Ainsi se pose la question de la mixité au sein du même parc social et des plafonds de ressources. En Europe, il existe trois modèles d’offre sociale. Le premier modèle, dit « universaliste », utilise le parc public comme outil de régulation globale du marché. Il impose des loyers faibles, grâce à une offre abondante et sans conditions d’accès. Des exemples incluent la Suède, les Pays-Bas et le Danemark. Cependant, ce modèle est en voie de disparition sous les effets de la doctrine de libre concurrence prônée par l’UE.

Le second modèle, dit « résiduel », réserve le parc social aux ménages les plus en difficulté, qui ne peuvent accéder au logement privé. Ce modèle entérine une approche d’un logement social totalement hors marché, souvent paupérisé et très stigmatisé. C’est ce modèle qui répond le mieux aux injonctions libérales de l’UE. Historiquement présent dans les pays du Sud de l’Europe, il s’est également diffusé au Royaume-Uni. Ce modèle tend à gagner les pays d’Europe orientale et l’Allemagne, où le parc social fond à grande vitesse, sous l’effet de ventes massives et de sorties de logements de leur statut social à durée limitée.

Enfin, le modèle intermédiaire entre les deux précédents, le modèle « généraliste », comporte des conditions d’accès relativement ouvertes. Il intègre la possibilité d’une mixité sociale en son sein, permettant le maintien dans les lieux même si les conditions initiales ne sont plus respectées. C’est le modèle adopté par la France ; cependant, cette mixité pose problème. Deux arguments sont évoqués, en particulier en Île-de-France : 

Le fort accroissement des inégalités territoriales en matière de logement depuis le début des années 2000, sous l’effet de hausses de prix généralisées qui ont creusé les écarts géographiques de capacité d’accès au logement privé.

Le constat généralisé de l’écart entre les grands principes posés par la réglementation en matière de plafonds de ressources et la réalité de la demande de logements sociaux ainsi que du profil des ménages qui y accèdent.

\subsubsection{Prendre en compte les questions environnementales et sociales}

Des réformes ont été entreprises pour réduire les conditions indignes de logement. Cependant, depuis 2013, où le taux était à 1,8 \%, beaucoup reste encore à faire, car les résultats sont insuffisants. La précarité énergétique concerne 30 \% des Français les plus pauvres. Deux instruments de rénovation électrique ont été utilisés : Le crédit d’impôt pour la transition énergétique (CITE), qui a surtout profité aux ménages aisés. « MaprimeRénov », qui a succédé au CITE et est ouvert à un plus grand nombre de travaux énergétiques. Ces deux instruments ont été à peu près satisfaisants.

Une approche qualitative de la politique du logement vise à restreindre l’aide aux zones tendues, avec des prix supérieurs à 2000 €/m² et un taux d’occupation de 90 \%. L’idée est de permettre l’adaptation du parc aux défis environnementaux ou sociétaux, par un financement de travaux énergétiques et le soutien aux personnes vulnérables, ainsi que l’adaptation des logements au vieillissement de la population.

\subsubsection{Conclusion}

Les politiques du logement ont-elles échoué à maîtriser les marchés immobiliers ? On peut considérer que le niveau élevé des prix satisfait à la fois une majorité de propriétaires et les responsables des budgets publics, qui y voient une manne fiscale fort opportune. Les difficultés pour se loger persistent et s'aggravent parfois. Les questions sur l’efficacité des politiques du logement restent très complexes et nécessitent des réflexions approfondies. Cependant, les expériences des voisins européens pourraient mieux éclairer les grands débats sur la construction, la propriété, la régulation des marchés et le rôle du logement social.

\chapter{Conditions de logement en France}

\section{Évolution des conditions de logement depuis 1960}

\subsection{État des lieux}

Depuis la fin des années 1960, le parc de logements en France métropolitaine a connu une expansion significative, passant de 18,7 millions à plus de 35 millions entre 1968 et 2018. Cette croissance s'accompagne de transformations profondes des caractéristiques des logements, avec l'introduction d'équipements modernes tels que le téléphone et l'eau courante, ainsi qu'une augmentation du nombre de pièces et des changements dans les modes de chauffage. Ces évolutions ont été rendues possibles grâce à l'accroissement du parc, qui a impliqué la construction sur de nouveaux terrains, au renouvellement des bâtiments, remplaçant ainsi les anciennes constructions par des structures modernes, et à la réhabilitation de logements anciens, permettant ainsi de préserver le patrimoine tout en améliorant le confort des habitants.

Après la Seconde Guerre mondiale, la France fait face à un déficit de logements marqué par une faiblesse de la construction à la fin des années 1930, avec seulement 60 000 logements neufs par an. Ce phénomène s'explique par plusieurs facteurs, notamment le gel des loyers et l'ignorance de l'exode rural par les autorités publiques, ainsi que par les destructions massives causées par le conflit, avec 500 000 logements détruits et 1 400 000 endommagés. La reprise démographique entre 1946 et 1954 aggrave encore les conditions de logement, avec une augmentation de la population de 285 000 personnes par an, après une baisse de 100 000 entre 1936 et 1946. En conséquence, une personne sur trois vit dans une situation de surpeuplement. Les conditions sanitaires sont également préoccupantes, selon le recensement de 1946 : 9 résidences principales sur 10 disposent de l'électricité, mais seulement 6 \% possèdent une douche ou baignoire, 37 \% l'eau courante (13 \% dans les communes rurales) et 20 \% des logements ont des WC intérieurs.

À partir de 1953, l'État français commence à investir directement dans le logement à travers une politique de construction axée sur plusieurs mesures, notamment une forte aide à la pierre et des financements publics diversifiés. L'accent est alors mis sur la construction neuve, considérant le logement existant comme largement hors normes. Cependant, dès les années 1970, la perception du parc ancien évolue : il est désormais considéré comme un patrimoine à sauvegarder, entretenir et réhabiliter. L'État s'engage davantage dans l'amélioration du confort des logements, délaissant l'idée d'une simple croissance en volume pour se concentrer sur la qualité de vie des habitants [Fribourg, 2008].

\subsection{Accroissement du parc immobilier}

\begin{wrapfigure}{r}{0.6\textwidth}
	\centering
\includegraphics[scale=0.4]{../../../Pictures/Screenshots/Capture d'écran 2024-10-08 193523}
\end{wrapfigure}
Entre 1968 et 2015, le nombre de résidences principales en France a connu une augmentation significative, passant de 15,6 à 28,1 millions, soit une hausse de 80 \%. Ce phénomène s'explique par plusieurs facteurs, notamment l'accroissement de la population, qui est passée de 49,5 à 64 millions de personnes, ainsi que par la baisse de la taille moyenne des ménages. Le taux d'accroissement annuel moyen du nombre de logements s'élève à + 1,3 \%, bien que ce taux ne soit pas constant sur l'ensemble de la période. Les évolutions des logements présentent des contrastes marqués par rapport à celles de la population, révélant des dynamiques complexes dans le secteur du logement.

\subsection{Évolution de la composition du parc de logements}

Initialement, le cumul des aides à la pierre (destinées à la construction neuve), des aides à la personne (pour les logements conventionnés, généralement neufs) et des aides fiscales (favorisant l'accession à la propriété dans le neuf) favorisait la construction de nouveaux logements. Cependant, à la suite du rapport Nora-Eveno en 1976 sur l'amélioration de l'habitat ancien, la perception du parc ancien évolue. Il est désormais considéré comme un patrimoine à sauvegarder, englobant des structures telles que les anciens HLM, les maisons de ville, les immeubles haussmanniens et les pavillons des années 1930. Cette nouvelle approche valorise la durabilité des logements, entraînant un quadruplement du montant des aides budgétaires consacrées à l'amélioration. De plus, des organismes HLM se voient désormais autorisés à acquérir et réhabiliter des logements anciens, marquant un tournant dans la politique du logement.

\subsection{Le développement des maisons individuelles}

L'augmentation du nombre de résidences principales en France a été accompagnée d'une progression du pourcentage de maisons individuelles, au détriment des logements collectifs, ainsi que d'une hausse du pourcentage de propriétaires, au détriment des locataires. Ces évolutions sont en lien avec la périurbanisation du territoire et le développement de l'habitat pavillonnaire. 
\begin{wrapfigure}{r}{0.6\textwidth}
	\centering
\includegraphics[scale=0.4]{../../../Pictures/Screenshots/Capture d'écran 2024-10-08 194659}
\end{wrapfigure}
Ces mouvements ont été favorisés par de nouveaux modes d'intervention de l'État, des banques et des promoteurs dans la production de logements à partir des années 1970. Parmi ces changements, on note une transition d'une logique d'aide à la pierre vers une logique d'aide à la personne, la banalisation des financements de l'accession à la propriété, désormais accessibles par les circuits bancaires ordinaires, et un accroissement continu de la part des aides fiscales.
\newpage
\begin{wrapfigure}{r}{0.6\textwidth}
	\centering
\includegraphics[scale=0.4]{../../../Pictures/Screenshots/Capture d'écran 2024-10-08 194850}
\end{wrapfigure}
La part des maisons individuelles a augmenté entre 1968 et 1999, puis s'est stabilisée, atteignant 56 \% des résidences principales en 2015, contre 40 \% en 1968. Les différences sont plus marquées selon la période de construction. En effet, la part des maisons individuelles est plus importante parmi les résidences principales construites avant 1949 ou depuis les années 1980. Par ailleurs, la reconstruction du parc immobilier pendant les Trente Glorieuses a été caractérisée par la mise sur le marché d'un pourcentage élevé d'appartements, atteignant 60 \%.

\subsection{Évolution des statuts d'occupation}
\begin{wrapfigure}{l}{0.6\textwidth}
	\centering
\includegraphics[scale=0.3]{../../../Pictures/Screenshots/Capture d'écran 2024-10-08 195138}
\end{wrapfigure}
Le statut d'occupation du logement est lié au type de résidences principales. Depuis le recensement de 1968, le pourcentage de propriétaires n'a cessé d'augmenter, atteignant 58 \% des résidences principales en 2015, contre 35,5 \% en 1954. La répartition géographique des propriétaires reflète la répartition des maisons et des appartements. Ainsi, la part des propriétaires est plus élevée dans les départements ruraux et plus faible dans ceux abritant une grande métropole.

\subsection{Uniformisation des conditions sanitaires}

Moins d'un logement sur deux comprenait une baignoire ou une douche en 1968, tandis qu'en 2015, la quasi-totalité des résidences principales en était équipée (99,5 \%). La majorité des logements antérieurs à 1949 ont été mis aux normes au cours des années 1970 et 1980. En 1968, plus de 95 \% des logements construits depuis 1949 possédaient déjà leur propre installation. Par ailleurs, parmi les logements antérieurs à 1949 dans les communes rurales de moins de 1 000 habitants, la part de ceux équipés d'une baignoire ou d'une douche est passée de 19 \% à 96 \% entre 1968 et 2006.
\begin{wrapfigure}{r}{0.65\textwidth}
	\centering
\includegraphics[scale=0.45]{../../../Pictures/Screenshots/Capture d'écran 2024-10-08 195538}
\end{wrapfigure}

En 1968, seules 36 \% des résidences principales étaient dotées du chauffage central (collectif ou individuel). En 2015, ce chiffre a atteint 89 \%, dont 18 \% bénéficient du chauffage central collectif (15 \% en 1968), 42 \% du chauffage central par une chaudière individuelle (21 \% en 1968), 29 \% du chauffage électrique, et 11 \% utilisent d'autres moyens. 

Les combustibles ont évolué : le charbon a disparu (1 \% en 1999), la part du mazout/fioul a diminué (25 \% en 1990, 12 \% en 2015), tandis que les pourcentages du gaz et de l'électricité ont augmenté, passant de 32 \% et 25 \% en 1990 à 37 \% et 33 \% en 2015.

\subsection{Opinions des ménages sur les conditions}

Malgré ces rattrapages, l'expression « crise du logement » est d'actualité depuis les années 2000. La forte hausse des loyers dans les parcs privé et social, ainsi que celle du prix des logements au cours des années 2000 et 2010, ont engendré de nouvelles inégalités. 

Entre 1992 et 2013, le taux d'effort net moyen des locataires du parc privé est passé de 23 \% à 30 \%, tandis que pour ceux du parc social, il est passé de 19 \% à 24 \%. Le taux d'effort des accédants à la propriété varie peu (proche de 25 \%), mais la sélectivité s'est accrue. De plus, le revenu annuel moyen des accédants est de 47 800 euros en 2013, contre 37 800 euros en 2002.
\newpage
\begin{wrapfigure}{r}{0.65\textwidth}
	\centering
	\includegraphics[scale=0.7]{../../../Pictures/Screenshots/Capture d'écran 2024-10-08 200201}
\end{wrapfigure}
Si les principales normes de confort intérieur ne distinguent plus les logements urbains des autres résidences principales (rurales ou périurbaines), l'environnement de ces dernières les rend plus attractives. En effet, des facteurs tels que le bruit, l'insécurité, le surpeuplement et la qualité de l'air sont autant de critères défavorables à la ville.

\section{Les conditions de logement en Ile-de-France}

\subsection{Le parc de logements}

En 2018, la région Île-de-France comptait 5 220 000 logements, contre 5 122 000 en 2013, 4 891 000 en 2006 et 4 109 000 en 1984. Cela représente 18 \% du parc principal national. Paris en comporte 1 194 500, mais sa part a diminué, passant de 23 \% en 2013 à 28 \% en 1984. 

L'essentiel de la croissance régionale a eu lieu en grande couronne, avec 40 \% en 2013, contre 35 \% en 1984. La région se caractérise par une proportion élevée d'appartements (72 \%), de 2,8 pièces et d'une superficie moyenne de 60 m². En 2018, 21 \% des résidences avaient une taille inférieure à 40 m², contre à peine plus de 8 \% en province. Cette part est plus élevée à Paris (39 \%) et moindre en grande couronne (11 \%).
\newpage
\begin{wrapfigure}{l}{0.5\textwidth}
	\centering
	\includegraphics[scale=0.4]{../../../Pictures/Screenshots/Capture d'écran 2024-10-08 204458}
\end{wrapfigure}
Entre 2006 et 2013, le nombre de résidences principales n’a augmenté que de 231 000 unités, soit 31 500 logements par an. Cette progression annuelle est l’une des plus faibles des 30 dernières années. 

La croissance des résidences principales est atténuée par plusieurs facteurs : l’amplification des destructions d’immeubles et des restructurations de logements, l’augmentation des regroupements d’appartements (pour accroître la surface), ainsi que la légère hausse de la vacance et la transformation de résidences principales en résidences secondaires.

\subsection{Le mal-logement}

La quasi-totalité des ménages franciliens vivent dans des logements équipés du confort sanitaire (eau, toilettes intérieures et installations sanitaires), avec seulement 72 400 Franciliens (0,6 \%) n’en disposant pas. Près de 10 \% souffrent du mal logement, qui intègre des notions de qualité (équipements sanitaires, défauts affectant le logement ou l’immeuble, etc.), de surpeuplement, de coût par rapport au revenu, d’environnement du logement et de situation de sans-abri.

Deux catégories de situations peuvent être distinguées : les personnes disposant d’un logement, mais ayant des conditions de logement difficiles (976 900) et les personnes privées de domicile personnel (159 000).

La première catégorie de mal-logés regroupe deux grands sous-ensembles : les personnes vivant dans des logements privés de confort (47,3 \%) et celles occupant des logements fortement surpeuplés (52,7 \%). Parmi eux, 72 000 (7,4 \%) cumulent les deux difficultés.

Les personnes privées de confort vivent dans des logements qui présentent un danger pour la santé ou la sécurité de leurs habitants, ou qui ne possèdent pas les équipements sanitaires élémentaires. Ces logements se trouvent souvent dans des immeubles considérés comme vétustes, dont 45 \% datent d’avant 1949. Ils sont plus souvent localisés à Paris et en petite couronne (respectivement 40 \% et 34 \%). La majorité relève du secteur locatif (72 \%), qu’il soit social ou privé (36 \% chacun).

Les personnes vivant dans des logements surpeuplés manquent en moyenne d'au moins deux pièces par rapport au nombre de personnes qui les occupent. La plupart des ménages en surpeuplement accentué se trouvent dans des appartements (95 \%) et sont majoritairement localisés en petite couronne (50 \%, contre 37 \% de ménages vivant en petite couronne). 

Ces ménages relèvent du secteur locatif dans 81,5 \% des cas, à parts égales entre le social et le privé. Dans le parc social, les logements font en moyenne 65 m², contre 34 m² dans le parc locatif privé.

La deuxième catégorie de mal-logés regroupe deux grands sous-ensembles.

Le premier sous-ensemble concerne les enfants non étudiants, avec 63 300 enfants de plus de 25 ans qui ont quitté le domicile parental et qui s’y réinstallent. Cette population est plutôt jeune (36 ans) et exerce souvent des métiers d’employés ou d’ouvriers. Les principales causes de retour incluent la séparation (41 \%), la perte d’emploi ou des problèmes financiers. De plus, 53 600 enfants ne peuvent pas partir du domicile parental, leur situation économique étant plus difficile que celle des enfants qui reviennent chez leurs parents. Les causes du maintien chez les parents sont des difficultés économiques, les parents eux-mêmes étant relativement fragiles.

Le second sous-ensemble concerne les personnes sans lien familial et hébergées. Parmi elles, 45,5 \% sont des moins de 60 ans dont les ressources ne suffisent pas à acquitter un loyer. Ces individus sont souvent accueillis suite à une séparation, des problèmes financiers ou un désir de se rapprocher de leur emploi. Les difficultés économiques expliquent ces installations chez des tiers. Pour les plus de 60 ans, quel que soit leur niveau de ressources, ils représentent 54,5 \%. Retraités pour la plupart et âgés de 74 ans en moyenne, ils affichent un profil moins modeste. Ces personnes sont fréquemment hébergées par des ménages également âgés (69 ans) et souvent par des ménages sans enfant (87 \%), bénéficiant de meilleures conditions de logement, avec une très grande majorité (8 \%) ne vivant pas en surpeuplement.

\chapter{Le secteur locatif social}

\section{Caractéristiques et production du logement social}

\subsection{Qu'est-ce qu'un logement social ?}

Les logements sociaux, également connus sous le nom de HLM (Habitations à Loyer Modéré), répondent à plusieurs critères essentiels. Ils sont agréés par l'État et financés par des fonds publics, garantissant ainsi leur accessibilité. Leur attribution est soumise à une procédure contrôlée par l'État, et les loyers sont plafonnés, permettant de soutenir les ménages à faibles revenus. Les locataires doivent disposer de ressources inférieures à un certain plafond et peuvent bénéficier de l'Aide Personnalisée au Logement (APL). 

Parmi les avantages de la location d'un logement HLM, on note que les prix sont réglementés par l'État, ce qui les rend inférieurs à ceux du parc privé. De plus, le loyer est dû à terme échu, et le locataire jouit d'un droit au maintien dans les lieux tant qu'il respecte son contrat. Enfin, un suivi individualisé est proposé aux locataires en cas de difficultés, renforçant ainsi la solidarité et le soutien social.

Les principaux financeurs des logements sociaux jouent un rôle crucial dans leur développement et leur accessibilité. Parmi eux, l'État est un acteur majeur, apportant des ressources financières essentielles. Les collectivités territoriales, telles que les régions, départements, agglomérations et communes, contribuent également au financement de ces projets, en soutenant des initiatives locales. La Caisse des Dépôts et Consignations intervient également en fournissant des fonds spécifiques pour la construction et la rénovation de logements sociaux. De plus, Action Logement, qui regroupe le 1\% logement versé par les employeurs, participe activement à ce financement. Enfin, le bailleur lui-même peut investir ses propres fonds pour soutenir la réalisation de logements sociaux, garantissant ainsi une diversité de sources de financement pour répondre aux besoins en matière de logement abordable.

Les logements sociaux se déclinent en différentes catégories, chacune répondant à des besoins spécifiques. Les PLAI (Prêt Locatif Aidé d’Intégration) sont réservés aux personnes en situation de grande précarité, offrant ainsi un soutien crucial pour les ménages les plus vulnérables. Les PLUS (Prêt Locatif à Usage Social) correspondent aux HLM traditionnelles, garantissant un accès à des logements abordables pour un large public. En outre, les PLS (Prêt Locatif Social) et les PLI (Prêt Locatif Intermédiaire) s'adressent aux familles dont les revenus sont trop élevés pour accéder aux HLM, mais trop faibles pour se loger dans le secteur privé. Ces différentes catégories de logements sociaux illustrent la diversité des solutions mises en place pour répondre aux enjeux de l'accès au logement pour tous.

\subsection{Cadre juridique du logement locatif en France}

Le logement locatif privé constitue une part importante du parc immobilier, avec les logements loués vides représentant la majorité de ce secteur. Le loyer initial est fixé librement entre le propriétaire et le locataire, et peut être révisé une fois par an, sous réserve d'une clause dans le contrat, bien que son augmentation ne puisse excéder l'indice de référence des loyers (IRL) de l'Insee. Dans certaines régions, où le loyer est encadré, les propriétaires doivent respecter des plafonds spécifiques établis par la loi de 1948, ce qui inclut les logements situés à Paris et dans les agglomérations de Bordeaux, Grenoble, Lille ou Lyon. En outre, les logements atypiques, tels que les logements loués meublés ou les chambres d’hôtel, présentent des caractéristiques spécifiques et complexes concernant la fixation du loyer, la durée du bail et la protection du locataire. Enfin, la relative rareté des logements dans ce secteur rend leur accès plus difficile pour de nombreux ménages.

Les logements sociaux se divisent en plusieurs catégories. Les logements HLM sont ouverts à une part importante de la population, grâce à des plafonds de revenus relativement élevés. Les ménages éligibles peuvent déposer un dossier auprès d’un organisme HLM, mais les temps d’attente peuvent être longs, variant entre 18 et 30 mois, voire 4 à 5 ans à Paris. Le loyer d’un logement HLM dépend de la surface utile, avec un tarif fixé par m² selon une fourchette déterminée par l’État, en tenant compte de la zone géographique et du conventionnement avec l’État. 

En ce qui concerne les logements non HLM, ils regroupent de nombreux dispositifs, comme le Prêt Locatif Social (PLS), et se situent à mi-chemin entre le logement privé et les logements HLM, avec des plafonds de ressources souvent plus élevés pour les locataires. Les contraintes pesant sur les propriétaires peuvent être temporaires, leur permettant de fixer des loyers plus élevés que dans le parc HLM. 

Enfin, les logements loués selon la loi de 1948 sont soumis à un encadrement strict par l’État, avec des loyers parmi les plus faibles du parc français, bien qu'il ne soit plus possible depuis 1986 de conclure de nouveaux contrats de bail de ce type.

\subsection{Parc locatif}

En 2021, 43 \% des ménages en France sont locataires, un pourcentage stable depuis 1984. Le parc locatif se divise en deux composantes principales : le secteur privé, qui représente 58 \% en 2021, contre 63 \% en 1985, et le secteur social. Dans le secteur privé, la quasi-totalité des bailleurs sont des particuliers, tandis que 4 \% sont des personnes morales, telles que des associations, des mutuelles ou des sociétés immobilières d’investissement. Parmi les bailleurs personnes physiques, qui incluent les particuliers et les sociétés civiles immobilières, 21 \% ont bénéficié d’un dispositif fiscal au moment de l’achat, comme les dispositifs Robien et Borloo.

Le bailleur privé a la liberté de choisir son locataire et de fixer le montant du loyer, dans la limite des plafonds applicables dans certaines communes en zone tendue. Pour un logement privé, le bail est conclu pour une durée de 3 ans en location vide et de 1 an en location meublée. De plus, le bailleur privé peut résilier le contrat en cas de loyers impayés, de cession du logement ou de reprise à son compte.

Le bailleur social ne choisit pas son locataire, car le logement est attribué sous conditions de ressources par une commission d'attribution, dont une partie est composée de représentants du bailleur social. Il applique strictement le loyer réglementaire et conclut un contrat de bail à durée indéterminée. De plus, il ne peut pas reprendre le logement pour y habiter ou y loger un proche.

\begin{center}
	\includegraphics[scale=0.45]{../../../Pictures/Screenshots/Capture d'écran 2024-10-12 091603}
\end{center}

L’aire urbaine de Paris, qui concentre 19\% des ménages, regroupe 28\% des locataires de logements sociaux.

\subsection{Caractéristiques du parc social}

En 2021, le parc locatif social, qui inclut les HLM et autres logements sociaux, représente 18 \% du parc de résidences principales en métropole, soit près de 5 millions de logements. Cependant, cette répartition est inégale sur le territoire français. Parmi les 100 aires urbaines françaises les plus peuplées, la part du parc social dans le parc locatif varie de 5 \% à Chalon-sur-Saône à 37 \% à Creil. Pour les aires urbaines de plus de 200 000 habitants, les pourcentages sont élevés dans celles de Reims (34 \%) et du Havre (29 \%). En revanche, la plupart des très grandes aires urbaines de la moitié sud affichent des taux inférieurs à la moyenne.
\begin{center}
\includegraphics[scale=0.5]{../../../Pictures/Screenshots/Capture d'écran 2024-10-12 092039}

\includegraphics[scale=0.5]{../../../Pictures/Screenshots/Capture d'écran 2024-10-12 092333}
\end{center}

La surface moyenne des logements sociaux est élevée par rapport à celle du secteur privé. La taille moyenne d’un HLM est de 66 m², contre 63 m² pour les logements privés. De plus, 63 \% des logements sociaux sont des 3-4 pièces, tandis que ce chiffre est de 52 \% dans le secteur privé. Le loyer au m² varie fortement selon le statut d’occupation. Les logements loués vides à loyer libre affichent un loyer moyen plus élevé (8,5 euros/m²) que ceux du secteur social. En revanche, les logements HLM proposent le loyer le plus faible (4,7 euros/m²), suivis des logements loués sous la loi de 1948 (5,2 euros/m²). Enfin, le loyer moyen des logements sociaux non HLM se situe à 7 euros/m², position intermédiaire entre les secteurs privé et HLM.

En France, 46 \% des logements sociaux sont situés dans des quartiers où le revenu fiscal moyen annuel par unité de consommation est inférieur à 15 000 euros/an, contre 18,5 \% des logements dans le secteur privé. Les logements locatifs, en particulier les HLM, sont plus rarement localisés dans les communes rurales, représentant seulement 6 \%. La majorité des HLM se trouve dans les communes centres et ils sont sur-représentés en banlieue, avec 41,6 \% de leur localisation.

\subsection{Caractéristiques des occupants}

Les familles mono-parentales sont sur-représentées dans les secteurs HLM et social non-HLM, représentant respectivement 16,1 \% et 12 \% de ces logements, alors qu'elles ne constituent que 6,7 \% des ménages. L’ancienneté d’occupation varie fortement selon le statut d’occupation : les propriétaires sont les moins mobiles, avec une ancienneté moyenne de 19,7 années, tandis que les locataires du parc privé sont les plus mobiles, avec une ancienneté de 5,8 années. Les ménages en HLM ont une durée d’occupation moyenne importante, s'élevant à 10,7 années.

\begin{center}
	\includegraphics[scale=0.5]{../../../Pictures/Screenshots/Capture d'écran 2024-10-12 093600}
\end{center}

\subsection{Les organismes de producteurs de logement social}

Les Offices Publics de l’Habitat (OPH) sont des établissements publics locaux à caractère industriel et commercial (EPIC). Ils résultent de la fusion des Offices Publics d’HLM (OPHLM) et des Offices Publics d’Aménagement et de Construction (OPAC) depuis 2007. En 2022, ils représentent 47 \% du parc locatif social, soit plus de 5 millions de logements au total. La France compte 266 offices. Leur mission est de mettre en œuvre les politiques d’habitat et de logement des collectivités ou des établissements publics de coopération intercommunale (EPCI) auxquels ils sont rattachés. Cela inclut la construction, l’acquisition et la rénovation des logements sociaux, ainsi que la réalisation d’opérations d’urbanisme et d’aménagement dans un souci de mixité sociale.

Les Offices Publics de l’Habitat (OPH) sont des établissements publics locaux à caractère industriel et commercial (EPIC). Ils résultent de la fusion des Offices Publics d’HLM (OPHLM) et des Offices Publics d’Aménagement et de Construction (OPAC) depuis 2007. En 2022, ils représentent 47 \% du parc locatif social, soit plus de 5 millions de logements au total. La France compte 266 offices. Leur mission est de mettre en œuvre les politiques d’habitat et de logement des collectivités ou des établissements publics de coopération intercommunale (EPCI) auxquels ils sont rattachés. Cela inclut la construction, l’acquisition et la rénovation des logements sociaux, ainsi que la réalisation d’opérations d’urbanisme et d’aménagement dans un souci de mixité sociale.

Les Sociétés d’Économie Mixte (SEM) sont des sociétés anonymes dont le capital est majoritairement détenu par une ou plusieurs collectivités territoriales, combinant des partenaires publics et privés. Elles interviennent dans de nombreux domaines d’intérêt général, y compris la construction et la gestion de logements. En France, les 163 SEM immobilières logent 1,42 million de personnes dans un parc de 578 000 logements, représentant 10 \% du parc locatif total des organismes de logement social.



\section{Financement et fonctionnement du secteur locatif social}

\subsection{Financement du logement social}

La grande majorité des financements destinés au logement social provient des prêts accordés par la Caisse des Dépôts et Consignations (CDC), que ce soit pour la construction de nouveaux logements ou pour la réhabilitation du parc ancien. Une autre source de financement spécifique est la Participation des employeurs à l’effort de logement, souvent désignée sous le terme de 1\% logement. Cette contribution est cruciale, car elle représente une source de financement significative. Dans les deux cas, il s'agit d'une épargne privée mise à disposition pour soutenir le logement social, renforçant ainsi l'accès à des logements abordables pour les ménages.

En 2014, la répartition du financement des investissements en matière de logement social était dominée par les prêts de la Caisse des Dépôts et Consignations (CDC), représentant 74 \% du total. Les contributions d'Action Logement s'élevaient à seulement 2,5 \%, tandis que les fonds propres des organismes de logement constituaient 15 \% et les collectivités locales apportaient 5 \%. L'État, quant à lui, ne représentait que 1 \% des financements sous forme d'aides directes. Il est essentiel de souligner que l'obtention de ces aides de l'État est un déclencheur fondamental pour le reste des financements. Ainsi, le financement du logement social repose principalement sur des sources privées, avec une part relativement faible de dépenses publiques, même en tenant compte des aides indirectes, telles que les avantages fiscaux accordés aux organismes HLM.

Exemple : structure de financement pour un logement social moyen qui coûtait 140 000\euro~en 2014.

\newpage
\begin{wrapfigure}{r}{0.65\textwidth}
	\centering
	\includegraphics[scale=0.45]{../../../Pictures/Screenshots/Capture d'écran 2024-10-22 092053}
\end{wrapfigure}

En 2022, l'encours de la dette des organismes de logement social (HLM) s'élevait à environ 150 milliards d'euros, marquant une hausse significative de 81 \% par rapport aux 84 milliards d'euros en 2006. Près de 80 \% de cette dette a été contractée auprès de la Caisse des Dépôts et Consignations (CDC). Depuis 2011, le montant annuel des nouveaux prêts accordés par la CDC a dépassé les 20 milliards d'euros, contre seulement 5 milliards en 2006. En 2015, les organismes HLM ont bénéficié de 13 milliards d'euros de prêts de la CDC, dont 10,4 milliards étaient destinés à la construction et 2,3 milliards à la réhabilitation. Par ailleurs, les prêts d'autres organismes s'élevaient à 1 milliard d'euros, tandis que les fonds propres mobilisés atteignaient 3 milliards d'euros.

\subsection{Caractéristiques des prêts}

Chaque année, le gouvernement détermine le nombre de prêts pouvant être accordés par la CDC ainsi que leurs conditions d'utilisation, incluant l'objet, la tarification, la durée maximale et la liste des emprunteurs éligibles. Ces prêts présentent deux caractéristiques majeures. Premièrement, ils sont adaptés au financement de projets à long terme, étant accordés pour des durées très longues, avec une moyenne de 32 ans, 40 ans pour les bâtiments, et jusqu'à 50 ans pour le foncier. Cette longue durée d'amortissement permet de rendre le coût de construction ou de réhabilitation compatible avec les plafonds de loyers imposés aux HLM. Deuxièmement, ces prêts se déconnectent d'une logique de marché, s'éloignant sur trois aspects principaux : ils sont souvent accordés au coût de la ressource, voire en dessous ; le taux des prêts est généralement indépendant des durées offertes ; et leur tarification dépend du caractère social plus ou moins affirmé des biens financés (PLAI, PLUS ou PLS), plutôt que de la situation financière de l'organisme emprunteur.

\subsection{Les loyers d'équilibre}

La gestion des organismes de logement social (HLM) repose sur des loyers dits « d’équilibre », qui doivent permettre aux organismes de faire face à leurs charges courantes, y compris la charge de la dette jusqu'à son extinction. En moyenne, les annuités financières représentent la moitié des recettes de loyers. Le montant de la dette dépend du coût d'investissement initial, que la dette couvre à hauteur de 70 \%. Cela signifie que les loyers sont essentiellement liés aux coûts de construction, reflétant ainsi les dépenses engagées pour la réalisation des projets de logement.

Les loyers des organismes de logement social (HLM) doivent respecter des plafonds légaux. L'octroi de prêts de longue durée et à taux d'intérêt bas est essentiel pour garantir que les loyers restent en dessous de ces plafonds. Au moment de la livraison du programme, le loyer et sa progression au fil du temps doivent prendre en compte plusieurs éléments : les coûts de gestion, les dépenses d'entretien des immeubles, les risques de vacance et d'impayés, ainsi que l'assujettissement à la taxe foncière après 25 ans.

\begin{wrapfigure}{r}{0.65\textwidth}
	\centering
\includegraphics[scale=0.45]{../../../Pictures/Screenshots/Capture d'écran 2024-10-22 093013}
\end{wrapfigure}

Pour un programme de 65 logements destinés à un couple avec un enfant dans la région de Toulouse, le coût total s'élève à 7,2 millions d'euros, soit 111 000 euros par logement. Ce programme est financé à hauteur de 77 \% par des prêts de la Caisse des Dépôts et Consignations (CDC) sur 40 ans, 12 \% par des aides directes provenant de l'État et de la collectivité locale, et 11 \% par des fonds propres.

Dans cet exemple, l'équilibre de la gestion locative est établi pour un loyer de départ de 310 euros, avec une augmentation prévue de X2 sur une période de 40 ans. Ces loyers d'équilibre sont rendus abordables pour les ménages les plus modestes grâce à l'existence d'aides personnelles aux locataires, qui représentent en moyenne 230 euros par logement. Cela équivaut à 22 \% des dépenses en logement.

\subsection{Les niveaux de loyers}

La plupart des loyers des logements HLM ne sont pas fixés librement par le bailleur, mais font l'objet d'une convention entre celui-ci et l'État. La législation prévoit la possibilité d'augmenter les loyers une fois par an, dans la limite des conventions conclues. Cette révision peut avoir lieu chaque année au 1er janvier, en fonction de l'Indice de Référence des Loyers (IRL). De plus, la convention fixe un plafond de loyer, qui varie selon le type de financement du logement et sa zone d'implantation.

\begin{center}
	\includegraphics[scale=0.5]{../../../Pictures/Screenshots/Capture d'écran 2024-10-22 093602}
\end{center}
Les zones de fixation des loyers pour les logements HLM sont classées comme suit : 

Zone A bis : Paris et 76 communes des Yvelines, des Hauts-de-Seine, de Seine-Saint-Denis, du Val-de-Marne et du Val-d’Oise.

Zone A : Agglomération de Paris, la Côte d’Azur, la partie française de l’agglomération genevoise, ainsi que certaines agglomérations ou communes où les loyers et les prix des logements sont très élevés.

Zone B1 : Certaines grandes agglomérations où les loyers et les prix des logements sont élevés, ainsi qu'une partie de la grande couronne parisienne non située en zone A bis ou A.

Zone B2 : Villes-centre de certaines grandes agglomérations, grande couronne autour de Paris non située en zone A bis, A et B1, ainsi que certaines communes où les loyers et les prix des logements sont assez élevés.

Zone C : Reste du territoire.

\begin{center}
	\includegraphics[scale=0.5]{../../../Pictures/Screenshots/Capture d'écran 2024-10-22 093824}
\end{center}

Les loyers plafonds sont sensiblement différenciés entre les agglomérations par rapport aux loyers effectifs du secteur privé.

\begin{center}
	\includegraphics[scale=0.5]{../../../Pictures/Screenshots/Capture d'écran 2024-10-22 094025}
\end{center}


\subsection{Limites du modèle}

Le modèle économique du secteur locatif social a été mis à l'épreuve par une forte augmentation des prix du logement, avec une hausse de 100 \% en termes réels entre 1997 et 2007. Cette situation a également entraîné une augmentation significative des prix des terrains, ce qui a eu un impact direct sur la capacité des bailleurs sociaux à construire de nouveaux logements, étant donné que les loyers sont régulés. En conséquence, le coût unitaire d'un logement locatif social a augmenté de plus de 80 \% en 15 ans sur l'ensemble du territoire, passant de moins de 80 000 euros à plus de 140 000 euros.

\begin{wrapfigure}{r}{0.56\textwidth}
	\centering
\includegraphics[scale=0.40]{../../../Pictures/Screenshots/Capture d'écran 2024-10-22 094336}
\end{wrapfigure}

En Île-de-France, le coût de construction est supérieur de près de 40 \% à celui des autres régions. De plus, la charge foncière y est deux fois plus importante qu'ailleurs.

Bien que la part de l’endettement ait diminué en pourcentage, le montant nécessaire pour réaliser un logement a considérablement augmenté de plus de 75 \% en 15 ans, atteignant aujourd'hui une moyenne de 110 000 euros, contre 62 000 euros en 2000. Cette situation s'explique par la hausse des coûts de construction et l'augmentation des besoins en logements, entraînant une croissance spectaculaire de la dette des organismes, qui est passée à 130 milliards d’euros en 2014, contre 84 milliards d’euros en 2006. Cependant, cet endettement accru n’a pas été suffisant à lui seul pour financer le « choc d’investissement » qui a eu lieu à partir de 2008.

\begin{center}
	\includegraphics[scale=0.4]{../../../Pictures/Screenshots/Capture d'écran 2024-10-22 094707}
\end{center}

\begin{center}
	\includegraphics[scale=0.4]{../../../Pictures/Screenshots/Capture d'écran 2024-10-22 095216}
\end{center}

Pour faire face à la hausse conjointe des volumes et des coûts, plusieurs ressources complémentaires ont été mobilisées. Tout d'abord, il y a eu un recours croissant aux fonds propres, passant de 5 \% à 14 \% du total du financement. De plus, les contributions des collectivités locales ont augmenté de manière significative. Par ailleurs, la mobilisation de terrains publics au profit des bailleurs sociaux, accélérée à partir de 2012, a également facilité la réalisation de certains programmes.

Les bailleurs sociaux ont pu exploiter d'autres leviers pour s'adapter à la situation. Ils ont notamment réalisé des opérations d'acquisition-réhabilitation, permettant de pallier la pénurie d’offre foncière dans les zones urbaines denses. De plus, dans un contexte de très bas taux d’intérêt depuis 2012, les HLM ont préféré recourir aux emprunts bancaires et ont également restructuré leur dette auprès de la CDC. Cependant, ces changements récents traduisent une fragilisation du modèle, rendant les HLM plus dépendants de l’engagement des municipalités en leur faveur. La réduction des dotations budgétaires aux communes depuis 2015 a également rendu plus difficile le maintien de leur soutien au logement social.

\section{Vivre en HLM : quelles conséquences ?}

\subsection{Avantages monétaires et conditions de vie}

L’occupation d’un logement social permet aux ménages de bénéficier de loyers inférieurs à ceux pratiqués par les bailleurs privés, ce qui constitue une forme de subvention implicite. À ce sujet, quatre questions peuvent être soulevées : quel est le montant de la subvention implicite associée à l’occupation d’un logement social ? Dans quelle mesure l’occupation d’un logement social modifie-t-elle les conditions de logement ? Comment l’occupation d’un logement social impacte-t-elle le niveau de vie hors logement des ménages ? L’occupation d’un logement social entraîne-t-elle une surconsommation de logement ? 

(Source : Trevien C. (2013), « \textit{Habiter en HLM : quel avantage monétaire et quel impact sur les conditions de logement ?} », Document de travail de l’Insee, n°2013-2.)

Ce travail confirme les résultats de plusieurs études portant sur le lien entre logement social et redistribution des richesses. La subvention implicite associée à l’occupation des logements sociaux bénéficie en partie aux ménages aisés. En effet, selon les plafonds de revenu à l’entrée, 80 \% des ménages sont éligibles pour les logements sociaux, et le loyer ne dépend des revenus des locataires que de façon marginale. De plus, il existe un droit au maintien dans les lieux, indépendamment de l’évolution des revenus des occupants.

L’emménagement en HLM n’a pas le même effet sur les conditions de logement selon le niveau de vie du ménage. Les caractéristiques des logements sociaux occupés diffèrent sensiblement selon le niveau socio-économique des locataires. Les ménages à faible revenu se voient attribuer des logements de moins bonne qualité, car ils sont souvent les seuls compatibles avec leurs ressources ou parce que les ménages plus aisés refusent d’y emménager. En revanche, les ménages aisés peuvent quitter plus facilement les logements sociaux de mauvaise qualité, ce qui entraîne une attrition différenciée selon les revenus. Enfin, les bailleurs sociaux favorisent l’emménagement de ménages plus aisés pour des raisons de solvabilité.

\subsection{Résultats de l'étude}

\begin{center}
	\includegraphics[scale=0.4]{../../../Pictures/Screenshots/Capture d'écran 2024-10-22 095914}
\end{center}

\subsection{Définition des effets}

Subvention implicite : il s'agit de la différence entre le loyer potentiel d’un logement HLM s’il avait été proposé dans le parc locatif privé et le loyer demandé par le bailleur social.

Valeur du logement : cela représente la différence de valeur, aux prix du marché privé, entre le logement HLM occupé par le ménage et celui qu’il aurait occupé dans le parc locatif privé.

Effet niveau de vie hors logement : c'est la différence entre le loyer du logement HLM et les dépenses de logement auxquelles le ménage aurait dû consentir pour le logement qu’il aurait occupé dans le parc locatif privé.

Surconsommation : cela correspond à la différence de la quantité de logement consommée entre le logement qu’il aurait occupé s’il bénéficiait d’une allocation égale à la subvention implicite et le logement HLM occupé.

\subsection{Résultats de l'étude}

\begin{center}
	\includegraphics[scale=0.5]{../../../Pictures/Screenshots/Capture d'écran 2024-10-22 100131}
\end{center}

Les gains associés à l’occupation d’une HLM varient fortement selon la taille de l’aire urbaine. Les loyers des logements sociaux progressent moins vite que les loyers libres en fonction de la taille de l’aire urbaine. Par exemple, le loyer libre moyen est supérieur de 60 \% dans l’AU (Aire Urbaine) de Paris par rapport à l’extérieur, tandis que cet écart est limité à 22 \% dans le parc social.

La hausse de niveau de vie résultant de l’emménagement en HLM croît avec l’augmentation de la taille de l’aire urbaine, passant de 142 euros hors AU à 318 euros dans l’AU de Paris. Dans cette même aire urbaine, les ménages en HLM consommeraient moins de logement s’ils disposaient de l’équivalent de la subvention implicite pour se loger dans le privé. Cependant, la surconsommation de logement reste faible, s’élevant à 48 euros par mois.

\begin{center}
	\includegraphics[scale=0.5]{../../../Pictures/Screenshots/Capture d'écran 2024-10-22 100445}
\end{center}

Le gain associé à l’occupation d’une HLM croît avec le niveau de vie. Les ménages du 1er quintile bénéficient d’une subvention implicite de 239 euros, tandis qu’elle s’élève à 319 euros pour les ménages du dernier quintile. Cette subvention représente en moyenne 23,6 \% des revenus des ménages du 1er quintile, mais seulement 6,9 \% pour ceux du dernier. On ne peut donc pas parler d’effet anti-redistributif.

L’effet niveau de vie hors logement de l’occupation d’un logement social croît encore plus fortement que la subvention implicite avec le niveau de vie du ménage. Il progresse de 190 euros pour le 1er quintile à 330 euros pour le dernier quintile.

L’emménagement en HLM n’entraîne pas un rapprochement significatif du centre de l’aire urbaine. Il réduit de 15,4 points de pourcentage la probabilité d’habiter une maison, avec un effet particulièrement fort dans les aires urbaines de moins de 50 000 habitants. De plus, il diminue de 3 000 euros le revenu médian annuel du voisinage, cette diminution étant d’autant plus forte que l’aire urbaine est grande.

En revanche, l’emménagement en HLM augmente la surface du logement de 2,5 m². L’incrément est positif dans les aires urbaines de plus de 500 000 habitants (+5,9 m²) et de Paris (+10,7 m²), tandis qu’il est négatif en dehors des aires urbaines (-10,9 m²). L’augmentation est également plus importante pour les ménages modestes (+2,9 m²) et devient non significative pour les ménages des deux derniers quintiles.

\begin{center}
	\includegraphics[scale=0.7]{../../../Pictures/Screenshots/Capture d'écran 2024-10-22 100752}
\end{center}

\subsection{Accession à la propriété}

Selon Goffette-Nagot et Sidibé (2012), les loyers des logements HLM sont d'environ 40 \% inférieurs à ceux du marché, ce qui permet aux locataires de bénéficier d'un pouvoir d'achat supplémentaire. En effet, ce statut de locataire en HLM permettrait de consommer un supplément de 10 \% de services de logement et 11 \% d'autres dépenses par rapport aux locataires du secteur libre. Cela soulève la question de l'impact de ce statut sur les capacités d'épargne des ménages. On peut se demander si, en favorisant l'épargne, le statut de locataire en HLM pourrait influencer la dynamique d'accession à la propriété, en facilitant l'accès à la propriété pour ces ménages.

Source : Goffette-Nagot F. et Sidibé M. (2012), « \textit{Logement social et accession à la
propriété }», Economie et Prévision, n°200-201.




















\chapter{La dynamique des marchés immobiliers}

L'analyse des marchés immobiliers se réalise à la fois de manière statique, c'est-à-dire à un instant donné, et dynamique, en tenant compte des variations des prix et des quantités au fil du temps. Ces fluctuations des variables immobilières résultent de l'ajustement continu des marchés, qui peut se manifester de plusieurs façons. Tout d'abord, il y a l'adaptation par les mutations résidentielles, où les changements de propriétaires et de locataires influencent l'offre et la demande. Ensuite, l'adaptation par la construction neuve, qui prend en compte les nouvelles constructions tout en excluant les destructions et les désaffectations-réaffectations, joue un rôle essentiel dans l'évolution du marché. De plus, les travaux de rénovation, nets de la dégradation, permettent également d'améliorer l'état du parc immobilier. Enfin, les ajustements par les prix sont cruciaux, car ils reflètent directement l'équilibre entre l'offre et la demande sur le marché immobilier. Ces différentes formes d'adaptation illustrent la complexité et la dynamique des marchés immobiliers.

Les différents modes d'adaptation de l'offre et de la demande ne sont pas indépendants les uns des autres. En effet, les trois premiers modes d'adaptation se concentrent sur les quantités, ce qui souligne l'interdépendance de ces éléments. Il est important de noter que l'adaptation par les prix ne peut être considérée séparément de celle par les quantités que si les composantes du marché sont élastiques. Dans l'analyse économique, on distingue également les évolutions à long terme des évolutions à court terme. On considère que l'évolution d'une grandeur est la somme de trois mouvements : la composante saisonnière, qui reflète les variations prévisibles au cours de l'année, la composante cyclique, qui correspond aux fluctuations de court terme, et enfin la tendance, accompagnée de la composante irrégulière résiduelle, qui englobe les variations imprévues. Cette approche permet de mieux comprendre la dynamique des marchés et les interactions entre les différents facteurs influençant l'offre et la demande.

La plupart des séries statistiques sont fournies corrigées des variations saisonnières, ce qui permet d'isoler chaque composante et d'expliquer son évolution. Dans ce processus, on commence par isoler la composante tendancielle afin d'obtenir une vision claire de l'évolution à court terme ou conjoncturelle. Pour ce faire, on utilise fréquemment la méthode des moyennes mobiles, qui consiste à remplacer une valeur par la moyenne d'un certain nombre de valeurs qui l'encadrent, ces valeurs pouvant être éventuellement pondérées. Cette approche permet d'analyser la dynamique des marchés immobiliers sous trois dimensions : produit, actif et service. Dans cette analyse, l'étude à long terme sera présentée de manière succincte, tandis que l'étude à court terme sera explorée de façon plus approfondie, offrant ainsi une compréhension complète des fluctuations et des tendances du marché immobilier.

\section{La dynamique de long terme du logement}

Les informations concernant les caractéristiques à long terme du logement sont souvent lacunaires et peu fiables. Dans ce contexte, le raisonnement théorique joue un rôle crucial, car il a autant pour fonction d'expliquer les observations que d'indiquer les évolutions possibles. Ainsi, une analyse rigoureuse est nécessaire pour mieux comprendre les dynamiques du marché. Dans cette étude, nous allons examiner successivement les volumes et les prix, afin d'évaluer comment ces deux éléments interagissent et influencent les caractéristiques du logement sur le long terme. Cette approche permettra d'apporter des éclairages sur les tendances observées et d'identifier les facteurs sous-jacents qui façonnent le marché immobilier.

\subsection{Les volumes}

On utilise également les données établies par Villa (1994), qui fournit un siècle de données macroéconomiques pour la France. Parmi ces informations, il présente notamment la consommation de services du logement depuis 1890 ainsi que le produit intérieur brut (PIB). Ces données sont représentées dans la figure suivante, où un ajustement exponentiel met en évidence le taux de croissance moyen sur la période. Cette représentation permet d'analyser les tendances historiques et d'évaluer l'évolution du marché du logement en relation avec l'économie française au fil du temps.

\begin{center}
	\includegraphics[scale=0.4]{../../../Pictures/Screenshots/Capture d'écran 2024-10-21 183716}
\end{center}

Il apparaît que la consommation de logements croît en tendance, tout comme le PIB. L'ajustement des données à long terme révèle un taux de croissance annuel de la location immobilière inférieur à celui du PIB, soit 2,14 \% contre 2,56 \%. Cependant, cette évolution est variable dans le temps. Rapportées au PIB, les locations immobilières ont diminué entre 1880 et la Seconde Guerre mondiale, pour ensuite augmenter depuis cette période. Il est vraisemblable qu'en moyenne, la consommation de logements évolue à peu près parallèlement au PIB, comme c'est le cas pour la consommation en général. Cela constitue un des faits stylisés de la croissance, soulignant l'interconnexion entre ces deux indicateurs économiques.

Le stock de logements est proportionnel à la consommation. Si cette consommation est approximativement proportionnelle au PIB, alors le stock de logements l'est également, tout comme le capital fixe productif. Cependant, étant donné que la durée de vie du capital logement est plus longue que celle du capital fixe productif, il est nécessaire de réaliser proportionnellement moins d'investissements immobiliers que d'investissements productifs pour maintenir le stock sur son sentier de croissance. Cette dynamique souligne l'importance des investissements dans le secteur immobilier pour soutenir l'évolution du marché tout en tenant compte des différences de durée de vie des capitaux.

\subsection{Les prix}

La reconstitution d'une série de prix immobiliers sur une longue période n'est pas une tâche aisée. En France, par exemple, l'OCDE (2005) produit une série de prix depuis 1970, en indiquant comme source l'indice INSEE (1994). Cela soulève des doutes quant à la fiabilité des valeurs antérieures. Friggit (2010) compile différentes données, notamment celles de Duon (1946), pour construire une série depuis le XIXe siècle. Il en conclut que le prix des logements évolue en parallèle avec le revenu disponible par ménage. Cette constatation met en lumière l'interaction entre le marché immobilier et les capacités financières des ménages au fil du temps.

Si le prix des logements évolue comme le revenu disponible par ménage, cela implique que le rapport entre le revenu disponible et le prix des logements reste constant. Or, ce rapport représente le pouvoir d'achat logement, qui serait donc constant dans le temps. Cependant, d'après les études de Duon (1946), on peut interpréter les données différemment, en les considérant comme des données de prix moyen par logement. Autrement dit, ces données possèdent une dimension de volume. Il en résulte que ce n'est pas le pouvoir d'achat logement qui est stable, mais plutôt la dépense moyenne de logement par ménage. Cette distinction souligne l'importance de considérer les variations de la consommation en parallèle avec l'évolution des prix.

Par ailleurs, l'élasticité de long terme de la consommation de logements par rapport au pouvoir d'achat logement est égale à un. Les observations concernant l'évolution des prix ne sont valables que pour le long terme. À court terme, et en particulier durant les années récentes, les prix ont connu des évolutions bien supérieures aux revenus des ménages. Ce sont ces fluctuations cycliques ou conjoncturelles que nous allons analyser maintenant, en évoquant d'abord les deux derniers cycles immobiliers. Cette analyse permettra de mieux comprendre les dynamiques à court terme du marché immobilier et leur impact sur les ménages.

\section{Le cycle depuis 2000}

Entre 2000 et 2007, les prix de l'immobilier ont connu une croissance exceptionnelle, conduisant à un doublement des prix en France. Les volumes de construction ont également atteint un niveau remarquable durant cette période. Cependant, les années 2008 à 2009 ont été marquées par une contraction des marchés. La récente augmentation des prix est en partie due à la baisse des taux d'intérêt, mais la remontée de ces derniers entraîne une tension sur le marché immobilier. Ce cycle affecte presque tous les pays développés, à l'exception notable de l'Allemagne et du Japon, principalement pour des raisons démographiques. L'intensité de ce cycle varie d'un pays à l'autre.

\subsection{Les faits}

Les prix réels de l'immobilier ancien ont commencé à croître à la fin des années 90, atteignant un pic en 2007. Par la suite, les prix ont connu un décrochage, avant de se redresser et de dépasser ce pic, du moins pour les appartements, puis de fléchir à nouveau. Pendant cette même période, les loyers réels ont affiché une faible croissance, témoignant d'une nette déconnexion par rapport aux prix. Sur le marché des logements neufs, l'augmentation des prix a été d'environ 40 \% entre 2000 et 2007, ce qui représente une forte hausse, mais de moindre amplitude que celle observée pour les logements anciens. Cette dynamique souligne les différences de comportement entre les segments du marché immobilier.

\begin{center}
	\includegraphics[scale=0.4]{../../../Pictures/Screenshots/Capture d'écran 2024-10-21 184458}
\end{center}

Les volumes de construction ont également atteint un niveau très élevé. Plus de 400 000 logements ont été commencés durant les années 2005-2007, avant que ces chiffres ne retombent pour revenir en 2011 à leur niveau tendanciel. Cette évolution des volumes de construction reflète les dynamiques du marché immobilier et les fluctuations économiques de la période.

\begin{center}
	\includegraphics[scale=0.4]{../../../Pictures/Screenshots/Capture d'écran 2024-10-21 184634}
\end{center}

\subsection{La phase croissante du cycle}

De nombreux travaux ont fourni des explications à ces phénomènes (OCDE, 2005 ; Hissler, 2005 ; Wilhem, 2005 ; Di Martino et Duca, 2007 ; Arthus et Alii, 2008). Certains analystes interprètent l'augmentation des prix comme la manifestation d'une crise du logement, résultant d'un accroissement de la demande que l'offre n'a pu satisfaire. Toutefois, si tel était le cas, le choc de demande se serait manifesté de manière quantitative, par un accroissement de la population ou du nombre de ménages. Or, ces indicateurs ont crû de manière régulière, sans accélération particulière, durant les années 2000. De plus, si un choc de demande était à l'origine de la hausse des prix, cela se serait également reflété dans l'évolution des loyers. Cette analyse soulève des questions sur les véritables causes de l'inflation des prix immobiliers.

Les loyers ont augmenté, mais de manière modérée durant les années 2000. Une hausse significative s'était produite antérieurement, entre 1985 et 1995, lors du précédent cycle. De plus, étant donné que la flambée des prix est un phénomène mondial, le choc de demande devrait également être mondial. Or, nous n'avons pas observé ce type de phénomène à cette échelle. Le fait majeur du cycle des années 2000 est l'évolution divergente des prix et des loyers. L'explication de ce cycle réside dans cette divergence, qui est liée aux conditions financières. Cette situation met en lumière les dynamiques complexes qui influencent le marché immobilier.

Quand on considère le rôle des conditions de financement, le point de vue couramment défendu est que c'est l'amélioration de ces conditions qui a permis de maintenir le pouvoir d'achat logement des ménages. De ce point de vue, la baisse des taux d'intérêt aurait permis de contrebalancer l'augmentation des prix, souvent qualifiée de bulles. La théorie économique stipule que c'est la demande qui détermine le prix. Ainsi, c'est la solvabilité des acquéreurs et leur capacité d'endettement qui conditionnent les prix des biens, et non l'inverse. Depuis 2000, les conditions des prêts au logement ont connu une évolution très favorable concernant les trois caractéristiques principales des prêts : les taux, la durée et le taux d'apport personnel. Cette dynamique a eu un impact significatif sur le marché immobilier.

Au niveau macroéconomique, la diminution des exigences d'apport personnel permet à un plus grand nombre de ménages de s'engager dans une démarche d'accession à la propriété. Indépendamment des conditions de financement, la croissance des revenus a également contribué à une augmentation des dépenses allouées au logement. Entre 2001 et 2006, le revenu disponible des ménages a crû de 21 \%. Parallèlement, le montant des annuités a augmenté de 32 \%, ce qui peut être attribué au fait que les accédants se recrutent principalement parmi des tranches de revenus plus élevées. Cette évolution souligne l'impact des conditions économiques sur l'accessibilité au logement.


En définitive, la capacité d'endettement, et donc la capacité d'achat des ménages, a quasiment doublé entre 2001 et 2006. Ce doublement est attribuable à 30 \% à l'augmentation de la dépense de logement, à 30 \% à l'allongement de la durée des prêts, et à 40 \% à la baisse des taux d'intérêt. Ainsi, le doublement de la capacité d'endettement et donc le paiement des ménages a été absorbé par le doublement des prix. L'augmentation des prix s'est réalisée dans la proportion où les ménages ont pu la supporter. Toutefois, s'il y a eu exclusion de certaines catégories de ménages, cela résulte principalement de leur incapacité à accéder au crédit. Cette situation met en lumière les défis d'accessibilité au logement pour certains segments de la population.

L'augmentation de la capacité à payer n'a pas entraîné d'augmentation du nombre de transactions dans l'ancien. En d'autres termes, la rigidité de l'offre a conduit à une réponse en prix face à l'accroissement de la demande. Parallèlement, l'amélioration de la capacité d'endettement des ménages s'est également traduite par une augmentation de la demande de logements neufs. Sur la décennie 2001–2010, on a construit 3,487 millions de logements, contre 2,790 millions sur la décennie précédente. Cette dynamique souligne les différences entre le marché de l'ancien et celui du neuf, ainsi que l'impact des conditions économiques sur la construction.

On peut utiliser le schéma de Di Pasquale et Wheaton pour analyser l'enchaînement des phénomènes. Le choc de taux d'intérêt sur le marché de l'actif entraîne une hausse des prix par capitalisation des loyers. Cette hausse des prix entraîne à son tour un accroissement de la construction. Cet accroissement de la construction est absorbé par une augmentation de la demande de permis, ainsi que par la baisse des taux d'intérêt et l'amélioration générale des conditions de financement. Ce processus met en lumière les interactions complexes entre les différents éléments du marché immobilier.

\begin{center}
	\includegraphics[scale=0.4]{../../../Pictures/Screenshots/Capture d'écran 2024-10-21 185535}
\end{center}

L'accroissement de la demande, dû à l'amélioration de la capacité d'endettement des ménages, a également été influencé par des investisseurs immobiliers. Les données de l'observatoire du financement du logement montrent que l'accroissement des transactions le plus important a concerné l'acquisition de résidences secondaires et de logements locatifs, avec des taux d'accroissement respectifs de 58 \% et 79 \% entre 2001 et 2006. En comparaison, l'accession à la propriété dans le neuf a augmenté de 29 \%, tandis que celle dans l'ancien a connu une hausse de 17 \%. Cette dynamique illustre l'importance croissante des investissements immobiliers dans le marché.

La même croissance de la dette hypothécaire s'est produite aux États-Unis. Comme en France, le développement des crédits hypothécaires a principalement concerné les logements « non owner-occupied », c'est-à-dire les résidences secondaires et les logements locatifs. Le nombre de prêts dans cette catégorie a été multiplié par 2,9 entre 2000 et 2005, tandis que les logements owner-occupied ont vu leur nombre de prêts augmenter de 41 \% (Housing Mortgage Report, 2009). Par ailleurs, les crédits pour l'acquisition de logements se sont également développés dans la catégorie des sub-primes, c'est-à-dire des prêts les plus risqués. Grossman-Wirth et Alii (2010) montrent le rôle de la croissance des sub-primes dans l'augmentation des prix. Cette situation met en évidence les dynamiques similaires entre les marchés immobilier américain et français.

\subsection{Évolution depuis 2005}

Aux États-Unis, les crédits hypothécaires sont accordés moins en fonction de la capacité de remboursement des ménages que du ratio Loan-to-value, c'est-à-dire la quotité. Ce système de financement fonctionne tant que le marché ne se retourne pas. La crise s'est d'abord manifestée aux États-Unis et dans les pays engagés dans une expansion excessive du crédit. Elle s'est ensuite propagée aux pays moins endettés, comme la France ou l'Allemagne, par l'intermédiaire des sub-primes, qui ont mis en difficulté certaines banques. Cette situation souligne les risques systémiques liés à des pratiques de financement peu rigoureuses.

Les prix sur le marché de l'immobilier s'effondrent, car l'offre excède largement la demande. Par ailleurs, les actifs sub-prime offerts sur les marchés financiers perdent également de leur valeur, entraînant une crise financière qui s'étend au monde entier. Il en résulte une crise de liquidité sur les marchés financiers. Pourquoi les banques centrales n'ont-elles pas réagi ? D'une part, il y a des difficultés à discerner une bulle lorsqu'elle débute. D'autre part, il existe un risque d'étouffer la croissance en réhaussant les taux d'intérêt. De plus, les États ont dû porter secours aux banques affectées par la défaillance des emprunteurs, ce qui a conduit à une crise de la dette souveraine en Europe. Cette situation met en lumière les défis complexes auxquels sont confrontés les décideurs économiques en période de crise.

\section{Le cycle des années 1985-1998}

Entre 1985 et 1998, le secteur immobilier a connu un cycle d'une ampleur nouvelle pour l'époque. La plupart des pays industrialisés ont connu un phénomène semblable (Renault, 1995). Ce cycle s'est manifesté par une flambée des prix et des quantités, accompagnée d'un gonflement de l'endettement des agents impliqués dans l'immobilier : investisseurs, promoteurs, marchands de biens et ménages. Ce phénomène cyclique peut être illustré par l'évolution conjointe des prix et des volumes sur le marché des logements anciens à Paris. On peut considérer que le dernier cycle commence en 1985, marquant ainsi une période significative dans l'histoire du marché immobilier.

Il se manifeste par une fluctuation très sensible des volumes, à la différence du cycle plus récent, où les volumes ont peu varié. On est donc porté à croire que les facteurs explicatifs sont différents. Pour le cycle immobilier des années 1985-1998, l'hypothèse de bulles a été fréquemment avancée. On l'examine en premier avant d'envisager d'autres facteurs explicatifs. Cette analyse permet de mieux comprendre les dynamiques spécifiques qui ont influencé le marché immobilier durant cette période.

\subsection{L’hypothèse de bulle}

Une bulle est un phénomène spéculatif, purement nominal, qui consiste en un écart entre le prix observé et la valeur fondamentale, celle-ci étant la valeur actualisée des recettes futures correctement anticipées. Le plus souvent, la valeur fondamentale est estimée à partir des variables macro-économiques. On suppose que les prix effectifs avant la flambée des prix représentent la valeur fondamentale et on les ajuste sur les variables exogènes. Les coefficients obtenus et les valeurs des variables exogènes sont utilisés pour déterminer la valeur fondamentale lors de la flambée des prix. Si on constate que les prix s'écartent de la valeur fondamentale, on en déduit qu'il y a bulle. Cette approche permet d'identifier les déséquilibres sur le marché immobilier.

C’est par exemple la démarche de Case (1986), qui utilise comme déterminants fondamentaux la croissance de l’emploi, de la population, les taux d’intérêts, du revenu, les coûts de construction et un certain nombre d’autres variables. Il existe une méthode simple pour identifier l’existence d’une bulle : c’est d’observer le prix de sortie du cycle. En effet, Poterba (1980) montre que, dans le cas d’une bulle, le prix de sortie est non pas égal mais inférieur au prix d’entrée, dans la mesure où la flambée des prix a entraîné une production de logements neufs ne correspondant à aucune demande, ce qui va faire baisser les prix. Cette analyse permet de mieux comprendre les mécanismes à l'œuvre dans les cycles immobiliers.

On peut tester l’existence d’une bulle a posteriori en utilisant les loyers effectivement observés. Cornuel (1999 et 2003) montre que, pour le marché parisien, les prix sont bien expliqués par les loyers, et même que, de manière générale, les prix sont un peu en dessous de la valeur fondamentale. En particulier, le niveau des loyers rend bien compte du pic des prix de 1992. En définitive, au-delà de la flambée des prix, le fait marquant du marché du logement est qu’il y a eu également une augmentation des loyers. Pour la France entière, ils ont cru de 20 points de plus que l’inflation entre 1985 et 1995, ce qui souligne l’évolution dynamique du marché immobilier durant cette période.

Par conséquent, c’est la valeur fondamentale qui a connu une fluctuation. Cette flambée de loyers ne peut résulter elle-même d’une bulle, puisque le service n’est pas stockable, condition requise pour que l’on puisse avoir affaire à une bulle. C’est donc une fluctuation due à un phénomène réel, c’est-à-dire une chose dont il faut identifier la nature. Cette distinction est essentielle pour comprendre les dynamiques du marché immobilier et les facteurs qui influencent les prix et les loyers.









\end{document}