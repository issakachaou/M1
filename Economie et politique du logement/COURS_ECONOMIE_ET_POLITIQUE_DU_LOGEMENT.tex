\documentclass[a4paper, 12pt]{report}
\usepackage{graphicx}
\usepackage[utf8]{inputenc} 
\usepackage[french]{babel}
\usepackage[T1]{fontenc}
\usepackage{fancyhdr}
\usepackage{amsmath,amsfonts,amssymb, empheq}
\usepackage{eurosym}
\usepackage{booktabs}
\usepackage{wrapfig}
\pagestyle{fancy}
\fancyhead[R]{Université Paris-Est Créteil}
\fancyhead[L]{Économie et politique du logement}
\usepackage{array,multirow,makecell}
\usepackage{hyperref}
\setcellgapes{1pt}
\makegapedcells
\newcolumntype{R}[1]{>{\raggedleft\arraybackslash }b{#1}}
\newcolumntype{L}[1]{>{\raggedright\arraybackslash }b{#1}}
\newcolumntype{C}[1]{>{\centering\arraybackslash }b{#1}} 
%\renewcommand{\thechapter}{\Roman{chapter}}
%\setcounter{chapter}{1} % pour numéroter le chapitre 
\begin{document}
	\chapter*{Introduction}
	
\section{Logement et localisation}

Le logement est un bien de consommation aux multiples dimensions, dont le choix est avant tout déterminé par la localisation. Cette dernière influence fortement l'agrément du voisinage immédiat, l'accès aux écoles, aux emplois et aux services.

En France, une tendance marquée vers l’éloignement des centres urbains se dessine, avec une préférence croissante pour la maison individuelle par rapport à l’appartement, phénomène que l'on appelle la périurbanisation. Au milieu des années 1980, la construction de maisons individuelles était deux fois plus fréquente que celle des appartements.

Cependant, depuis 2008, on observe un repli de la construction de maisons individuelles au profit des logements collectifs, en raison de divers facteurs tels que les coûts environnementaux liés au transport automobile et le vieillissement de la population. Entre 2016 et 2021, le nombre de logements individuels a augmenté en moyenne de 0,6 \% par an, contre 1,3 \% pour les logements collectifs, montrant une tendance similaire à celle observée entre 2011 et 2016 (respectivement 0,9 \% et 1,2 \%).

Cette évolution contribue à une augmentation des écarts géographiques de prix et de loyers, révélant ainsi les défis croissants du marché immobilier.

Il existe au moins deux types d’inadéquations spatiales qui impactent le marché du travail.

La première concerne la situation où les emplois sont trop éloignés des logements, ce qui peut être dû à un manque de transports adéquats ou aux coûts de mobilité qui empêchent les individus de déménager vers les zones d'emploi. Cette distance géographique crée des difficultés pour les travailleurs qui souhaitent accéder à des opportunités professionnelles. 

La deuxième inadéquation réside dans le fait que l’offre de travail ne correspond pas à la demande en termes de qualifications. Ainsi, même si des emplois sont disponibles, les candidats peuvent ne pas posséder les compétences requises, ce qui entraîne un déséquilibre sur le marché de l'emploi et limite les possibilités d'embauche pour les entreprises. 

Ces deux problématiques soulignent l'importance d'une planification urbaine et d'une formation professionnelle adaptées pour favoriser une meilleure adéquation entre le logement et l'emploi.

Ce type d’inadéquation est renforcé par la ségrégation spatiale, qui crée des barrières supplémentaires à l'accès à l'emploi et à la formation. Elle peut également être à l’origine d’effets de pairs négatifs, où les caractéristiques du voisinage influencent défavorablement l'accumulation de capital humain, comme l'ont souligné Goux et Maurin en 2005. De plus, la ségrégation spatiale peut engendrer une discrimination territoriale, où le lieu de résidence envoie un signal négatif à un employeur potentiel, comme l'ont observé Petit et al. en 2011.

Ces dynamiques contribuent à perpétuer les inégalités sociales et économiques, rendant encore plus difficile la mobilité professionnelle et l'accès à des opportunités de développement personnel.

La mobilité résidentielle a diminué depuis le début du XXIe siècle, principalement sous l’effet du vieillissement de la population. Cela est particulièrement visible dans le secteur locatif social, où les locataires en place vieillissent et sont moins enclins à déménager. Dans les années 1980, le logement social a joué un rôle de tremplin vers la propriété occupante, comme l'ont noté Goffette-Nagot et Sidibé en 2016. Cependant, ce rôle s’atténue progressivement, limitant ainsi les possibilités de transition vers la propriété et contribuant à une stagnation dans les dynamiques de mobilité résidentielle.

La mobilité a aussi davantage diminué pour les personnes plus âgées, dont les logements sont mieux adaptés et pour qui les déménagements contraints par les revenus ou la santé sont moins fréquents, comme l'indique Laferrère en 2007. De plus, les mobilités de longue distance ont également diminué en raison de la crise économique et des difficultés qu’elle a engendrées sur le marché du travail.

\section{Le parc de logements}

De 1968 à 2013, la population de France métropolitaine a augmenté de 14 millions d’habitants, soit une hausse de 28 \%. Le nombre de résidences principales a quant à lui augmenté de 12 millions, représentant une augmentation de 76 \%. Par ailleurs, le parc total de logements, incluant les résidences secondaires, occasionnelles ou vacantes, a connu une augmentation de 80 \%.

De 2000 à 2016, entre 300 000 et 500 000 logements ont été construits chaque année, totalisant environ 6 millions. Pendant cette même période, la population a augmenté de 352 000 habitants par an en métropole.

La croissance du parc de logements est donc dynamique, avec un accroissement moyen de 1,1 \% par an en France depuis le début des années 1980. Le nombre de logements par habitant, qui était de 519 pour mille en 2012, est proche de celui de l’Allemagne et de l’Autriche, mais reste en dessous de celui de l’Espagne.

Depuis 1982, le nombre de résidences principales s’est accru de 52 \%, une hausse portée par la croissance démographique, qui a augmenté de 20 \% sur cette même période, ainsi que par la baisse de la taille des ménages, due à des facteurs tels que l'arrivée de couples plus tardifs, les ruptures d’union et le vieillissement démographique.

Conclusion : le taux d’occupation des résidences principales baisse de façon continue, illustrée par la diminution du nombre de personnes par logement, qui est passé de 3 personnes en moyenne en 1968 à 2,3 en 2016. Par ailleurs, le nombre de mètres carrés par personne a augmenté, passant de 31 à 40 entre 1984 et 2006, et reste stable depuis.

Ces moyennes masquent des inégalités entre des ménages plus âgés, qui occupent, en tant que propriétaires ou locataires dans le parc social, des logements rendus spacieux par le départ des enfants, et les plus jeunes, qui peinent à accéder au parc social et sont souvent locataires de logements plus petits.

Depuis le milieu des années 2000, le nombre de résidences principales croît à un rythme un peu moins soutenu que l’ensemble du parc, alors qu'il avait augmenté un peu plus vite entre 1990 et le milieu des années 2000.

Conséquence : la part des résidences principales diminue légèrement, passant de 82,6 \% en 1982 à 81,8 \% en 2021.

Depuis le début des années 2010, le nombre de résidences secondaires et de logements occasionnels augmente plus vite que l’ensemble du parc, alors qu'il avait progressé de façon moins soutenue que ce dernier à partir de 1990.

Le territoire n’est pas homogène, avec un taux de vacance élevé dans certaines zones, tandis que d'autres, comme la région Île-de-France ou PACA, connaissent un marché du logement tendu. La vacance désigne la proportion de logements inoccupés, souvent en attente de locataires ou de propriétaires.

Taux de vacance : 8 \% du stock de logements. Les logements vacants augmentent nettement depuis 2006, à un rythme supérieur à celui de l’ensemble du parc.

Taux de vacance : 8 \% du stock de logements. Les logements vacants augmentent nettement depuis 2006, à un rythme supérieur à celui de l’ensemble du parc. La vacance est forte surtout dans les villes moyennes et les unités urbaines de moins de 100 000 habitants. Elle peut provenir d’une inadéquation entre l’offre et la demande en matière de taille, de localisation, de prix, ou correspondre à des logements en mauvais état.

\section{Les conditions de logement}

Les conditions de logement montrent une ancienneté des logements variée : 30 \% d'entre eux datent encore d’avant 1949, tandis que 24 \% du parc actuel (31 \% du parc collectif) a été bâti entre 1949 et 1974. Environ 40 \% des logements (43 \% des maisons) ont été construits après 1975. Aujourd’hui, le confort de base est généralisé à la quasi-totalité des logements. En 2013, seulement 1 \% des logements manquaient du confort sanitaire de base, et 3 \% présentaient plus d’un défaut grave d’isolation thermique, d’étanchéité ou d’installation électrique.

La taille des résidences principales a augmenté, mais peut rester inadaptée au nombre d’occupants. Le surpeuplement, qui a été divisé par deux entre 1984 et 2006 et s'est stabilisé à 8 \%, concerne 21 \% des ménages dans l’unité urbaine de Paris, 16 \% des personnes de moins de 40 ans et 18 \% des ménages les plus modestes.

L’amélioration des conditions de logement n’occulte pas la question de la privation de domicile. En 2020, 3,6 millions de personnes étaient soit privées de domicile personnel (895 000), soit vivaient dans des conditions très difficiles (privation de confort ou surpeuplement) (2 880 000), soit étaient en situation d’occupation précaire (hôtel, caravanes, etc.), selon la Fondation Abbé-Pierre.

Les difficultés de sortir des situations de précarité sont particulièrement marquées pour ceux qui ont connu l’absence de logement personnel. Dans les établissements pour personnes en difficulté sociale, 52 \% des résidents sont des actifs, mais seulement 20 \% d'entre eux sont en emploi ou en stage, tandis que 32 \% se trouvent au chômage. De plus, il est préoccupant de constater que seulement 13 \% des individus avaient un logement personnel auparavant ; la majorité était hébergée par la famille, un tiers ou dans un établissement. Cette situation souligne les obstacles persistants auxquels font face les personnes en précarité pour retrouver une stabilité résidentielle.

\section{Le logement comme placement}

Le logement est également considéré comme un actif, en concurrence avec d'autres formes d’épargne, mais il présente la particularité d’être moins liquide. Les coûts de mobilité, qu'ils soient directs (droits de mutation, frais) ou indirects, sont élevés, ce qui complique les transactions. De plus, un logement nécessite un temps significatif pour être construit, ce qui limite la flexibilité des propriétaires et des investisseurs sur le marché immobilier.

Le taux de propriétaires atteint environ 75 \% à l’âge de la retraite et baisse peu par la suite. Le logement est perçu comme un placement qui protège contre le risque de hausses de loyers plus rapides que l’évolution des pensions de retraite, offrant ainsi une certaine sécurité financière aux retraités.

Le logement peut prendre la forme d'une maison individuelle ou d'un appartement. Environ 80 \% des propriétaires vivent dans une maison, tandis que 75 \% des locataires résident dans un appartement. Cette répartition souligne les préférences distinctes en matière de logement selon le statut de propriété.

Dans le secteur locatif, les mono-propriétaires d’immeubles ont pratiquement disparu. Les mono-propriétaires privés ont été affectés par des facteurs tels que les partages successoraux, les taxes et l'accès à la propriété. Parallèlement, les investisseurs institutionnels ont réorienté leurs actifs dans les années 2000, comme l'indiquent Bessière et Laferrère (2002), en vendant leurs immeubles à la découpe, ce qui a également contribué à cette évolution du marché locatif.

Les particuliers bailleurs sont généralement de petits propriétaires d’appartements. En 2013, 93,5 \% des propriétaires du parc locatif libre étaient des particuliers, contre 83,2 \% en 1996. Cet émiettement de l’investissement privé a des conséquences sur la mise en œuvre de travaux de réhabilitation énergétique. En effet, seulement 14 \% des résidences principales affichent une étiquette énergie A, B ou C, selon François (2014), ce qui souligne les défis en matière d'efficacité énergétique dans le parc locatif.

\section{Le statut d'occupation du logement}

En 2021, 58 \% des ménages étaient propriétaires de leur résidence principale, un chiffre stable depuis 2010 après une augmentation continue depuis 1982, où il était de 50 \%. La part des propriétaires sans charges de remboursement a crû jusqu’en 2010, atteignant 38 \% contre 27 \% en 1982, avant de se stabiliser à 38 \% en 2021 en raison du vieillissement de la population. Par ailleurs, la part des propriétaires accédants est restée stable à 20 \% depuis 15 ans, après avoir diminué de façon continue entre 1990 et le milieu des années 2000, passant de 26 \% à 20 \%. La part des ménages locataires de leur résidence s'est maintenue autour de 40 \% depuis 1990, un niveau légèrement inférieur à celui de 1982 (41 \%). Enfin, les logements détenus par des bailleurs publics représentent 17 \% du parc, tandis que ceux détenus par des bailleurs privés en représentent 23 \%.

Le choix du statut d’occupation est influencé par plusieurs facteurs. Parmi eux, on trouve les contraintes de crédit, telles que le taux d’apport personnel, le taux d’intérêt et la durée des emprunts. Le revenu joue également un rôle crucial, tout comme les anticipations concernant les prix et les loyers. De plus, la position dans le cycle de vie et les anticipations de mobilité sont des éléments déterminants qui impactent ce choix.

L’évolution de l’âge auquel la moitié d’une génération a accédé à la propriété offre un aperçu des tendances à long terme. Ainsi, l'âge était de 47 ans pour la génération née en 1924, de 36 ans pour celle née en 1944, de 33,5 ans pour celle née en 1952, et de 38 ans pour celle née en 1964. Pour les générations suivantes, cet âge est resté stable : les jeunes des générations nées à partir du milieu des années 1970 accédaient à la propriété entre 35 et 39 ans.

Entre 1996 et 2016, plusieurs évolutions économiques significatives ont été observées. Les prix à la consommation ont augmenté de 31 \%, tandis que le revenu disponible brut par ménage a connu une hausse de 40 \%. Parallèlement, le prix des logements anciens a été multiplié par 2,52, avec une augmentation encore plus marquée en Île-de-France, atteignant 2,67.

Après une baisse liée à la crise de 2008, la reprise du marché immobilier a été soutenue, notamment en Île-de-France. Les prix du logement ont légèrement diminué à partir de 2012, mais ils connaissent une reprise depuis 2016. Pour les primo-accédants, la hausse des prix freine l’achat, bien que cela soit en partie compensé par l’allongement de la durée des emprunts, qui est passée à 19,6 ans entre 2009 et 2013, contre 14,6 ans entre 1997 et 2001. De plus, la baisse des taux d’intérêt, passant de 5,2 \% entre 1997 et 2001 à 3,5 \% entre 2009 et 2013, a également contribué à atténuer cette pression.

Au fil du temps, les ménages non propriétaires tendent à être de moins en moins aisés. En effet, le pourcentage de ménages propriétaires dans le premier quartile de revenu a diminué de 8,9 points depuis 1984. Aujourd'hui, le revenu est un facteur encore plus déterminant pour accéder à la propriété qu'il ne l'était il y a 30 ans. Plusieurs conditions favorisent cet accès, notamment la présence de deux apporteurs de ressources au sein du couple, la stabilité d'un emploi en CDI, ainsi que l'aide financière des parents, devenue plus courante et plus importante. En 2013, un quart des accédants âgés de 25 à 44 ans ont reçu un don familial lors de l'achat, et l'apport personnel représentait environ un tiers du montant total de l'acquisition, un chiffre stable par rapport à 2001.

\section{Les dépenses de logement}

En 2018, 26,6 \% de la dépense de consommation finale des ménages était consacrée aux services liés au logement. Cela en fait le premier poste de dépense, bien loin devant les parts allouées à l'alimentation (17,1 \%) et aux transports (14,3 \%). Depuis les années 1990, la part des dépenses dédiées au logement a augmenté de manière significative, étant inférieure de 6,5 points par rapport à son niveau actuel.

\begin{wrapfigure}{r}{0.6\textwidth}
	\centering
\includegraphics[scale=0.5]{../../../Downloads/Screenshot 2024-09-17 at 18-23-33 Microsoft PowerPoint - CM_EL_intro_2024_2025 - CM_EL_intro_2024_2025.pdf}
\end{wrapfigure}

En 2013, 4,5 \% des locataires étaient en situation d'impayé de loyer ou de charges, un pourcentage similaire à celui de 2006. Par ailleurs, 11,5 \% des accédants rencontraient des difficultés de remboursement d'emprunt, contre 8,9 \% en 2006. Depuis le milieu des années 1980, les dépenses de logement des locataires ont considérablement augmenté. Les indices des loyers, à qualité constante, ont progressé plus rapidement que les prix à la consommation et le revenu disponible brut par ménage. Après déduction des aides au logement, un locataire du secteur privé dépensait en moyenne 50 \% de plus au m² pour son logement qu’un locataire du secteur social, alors qu’en 1984, cet écart était plus faible (43 \%). Cette hausse a été particulièrement marquée pour les ménages du premier quartile, en partie en raison d'un rattrapage de leurs conditions de confort par rapport aux autres ménages, comme l’a souligné Briant en 2010.

Depuis le début des années 2000, le niveau de revenu des locataires a diminué par rapport à celui des propriétaires. Dans le même temps, le taux d’effort, c’est-à-dire la part du revenu consacrée aux dépenses de logement, a augmenté. Cette hausse a été plus marquée pour les locataires du secteur privé, passant de 23,6 \% en 2001 à 28,4 \% en 2013, que pour ceux du secteur social, où il est passé de 20,2 \% à 24,1 \%. Cet accroissement de l'effort a particulièrement pesé sur les ménages les plus modestes, malgré le soutien des aides personnelles au logement, qui se sont concentrées sur ces foyers. Pour ces derniers, les aides au logement représentent plus de 30 \% du montant total de leurs dépenses de logement.
\newpage
\begin{wrapfigure}{r}{0.6\textwidth}
	\centering
\includegraphics[scale=0.5]{../../../Downloads/Screenshot 2024-09-17 at 18-30-05 Microsoft PowerPoint - CM_EL_intro_2024_2025 - CM_EL_intro_2024_2025.pdf}
\end{wrapfigure}

Depuis 1990, la part des dépenses de logement des locataires couverte par les prestations sociales s'est accrue. Cependant, en 2018, cette part a diminué, atteignant 14,6 \%. Cette baisse s'explique par deux facteurs principaux : l'abaissement de 5 € des différentes prestations sociales liées au logement et la mise en place du dispositif de réduction de loyer de solidarité (RLS) dans le secteur social.

\section{Politique du logement}

L'État joue un rôle important en régulant la construction, en définissant le droit de propriété, en encadrant les contrats de location et les loyers, et en subventionnant l'investissement ou la consommation. Historiquement, l'intervention publique s'est concentrée sur plusieurs aspects : la construction, notamment avec le développement du logement social, l'aide à l'accession à la propriété via le crédit et les taux aidés, ainsi que les aides monétaires directes au revenu des locataires, comme les APL. Dans les années 1970, la faillite des grands ensembles de logements sociaux a révélé une concentration des problèmes économiques et sociaux dans certaines zones. En réponse, les aides directes ont été étendues à de nouvelles catégories de bénéficiaires, tels que les étudiants.

Depuis 1984, pour contrer la baisse de l’offre locative privée, des incitations fiscales à l’investissement locatif ont été instaurées, allant du dispositif Quilès-Méhaignerie en 1984 au dispositif Pinel en 2014. Plus récemment, d’autres lois, telles que la loi DALO et la loi SRU, ont été mises en place pour réguler les baux, encadrer l’évolution et le niveau des loyers, taxer les logements vacants, et encourager la mixité sociale.

Les politiques du logement ont des effets complexes, rendant leur évaluation difficile. L’encouragement à la construction, qui vise à la fois à développer l’offre de logements et à soutenir l’emploi dans le secteur de la construction, comporte le risque de surinvestissement dans des zones où la demande est faible. Par ailleurs, les politiques de réglementation des loyers doivent trouver un équilibre entre une liberté totale, qui favorise la mobilité mais peut entraîner des loyers excessifs si le bailleur est en position de monopole, et un contrôle trop strict, qui risque de créer une pénurie de logements et de freiner la mobilité nécessaire.

Les aides « personnelles » à la consommation de logement s'élèvent en moyenne à 266 euros par mois. Elles bénéficient à 40 \% des locataires, dont la moitié dans le secteur social, un pourcentage qui a augmenté après 1984 avant de se stabiliser depuis 1996. En revanche, seulement 6 \% des accédants en bénéficient. Ces aides sont en baisse en raison de leur recentrage sur les ménages les plus modestes depuis la fin des années 1980, ainsi que de la diminution de l'accès à la propriété pour ces ménages.

Des études montrent que les aides au logement ont été partiellement absorbées par des hausses de loyer [Laferrère et Le Blanc, 2002 ; Fack, 2005]. L’encouragement à la construction locative privée n’est pas exempt d’effets pervers. En plus d'une mauvaise répartition géographique de l'offre, un impact sur les prix a également été observé dans certaines zones [Bono et Trannoy, 2013].

\end{document}