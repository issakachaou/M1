\documentclass[a4paper, 12pt]{report}
\usepackage{graphicx}
\usepackage[utf8]{inputenc} 
\usepackage[french]{babel}
\usepackage[T1]{fontenc}
\usepackage{fancyhdr}
\usepackage{amsmath,amsfonts,amssymb, empheq}
\usepackage{eurosym}
\usepackage{booktabs}
\usepackage{cancel}
\usepackage{hyperref}
\pagestyle{fancy}
\fancyhead[R]{Université Paris-Est Créteil}
\fancyhead[L]{Économie de l'incertitude}
\usepackage{array,multirow,makecell}
\setcellgapes{1pt}
\makegapedcells
\newcolumntype{R}[1]{>{\raggedleft\arraybackslash }b{#1}}
\newcolumntype{L}[1]{>{\raggedright\arraybackslash }b{#1}}
\newcolumntype{C}[1]{>{\centering\arraybackslash }b{#1}} 
%\renewcommand{\thechapter}{\Roman{chapter}}
%\setcounter{chapter}{1} % pour numéroter le chapitre 
\begin{document}
	\chapter*{Introduction}
	
\section{Hypothèse d'information parfaite}

Jusqu'à présent, vous avez étudié en microéconomie les comportements des agents (consommateur, producteur) sous l'hypothèse d'information parfaite. Acheteurs et vendeurs sont parfaitement informés des caractéristiques des produits et des facteurs de production et des prix auxquels ils sont proposés. 

Cette hypothèse exclut toute forme incertitude : aucun agent n'a jamais de doute, que ce soit sur la qualité des produits qu'il achète ; la qualification des travailleurs qu'il
embauche ; les conditions de travail dans l'entreprise dans laquelle il envisage de travailler ; l'évolution future des prix, des salaires, des taux d'intérêt ; l'âge auquel il prendra sa retraite… ou même la date de son décès. 

\section{Omniprésence de l'incertitude}

L'incertitude et l'imperfection de l'information jouent pourtant un rôle important dans de nombreux champs de l’économie : finance, assurance, retraites, économie du travail, économie de la santé, économie de l’environnement, etc.

L'incertitude est ainsi omniprésente dans la réalité : le salarié peut être licencié (ou promu), le chef d'entreprise peut faire faillite à la suite de la défaillance d'un fournisseur ou bénéficier des ennuis judiciaires d'un concurrent, l'automobiliste peut avoir un accident, certaines consommations peuvent rendre malade à plus ou moins long-terme, les maisons peuvent bruler, la terre peut trembler… une pandémie peut survenir, etc.

Dans certaines situations, l'incertitude est même un élément essentiel du problème qui se pose à l'agent économique qui doit prendre une décision : Assurances, garanties, placements financiers, sociétés d'intérim, agences immobilières, publicités… tout cela n'existe que car l'incertitude existe !

Ce cours a pour objectif d’étudier dans quelle mesure les résultats établis en microéconomie sous l'hypothèse simplificatrice d’information parfaite sont modifiés par la prise en compte de l'incertitude.

\section{Microéconomie de l'incertitude}

La microéconomie de l'incertitude n'est pas une rupture complète avec ce que vous avez appris sous l'hypothèse d'information parfaite, mais plutôt une généralisation de vos connaissances : l'univers certain que vous connaissez s'analyse, nous le verrons, comme un cas particulier de l'univers incertain. 

Dans certains cas, il est par ailleurs évidemment raisonnable de négliger l'incertitude. Par exemple, quand vous achetez votre café, on peut supposer que vous savez de manière certaine l'utilité que vous apportera sa consommation (même s'il existe, certes, un risque infime que ce café contienne de l'arsenic ou tout autre produit inattendu).

Les résultats d'univers certain seront seulement reformulés dans un cadre d’analyse plus large. Vous allez voir apparaitre un aspect des préférences qui n'a pas de sens en univers certain et qui est omniprésent en univers incertain : l'attitude face au risque des agents.

Vous verrez aussi pourquoi il est raisonnable de penser que la plupart des individus n'aiment pas le risque, sont prêts à payer pour le diminuer et ont intérêt à le partager avec d'autres agents. 

\section{Question de l’intervention de l'État}

L'économie de l'incertitude définit précisément la notion de risque puis montre dans quelle mesure les individus ont intérêt à partager les risques qu'ils supportent (par exemple via des assurances, une diversification des placements financiers ou encore la famille).

L'économie de l'incertitude fait donc surgir une question familière aux économistes : comment les individus, laissés à eux-mêmes, vont-ils partager les risques qu'ils supportent ? Le marché conduit-il à une répartition optimale de ces risques ? Ou bien l'intervention de l'État est-elle souhaitable ? Peut-on créer des instruments ou des institutions qui permettraient un partage de ces risques meilleur que celui qui se forme spontanément ?

Nous retrouvons donc, en économie de l'incertitude, la grande question de l'intervention de l'État : est-il légitime qu'il intervienne ? Améliore-t-il la situation lorsqu'il le fait ? Doit-il réglementer, taxer ou subventionner, ou produire lui-même ?

Nous ne répondrons pas directement à ces questions dans ce cours, mais nous définirons précisément la notion de risque et en donnerons le cadre et les outils théoriques. C'est le travail préliminaire indispensable pour répondre à ces questions.

\chapter{Concepts de base}

L'incertitude qui pèse sur un problème économique peut être résumée par trois éléments. 

Les états de la nature : les événements qui peuvent se réaliser. Ils peuvent être écrits sous forme discrète (exemple : faire face à un sinistre ou non) ou continue (ex : taux de sinistre entre 0 et 1)

Les actions réalisables par l'agent étudié : s'assurer ou non contre un sinistre (2 actions), s'assurer un taux de remboursement en cas de sinistre (appartenant à l'intervalle $[0;1]$)

Les conséquences des actions pour un état de la nature donné : le montant de richesse finale selon qu'un sinistre a lieu ou non et que l'on s'est assuré ou non. Attention : on fait ici l'hypothèse simplificatrice que les conséquences des actions peuvent être ramenées à des montants monétaires.

\section{Les loteries}

Toutes ces informations (états de la nature, actions, conséquences des actions) peuvent être représentées sous la forme d'une matrice d'information et de loteries.

Notons $E$ l'ensemble des différents états de la nature $e_j$. 

Exemple avec trois états de la nature $E=\left\{ e_1,e_2,e_3 \right\}$

Notons $p_j$ les probabilités (objectives ou subjectives) associées aux états de la nature : $p_j=P\left[ e=e_j \right]$ avec $\sum_j^Jp_j=1$

Notons $A$ l'ensemble des actions possibles $a_i$ de l'agent

Notons $x_{ij}$ les conséquences des actions $a_i$ selon les différents
états de la nature $e_j$ où $i$ est l'indice de l'action et $j$ celui de l'état de
la nature

Dans notre exemple : $i\in \left\{ 1,2,3 \right\}\,;\,j\in \left\{ 1,2,3 \right\}$

\subsection{Écriture matrice d'information}

La matrice d'information s'écrit de la manière suivante :

\begin{center}
	\includegraphics[scale=0.5]{../../../Pictures/Screenshots/Capture d'écran 2024-09-09 225833}
\end{center}

Choisir une action $a_i$ revient à choisir des gains $x_{ij}$ quand l'état de la nature $e_j$ se réalise, sachant que cet événement aura lieu avec une probabilité $p_j$

\subsection{Remarque 1 : Risque vs incertitude ?}

On oppose souvent les notions de risque et d’incertitude. Le risque est défini par un ensemble d’états de la nature et par les probabilités associées à ces états. L'incertitude est cas où une quantification des probabilités associées
aux différents états de la nature n’est pas possible. On parle même parfois d’incertitude radicale lorsque nous ne sommes pas en mesure de savoir les différents états de la nature possible. 

Cette distinction remonte à l'économiste américain Franck Knight (1921) et est donc antérieure à la démonstration de Savage sur les probabilités subjectives (1954). Nous ne reprendrons pas cette distinction aujourd'hui considérée comme quelque peu dépassée : risque et incertitude seront ici considérés comme synonymes. Ainsi, incertitude n'est pas synonyme d'ignorance totale.

Tout au long de ce cours, nous supposerons que l’agent étudié sait que l'état de la nature qui se réalisera est l'un de ceux de la matrice d'information.Nous faisons par ailleurs l'hypothèse simplificatrice qu'il est capable d'attribuer une probabilité subjective à chaque état de la nature (Savage, 1954)

\subsection{Remarque 2 : Univers certain vs incertain}

Lorsqu'au moment de la décision, une au moins des actions possibles a plus d'une conséquence possible, on dit que le décideur prend sa décision en univers incertain.

A l’inverse, lorsque toutes les actions possibles $a_i$ pour un décideur ont chacune une seule conséquence possible $x_{ij}$, on dit que le décideur prend sa décision en univers certain.

Il s’agit d’un cas particulier de l'univers incertain dans lequel la matrice d'information ne comprend qu'une colonne : celle de l'état de la nature dont on sait qu'il va se réaliser !

\subsection{Écriture loteries}

A chaque action $a$ (chaque ligne de la matrice d'information) on associe une loterie

Loterie discrète (cas le plus commun dans ce cours)

$$
\begin{matrix}
	a=\begin{cases}
		z_1 \quad z_2 \quad \cdots \quad z_I & \\
		p_1 \quad p_2 \quad \cdots \quad p_I &
	\end{cases} &  0<p_i \leq 1 & \sum_{i=1}^I p_i=1 \\
\end{matrix}
$$

Les $z_i$ sont les réalisations de la variable d'intérêt (gains ou pertes)
qui surviennent avec des probabilités $p_i$, $I$ événements possibles

Loterie continue : dans ce cas, on n'utilise plus des probabilités discrètes mais une loi de probabilité continue, caractérisée par une fonction de répartition $F_a(z)$ ou une densité $f_a(z)$

$$
F_a(z)=\mathbb{P}\left[ Z\le z \right]=\int_{-\infty }^{Z}f_a(z)dz \; z \in A
$$

$A$ ensemble des réalisations possibles $z$ de la variable aléatoire $Z$

\subsection{Exemple 1}

\begin{center}
	\includegraphics[scale=0.5]{../../../Pictures/Screenshots/Capture d'écran 2024-09-10 001051}
\end{center}

Écrire les 3 loteries $a_1,a_2,a_3$ associées aux trois actions possibles de l'agent

$$
a_1=\begin{cases}
	\begin{matrix}z_1 & z_2 & z_3 \\ 0,1 & 0,4 & 0,5 \end{matrix}
\end{cases}
$$
$$
a_2=\begin{cases}
	\begin{matrix}z_1 & z_2 \\ 0,1 & 0,9 \end{matrix}
\end{cases}
$$
$$
\underbrace{a_3=\begin{cases}
		\begin{matrix}z_1 \\ 1 \end{matrix}
\end{cases}}_ {\text{Loterie certaine}}
$$

\subsection{Exemple 2 : assurance partielle}

Un ménage veut assurer sa voiture de valeur $v$ sachant que la probabilité d'accident est $p$. Le ménage s'assure pour un montant $z\le v$ et doit payer une prime d'assurance égale à $\beta z$ , avec $\beta \in ]0;1[$. En cas d'accident, son capital sera égal au montant remboursé $z$ moins la prime d'assurance $\beta z$. S'il n'y a pas d'accident, son capital sera de $v$ moins la prime d'assurance.

Écrire la loterie associée au problème
$$
a(z)=\begin{cases}
	\begin{matrix}z-\beta z & v-\beta z \\ p & 1-p \end{matrix}
\end{cases}
$$
Avec $z$ la variable de choix de l'agent

\subsection{Exemple 3 : jeu d'argent}

Une personne achète un jeu à gratter d'un montant $m$. Elle peut gagner le montant $z>m$ avec une probabilité $p = 0,25$.

Écrire la loterie associée au problème

$$
a=\begin{cases}
	\begin{matrix}z-m & -m \\ \underbrace{0,25}_{\text{Vous gagnez}}& \underbrace{0,75}_{\text{Vous perdez}}\end{matrix}
\end{cases}
$$

Remarque : Cette manière d'écrire les conséquences du jeu suppose qu'elles sont purement monétaires ce qui exclut qu'on prenne du plaisir à jouer, ou la répugnance pour les jeux d'argent. Plus généralement, nous supposerons dans ce cours que les actions, en elles-mêmes, ne procurent ni utilité ni désutilité

\subsection{Exemple 4 : risque de chômage}

Une personne peut être au chômage avec une probabilité $p$. Si elle travaille, son salaire est $w$, sinon il est égal à $\gamma w$ avec $0\le \gamma < 1$. La cotisation chômage est égale à $\tau w$ avec $0 < \tau < 1$.

Écrire la loterie associée au problème

$$
a=\begin{cases}
	\begin{matrix}\gamma w & w-\tau w \\ \underbrace{p}_{\text{Chômage}}& \underbrace{1-p}_{\text{Pas chômage}}\end{matrix}
\end{cases}
$$

\subsection{Exemple 5 : fonction de profit}

Une entreprise produit un bien qu'elle vend au prix aléatoire $P$. Pour le produire elle embauche $L$ travailleurs qu'elle rémunère au salaire certain $w$. Sa fonction de production est $Q=\sqrt{\ell}$ et le prix $P$ apparait avec une densité $f_P(p)$ (il s'agit de la distribution du prix).

Après avoir défini le programme de l'entreprise, la condition de premier ordre et le profit aléatoire en fonction du prix $\prod (p)$ , écrire la loterie continue associée au profit de l'entreprise.

Le programme de l'entreprise
$$
\begin{cases}\underset{\ell}\max \,\prod(p)=p\cdot q - w \cdot \ell \\ s.c. \, Q=\sqrt{\ell}\end{cases}
$$
Condition de premier ordre 
$$
\underset{\ell}\max \, p \cdot \sqrt{\ell}-w\cdot\ell
$$
$$
\frac{d \prod(p)}{d \ell}=\frac{1}{2}p\cdot \ell^{-\frac{1}{2}}-w=0
$$
$$
w=\frac{1}{2}p\cdot \ell^{-\frac{1}{2}}
$$
$$
2w=p\cdot \ell^{-\frac{1}{2}}
$$
$$
\frac{1}{2w}=\frac{1}{p}\ell^{\frac{1}{2}}
$$
$$
\ell^{\star}=\left( \frac{1}{2}\cdot\frac{p}{w}\right)^2
$$
Pour trouver le profit aléatoire on remplace $\ell$ par $\ell^{\star}$ dans la fonction de profit. 
$$
\prod(p)=p\cdot \sqrt{\ell^{\star}}-w\cdot\ell^{\star}
$$
$$
p\cdot \left( \frac{p}{2w} \right)-w\cdot\left( \frac{p}{2w} \right)^2
$$
$$
\frac{p^2}{2w}-\cancel{w}\cdot \frac{p^2}{2w^{\cancel{2}}}
$$
$$
\frac{p^2}{4w}>0
$$
La loterie finale de l'entreprise est donc :
$$
a=\begin{cases}\begin{matrix}\prod(p)=\frac{p^2}{4w}\\ f_P(p)\to \text{distribution prix}\end{matrix}\end{cases}
$$

\section{Le critère d'espérance mathématique}

Dans la sous-section 1.1, nous avons exploré la manière d'écrire un problème économique dans un contexte d'incertitude, en utilisant des outils tels que la matrice d'information et les loteries. Une question essentielle qui se pose est celle de la prise de décision par l'agent : comment choisit-il une action parmi les différentes options disponibles et comment compare-t-il les loteries ? Pour faciliter cette comparaison, un critère simple est souvent employé : l'approche par l'espérance mathématique, qui consiste à évaluer le gain « moyen » associé à chaque loterie. Cependant, il est crucial de noter que ce critère présente des limites significatives, comme le démontrent les paradoxes 1.3 et 1.4, qui soulignent l'insatisfaction que peut engendrer cette méthode d'évaluation.

\subsection{Définition}

L'espérance mathématique d'une variable aléatoire discrète $X$ de réalisations $(x_1,\cdots,x_I)$ qui surviennent avec des probabilités $(p_1,\cdots,p_I)$ est définie par :
$$
\mathbb{E}(X)=\sum_{i=1}^{I}p_i\cdot x_i
$$
L'espérance mathématique d'une variable aléatoire continue $X$ de réalisations $x \in A$ réel est définie par :
$$
\mathbb{E}(X)=\int_{x \in A}^{}xf(x)dx
$$
Avec $f(x)$ densité de probabilité de $X$.

\section{Le paradoxe de St Pétersbourg}

Le paradoxe de St Pétersbourg, formulé par Daniel Bernoulli en 1738, remet en question la validité du critère d'espérance mathématique. Dans ce jeu, une pièce équilibrée est lancée, et le joueur compte le nombre de fois où elle tombe sur pile avant qu'elle ne tombe sur face. Si le joueur obtient $I$ jets successifs sur pile, il remporte $2^{I}$ ducats. À la fin de l'expérience, Bernoulli interroge ses interlocuteurs sur le montant qu'ils seraient prêts à payer pour participer à ce jeu. Étonnamment, la plupart d'entre eux proposent des sommes relativement faibles, autour de 4 ducats, ce qui illustre un décalage entre l'espérance mathématique théorique du jeu et la valeur perçue par les joueurs.

\subsection{Loterie associée}

Quel montant le critère d'espérance mathématique conduirait-il à proposer ?
On écrit la loterie et calcule l'espérance mathématique. La probabilité de tomber $I$ fois de suite sur pile est égale à $(\frac{1}{2})^I$. Dans ce cas, le gain est alors de $2^I$. On obtient donc la loterie suivante :
$$
 \left\{\begin{matrix}
	1 & 2 & \cdots & 2^{I} & \cdots \\
	\frac{1}{2} &  \frac{1}{4}  & \cdots &  \frac{1}{2^{I+1}} & \cdots \\
\end{matrix}\right.
$$
\subsection{Espérance mathématique}

Espérance mathématique de cette loterie ? On vérifie que la somme des probabilités est bien égale à 1 :
$$
\sum_{i=1}^{+\infty}\left(\frac{1}{2}\right)^{i+1}=\frac{1}{2}\cdot\frac{1}{1-\frac{1}{2}}=1
$$
Remarque : c'est une suite géométrique de raison $\frac{1}{2}$

Espérance mathématique :
$$
\mathbb{E}(B)\sum_{i=1}^{+\infty}\left(\frac{1}{2}\right)^{i+1}2^i=\lim_{I \to +\infty} \sum_{i=1}^{I}\frac{1}{2}\cdot\left(\frac{1}{2}\right)^i\cdot 2^i=\lim_{I \to +\infty} \sum_{i=1}^{I}\frac{1}{2}=+\infty
$$
Un joueur devrait donc être prêt à donner tout ce qu'il possède pour jouer à ce jeu… ce qui ne correspond pas du tout à ce que l'on observe dans la réalité. C'est ce qu'on appelle un paradoxe expérimental. 

\section{Le paradoxe de l'assurance}

Le paradoxe de l'assurance met en lumière une contradiction intrigante : bien que le critère d'espérance mathématique soit un outil fondamental pour évaluer les risques, il ne suffit pas à expliquer pourquoi un marché de l'assurance existe. En effet, selon ce critère, les individus rationnels devraient choisir d'auto-assurer leurs risques, car la probabilité de perte et le montant de la prime ne justifient pas toujours le recours à un assureur. Pourtant, la réalité montre que les gens préfèrent souvent souscrire des polices d'assurance, motivés par des facteurs psychologiques, sociaux et économiques, tels que la peur de l'incertitude, le besoin de sécurité et la solidarité collective. Ainsi, le paradoxe soulève des questions sur la nature des décisions économiques et les comportements humains face au risque.

Considérons un particulier qui dispose d'une richesse non risquée $\omega$ et qui souhaite assurer un bien risqué de valeur $v$ pour un montant $z$. Pour obtenir cette indemnité $z$ en cas de sinistre, il doit régler une prime d'assurance d'un montant $\beta z$ avec $0 < \beta < 1$. Le sinistre survient avec une probabilité $p$.

Loterie associée à la richesse du particulier :
$$
W= \left\{\begin{matrix}
	\omega + (1-\beta)z& \omega + v -\beta z \\
	p & 1-p \\
\end{matrix}\right.
$$

\subsection{L'assureur}

L'assureur perçoit la prime d'assurance $\beta z$ que le sinistre ait lieu ou non et fait face à un coût de fonctionnement $c$,
en plus du remboursement $z$ qu'il doit effectuer en cas de
sinistre. Si le sinistre a lieu il fait une perte $\beta z - z - c < 0$
et s'il n'a pas lieu il réalise un gain de $\beta z - c$.

Loterie associée au profit de l'assureur :
$$
W= \left\{\begin{matrix}
	(\beta-1)z-c & \beta z-c \\
	p & 1-p \\
\end{matrix}\right.
$$

\subsection{Espérances mathématiques}

Espérance de richesse du particulier :
$$
\mathbb{E}(W)=p[\omega+(1-\beta)z]+(1-p)[\omega+v-\beta z]=\omega+(1-p)v+(p-\beta)z
$$
Espérance de profit de l'assureur :
$$
\mathbb{E}(\pi)=p[(\beta-1)z-c]+(1-p)[\beta z-c]=(\beta-p)z-c
$$

Il s'agit de fonctions linéaires de $z$. Le paramètre essentiel est la pente des droites.

\subsection{Demande d'assurance du particulier}

Le particulier va choisir le montant assuré $z$ qui maximise l'espérance de sa richesse :
$$
\begin{cases}\underset{z}\max \,\mathbb{E}(W)\\ s.c. \,0 \le z\le v\end{cases}
$$
Or, la dérivée de l'espérance de la richesse par rapport à \( z \) est la pente \( p - \beta \), il y a donc trois cas possibles. Si \( p < \beta \), l'espérance de la richesse décroît avec le montant assuré, ce qui implique que la demande d'assurance est nulle, \( z^d = 0 \). Si \( p = \beta \), l'espérance de la richesse ne dépend pas du montant assuré, rendant la demande d'assurance indéterminée, \( z^d \in [0, v] \). Enfin, si \( p > \beta \), l'espérance de la richesse croît avec le montant assuré, et le particulier choisit donc l'assurance complète, \( z^d = v \). Globalement, le particulier ne souhaite s'assurer que si \( p \geq \beta \).

\subsection{Choix de l'assureur}

L'assureur cherche à maximiser l'espérance du profit :
$$
\begin{cases}\underset{z}\max \,\mathbb{E}(\pi)\\ s.c. \, 0\le z\le v\,;\mathbb{E}(\pi)>0\end{cases}
$$
Or, la dérivée de l'espérance de la richesse par rapport à \( z \) est la pente \( \beta - p \), il y a donc trois cas possibles. Si \( p < \beta \), l'espérance de profit croît avec le montant assuré, et l'assureur a intérêt à offrir une assurance complète (sous réserve que l'espérance de profit soit positive), soit \( z^s = v \). Si \( p = \beta \), l'espérance de profit ne dépend pas du montant assuré, mais elle est surtout négative, soit $ (=(\beta-p) z - c = -c)$. Dans ce cas, l'assureur ne propose donc pas de contrat, soit \( z^s = 0 \). Enfin, si \( p > \beta \), l'espérance de profit de l'assureur est toujours négative, et l'assureur ne propose pas de contrat, soit \( z^s = 0 \).

\subsection{Offre et demande d'assurance}
En résumé, on a les fonctions d'offre et demande suivantes :
\begin{center}
\includegraphics[scale=0.5]{../../../Pictures/Screenshots/Capture d'écran 2024-09-23 161629}
\end{center}
Il ne peut donc pas y avoir de marché de l'assurance si on applique le critère de l'espérance mathématique. Pourtant, ce marché existe et a un poids important dans l'économie ce qui est paradoxale

\subsection{Conclusion sur les paradoxes}

Le paradoxe de St Pétersbourg et celui de l'assurance soulignent les limites du critère d'espérance mathématique. Deux critiques principales peuvent être faites à son encontre : premièrement, ce n'est pas forcément le gain ou la richesse en tant que tel qui importe aux agents, mais l'utilité qui y est associée. Deuxièmement, le critère d'espérance mathématique ne tient pas compte du fait que l'attitude vis-à-vis du risque peut être très différente d'un individu à l'autre. Nous allons donc introduire des critères spécifiques de prise en compte du risque, tels que les fonctions de Markowitz. Par la suite, dans le Chapitre 2, nous introduirons la notion plus large d'utilité espérée, qui sera retenue dans le reste de ce cours.

\section{Neutralité face au risque}

Avec le critère d'espérance mathématique retenu jusqu'à présent, un décideur est indifférent entre les richesses :
$$
W_1= \left\{\begin{matrix}
	\omega+50\\
	1 \\
\end{matrix}\right.
$$
$$
W_2= \left\{\begin{matrix}
	\omega-100 & \omega+200 \\
	0,5 & 0,5 \\
\end{matrix}\right.
$$

En effet, $\mathbb{E}(W_1) =\mathbb{E}(W_2) = \omega +50$ ne tient compte que du rendement moyen. Le décideur est donc indifférent entre la richesse $W_1$ ,certaine, sans risque, et la richesse $W_2$, aléatoire, risquée.  Le critère d'espérance mathématique représente les préférences d'un individu indifférent au risque ou encore neutre face au risque

On retiendra ainsi qu'un individu est neutre face au risque si et seulement si ses préférences sont représentées par le critère d'espérance mathématique. Pour améliorer la représentation des préférences des agents, il faudrait donc introduire, en plus de l'espérance mathématique de la richesse, une mesure du risque. C'est l'objet des fonctions de Markowitz.

\section{Fonctions de Markowitz}

Puisqu'une richesse est risquée lorsqu'elle peut prendre différentes valeurs, une première mesure du risque est la dispersion de la variable aléatoire représentant cette richesse. 

La variance constitue une mesure connue et intuitive de la dispersion et donc du risque :
$$
\mathbb{V}(W)=\mathbb{E}\left[(W-\mathbb{E}(W))^2\right]= \mathbb{E}(W^2)-\mathbb{E}(W)^2
$$
= Valeur moyenne de la distance au carré entre les réalisations de
$W$ et leur moyenne théorique $\mathbb{E}(W)$. Plus les réalisations s'écartent de leur valeur moyenne, plus la richesse est risquée = Moyenne des carrés - carré de la moyenne

\subsection{Définition}

Les fonctions de Markowitz sont une manière de représenter les préférences des agents par une fonction d'utilité à deux éléments, l'espérance et la variance de la richesse :
$$
U(W)=f\left(\mathbb{E}(W),\mathbb{V}(W)\right)
$$
La fonction de Markowitz est supposée croissante avec l'espérance de la richesse : pour un risque donné (i.e. à variance donnée), le décideur préférera la richesse qui a l'espérance la plus élevée : $\frac{\partial U(W)}{\partial \mathbb{E}(W)}>0$. L'effet du risque sur l'utilité dépend quant à lui de l'attitude vis-à-vis du risque de l'individu

\subsection{Attitude vis-à-vis du risque}

Trois hypothèses peuvent être faites concernant le lien entre utilité et risque :

L'utilité est une fonction décroissante du risque : $\frac{\partial U(W)}{\partial \mathbb{V}(W)}<0$. A espérance donnée, le décideur préfère la richesse qui présente. le risque le plus faible (i.e. la variance la plus faible). On dira que le décideur a de l'aversion pour le risque, ou encore
qu'il est riscophobe. 

L'utilité est indépendante du risque : $\frac{\partial U(W)}{\partial \mathbb{V}(W)}=0$. Dans ce cas, la fonction de Markowitz ne dépend plus que de
l'espérance de la richesse. On retrouve le cas d'un décideur neutre face au risque. Le critère d'espérance mathématique est un cas particulier des
fonctions de Markowitz

L'utilité est une fonction croissante du risque :  $\frac{\partial U(W)}{\partial \mathbb{V}(W)}>0$. Le décideur préfère la richesse qui présente le risque le plus élevé. On dira alors que le décideur a un goût pour le risque, ou encore qu'il est riscophile. Ce comportement reste cependant peu observé en pratique !

Attention : un riscophobe n'est pas une personne qui ne prend jamais de risque, ni même une personne qui ne prend pas de risque lorsque c'est possible… de même, un riscophile peut dans certains cas choisir une richesse certaine tout dépend des loteries proposées et du degré d'aversion pour le risque des individus, c'est l'arbitrage entre niveau de risque et richesse !

\subsection{Courbes d'indifférence de Markowitz}

\begin{center}
	\includegraphics[scale=0.5]{../../../Pictures/Screenshots/Capture d'écran 2024-09-23 183537}
\end{center}

\subsection{Fonction de Markowitz linéaire}

La spécification la plus usuelle et la plus simple de la fonction de Markowitz est la forme linéaire :
$$
U(W)= \mathbb{E}(W)-k\mathbb{V}(W)
$$
La constante $k$ représente le degré d'aversion pour le risque : plus $k$ est élevé, plus l'individu a d'aversion pour le risque

Si $k>0$ : l'utilité est d'autant plus faible que le risque (la variance)
est élevé, c'est le cas d'aversion pour le risque.

Si $k=0$ nous sommes dans le cas de la neutralité face au risque.

Si $k<0$ nous sommes dans le cas de goût pour le risque. 

La fonction de Markowitz linéaire est parfois appelée "critère espérance-variance" et est très utilisée. Elle impose la même sensibilité au risque de gain et de perte. 

\subsection{Exemple}

Supposons qu'un individu possède une richesse initiale certaine $\omega$ , que ses préférences soient correctement représentées par une fonction de Markowitz linéaire, et qu'il doive choisir entre deux loteries :
$$
X= \left\{\begin{matrix}
	10 & 40 \\
	0,5 & 0,5 \\
\end{matrix}\right.
$$
et 
$$
Y= \left\{\begin{matrix}
	-5 & 65 \\
	0,5 & 0,5 \\
\end{matrix}\right.
$$

Que peut-on dire du choix de cet individu selon son attitude vis-à-vis du risque ?
$$
W=\omega + \begin{matrix}
	X \\
	Y
\end{matrix}
$$
Critère  de choix : fonction linaire de Markowitz
$$
U(\omega+X)=\mathbb{E}(\omega+X)-k\mathbb{V}(\omega+X)
$$
$$
U(\omega+Y)=\mathbb{E}(\omega+Y)-k\mathbb{V}(\omega+Y)
$$
$$
\mathbb{E}(\omega+X)=\omega+\mathbb{E}(X)=\omega+10\cdot0,5+40\cdot0,5=\omega+25
$$
$$
\mathbb{E}(\omega+Y)=\omega+\mathbb{E}(Y)=\omega-5\cdot0,5+65\cdot0,5=\omega+30
$$
$$
\mathbb{V}(\omega+X)=\mathbb{V}(X)=\mathbb{E}(X^2)-\mathbb{E}(X)^2
$$
$$
0,5\cdot10^2+0,5\cdot40^2-25^2=225
$$
$$
\mathbb{V}(\omega+Y)=\mathbb{V}(Y)=\mathbb{E}(Y^2)-\mathbb{E}(Y)^2
$$
$$
0,5\cdot(-5)^2+0,5\cdot65^2-30^2=1225
$$

Un agent neutre au risque ne tient compte que de l'espérance et choisit la loterie qui conduit à l'espérance la plus élévé. Dans ce cas on a $Y \succ X$. 

L'agent riscophobe : $U(\omega+X) > U(\omega+Y)$
$$
U(\cancel{\omega}+X)=\mathbb{E}(\cancel{\omega}+X)-k\mathbb{V}(\cancel{\omega}+X)>U(\cancel{\omega}+Y)=\mathbb{E}(\cancel{\omega}+Y)-k\mathbb{V}(\cancel{\omega}+Y)
$$
$$
25-k\cdot225>30-k\cdot1225
$$
$$
1000k>5
$$
$$
\text{Si }k>0,005\; \text{Choix de $X$}
$$
$$
\text{Si }k<0,005\; \text{Choix de $Y$}
$$
$$
\text{Si }k = 0,005\; \text{Indiffrérence entre $X$ et $Y$}
$$

\subsection{Retour sur le pb de l'assurance}

Reprenons le problème de l'assurance vu en section 1.4. en supposant désormais que les préférences de l'assuré ne sont plus représentées par l'espérance mathématique mais par une fonction de Markowitz linéaire. Nous avions du côté du particulier la loterie suivante :
$$
W= \left\{\begin{matrix}
	\omega + (1-\beta)z& \omega + v -\beta z \\
	p & 1-p \\
\end{matrix}\right.
$$
Nous avions calculé son espérance :
$$
\mathbb{E}(W)=\omega+(1-p)v+(p-\beta)z
$$
Calculons maintenant sa variance pour introduire le risque :
$$
\mathbb{V}(W)=\mathbb{E}\left[(W-\mathbb{E}(W))^2\right]
$$
$$
p[\omega+(1-\beta)z-(\omega+(1-p)v)+(p-\beta)z]^2+(1-p)[\omega+v-\beta z-(\omega+(1-p)v+(p-\beta)z)]^2
$$
$$
\mathbb{V}(W)=p(1-p)(v-z)^2
$$
\subsection{Étude de la variance (i.e. du risque)}

Le risque est croissant avec l'écart entre la valeur du bien $v$ et le montant remboursé $z$ en cas de sinistre. Moins on s'assure, plus la variance est forte. Le risque varie également avec la probabilité de sinistre $p$.

$p(1-p)$ minimale en $p=0$ (sinistre impossible) et $p=1$ (sinistre certain). 

$p(1-p)$  maximale en $p=\frac{1}{2}$, c'est donc lorsque le sinistre a autant
de chances de survenir que de ne pas survenir que le risque est le
plus fort

$p(1-p)$  peut être interprété comme un indicateur d'incertitude sur les états de la nature
$$
\begin{cases}\underset{z}\max \,U(W)=\mathbb{E}(W)-k\mathbb{V}(W)=\omega+(1-p)v+(p-\beta)z-kp(1-p)(v-z)^2\\ s.c. \, 0\le z\le v\end{cases}
$$
Nous écartons le cas où la probabilité de sinistre $p$ est nulle ou égale à 1 : dans ce cas il n'y a plus d'incertitude et le choix devient trivial. Nous supposons donc que : $0<p<1$

De même, nous avons vu en section 1.4. qu'il est peu vraisemblable de supposer que l'assureur demande une prime d'assurance $\beta z$ inférieure à l'espérance de l'indemnité $pz$ (ce ne serait pas rentable pour lui). Nous supposons que : $p<\beta$

Enfin, l'assureur ne peut pas demander une prime $\beta z$ supérieure ou même égale à la somme $z$ qu'il s'engage à verser à l'assuré en cas de sinistre (aucun assuré n'accepterait de s'assurer dans ce cas !). Nous supposons donc que : $\beta<1$. Ainsi, nous avons la contrainte suivante sur les paramètres :
$$
0<p<\beta<1
$$


\subsection{Problème de l'assuré avec utilité Markowitz}
$$
\begin{cases}\underset{z}\max \,U(W)=\mathbb{E}(W)-k\mathbb{V}(W)=\omega+(1-p)v+(p-\beta)z-kp(1-p)(v-z)^2\\ s.c. \, 0\le z\le v\end{cases}
$$
Nous écartons le cas où la probabilité de sinistre $p$ est nulle ou égale à 1 : dans ce cas il n'y a plus d'incertitude et le choix devient trivial. Nous supposons donc que : $0<p<1$

De même, nous avons vu en section 1.4. qu'il est peu vraisemblable de supposer que l'assureur demande une prime d'assurance $\beta z$ inférieure à l'espérance de l'indemnité $pz$ (ce ne serait pas rentable pour lui). Nous supposons que : $p<\beta$

Enfin, l'assureur ne peut pas demander une prime $\beta z$ supérieure ou même égale à la somme $z$ qu'il s'engage à verser à l'assuré en cas de sinistre (aucun assuré n'accepterait de s'assurer dans ce cas !). Nous supposons donc que : $\beta<1$. Ainsi, nous avons la contrainte suivante sur les paramètres :
$$
0<p<\beta<1
$$

\subsection{Indemnité optimale}

Trois cas sont possible selon le degré d'aversion au risque :

$k=0$ (neutralité face au risque) : on se retrouve dans le cas de l'espérance mathématique que nous avons déjà étudié, donc il n'existe pas de marché de l'assurance. S'il n'existait que des propriétaires neutres face au risque, avec nos hypothèses, il n'existerait pas de marché de l'assurance

$k<0$ (goût pour le risque) : si le propriétaire est riscophile, son utilité de Markowitz est une fonction croissante à la fois de l'espérance et de la variance de la richesse. Or, ces deux fonctions, $\mathbb{E}(W)$ et $\mathbb{V}(W)$ sont décroissantes de l'indemnité $z$. Le propriétaire choisit donc de ne pas s'assurer : $z^d=0$. S'il n'existait que des propriétaires riscophiles, il n'y aurait pas non plus de marché de l'assurance. 

Le dernier cas est celui du propriétaire averse au risque ($k>0$). Toute hausse de $z$ a alors deux effets de sens contraire : la diminution de l'espérance $\mathbb{E}(W)$ , qui diminue l'utilité, et la diminution de la variance $\mathbb{V}(W)$, qui augmente l'utilité.

Dans ce cas, il faut résoudre le programme de l'assuré
$$
\begin{cases}\underset{z}\max \,U(W)=\mathbb{E}(W)-k\mathbb{V}(W)=\omega+(1-p)v+(p-\beta)z-kp(1-p)(v-z)^2\\ s.c. \, 0\le z\le v \\ 0 < p < \beta < 1 \end{cases}
$$
Condition de premier ordre
$$
\frac{d U(W)}{dz}=0\Leftrightarrow p-\beta+kp(1-p)2(v-z)=0
$$
Rappel : $g(f(x))'=g'(f(x))\cdot f'(x)$
$$
p-\beta + kp(1-p)2v-kp(1-p)2z=0
$$
$$
p-\beta + kp(1-p)2v=kp(1-p)2z
$$
$$
z^d=\frac{p-\beta + kp(1-p)2v}{kp(1-p)2}
$$
$$
z^d=\frac{p-\beta }{kp(1-p)2}+\frac{kp(1-p)2}{kp(1-p)2}v
$$
$$
z^d=v-\frac{\beta-p}{kp(1-p)2}
$$
Condition du second ordre 
$$
\frac{d^2U(W)}{dz^2}=-kp(1-p)2<0
$$
Remarque 
$$
z^d>0 \Leftrightarrow  v -\frac{\beta - p }{kp(1-p)2}>0
$$
$$
p>\beta - v(2kp(1-p))
$$
(Si cette valeur est strictement comprise entre 0 et $v$). Un marché de l'assurance peut donc bien exister lorsque les décideurs sont riscophobes

\subsection{Statique comparative}
$$
z^d=v-\frac{\beta-p}{kp(1-p)2}
$$
Le montant assuré $z^d$ augmente avec la valeur du bien risqué $v$ : toutes choses étant égale par ailleurs un riscophobe est prêt à payer une prime plus élevée pour un bien d'une plus grande valeur

Le montant assuré $z^d$ augmente avec la probabilité de sinistre

Le montant assuré $z^d$ baisse avec le taux de prime d'assurance qu'il faut payer $\beta$, c'est un effet prix classique

Le montant assuré $z^d$ augmente avec le degré d'aversion au risque $k$.

Ces résultats correspondent à l'intuition. Par contre, avec une fonction de Markowitz linéaire, le montant assuré ne dépend pas de la richesse non risquée $\omega$.

\section{Autres mesures du risque}

Les fonctions de Markowitz, qui évaluent le risque à travers la variance de la richesse aléatoire, constituent une approche classique en finance. Cependant, il existe d'autres manières d'appréhender le risque qui méritent d'être explorées. Parmi celles-ci, les fonctions de sécurité d'abord privilégient la protection du capital en minimisant les pertes potentielles. Les fonctions Maximin, ou critère de Wald, se concentrent sur le meilleur résultat possible dans le pire des scénarios, tandis que les fonctions Maximax visent à maximiser le gain dans le meilleur des cas. Les fonctions de Hurwicz, quant à elles, combinent ces deux perspectives en intégrant un coefficient d'optimisme. Enfin, les fonctions de regret évaluent le risque en tenant compte des opportunités manquées, permettant ainsi une prise de décision plus éclairée face à l'incertitude. Ces différentes approches enrichissent notre compréhension du risque et offrent des outils variés pour la gestion de portefeuille.

\subsection{Fonctions sécurité d'abord}

Soient les deux loteries suivantes auxquelles est soumis nun individu ayant une richesse certaine initiale $\omega$ :
$$
\begin{matrix}
	X=\left\{\begin{matrix}
		-5 & 5 & 10 \\
		0,1 & 0,2 & 0,7 \\
	\end{matrix}\right. & \text{Et} & Y=\left\{\begin{matrix}
		-20 & 5 & 10 \\
		0,01 & 0,29 & 0,7 \\
	\end{matrix}\right.
\end{matrix}
$$
$$
\mathbb{E}(\omega+X)=\omega+7,5 \; ; \;\mathbb{E}(\omega+Y)=\omega+8,25
$$
$$
\mathbb{V}(\omega+X)=21,25 \; ; \;\mathbb{V}(\omega+Y)=13,75
$$

Si on mesure le risque par la variance, un individu riscophobe choisira toujours la loterie $Y$ qui a une plus grande espérance de richesse et une plus petite variance. Toutefois, on peut imaginer qu'un individu, très inquiet à l'idée de perdre éventuellement 20, préférera la loterie $X$.

Dans ce cas, l'individu ne mesure pas le risque par la variance (i.e. la somme des carrés des écarts à l'espérance mathématique, pondérés par leur probabilité). Un tel individu mesurera le risque par la somme des carrés des pertes, pondérées par leur probabilité :
$$
\mathbb{V}^{\star}(\omega+X)=(-5)^2\cdot 0,1=2,5 \; ; \;\mathbb{V}^{\star}(\omega+X)=(-20)^2\cdot 0,01 =4
$$
Avec cette nouvelle mesure, la loterie $X$ devient moins risquée que $Y$ et les choix de l'individu riscophobe dépendront de la comparaison des espérances et des risques, en fonction du degré d'aversion au risque

On peut alors définir une nouvelle fonction d'utilité, sous la forme suivante :
$$
U(W)=\mathbb{E}(W)-k\mathbb{V}^{\star}(W)
$$
Ces fonctions d'utilité sont appelées fonctions d'utilité sécurité d'abord

\subsection{Fonctions Maximin (critère de Wald)}
Soit une richesse initiale certaine $\omega$ et trois loteries :
$$
\begin{matrix}
	X=\left\{\begin{matrix}
		1 & 20 & 30 \\
		0,1 & 0,2 & 0,7 \\
	\end{matrix}\right. & Y=\left\{\begin{matrix}
		2 & 3 & 4 \\
		0,1 & 0,2 & 0,7 \\
	\end{matrix}\right. & Z=\left\{\begin{matrix}
		-10 & 2 & 50 \\
		0,1 & 0,2 & 0,7 \\
	\end{matrix}\right.
\end{matrix}
$$
Les fonctions Maximin correspondent à un individu pessimiste qui ne considère que la plus faible valeur de chaque richesse :
$$
\begin{matrix}
	m(\omega+X)=\omega+1 & m(\omega+Y)=\omega+2 & m(\omega+Z)=\omega-10
\end{matrix}
$$
Choix de la loterie qui garantit le plus grand minimum : ici il s'agit de la loterie $Y$

\subsection{Fonctions Maximax}


Soit une richesse initiale certaine $\omega$ et trois loteries :
$$
\begin{matrix}
	X=\left\{\begin{matrix}
		1 & 20 & 30 \\
		0,1 & 0,2 & 0,7 \\
	\end{matrix}\right. & Y=\left\{\begin{matrix}
		2 & 3 & 4 \\
		0,1 & 0,2 & 0,7 \\
	\end{matrix}\right. & Z=\left\{\begin{matrix}
		-10 & 2 & 50 \\
		0,1 & 0,2 & 0,7 \\
	\end{matrix}\right.
\end{matrix}
$$
Les fonctions Maximax correspondent à un individu optimiste qui ne considère que la valeur la plus élevée de chaque richesse :
$$
\begin{matrix}
	M(\omega+X)=\omega+30 & M(\omega+Y)=\omega+4 & M(\omega+Z)=\omega+50
\end{matrix}
$$
Choix de la loterie qui garantit le plus grand maximum : ici la loterie $Z$

\subsection{Fonctions de Hurwicz}

Les fonctions de Hurwicz pondèrent Maximin et Maximax pour les différentes loteries, soit pour la loterie $X$ :
$$
U(\omega +X)= \alpha m (\omega +X)+(1-\alpha)M(\omega +X)
$$
$\alpha$ mesure le degré de pessimisme de l'individu

\subsection{Fonctions de regret }

Les fonctions vues jusqu'à présent attribuent à une richesse une utilité indépendante des autres richesses possibles… nous allons relâcher cette hypothèse

Réécrivons les trois loteries sous forme de matrice d'information :
\begin{center}
	\includegraphics[scale=0.4]{../../../Pictures/Screenshots/Capture d'écran 2024-09-25 213135}
\end{center}
Si l'état $e_1$ se réalise, le décideur saura (mais trop tard) que la meilleure action aurait été $Y$ : s'il a choisit $X$, il obtiendra une richesse de $\omega + 1$ au lieu de $\omega + 2$,  on dit qu'il regrettera alors $\omega+2-(\omega+1)=1$
\begin{center}
\includegraphics[scale=0.4]{../../../Pictures/Screenshots/Capture d'écran 2024-09-29 214207}
\end{center}
Avec ce critère, l'individu choisira la loterie associée à la plus petite somme des regrets, soit ici $X$

\subsection{Conclusion sur la mesure du risque}

Nous avons examiné plusieurs fonctions, telles que celles de Markowitz et Maximin, qui introduisent la notion de risque dans les décisions des agents économiques. La représentation des préférences par une fonction de Markowitz permet de résoudre le paradoxe de l'assurance, en conduisant à une fonction de demande d'assurance présentant des propriétés relativement satisfaisantes. Dans le chapitre 2, nous approfondirons la notion plus générale d'utilité espérée, qui inclut plusieurs des fonctions discutées comme des cas particuliers. C'est cette notion d'utilité espérée que nous retiendrons par la suite pour analyser les choix en situation d'incertitude, offrant ainsi un cadre théorique robuste pour comprendre les comportements des agents face au risque.

\chapter{L'utilité espérée}

Ce chapitre vise à présenter les fonctions d'utilité espérée, leurs fondements et leurs propriétés. Ce concept d'utilité espérée sera utilisé dans le reste du cours. Nous comparerons notamment les fonctions d'utilité espérée aux fonctions d'utilité que vous avez vu jusqu'à présent en microéconomie. Nous appliquerons ensuite la théorie de l'utilité espérée au cas de l'assurance. Enfin, nous présenterons les limites de l'utilité espérée

\section{Définition}

On appelle fonctions d'utilité espérée les fonctions d'utilité dont la valeur est l'espérance, non pas de la richesse, mais de l'utilité de cette richesse. Elles s'écrivent :
$$
U(W)=\mathbb{E}\left( u(W)\right) 
$$
Pour écrire la fonction d'utilité espérée, on transforme la richesse aléatoire $W$ par la fonction d'utilité $u$. On obtient alors une nouvelle
variable aléatoire $u(W)$. L'utilité espérée de $W$ est l'espérance de cette nouvelle variable aléatoire. Cela permet de prendre en compte le fait que ce n'est pas la richesse qui compte, mais l'utilité qui y est associée. L'utilité espérée étend le concept de fonction d'utilité à la décision dans l'incertitude… d'où l'introduction de l'espérance et des probabilité. 

Considérons une richesse aléatoire $W$ comportant une partie certaine $\omega$ et une partie aléatoire $X$ :
$$
W=\omega+X
$$
Dans le cas discret, i.e. lorsque la richesse $W$ prend un nombre fini de valeurs, l'utilité espérée s'écrit :
$$
U(W)=\mathbb{E}\left( u(W)\right)=\sum_{i=1}^{I}p_iu(w_i)=\sum_{i=1}^{I}p_iu( \omega+x_i)
$$
Dans le cas continu, l'utilité espérée s'écrit :
$$
U(W)=\mathbb{E}\left( u(W)\right)=\int_{a}^{b}f(x)u(w)dx=\int_{a}^{b}f(x)u(\omega+x)dx
$$
Où $a$ et $b$ désignent les valeurs minimales et maximales de la richesse aléatoire $x$ et $f(x)$ sa densité (distribution)

Il existe autant de fonctions d'utilité espérée que de fonctions d'utilité $u$
$$
u'>0
$$
Rappel de l'hypothèse de non satiété : un agent préférera toujours avoir une richesse plus grande (consommer plus)
\section{Fondements}

L'histoire des fonctions d'utilité espérée repose sur les
travaux de Daniel Bernoulli (1738), John von Neumann et Oskar Morgenstern (1944) et de Leonard J. Savage (1954)

\subsection{Bernoulli (1738)}

Afin de résoudre le paradoxe de St Pétersbourg (\textit{cf.} 1.3.), Bernoulli a supposé que les préférences des agents étaient représentées par l'espérance d'une fonction de la richesse, fonction croissante (hypothèse de non satiété) mais à un taux décroissant (utilité marginale décroissante)

Par exemple, en prenant la fonction d'utilité logarithmique, on obtient la fonction d'utilité espérée suivante :
$$
U(W)=\mathbb{E}\left( u(W)\right)=\mathbb{E}\left( \ln(W) \right)
$$
Supposer une fonction d'utilité logarithmique lève le paradoxe de St Pétersbourg : le prix que les agents sont prêts à payer pour participer au jeu devient fini

\subsection{Von Neumann et Morgenstern (1944)}

Ces deux auteurs sont considérés comme les pères fondateurs de la théorie de l'utilité espérée… à tel point que les fonctions d'utilité espérée sont souvent appelées fonctions d'utilité de Von Neumann \& Morgenstern

ls ont démontré l'existence des fonctions d'utilité espérée, à partir d'une liste d'hypothèses relatives aux préférences des agents sur les loteries, en supposant les probabilités des différents états de la nature comme données

Ils ont donné des fondements axiomatiques aux fonctions d'utilité espérée : si l'on admet leurs axiomes (laissés de côté ici par souci de simplicité), on doit admettre que les préférences des agents sont correctement représentées par des fonctions d'utilité espérée

\subsection{Savage (1954)}

Von Neumann et Morgenstern raisonnaient sur des probabilités données des états de la nature, donc supposées objectives

Comme nous l'avons déjà souligné dans le Chap. 1, Savage a donné des fondements axiomatiques à l'existence de probabilités subjectives

\section{Quelques propriétés}

\subsection{Fonctions globales / élémentaires}

Nous allons voir dans cette sous-section 2.3. quelques propriétés des fonctions d'utilité espérée. Commençons par nous attarder sur la distinction entre fonction d'utilité globale et fonction d'utilité élémentaire.

La fonction d'utilité espérée $U(W)=\mathbb{E}\left( u(W)\right)$ est une
fonction composée : sa valeur dépend des valeurs possibles de la richesse par l'intermédiaire de la fonction $u$.

$u$ est appelée fonction d'utilité élémentaire

$U$ est appelée fonction d'utilité globale

En pratique, il n'y a pas de distinction marquée entre $U$ et $u$ (nous la ferons toutefois dans l'écriture pour éviter toute confusion dans ce cours !). En effet :

Il suffit de donner la fonction $u$ pour connaitre parfaitement la forme de $U$. Par exemple, lorsqu'on suppose une fonction d'utilité espérée logarithmique, on sait sans ambiguïté que l'on désigne la fonction : $U(W)=\mathbb{E}\left( \ln(W) \right)$

Dans le cas d'un univers certain, les fonctions $U$ et $u$ sont confondues. Ainsi,
$$
\text{Si}\;W\left\{\begin{matrix}
	w\\1
\end{matrix}\right.
$$
On a : 
$$
U(W)=\mathbb{E}\left ( u(W) \right )=\mathbb{E}\left ( u(w) \right )=u(w)
$$
On voit bien que l'univers certain (et ses fonctions d'utilité $u$) constitue un cas particulier de l'univers incertain (et ses fonctions d'utilité espérée $U$) !

\subsection{Unicité fonction d'utilité globale}

Vous savez, grâce à vos cours de microéconomie, qu'une fonction d'utilité n'est définie qu'à une transformation strictement croissante près : les mêmes préférences peuvent être représentées par une multitude de fonctions d'utilité

C'est également vrai pour la fonction d'utilité espérée $U$ : toute transformation strictement croissante d'une fonction d'utilité espérée $g(U)$ représentera les mêmes préférences

MAIS cette nouvelle fonction $g(U)$ n'aura pas nécessairement la forme d'une fonction d'utilité espérée (forme qui est commode en pratique)…

Pour continuer à avoir une fonction d'utilité espérée, il faudra se limiter aux transformations affines strictement croissantes de la forme :
$$
g(U)=aU+b, \quad a>0
$$
On retiendra que les fonctions d'utilité espérée ne sont définies qu'à une transformation affine strictement croissante près

\subsection{Neutralité face au risque}

Supposons une fonction d'utilité élémentaire de la forme :
$$
u_1(w)=aw+b\;;\;a>0 \equiv U_1(W)=\mathbb{E}(aW+b)
$$
La fonction d'utilité espérée n'étant définie qu'à une transformation affine strictement croissante près, les mêmes préférences peuvent être représentées par :
$$
u_2(w)=w\ \equiv U_2(W)=\mathbb{E}(W)
$$
On retombe sur le critère d'espérance mathématique de la richesse vu dans le Chap. 1, utilisé pour décrire les préférences d'individus neutres face au risque

On retiendra les deux points suivants :

Les fonctions espérance de la richesse constituent un cas particulier des fonctions d'utilité espérée. De même, nous verrons plus loin que certaines fonctions de Markowitz sont des cas particuliers des fonctions d'utilité espérée.

Les fonctions d'utilité espérée dont la fonction élémentaire $u$ est affine, et donc a fortiori la fonction identité $u(w)=w$, représentent les préférences d'un agent neutre face au risque

\section{Application à l'assurance}

Reprenons le problème de l'assurance dans le cadre de la théorie de l'utilité espérée et supposons des préférences logarithmiques $u(w)=\ln(w)$. La loterie associée à l'assurance est toujours :
$$
W= \left\{\begin{matrix}
	\omega + (1-\beta)z& \omega + v -\beta z \\
	p & 1-p \\
\end{matrix}\right.
$$
Écrire l'utilité espérée associée à cette loterie

Écrire le programme de l'assuré

Résoudre le programme pour en déduire la demande
d'assurance
$$
U(w)=\mathbb{E}(u(w))=\mathbb{E}(\ln(w))
$$
$$
p\ln \left[ \omega + (1-\beta)z\right]+(1-p)\ln\left[ \omega + v - \beta_z \right] 
$$
$$
\begin{cases}\underset{z}\max U(W)\\ s.c. \, 0\le z\le v \\ 0 < p < \beta < 1 \end{cases}
$$
$$
\frac{dU(W)}{dz}=0 \Leftrightarrow p \cdot \frac{1-\beta}{\omega + (1-\beta)z}+(1-p)\frac{\beta}{\omega + v - \beta_z}=0
$$
$$
\Leftrightarrow  p \cdot \frac{1-\beta}{\omega + (1-\beta)z} = (1-p)\frac{\beta}{\omega + v - \beta_z}
$$
$$
p(1-\beta)(\omega + v - \beta_z) = (\omega +(1-\beta)z)(1-p)\beta
$$
$$
z^d = \frac{p}{\beta}v-\left( \frac{\beta-p}{(1-\beta)\beta} \right) \omega
$$

Si $z^d<0$ solution en coin $z^d=0$

Si $z^d>v$ solution en coin $z^d=v$

Pour tous les autres cas voir la fonction de demande.

\subsection{Statique comparative}
$$
z^d = \frac{p}{\beta}v-\left( \frac{\beta-p}{(1-\beta)\beta} \right) \omega
$$
Le montant assuré $z^d$ est une fonction croissante de la probabilité
de sinistre $p$ et de la valeur $v$ du bien risqué (trivial)

$z^d$ est une fonction décroissante du prix $\beta$ de l'euro d'indemnité,
c'est un effet prix classique (dérivée négative)

$z^d$ dépend négativement de la richesse certaine $\omega$. Ce résultat, nouveau, souligne que les individus les plus riches sont leur propre assureur.

Il n'existe pas d'équivalent à la mesure $k$ d'aversion face au risque des fonctions de Markowitz. Nous verrons plus tard (Chap. 3) que la fonction d'utilité espérée logarithmique représente les préférences d'un agent riscophobe. Nous verrons également comment mesurer son degré d'aversion au risque.

\section{Critiques}

Les comportements déduits des fonctions d'utilité espérée sont généralement compatibles avec ce qui est observé dans la réalité

Il existe toutefois des invalidations expérimentales à la théorie de l'utilité espérée, les deux plus connues sont le paradoxe d'Allais et le paradoxe d'Ellsberg

Ces paradoxes remettent en cause les axiomes sur lesquels repose la théorie de l'utilité espérée (axiomes non détaillés jusqu'à présent)

\subsection{Axiomes de rationalité univers certain}

"Rationnel" $\neq$ "Raisonnable" : supposer qu'un individu est rationnel, c'est supposer que ses actions sont en adéquation avec le but qu'il poursuit (atteindre ce qu'il préfère), il a une cohérence dans son comportement

En univers certain, supposer un individu rationnel c'est supposer qu'il a des préférences sur l'ensemble des conséquences possibles $A$. Il existe une relation notée $\succsim$, permettant de classer les conséquences, qui satisfait :

L'axiome de réflexivité : chaque conséquence $A_i$ est bien définie

L'axiome de complétude : $\forall i,j $ soit $A_i\succ A_j$ soit $A_j\succ A_i$ soit $A_i \sim A_j$

L'axiome de transitivité : $\forall i,j,k $ si $A_i\succsim A_j$ et $A_j\succsim A_k$ alors $A_i\succsim A_k$

(+ Axiome de continuité pour pouvoir représenter les préférences
par une fonction d'utilité : les préférences ne font pas de saut)

\subsection{Axiomes fonctions d'utilité espérée}

Absence d'illusion stochastique : tous les agents sont indifférents entre deux loteries (éventuellement composées) qui donnent les mêmes lots ultimes avec les mêmes probabilités

Probabilité croissante :
Soient 
$$
\begin{matrix}
W_1= \left\{\begin{matrix}
	w^{min} & w^{max} \\
	1-p & p \\
\end{matrix}\right. & Et & 
W_2= \left\{\begin{matrix}
	w^{min} & w^{max} \\
	1-q & q\\
\end{matrix}\right.
\end{matrix}
$$
$$
\begin{matrix}
W_1 \succ W_2 \Leftrightarrow p > q & Et &  W_1 \sim W_2 \Leftrightarrow p=q
\end{matrix}
$$
Cohérence dans les préférences
$$
\exists \alpha,\beta \in[0,1] |W_1 \succ W_2 \succ W_3 
$$
$$
\alpha W_1 + (1-\alpha)W_3 \succ W_2 \text{ et } W_2 \succ \beta W_1 + (1-\beta)W_3
$$
Indépendance 
$$
W_1 \succ W_2 \Rightarrow \alpha W_1 + (1-\alpha)W_3 \succ \alpha W_2 +(1-\alpha)W_3
$$
Pas de complémentarité ni de substituabilité entre les différentes richesses aléatoires

\subsection{Paradoxe d'Allais, 1952}

Soit un individu soumis à un 1\ier~choix entre 2 loteries :
$$
\begin{matrix}
	A= \left\{\begin{matrix}
		100 \\
		1 \\
	\end{matrix}\right. & Et & 
	B = \left\{\begin{matrix}
		0 & 100 & 500 \\
		0,01 & 0,89 & 0,10 \\
	\end{matrix}\right.
\end{matrix}
$$
Puis à un 2\ieme~choix entre :
$$
\begin{matrix}
	C= \left\{\begin{matrix}
		0 & 100 \\
		0,89 & 0,11 \\
	\end{matrix}\right. & Et & 
	D = \left\{\begin{matrix}
		0 & 500  \\
		0,90 & 0,10 \\
	\end{matrix}\right.
\end{matrix}
$$
Si les préférences sont représentables par une fonction d'utilité espérée, alors un individu qui préfère $A$ à $B$ doit préférer $C$ à $D$ :
$$
\mathbb{E}\left( u(A) \right)>\mathbb{E}\left( u(B) \right)
$$
$$
\equiv u(100)> 0,01 u(0)+0,89u(100)+0,1u(500)
$$
$$
\equiv 0,89u(0)+0,11u(100)>0,9u(0)+0,1u(500)
$$
$$
\mathbb{E}\left( u(C) \right)>\mathbb{E}\left( u(D) \right)
$$
Ce choix a été soumis aux participants d'un colloque à Paris. Parmi les personnes présentes, certaines ont déclaré préférer $A$ à $B$ …mais également $D$ à $C$, ce qui est contraire aux prédictions de la théorie de
l'utilité espérée. Il s'agit d'une invalidation expérimentale de l'axiome
d'indépendance

\subsection{Paradoxe d'Ellsberg, 1961}

"Voici 2 urnes. La première contient 50 boules rouges et 50 boules
noires. La seconde contient 100 boules, les unes rouges, les autres
noires, mais dans une proportion qu'on ne vous indique pas. On
s'apprête maintenant à tirer une boule d'une des deux urnes. Vous
avez le choix entre deux paris sur la couleur de la boule qui sera tirée.
Vous recevrez 100\euro~si vous gagnez votre pari. Il ne se passera rien si
vous perdez."

1\iere~étape : choix entre les deux paris suivants :

$R1$ : je parie que la boule est rouge si elle est tirée dans l'urne 1

$R2$ : je parie que la boule est rouge si elle est tirée dans l'urne 2

2\ieme~étape : choix entre :

$N1$: je parie que la boule est noire si elle est tirée dans l'urne 1

$N2$ : je parie que la boule est noire si elle est tirée dans l'urne 2

On constate que beaucoup de ceux qui ont préféré $R1$ à $R2$ préfèrent également $N1$ à $N2$. Pourtant, s'ils pensent qu'ils ont plus de chance de
gagner en pariant $R1$ plutôt que $R2$, c'est qu'ils pensent que dans la seconde urne il y a plus de boules noires que de boules rouges. Ils devraient donc penser qu'ils ont plus de chance de gagner en pariant $N2$ qu'en pariant $N1$. Ce paradoxe expérimental invalide l'axiome de la
théorie de l'utilité espérée portant sur les probabilités

\subsection{Portée des paradoxes}

Faut-il déduire de ces paradoxes que les fondements axiomatiques de la théorie de l'utilité espérée sont trop restrictifs ? Faut-il remettre complètement en question la théorie de l'utilité espérée ?

La question est ouverte

Ici, nous donnons quelques éléments permettant de nuancer ces paradoxes et de les réconcilier avec la théorie de l'utilité espérée

\subsection{Expérience de lab v.s. vie courante}

Dans le domaine de l'économie expérimentale, il est essentiel de ne pas se fier uniquement aux déclarations des agents économiques, mais plutôt à leurs actions concrètes. En effet, la manière dont les individus se comportent lors des expériences peut être influencée par divers facteurs, notamment leur niveau d'incitation à réfléchir. Il est plausible que les participants, confrontés à des enjeux perçus comme peu significatifs dans un cadre de laboratoire, n'engagent pas l'effort intellectuel nécessaire. Cependant, dans des situations réelles où les enjeux sont plus tangibles, leur comportement pourrait être radicalement différent. Ce phénomène soulève des questions sur l'organisation même des expériences, comme l'illustre le paradoxe d'Ellsberg, où la transparence d'une urne contraste avec l'opacité d'une autre. Ainsi, l'économie expérimentale s'efforce de prendre de nombreuses précautions afin de réduire la distance entre les réponses fournies par les participants et leurs véritables préférences.

\subsection{Argument des illusions d'optiques}

Les paradoxes expérimentaux peuvent tout simplement montrer, non pas que la théorie de l'utilité espérée est invalidée mais que les agents économiques peuvent se tromper (ils peuvent être victimes d’illusion cognitive dans leur prise de décision en univers incertain).

La question est de savoir si de telles erreurs sont fréquentes ou non dans les actes économiques réels. En effet, ce qui nous intéresse en tant qu'économistes, c'est ce qui détermine le comportement des individus (pas
ce qui devrait le déterminer) : si la plupart des agents font des erreurs par rapport à la théorie, alors la théorie de l'utilité espérée n'est sans doute pas adaptée.

\subsection{Disparition des individus irrationnels}

Un autre argument serait de dire que les individus irrationnels, c'est-à-dire ne respectant pas les axiomes de l'utilité espérée tendent à disparaitre parce qu'ils changent de comportement avant d'être ruinés ou par ruine

\subsection{Théories alternatives}

Plutôt que de tenter de « sauver » la théorie de l'utilité espérée, il est possible d'envisager des approches qui la transcendent. Une telle démarche se manifeste par la théorie de l'utilité non espérée, qui propose de transformer les probabilités plutôt que les valeurs de la richesse. Cette perspective est enrichie par la théorie des perspectives développée par Kahneman et Tversky en 1979, qui met en lumière l'impact des biais cognitifs et l'aversion à la perte sur les décisions économiques. En outre, l'économie expérimentale s'est ouverte à des collaborations interdisciplinaires, intégrant les contributions de psychologues, de neuroscientifiques et d'autres experts pour mieux comprendre les comportements humains face à l'incertitude et aux choix. Cette approche holistique permet d'approfondir notre compréhension des mécanismes décisionnels au-delà des modèles traditionnels.

\chapter{Mesure du risque \& degré d'aversion}

\section{Équivalent certain}


\section{Prix de vente}


\section{Prime de risque}


\section{Prime de risque et degré d'aversion}


\section{ Les trois primes de risque et les indices d'aversion associés}



\end{document}
