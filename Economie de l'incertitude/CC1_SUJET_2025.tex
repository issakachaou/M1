\documentclass[12pt, a4paper]{exam}
\usepackage{graphicx}
\usepackage[utf8]{inputenc} 
\usepackage[french]{babel}
\usepackage[T1]{fontenc}
\usepackage{amsmath,amsfonts,amssymb, empheq}
\usepackage{array,multirow,makecell}
\pagestyle{head}
\pagestyle{foot}
%\printanswers
\usepackage{geometry} % Pour les marges
\geometry{hmargin=2.5cm, vmargin=2.5cm}
\footer{}{\thepage}{}
% Configuration des en-têtes et pieds de page
\pagestyle{headandfoot} % Change le style de page pour avoir en-têtes et pieds de page
\firstpageheader{\textsc{Master 1}}{}{Nom :\hspace{5cm} \\ Prénom :  \hspace{5cm} \\ Groupe :\hspace{5cm} }
\footer{}{\thepage}{}
\cfoot{\thepage\ / \pageref{LastPage}}
	\firstpageheader{\textsc{Master 1}}{}{
		\begin{tabular}{ll}
			\text{Nom :} & \underline{\hspace{5cm}} \\
			\text{Prénom :} & \underline{\hspace{5cm}} \\
			\text{Groupe :} & \underline{\hspace{5cm}} \\
		\end{tabular}}
\begin{document}
	\checkboxchar{$\Box$}
	\checkedchar{$\blacksquare$}


\begin{center}
		\includegraphics[scale=0.25]{../../../Pictures/FAC_SEG_rvb}
\end{center}

	\begin{center}
		
		\vspace*{1cm}
		{\textbf{\textsc{\Huge{Contrôle continu  no. 1 \\Economie de l'incertitude}}}}\\
		
		Document autorisé : aucun \\
		Calculatrice autorisée\\
				\vspace*{1cm}
					\end{center}
	\section*{Question de cours (5 points)}
		 
\begin{questions}
\question[1,25] Quelles critiques les paradoxes vus en cours font-ils au critère de l'espérance mathématique ?
	\makeemptybox{4cm}

\begin{solutionorbox}
	la réponse
\end{solutionorbox}

\question[1,25] Définissez le concept d'aversion au risque.
	\makeemptybox{3cm}
	
\begin{solutionorbox}
	la réponse
\end{solutionorbox}	
	
\question[1,25] Est-ce qu'un agent averse au risque ne prendra jamais de risque ?
	\makeemptybox{3cm}
	
\begin{solutionorbox}
	la réponse
\end{solutionorbox}	
	
\question[1,25] En quoi des préférences fondées sur le critère de l'espérance mathématiques est un cas particulier de la fonction de Markowitz ?
\makeemptybox{3cm}

\begin{solutionorbox}
	la réponse
\end{solutionorbox}

\end{questions}

\section*{Exercice 1 : Projet d'investissement en univers incertain (5 points)}

Un investisseur à le choix en différent choix d'investissement. On suppose qu'il est rationnel et que son but est de maximiser sa richesse. On suppose également que l'investisseur est dans un univers incertain, en d'autre termes, après avoir investit son gains dépends de l'évolution des marchés que ce soit à la hausse ou à la baisse. 












\label{LastPage}
\end{document}
