\documentclass[12pt, a4paper]{exam}
\usepackage{graphicx}
\usepackage[utf8]{inputenc} 
\usepackage[french]{babel}
\usepackage[T1]{fontenc}
\usepackage{amsmath,amsfonts,amssymb, empheq}
\usepackage{array,multirow,makecell,eurosym}
\usepackage{cancel}
\usepackage{booktabs}
\pagestyle{head}
\pagestyle{foot} 
\printanswers
\usepackage{geometry} % Pour les marges
\geometry{hmargin=2.5cm, vmargin=2.5cm}
\footer{}{\thepage}{}
% Configuration des en-têtes et pieds de page
\pagestyle{headandfoot} % Change le style de page pour avoir en-têtes et pieds de page
\firstpageheader{\textsc{Master 1}}{}{}
\footer{}{\thepage}{}
\cfoot{\thepage\ / \pageref{LastPage}}
	\firstpageheader{\textsc{Master 1}}{}{}
		%\begin{tabular}{ll}
		%	\text{Nom :} & \underline{\hspace{5cm}} \\
		%	\text{Prénom :} & \underline{\hspace{5cm}} \\
		%	\text{Groupe :} & \underline{\hspace{5cm}} \\
		%\end{tabular}}
\begin{document}
	\checkboxchar{$\Box$}
	\checkedchar{$\blacksquare$}


\begin{center}
		\includegraphics[scale=0.25]{../../../Pictures/FAC_SEG_rvb}
\end{center}

	\begin{center}
		
		\vspace*{1cm}
		{\textbf{\textsc{\Huge{Contrôle continu  no. 1 \\Economie de l'incertitude}}}}\\
		
		Document autorisé : aucun \\
		Calculatrice autorisée\\
				\vspace*{1cm}
					\end{center}
	\section*{Question de cours (5 points)}
		 
\begin{questions}
\question[1,25] Quelles critiques les paradoxes vus en cours font-ils au critère de l'espérance mathématique ?

\begin{solution}
Le paradoxe de Saint-Pétersbourg et le paradoxe de l'assurance soulignent les limites du critère de l'espérance mathématique. La première limite est que la représentation des préférences des agents par l'espérance ne prend en compte que le gain, alors qu'en réalité, ce ne sont pas les gains qui importent aux individus, mais l'utilité qui y est associée. La deuxième limite est que le critère de l'espérance mathématique ne prend pas en compte le degré d'aversion au risque des agents.
\end{solution}

\question[1,25] Définissez le concept d'aversion au risque.

\begin{solution}
L'aversion au risque désigne la préférence d'un individu pour un gain certain plutôt que pour un gain incertain, à espérance égale. 
\end{solution}	

\question[1,25] Est-ce qu'un agent averse au risque ne prendra jamais de risque ?

\begin{solution}
Un agent averse au risque pourra prendre des risques si son aversion pour le risque est plus faible que son attrait pour la richesse
\end{solution}	
%\newpage

\question[1,25] En quoi des préférences fondées sur le critère de l'espérance mathématiques est un cas particulier de la fonction de Markowitz ?

\begin{solution}
Un agent neutre au risque a des préférences représentées par l'espérance mathématique soit : \[ U(X)=\mathbb{E}(X) \] Or une fonction de Markowitz est de la forme : \[ U(X)=\mathbb{E}(X)-k\mathbb{V}(X) \] Avec \( k \) le degré d'aversion au risque. Si \( k = 0 \) l'agent est neutre au risque et on retrouve : \[ U(X)=\mathbb{E}(X) \]
\end{solution}
\end{questions}
\section*{Exercice 1 : Projet d'investissement en univers incertain (5 points)}

Un investisseur avec un capital de 1000\euro~a le choix entre différents projets d'investissement. On suppose qu'il est rationnel et que son but est de maximiser sa richesse. On suppose également que l'investisseur évolue dans un univers incertain. En d'autres termes, après avoir investi son capital, les gains de son investissement dépendent de l'évolution des marchés, que ce soit à la hausse ou à la baisse.

Il a le choix entre un premier projet noté  \( X_1 \)  qui rapporte 2000\euro~avec une probabilité de 0,3 et 1500\euro~avec une probabilité de 0,7. Le second projet noté  \( X_2 \) rapporte 3000\euro~avec une probabilité de 0,2 et 1900\euro~avec une probabilité de 0,8.

\begin{questions}
\question[1,25] Écrire les loteries associés aux deux projets. 

\begin{solution}
	\[ 
	X_1 =\left\{\begin{matrix}
		2000-1000 & 1500-1000  \\
		0,3 &  0,7\\
	\end{matrix}\right.\Rightarrow 
	X_1 =\left\{\begin{matrix}
		1000 & 500  \\
		0,3 &  0,7\\
	\end{matrix}\right.	
	 \]
	  \[ 
	  X_2 =\left\{\begin{matrix}
	  	3000-1000 & 1900-1000  \\
	  	0,2 &  0,8\\
	  \end{matrix}\right.\Rightarrow 
	   X_2 =\left\{\begin{matrix}
	  	2000 & 900  \\
	  	0,2 &  0,8\\
	  \end{matrix}\right.
	   \]
\end{solution}

\question[1,25] Calculer les espérances et les variances des deux loteries. Commentez.

\begin{solution} Les espérances des loteries sont :
\[ 
\mathbb{E}(X_1)=1000\cdot0,3+500\cdot0,7=650\text{\euro}
 \]
\[ 
\mathbb{E}(X_2)=2000\cdot0,2+900\cdot0,8=1120\text{\euro}
 \] 
Les variances des loteries sont :
\[ 
\mathbb{V}(X_1)=1000^2\cdot0,3+500^2\cdot0,7-(650)^2=52500
 \]
\[ 
\mathbb{V}(X_2)=2000^2\cdot0,2+900^2\cdot0,8-(1120)^2=193600
\]
La loterie qui rapporte le plus en moyenne est la loterie \(X_2\) cependant c'est également la loterie plus risquée.
\end{solution}

\question[1,25] On suppose que l'investisseur à des préférences qui peuvent être représentés par une fonction de Markowitz. Quel loterie choisira-t-il si son degré d'aversion est \( k=-4 \) ? Même question pour \( k=0 \)  et  \( k=4 \) ?
\begin{solution}
Lorsque \( k=-4 \) on a la fonction suivante : 
\[ 
U(X_i)=\mathbb{E}(X_i)+4\mathbb{V}(X_i)\quad \forall i \in \{1,2\}
 \]
Alors : 
\[ 
U(X_1)=\mathbb{E}(X_1)+4\mathbb{V}(X_1)=650+4\cdot52500=210650
\]
\[ 
U(X_2)=\mathbb{E}(X_2)+4\mathbb{V}(X_2)=1120+4\cdot193600=775520
\]
Lorsque \( k=-4 \), la fonction d'utilité est une fonction croissante de la variance, l'agent à un goût pour le risque. Dans ce cas, \( U(X_2)>U(X_1) \), l'investisseur choisira \( X_2 \).\\

Lorsque \( k=0 \), on a la fonction suivante : 

\[ 
U(X_i)=\mathbb{E}(X_i)\quad \forall i \in \{1,2\}
\]
Alors : 
\[ 
U(X_1)=\mathbb{E}(X_1)=650
\]
\[ 
U(X_2)=\mathbb{E}(X_2)=1120
\]
Lorsque \( k=0 \), la fonction d'utilité représente les préférences d'un agent neutre au risque. Dans ce cas \( U(X_2)>U(X_1) \), l'investisseur choisira \( X_2 \).\\

Lorsque \( k=4 \), on a la fonction suivante : 

\[ 
U(X_i)=\mathbb{E}(X_i)-4\mathbb{V}(X_i)\quad \forall i \in \{1,2\}
\]
Alors : 
\[ 
U(X_1)=\mathbb{E}(X_1)-4\mathbb{V}(X_1)=650-4\cdot52500=-209350
\]
\[ 
U(X_2)=\mathbb{E}(X_2)-4\mathbb{V}(X_2)=1120-4\cdot193600=-773280
\]
Lorsque \( k=0 \), la fonction d'utilité représente les préférences d'un agent neutre au risque. Dans ce cas \( U(X_1)>U(X_2) \), l'investisseur choisira \( X_1 \).
\end{solution}
	\question[1,25] A partir de quel valeur de \( k \) l'investisseur choisira le projet 1 ?

\begin{solution}
Il faut résoudre l'inégalité suivante :
\[ 
U(X_1)>U(X_2)
\]
\[ 
\mathbb{E}(X_1)-k\mathbb{V}(X_1)>\mathbb{E}(X_2)-k\mathbb{V}(X_2)
\]
\[ 
 -k\mathbb{V}(X_1)>\mathbb{E}(X_2)-k\mathbb{V}(X_2)-\mathbb{E}(X_1)
\]
\[ 
 -k\mathbb{V}(X_1)+k\mathbb{V}(X_2)>\mathbb{E}(X_2)-\mathbb{E}(X_1)
\]
\[ 
k\left(\mathbb{V}(X_2)-\mathbb{V}(X_1)\right) > \mathbb{E}(X_2)-\mathbb{E}(X_1)
 \]
 \[ 
 k>\frac{\mathbb{E}(X_2)-\mathbb{E}(X_1)}{\mathbb{V}(X_2)-\mathbb{V}(X_1)}
  \]
  \[ 
 k>0,003
\]
A partir  \( k =0,003 \) l'investisseur choisira le projet 1.
\end{solution}

\end{questions}
\section*{Exercice 2 : Problème du producteur en univers incertain (5 points)}

On considère une entreprise dotée d'une fonction de production \(q=\ell^{\frac{2}{3}}\) où \(\ell\) est le nombre de travailleurs embauchés et \(q\) est le niveau de production. Le bien est vendu au prix \(p\) et le salaire est égal à \(w\). Dans un premier temps, on se situe en environnement certain. 

\begin{questions}
	\question[1,25] Rappelez quelles sont les valeurs de la demande de travail, de la production et du profit maximal de l'entreprise notée respectivement \(\ell^{\star}\),\,\(q^{\star}\) et \(\pi^{\star}\). 

\begin{solution}
Le programme du producteur
\[  
\begin{cases}
	\max \pi(q)= p\cdot q -w\ell=p\ell^{\frac{2}{3}}-w\ell\\
	s.c. \, q = \ell^{\frac{2}{3}}
\end{cases}
\]
Condition de premier ordre 
\[ 
\frac{d \pi(\ell)}{d \ell}=0 \Rightarrow p\frac{2}{3}\ell^{\frac{-1}{3}}-w=0\Rightarrow \ell^{\star}=\left(\frac{2p}{3w} \right)^3 
 \]
\[ 
q^{\star}(\ell^{\star})=\left( \left(\frac{2p}{3w} \right)^3\right)^{\frac{2}{3}} \Rightarrow q^{\star}=\left(\frac{2p}{3w} \right)^2
\]
\[ 
\pi^{\star}(q^{\star})=p\left(\frac{2p}{3w} \right)^2- w\left(\frac{2p}{3w} \right)^3 
\]
\[ 
\pi^{\star}(q^{\star})= \cancel{p} \left(\frac{4p^{\cancel{2}}}{9w^2} \right)- \cancel{w}\left(\frac{8p^3}{27w^{\cancel{3}}} \right) 
 \]
\[ 
\pi^{\star}(q^{\star})=\frac{4p^3}{9w^2}-\frac{8p^3}{27w^2}=\frac{12p^3-8p^3}{27w^2}=\frac{4p^3}{27w^2}
 \]
\end{solution}

	\question[2,50] On suppose désormais que le producteur doit faire face à une incertitude sur le prix de vente du bien qu'il produit, il sait juste que ce prix est aléatoire que sa valeur moyenne est \(\tilde{p}\) et sa variance \(\sigma^2_{\tilde{p}}\). On suppose que l'entrepreneur est neutre face au risque. Déterminez la demande de travail, l'offre de biens et le profit maximum notées respectivement \(\ell^{\star\star}\),\,\(q^{\star\star}\) et \(\pi^{\star\star}\). 


\begin{solution}
	Le programme du producteur
	\[  
	\begin{cases}
		\max \mathbb{E}(\pi)= \mathbb{E}\left( p\ell^{\frac{2}{3}}-w\ell\right) =\mathbb{E}(p)\ell^{\frac{2}{3}}-w\ell=\tilde{p}\ell^{\frac{2}{3}}-w\ell\\
		s.c. \, q = \ell^{\frac{2}{3}}
	\end{cases}
	\]
	Condition de premier ordre 
	\[ 
	\frac{d \mathbb{E}(\pi)}{d \ell}=0 \Rightarrow \tilde{p}\frac{2}{3}\ell^{\frac{-1}{3}}-w=0\Rightarrow \ell^{\star\star}=\left(\frac{2\tilde{p}}{3w} \right)^3 
	\]
	\[ 
	q^{\star\star}(\ell^{\star\star})=\left( \left(\frac{2\tilde{p}}{3w} \right)^3\right)^{\frac{2}{3}} \Rightarrow q^{\star\star}=\left(\frac{2\tilde{p}}{3w} \right)^2
	\]
	\[ 
	\pi^{\star\star}(q^{\star\star})=\tilde{p}\left(\frac{2\tilde{p}}{3w} \right)^2- w\left(\frac{2\tilde{p}}{3w} \right)^3 
	\]
	\[ 
	\pi^{\star\star}(q^{\star\star})= \cancel{\tilde{p}} \left(\frac{4\tilde{p}^{\cancel{2}}}{9w^2} \right)- \cancel{w}\left(\frac{8\tilde{p}^3}{27w^{\cancel{3}}} \right) 
	\]
	\[ 
	\pi^{\star\star}(q^{\star\star})=\frac{4\tilde{p}^3}{9w^2}-\frac{8\tilde{p}^3}{27w^2}=\frac{12\tilde{p}^3-8\tilde{p}^3}{27w^2}=\frac{4\tilde{p}^3}{27w^2}
	\]
\end{solution}

	\question[1,25] Comment la production d'univers incertain se situe-t-elle par rapport à celle d'univers certain ? 


\begin{solution}
\[ 
q^{\star\star}-q^{\star}=\left(\frac{2\tilde{p}}{3w} \right)^2-\left(\frac{2p}{3w} \right)^2=\frac{{4\tilde{p}^2-4p^2}}{9w^2}
\]
Si \(\tilde{p}>p\), surestimation \\ 
Si \(\tilde{p}<p\), sous-estimation \\
Si \(\tilde{p}=p\), pas d'écart
\end{solution}
\end{questions}

\section*{Exercice 3 : Action, ETF, ou Forex ? (5 points)}
Un agent à le choix entre investir dans panier d'action noté \(X_1\), un ETF noté \(X_2\) et sur le Forex noté \(X_3\). Lorsqu'il investit son capital, une tendance baissière peut apparaitre avec une probabilité de 0,7 et une tendance haussière peut apparaitre avec une probabilité de 0,1. Il existe également une probabilité qu'aucune tendance claire ne se dégage des marchés financiers avec une probabilité de 0,2. On a les loteries suivantes : 
\[ 
\begin{matrix}
	X_1=\left\{\begin{matrix}
		-5 & 10 & 30 \\
		0,7 & 0,2 & 0,1 \\
	\end{matrix}\right. & X_2=\left\{\begin{matrix}
		-2 & 9 & 15 \\
		0,7 & 0,2 & 0,1 \\
	\end{matrix}\right. & X_3=\left\{\begin{matrix}
		-20 & 0 & 5 \\
		0,7 & 0,2 & 0,1 \\
	\end{matrix}\right.
\end{matrix}
 \]
\begin{questions}
	\question[1,25] Quel sera le choix d'investissement de l'agent si ses préférences sont représentées par une fonction d'utilité de type sécurité d'abord avec un degré d'aversion au risque \(k=0,2\)?


\begin{solution}
Les espérances sont : 
\[ 
\mathbb{E}(X_1)=-5\cdot0,7+10\cdot0,2+30\cdot0,1=1,5
 \]
 \[ 
\mathbb{E}(X_2)=-2\cdot0,7+9\cdot0,2+15\cdot0,1=1,9
 \]
  \[ 
\mathbb{E}(X_3)=-20\cdot0,7+0\cdot0,2+5\cdot0,1=-13,5
 \]
Les variances sont : 
 \[ 
\mathbb{V}^{\star}(X_1)= 0,7\cdot(-5)^2=17,5
 \]
 \[ 
\mathbb{V}^{\star}(X_2)= 0,7\cdot(-2)^2=2,8
\]
 \[ 
\mathbb{V}^{\star}(X_3)= 0,7\cdot(-20)^2=280
\]
Utilisons une fonction d'utilité de type sécurité d'abord avec \(k=0,2\)
\[ 
U(X_1)=\mathbb{E}(X_1)-0,2\mathbb{V}^{\star}(X_1)=1,5 -0,2\cdot17,5=-2
 \]
 \[ 
U(X_2)=\mathbb{E}(X_2)-0,2\mathbb{V}^{\star}(X_2)=1,9 -0,2\cdot2,8=1,34
  \]
 \[ 
U(X_3)=\mathbb{E}(X_3)-0,2\mathbb{V}^{\star}(X_3)=-13,5-0,2\cdot280=-69,5
\] 

Il choisira \(X_2\)
 
\end{solution}


	\question[1,25] Quel sera le choix d'investissement de l'agent si ses préférences sont représentées par une fonction d'utilité de type maximin ?


\begin{solution}
Utilisons une fonction d'utilité de type maximin
\[ 
m(X_1)=-5 \quad m(X_2)=-2 \quad m(X_3)=-20
 \]
Il choisira \(X_2\)
\end{solution}

\question[1,25] Quel sera le choix d'investissement de l'agent si ses préférences sont représentées par une fonction d'utilité de type maximax ?


\begin{solution}
	Utilisons une fonction d'utilité de type maximax
	\[ 
	M(X_1)=30 \quad M(X_2)=15 \quad M(X_3)=5
	\]
	Il choisira \(X_1\)
\end{solution}


\question[1,25] Quel sera le choix d'investissement de l'agent si ses préférences sont représentées par une fonction de regret ?

\begin{solution}
	Utilisons une fonction de regret 
\begin{center}
\begin{tabular}{@{}ccccc@{}}
	\toprule
	& $e_1$ & $e_2$ & $e_3$ & Somme des regrets \\ \midrule
	$X_1$ & 3     & 0     & 0     & 3                 \\
	$X_2$ & 0     & 1     & 15    & 16                \\
	$X_3$ & 18    & 10    & 25    & 53                \\ \bottomrule
\end{tabular}
\end{center}
Il choisira \(X_1\)

\end{solution}

\end{questions}




\label{LastPage}
\end{document}
