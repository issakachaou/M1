\documentclass[a4paper, 12pt]{report}
\usepackage{graphicx}
\usepackage[utf8]{inputenc} 
\usepackage[french]{babel}
\usepackage[T1]{fontenc}
\usepackage{fancyhdr}
\usepackage{amsmath,amsfonts,amssymb, empheq}
\usepackage{eurosym}
\usepackage{booktabs}
\usepackage{wrapfig}
\pagestyle{fancy}
\fancyhead[R]{Université Paris-Est Créteil}
\fancyhead[L]{TD3}
\usepackage{array,multirow,makecell}
\setcellgapes{1pt}
\usepackage{mathptmx} %times aves le mode math
\usepackage{tikz}
\usepackage{hyperref}
\usepackage{import}
\makegapedcells
\newcolumntype{R}[1]{>{\raggedleft\arraybackslash }b{#1}}
\newcolumntype{L}[1]{>{\raggedright\arraybackslash }b{#1}}
\newcolumntype{C}[1]{>{\centering\arraybackslash }b{#1}} 

\begin{document}

\section*{Résolution de l'équation pour $r^\star$}

Nous cherchons à résoudre l'équation suivante pour exprimer $r^\star$ en fonction des constantes données :

\[
-\frac{1}{1-\alpha} \cdot \frac{1}{r} \cdot \left( \frac{r^{1-\gamma}}{1-\gamma} - \frac{b^{1-\gamma}}{1-\gamma} \right) + r^{-\gamma} = 0
\]

\subsection*{Étape 1 : Simplification de l'équation}

Pour simplifier cette équation, nous multiplions chaque côté par le facteur commun $(1-\alpha) \cdot r \cdot (1-\gamma)$ afin d'éliminer les dénominateurs. Nous obtenons alors :

\[
-\left( r^{1-\gamma} - b^{1-\gamma} \right) + (1-\alpha)(1-\gamma) \cdot r^{1-\gamma} = 0
\]

\subsection*{Étape 2 : Réorganisation des termes}

En regroupant les termes contenant $r^{1-\gamma}$, nous obtenons :

\[
-\left( r^{1-\gamma} \right) + b^{1-\gamma} + (1-\alpha)(1-\gamma) \cdot r^{1-\gamma} = 0
\]

Cela peut être réécrit comme suit :

\[
\left[ -(1-\alpha)(1-\gamma) + 1 \right] \cdot r^{1-\gamma} = b^{1-\gamma}
\]

\subsection*{Étape 3 : Isolement de $r^{1-\gamma}$}

Nous isolons $r^{1-\gamma}$ en divisant chaque côté par le facteur $\left[ -(1-\alpha)(1-\gamma) + 1 \right]$ :

\[
r^{1-\gamma} = b^{1-\gamma} \cdot \frac{1}{-(1-\alpha)(1-\gamma) + 1}
\]

\subsection*{Étape 4 : Résolution pour $r$}

Pour isoler $r$, nous appliquons la puissance $\frac{1}{1-\gamma}$ des deux côtés :

\[
r = b \cdot \left[ -(1-\alpha)(1-\gamma) + 1 \right]^{\frac{1}{\gamma-1}}
\]

\subsection*{Résultat final}

Ainsi, la solution finale est donnée par :

\[
r^\star = b \cdot \left[ -(1-\alpha)(1-\gamma) + 1 \right]^{\frac{1}{\gamma-1}}
\]




\end{document}