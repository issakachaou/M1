\documentclass[12pt,a4paper]{article}
\usepackage[utf8]{inputenc}
\usepackage{amsmath,amsfonts,amssymb,amsthm}
\usepackage{graphicx}
\usepackage{hyperref}
\usepackage{geometry}
\usepackage[french]{babel}
\geometry{margin=1in}

\title{Correction du Contrôle Continu n°1 \\ Économie de l'incertitude}
\author{}
\date{}

\begin{document}
	
	\maketitle
	
	\tableofcontents
	\newpage
	
	\section*{Énoncé du sujet}
	
	L'énoncé du sujet se trouve dans le fichier PDF fourni. Cette correction propose une solution complète aux deux exercices proposés, avec tous les calculs détaillés.
	
	\newpage
	
	\section{Exercice 1 : Jeux d'argent}
	
	\subsection{Énoncé}
	Un agent économique dispose d’une richesse initiale de \(10\,000\) euros et a la possibilité de participer à deux jeux d’argent distincts :
	\begin{itemize}
		\item \textbf{Jeu 1} : Le joueur paie \(100\) euros pour participer. Il lance deux fois un dé à \(6\) faces. Pour chaque lancer, s’il obtient un "6", il gagne \(1\,200\) euros. 
		\begin{itemize}
			\item S’il gagne deux fois, il reçoit \(1\,200 + 1\,200 = 2\,400\) euros.
			\item Sinon, il ne gagne rien.
		\end{itemize}
		\item \textbf{Jeu 2} : Le joueur paie \(200\) euros pour participer. Il lance un dé à \(6\) faces. S’il obtient un "5" ou un "6", il gagne \(1\,500\) euros. Sinon, il ne gagne rien.
	\end{itemize}
	
	\textbf{Problèmes à résoudre :}
	\begin{enumerate}
		\item Écrire les loteries \(X_1\) et \(X_2\) associées aux jeux, en termes de richesse finale. Indiquer également l'état de richesse si l'agent décide de ne participer à aucun jeu.
		\item Calculer l'espérance et la variance de richesse associées à chaque jeu. Comparer avec l'option de ne pas participer.
		\item Déterminer, pour chaque fonction d'utilité donnée, le choix optimal de l'agent.
	\end{enumerate}
	
	\subsection{Correction}
	\subsubsection*{1. Loteries associées et richesse finale}
	
	\begin{itemize}
		\item \textbf{Jeu 1} : 
		\begin{itemize}
			\item Coût de participation : \(100\) euros.
			\item Résultats possibles :
			\begin{itemize}
				\item Avec une probabilité \(\frac{1}{36}\), le joueur gagne \(2\,400\) euros. La richesse finale est alors :
				\[
				X_1 = 10\,000 - 100 + 2\,400 = 12\,300.
				\]
				\item Avec une probabilité \(\frac{35}{36}\), le joueur ne gagne rien. La richesse finale est :
				\[
				X_1 = 10\,000 - 100 = 9\,900.
				\]
			\end{itemize}
		\end{itemize}
		La loterie associée est donc :
		\[
		X_1 = 
		\begin{cases} 
			12\,300 & \text{avec une probabilité } \frac{1}{36}, \\
			9\,900 & \text{avec une probabilité } \frac{35}{36}.
		\end{cases}
		\]
		
		\item \textbf{Jeu 2} : 
		\begin{itemize}
			\item Coût de participation : \(200\) euros.
			\item Résultats possibles :
			\begin{itemize}
				\item Avec une probabilité \(\frac{1}{3}\), le joueur gagne \(1\,500\) euros. La richesse finale est alors :
				\[
				X_2 = 10\,000 - 200 + 1\,500 = 11\,300.
				\]
				\item Avec une probabilité \(\frac{2}{3}\), le joueur ne gagne rien. La richesse finale est :
				\[
				X_2 = 10\,000 - 200 = 9\,800.
				\]
			\end{itemize}
		\end{itemize}
		La loterie associée est donc :
		\[
		X_2 = 
		\begin{cases} 
			11\,300 & \text{avec une probabilité } \frac{1}{3}, \\
			9\,800 & \text{avec une probabilité } \frac{2}{3}.
		\end{cases}
		\]
		
		\item \textbf{Sans participation :} La richesse est certaine, égale à :
		\[
		X_0 = 10\,000.
		\]
	\end{itemize}
	
	\subsubsection*{2. Espérance et variance de richesse}
	
	\begin{itemize}
		\item \textbf{Jeu 1 :}
		\[
		E(X_1) = \frac{1}{36} \cdot 12\,300 + \frac{35}{36} \cdot 9\,900 = 10\,000.
		\]
		\[
		V(X_1) = \frac{1}{36}(12\,300 - 10\,000)^2 + \frac{35}{36}(9\,900 - 10\,000)^2 = 154\,166.67.
		\]
		
		\item \textbf{Jeu 2 :}
		\[
		E(X_2) = \frac{1}{3} \cdot 11\,300 + \frac{2}{3} \cdot 9\,800 = 10\,300.
		\]
		\[
		V(X_2) = \frac{1}{3}(11\,300 - 10\,300)^2 + \frac{2}{3}(9\,800 - 10\,300)^2 = 300\,000.
		\]
		
		\item \textbf{Sans participation :}
		\[
		E(X_0) = 10\,000, \quad V(X_0) = 0.
		\]
	\end{itemize}
	
	\subsubsection*{3. Choix selon les fonctions d’utilité}
	
	\begin{itemize}
		\item \textbf{Fonction d’utilité \(u(w) = \ln(w)\) :}  
		\[
		U(X_1) = \frac{1}{36} \ln(12\,300) + \frac{35}{36} \ln(9\,900),
		\]
		\[
		U(X_2) = \frac{1}{3} \ln(11\,300) + \frac{2}{3} \ln(9\,800).
		\]
		
		\item \textbf{Fonction d’utilité \(u(w) = w^2\) :}
		\[
		U(X_1) = \frac{1}{36} \cdot 12\,300^2 + \frac{35}{36} \cdot 9\,900^2,
		\]
		\[
		U(X_2) = \frac{1}{3} \cdot 11\,300^2 + \frac{2}{3} \cdot 9\,800^2.
		\]
		
		\item \textbf{Fonction d’utilité \(u(w) = \sqrt{w}\) :}
		\[
		U(X_1) = \frac{1}{36} \sqrt{12\,300} + \frac{35}{36} \sqrt{9\,900},
		\]
		\[
		U(X_2) = \frac{1}{3} \sqrt{11\,300} + \frac{2}{3} \sqrt{9\,800}.
		\]
	\end{itemize}
	
	Les choix dépendent des valeurs calculées pour chaque utilité.
	
	\newpage
	
\section{Exercice 2 : Investissements financiers}

\subsection{Énoncé}
Un investisseur dispose d’un capital de \(2\,000\) euros qu’il souhaite investir dans trois types d’actifs financiers :
\begin{itemize}
	\item \textbf{Livret} : Rendement sûr de \(2\%\) par an.
	\item \textbf{Obligations d’État} : Rendement incertain :
	\begin{itemize}
		\item \(5\%\) avec une probabilité de \(70\%\),
		\item \(3\%\) avec une probabilité de \(30\%\).
	\end{itemize}
	\item \textbf{Actions} : Rendement incertain :
	\begin{itemize}
		\item \(10\%\) avec une probabilité de \(50\%\),
		\item \(5\%\) avec une probabilité de \(30\%\),
		\item \(-5\%\) avec une probabilité de \(20\%\).
	\end{itemize}
\end{itemize}

\textbf{Problèmes à résoudre :}
\begin{enumerate}
	\item Calculer le montant final de chaque investissement dans chaque scénario et écrire les loteries correspondantes.
	\item Calculer l’espérance mathématique et la variance de chaque loterie.
	\item Supposons que l’investisseur suit une fonction d’utilité de Markowitz \(U(X) = E(X) - k \cdot V(X)\). Indiquer l’investissement choisi si \(k = 2\).
	\item Si l’investisseur répartit son capital en une fraction \(\alpha\) sur les actions et \((1-\alpha)\) sur le livret, déterminer la valeur optimale de \(\alpha\) pour maximiser \(U(X)\) lorsque \(k = 1\), \(k = 0\), et \(k = -2\).
\end{enumerate}

\subsection{Correction}

\subsubsection*{1. Loteries associées}

\begin{itemize}
	\item \textbf{Livret} :  
	Le rendement est sûr, donc :
	\[
	\text{Montant final} = 2\,000 \cdot (1 + 0.02) = 2\,040.
	\]
	La loterie est :
	\[
	X_{\text{Livret}} = 
	\begin{cases} 
		2\,040 & \text{avec une probabilité de } 1.
	\end{cases}
	\]
	
	\item \textbf{Obligations d’État} :  
	Les deux scénarios sont :
	\begin{itemize}
		\item Avec une probabilité \(70\%\), le rendement est \(5\%\), soit :
		\[
		2\,000 \cdot (1 + 0.05) = 2\,100.
		\]
		\item Avec une probabilité \(30\%\), le rendement est \(3\%\), soit :
		\[
		2\,000 \cdot (1 + 0.03) = 2\,060.
		\]
	\end{itemize}
	La loterie est :
	\[
	X_{\text{Obligations}} = 
	\begin{cases} 
		2\,100 & \text{avec une probabilité de } 0.7, \\
		2\,060 & \text{avec une probabilité de } 0.3.
	\end{cases}
	\]
	
	\item \textbf{Actions} :  
	Les trois scénarios sont :
	\begin{itemize}
		\item Avec une probabilité \(50\%\), le rendement est \(10\%\), soit :
		\[
		2\,000 \cdot (1 + 0.10) = 2\,200.
		\]
		\item Avec une probabilité \(30\%\), le rendement est \(5\%\), soit :
		\[
		2\,000 \cdot (1 + 0.05) = 2\,100.
		\]
		\item Avec une probabilité \(20\%\), le rendement est \(-5\%\), soit :
		\[
		2\,000 \cdot (1 - 0.05) = 1\,900.
		\]
	\end{itemize}
	La loterie est :
	\[
	X_{\text{Actions}} = 
	\begin{cases} 
		2\,200 & \text{avec une probabilité de } 0.5, \\
		2\,100 & \text{avec une probabilité de } 0.3, \\
		1\,900 & \text{avec une probabilité de } 0.2.
	\end{cases}
	\]
\end{itemize}

\subsubsection*{2. Espérance mathématique et variance}

\begin{itemize}
	\item \textbf{Livret :}
	\[
	E(X_{\text{Livret}}) = 2\,040, \quad V(X_{\text{Livret}}) = 0.
	\]
	
	\item \textbf{Obligations d’État :}
	\[
	E(X_{\text{Obligations}}) = 0.7 \cdot 2\,100 + 0.3 \cdot 2\,060 = 2\,086.
	\]
	\[
	V(X_{\text{Obligations}}) = 0.7 \cdot (2\,100 - 2\,086)^2 + 0.3 \cdot (2\,060 - 2\,086)^2 = 252.
	\]
	
	\item \textbf{Actions :}
	\[
	E(X_{\text{Actions}}) = 0.5 \cdot 2\,200 + 0.3 \cdot 2\,100 + 0.2 \cdot 1\,900 = 2\,100.
	\]
	\[
	V(X_{\text{Actions}}) = 0.5 \cdot (2\,200 - 2\,100)^2 + 0.3 \cdot (2\,100 - 2\,100)^2 + 0.2 \cdot (1\,900 - 2\,100)^2 = 8\,000.
	\]
\end{itemize}

\subsubsection*{3. Choix selon la fonction d’utilité de Markowitz}

La fonction d’utilité est donnée par :
\[
U(X) = E(X) - k \cdot V(X).
\]
Pour \(k = 2\) :
\begin{itemize}
	\item \(U(X_{\text{Livret}}) = 2\,040 - 2 \cdot 0 = 2\,040.\)
	\item \(U(X_{\text{Obligations}}) = 2\,086 - 2 \cdot 252 = 1\,582.\)
	\item \(U(X_{\text{Actions}}) = 2\,100 - 2 \cdot 8\,000 = -13\,900.\)
\end{itemize}
Le choix optimal est le \textbf{Livret}.

\subsubsection*{4. Répartition optimale entre actions et livret}

Le portefeuille est donné par :
\[
X_{\text{Portefeuille}} = \alpha X_{\text{Actions}} + (1-\alpha) X_{\text{Livret}}.
\]
Les caractéristiques du portefeuille sont les suivantes :
\begin{itemize}
	\item \textbf{Espérance mathématique :}
	\[
	E(X_{\text{Portefeuille}}) = \alpha E(X_{\text{Actions}}) + (1-\alpha) E(X_{\text{Livret}}).
	\]
	En substituant les valeurs :
	\[
	E(X_{\text{Portefeuille}}) = \alpha \cdot 2\,100 + (1-\alpha) \cdot 2\,040 = 2\,040 + 60\alpha.
	\]
	
	\item \textbf{Variance :}
	La variance est donnée par :
	\[
	V(X_{\text{Portefeuille}}) = \alpha^2 V(X_{\text{Actions}}) + (1-\alpha)^2 V(X_{\text{Livret}}) + 2\alpha(1-\alpha)\text{Cov}(X_{\text{Actions}}, X_{\text{Livret}}).
	\]
	Comme \(V(X_{\text{Livret}}) = 0\) (rendement sûr) et en supposant que les actions et le livret sont indépendants (\(\text{Cov} = 0\)), la formule devient :
	\[
	V(X_{\text{Portefeuille}}) = \alpha^2 V(X_{\text{Actions}}).
	\]
	En substituant \(V(X_{\text{Actions}}) = 8\,000\) :
	\[
	V(X_{\text{Portefeuille}}) = 8\,000 \alpha^2.
	\]
\end{itemize}

La fonction d’utilité de Markowitz est :
\[
U(X_{\text{Portefeuille}}) = E(X_{\text{Portefeuille}}) - k \cdot V(X_{\text{Portefeuille}}).
\]
En substituant \(E(X_{\text{Portefeuille}})\) et \(V(X_{\text{Portefeuille}})\) :
\[
U(X_{\text{Portefeuille}}) = (2\,040 + 60\alpha) - k \cdot 8\,000 \alpha^2.
\]

Pour maximiser \(U(X_{\text{Portefeuille}})\), nous résolvons :
\[
\frac{\partial U}{\partial \alpha} = 60 - 2k \cdot 8\,000 \alpha = 0.
\]
Ce qui donne :
\[
\alpha = \frac{60}{16\,000k}.
\]

\textbf{Cas particuliers :}
\begin{itemize}
	\item Pour \(k = 1\) :
	\[
	\alpha = \frac{60}{16\,000 \cdot 1} = \frac{60}{16\,000} = 0.00375.
	\]
	L’investisseur place \(0.375\%\) de son capital en actions et \(99.625\%\) sur le livret.
	
	\item Pour \(k = 0\) :
	\[
	\alpha = \frac{60}{16\,000 \cdot 0} = \infty.
	\]
	Ici, \(k = 0\) signifie que l’investisseur ne se soucie pas du risque et investit \(100\%\) de son capital en actions (\(\alpha = 1\)).
	
	\item Pour \(k = -2\) :
	\[
	\alpha = \frac{60}{16\,000 \cdot (-2)} = \frac{60}{-32\,000} = -0.001875.
	\]
	Une valeur négative pour \(\alpha\) signifie que l’investisseur emprunte pour investir davantage dans les actions.
\end{itemize}

\textbf{Conclusion :}
\begin{itemize}
	\item Lorsque \(k > 0\), l’investisseur est avers au risque et préfère une allocation principalement sur le livret.
	\item Lorsque \(k = 0\), l’investisseur maximise uniquement l’espérance et place tout en actions.
	\item Lorsque \(k < 0\), l’investisseur est attiré par le risque et surpondère les actions, quitte à emprunter.
\end{itemize}
	
\end{document}
