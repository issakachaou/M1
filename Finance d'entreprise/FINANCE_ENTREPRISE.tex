\documentclass[a4paper, 12pt]{report}
\usepackage{graphicx}
\usepackage[utf8]{inputenc} 
\usepackage[french]{babel}
\usepackage[T1]{fontenc}
\usepackage{fancyhdr}
\usepackage{amsmath,amsfonts,amssymb, empheq}
\usepackage{eurosym}
\usepackage{booktabs}
\usepackage{cancel}
\usepackage{wrapfig}
%\usepackage{tikz}
\usepackage{hyperref}
\pagestyle{fancy}
\usepackage{mathptmx} %times aves le mode math
\fancyhead[R]{Université Paris-Est Créteil}
\fancyhead[L]{Finance d'entreprise}
\usepackage{array,multirow,makecell}
\setcellgapes{1pt}
\makegapedcells
\newcolumntype{R}[1]{>{\raggedleft\arraybackslash }b{#1}}
\newcolumntype{L}[1]{>{\raggedright\arraybackslash }b{#1}}
\newcolumntype{C}[1]{>{\centering\arraybackslash }b{#1}} 
%\renewcommand{\thechapter}{\Roman{chapter}}
%\setcounter{chapter}{1} % pour numéroter le chapitre 

\begin{document}
	
\chapter*{Introduction}	

\section{Introduction}

La finance d'entreprise est un domaine essentiel qui concerne la gestion des ressources financières d'une entreprise. Elle englobe des décisions stratégiques liées à l'investissement, au financement et à la gestion des actifs. Les principaux objectifs de la finance d'entreprise incluent l'augmentation de la valeur de l'entreprise pour les actionnaires, la maximisation des profits et la gestion des risques financiers. Les outils et techniques utilisés dans ce domaine comprennent l'analyse des états financiers, l'évaluation des projets d'investissement et la gestion de la trésorerie. En outre, la compréhension des marchés financiers et des instruments financiers est cruciale pour prendre des décisions éclairées et optimiser la structure du capital.

Une entreprise est une organisation dont le but est de produire et d'offrir des biens et/ou des services à des consommateurs. Cela implique l'utilisation de ressources variées, telles que des ressources matérielles, humaines, financières, immatérielles et informationnelles, nécessitant ainsi la coordination de différentes fonctions, notamment l'achat, la commercialisation, la production, la finance et la recherche et développement (R\&D). L'objectif financier d'une entreprise est la création de valeur. On peut classer les entreprises en quatre catégories selon leur taille et effectifs : les Petites Entreprises (PE), qui comptent moins de 10 personnes ; les Moyennes Entreprises (ME), qui emploient entre 10 et 250 salariés ; les Entreprises de Taille Intermédiaire (ETI), qui engagent entre 250 et 5000 employés ; et enfin, les Grandes Entreprises (GE), qui, considérées comme des géants de l'économie, emploient plus de 5000 personnes.

En 2021, les secteurs principalement marchands non agricoles et non financiers comptent 3,7 millions d'entreprises (Source : \url{https://www.insee.fr/fr/statistiques/}). Ces entreprises affichent un chiffre d'affaires hors taxes global de 4142 milliards d'euros et une valeur ajoutée de 1179 milliards d'euros, représentant 60\% de la valeur ajoutée de l'économie française. Les 4200 entreprises de taille intermédiaire (ETI) et les grandes entreprises (GE) représentent 65\% du chiffre d'affaires, 61\% de la valeur ajoutée, 46\% des investissements et 86\% des exportations, illustrant une forte concentration de l'activité. En revanche, les 3,6 millions de petites entreprises (PE) contribuent à environ 21\% du chiffre d'affaires et à un quart de la valeur ajoutée, tout en n'ayant aucune part dans les exportations. Les grandes entreprises (GE) et les entreprises de taille intermédiaire (ETI), bien qu'elles ne représentent qu'une part infime des entreprises (environ 0,2\%), contribuent à 65\% du chiffre d'affaires, 61\% de la valeur ajoutée, 75\% des immobilisations corporelles et 86\% des exportations, soulignant ainsi une forte concentration de l'activité économique sur ces catégories. À l'inverse, les micro-entreprises et les PME hors micro-entreprises, représentant environ 99\% des entreprises, participent pour 35\% du chiffre d'affaires et 39\% de la valeur ajoutée, mais leur part dans les exportations reste marginale.

\section{Quel type d’entreprise ?}
	
Nous nous focaliserons sur les sociétés par action. Le capital de ces entreprises est divisé en actions, sans limite sur le nombre d'actionnaires. Chaque actionnaire (shareholder) détient une part de l'entreprise et a le droit de percevoir des dividendes. En 2021, ces sociétés ne représentaient que 7\% du nombre total d'entreprises, mais elles comptaient pour 38\% du nombre de salariés et 50\% de la valeur ajoutée, ce qui illustre leur importance en tant qu'entreprises de taille significative, notamment les entreprises de taille intermédiaire (ETI) et les grandes entreprises (GE).

Le capital de ces sociétés est divisé en actions, sans limite sur le nombre d'actionnaires. Cela permet une grande flexibilité dans leur gouvernance et leur financement. Chaque actionnaire détient une part de l'entreprise et a droit, en proportion de ses actions, à des dividendes, correspondant à une part des bénéfices.

\section{Qui prend les décisions ?}
	
Un conseil d'administration (CA) est élu par les actionnaires d'une société par actions lors des assemblées générales. Son rôle principal est triple : définir la politique générale de l'entreprise en fixant les grandes orientations stratégiques et en veillant à leur mise en œuvre ; contrôler les performances en surveillant les résultats financiers, la conformité aux lois et réglementations, ainsi que la bonne gestion des ressources ; et désigner et superviser le directeur général (CEO - \textit{Chief Executive Officer}), en nommant la personne responsable de la direction opérationnelle de l'entreprise et en évaluant son travail. Le CEO est en charge de la plupart des décisions impliquant la gestion de l'entreprise au quotidien. Ainsi, il existe une séparation entre la direction et la propriété de l'entreprise.
	
Les décisions des sociétés par actions cotées en bourse répondent à plusieurs objectifs stratégiques, qui peuvent parfois être en conflit. L'objectif principal des actionnaires est de maximiser la valeur boursière de l'entreprise, ce qui se traduit par une augmentation du cours de l'action et, potentiellement, des dividendes élevés. Cependant, il existe souvent un conflit d'intérêt entre les actionnaires et les dirigeants, ces derniers pouvant avoir des objectifs différents, comme la préservation de leur emploi ou des ambitions personnelles. De plus, les créanciers, qui fournissent également des capitaux, cherchent à minimiser le risque de défaut, ce qui peut entrer en contradiction avec les stratégies visant à maximiser la valeur boursière. Ainsi, les sociétés par actions doivent naviguer entre ces divers objectifs tout en gérant les tensions qui peuvent surgir entre les différentes parties prenantes.

L'asymétrie d'information entre les dirigeants, les actionnaires et les créanciers a des implications significatives pour la stratégie financière de l'entreprise. En raison de cette asymétrie, les dirigeants peuvent prendre des décisions qui ne sont pas toujours alignées avec les intérêts des actionnaires, entraînant des conflits d'intérêts. De plus, les créanciers, n'ayant pas accès aux mêmes informations que les dirigeants, peuvent percevoir un risque plus élevé, ce qui peut augmenter le coût du capital. Pour atténuer ces problèmes, les entreprises doivent adopter des pratiques de transparence, établir des mécanismes de contrôle efficaces et maintenir une communication ouverte avec toutes les parties prenantes, tout en choisissant des stratégies d'investissement prudentes.

\chapter{Évaluation d’un projet}

\section{Principes de l’évaluation}

Un projet d'investissement peut être défini comme une séquence de flux financiers (cash flows). Les flux positifs, appelés inflows, sont les revenus générés par le projet, tandis que les flux négatifs, ou outflows, représentent les coûts associés. Le principe fondamental de l'évaluation d'un projet repose sur la comparaison entre coûts et bénéfices : un projet n'est rentable que si les bénéfices dépassent les coûts. De plus, il est important de noter que les coûts et les bénéfices sont échelonnés dans le temps, ce qui implique qu'une analyse actualisée des flux financiers doit être réalisée pour évaluer la viabilité du projet.

\begin{center}
	\includegraphics[scale = 0.5]{../../../Pictures/Screenshots/Capture d'écran 2025-01-20 150035}
\end{center}

Le principe de base en finance stipule que toute comparaison doit se faire dans la même unité monétaire et temporelle. Par exemple, 1 euro aujourd'hui n'a pas la même valeur qu'1 euro dans un an ou plus, en raison du taux de préférence pour le présent. En effet, 1 euro aujourd'hui peut être placé et vaudra davantage dans un an (ou plus). Pour comparer des flux financiers, il est nécessaire de les "faire voyager dans le temps", c'est-à-dire d'exprimer leurs valeurs à une seule et même date (et unité). On notera \( r \) le taux d'intérêt nominal, exprimé sur une base annuelle (1 période dans l'échéancier) et supposé constant.

\section{Voyage dans le futur}

L'opération visant à cumuler des flux dans le futur est appelée capitalisation. La valeur future d'un flux initial est définie comme la valeur, exprimée dans \( n \) périodes, du flux initial \( f \). Cette valeur future peut être calculée en tenant compte du taux d'intérêt et du temps écoulé.

\[ VF_n(f)=f\cdot(1+r)^n \]

L'opération visant à exprimer des flux futurs en valeur présente est appelée actualisation. La valeur actuelle d'un flux \( f \) qui sera reçu dans \( n \) périodes est déterminée en tenant compte du taux d'intérêt, permettant ainsi de comparer des flux à des moments différents dans le temps.

\[ VA_n(f)= \frac{f}{(1+r)^n} \]

Ce que vaut aujourd’hui le flux \( f \) obtenu dans \( n \) périodes.

\( r \) est aussi appelé le taux  d'actualisation (taux actuariel si les périodes considérées sont des années).
Ce que vaut aujourd’hui le flux \( f \) obtenu dans n périodes.

\( r \) est aussi appelé le taux d'actualisation (taux actuariel si les périodes considérées sont des années).

\[ \delta = \frac{1}{(1+r)^n} \]

\section{Valeur d'une séquence de flux}

Un projet \( P \) est défini comme une séquence de flux (positifs ou négatifs) représentés par l'échéancier. La valeur nette du projet correspond à la somme des valeurs des flux exprimées à une même date. On parle de Valeur Actuelle Nette (VAN) du projet si cette date est aujourd'hui, ce qui signifie qu'il s'agit de la somme actualisée des différents flux. Ainsi, la valeur actuelle nette de la séquence de flux \( P \) est un indicateur clé pour évaluer la rentabilité d'un projet.

\[ VAN_n(P)=f_0+\frac{f_1}{(1+r)}+\frac{f_2}{(1+r)^2}+\frac{f_3}{(1+r)^3}+\cdots+\frac{f_n}{(1+r)^n} = \sum_{t=0}^{n}\frac{f_t}{(1+r)^t}\]

Ce que vaut aujourd’hui la richesse du projet \( P \) réalisé sur \( n \) périodes.






\end{document}
	
