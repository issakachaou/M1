\documentclass[a4paper, 12pt]{report}
\usepackage{graphicx}
\usepackage[utf8]{inputenc} 
\usepackage[french]{babel}
\usepackage[T1]{fontenc}
\usepackage{fancyhdr}
\usepackage{amsmath,amsfonts,amssymb, empheq}
\usepackage{eurosym}
\usepackage{booktabs}
\usepackage{cancel}
\usepackage{wrapfig}
%\usepackage{tikz}
\usepackage{hyperref}
\pagestyle{fancy}
\usepackage{mathptmx} %times aves le mode math
\fancyhead[R]{Université Paris-Est Créteil}
\fancyhead[L]{Finance d'entreprise}
\usepackage{array,multirow,makecell}
\setcellgapes{1pt}
\makegapedcells
\newcolumntype{R}[1]{>{\raggedleft\arraybackslash }b{#1}}
\newcolumntype{L}[1]{>{\raggedright\arraybackslash }b{#1}}
\newcolumntype{C}[1]{>{\centering\arraybackslash }b{#1}} 
%\renewcommand{\thechapter}{\Roman{chapter}}
%\setcounter{chapter}{1} % pour numéroter le chapitre 

\begin{document}
	
	
\begin{titlepage}
	\centering
	\begin{center}
		\includegraphics[scale=0.3]{../../../Pictures/FAC_SEG_rvb}
	\end{center}
	\vspace*{2cm}
	
	\Huge
	
	\textbf{Finance d'entreprise}
	\vspace{1.5cm}
	
	\Large
	Cours intégral
	
	\vspace{2cm}
	
	\textbf{Issa KACHAOU} \\
	{\normalsize Délégué général du Master 1 MBFA}
	
	
	\vfill
	
	\Large
	
	\textsc{\textbf{Université Paris Est-Créteil}}	 \\
	\textbf{Département d'\'Economie} \\
	\textbf{2025}
	
\end{titlepage}
\thispagestyle{empty}
\newpage
\clearpage
\mbox{}
\thispagestyle{empty}

\tableofcontents

\thispagestyle{empty}
\newpage
\mbox{}
\thispagestyle{empty} %Dernière page vide
%\backmatter	
	
\pagestyle{plain} 
\chapter*{Introduction}	

\section{Introduction}

La finance d'entreprise est un domaine essentiel qui concerne la gestion des ressources financières d'une entreprise. Elle englobe des décisions stratégiques liées à l'investissement, au financement et à la gestion des actifs. Les principaux objectifs de la finance d'entreprise incluent l'augmentation de la valeur de l'entreprise pour les actionnaires, la maximisation des profits et la gestion des risques financiers. Les outils et techniques utilisés dans ce domaine comprennent l'analyse des états financiers, l'évaluation des projets d'investissement et la gestion de la trésorerie. En outre, la compréhension des marchés financiers et des instruments financiers est cruciale pour prendre des décisions éclairées et optimiser la structure du capital.

Une entreprise est une organisation dont le but est de produire et d'offrir des biens et/ou des services à des consommateurs. Cela implique l'utilisation de ressources variées, telles que des ressources matérielles, humaines, financières, immatérielles et informationnelles, nécessitant ainsi la coordination de différentes fonctions, notamment l'achat, la commercialisation, la production, la finance et la recherche et développement (R\&D). L'objectif financier d'une entreprise est la création de valeur. On peut classer les entreprises en quatre catégories selon leur taille et effectifs : les Petites Entreprises (PE), qui comptent moins de 10 personnes ; les Moyennes Entreprises (ME), qui emploient entre 10 et 250 salariés ; les Entreprises de Taille Intermédiaire (ETI), qui engagent entre 250 et 5000 employés ; et enfin, les Grandes Entreprises (GE), qui, considérées comme des géants de l'économie, emploient plus de 5000 personnes.

En 2021, les secteurs principalement marchands non agricoles et non financiers comptent 3,7 millions d'entreprises (Source : \url{https://www.insee.fr/fr/statistiques/}). Ces entreprises affichent un chiffre d'affaires hors taxes global de 4142 milliards d'euros et une valeur ajoutée de 1179 milliards d'euros, représentant 60\% de la valeur ajoutée de l'économie française. Les 4200 entreprises de taille intermédiaire (ETI) et les grandes entreprises (GE) représentent 65\% du chiffre d'affaires, 61\% de la valeur ajoutée, 46\% des investissements et 86\% des exportations, illustrant une forte concentration de l'activité. En revanche, les 3,6 millions de petites entreprises (PE) contribuent à environ 21\% du chiffre d'affaires et à un quart de la valeur ajoutée, tout en n'ayant aucune part dans les exportations. Les grandes entreprises (GE) et les entreprises de taille intermédiaire (ETI), bien qu'elles ne représentent qu'une part infime des entreprises (environ 0,2\%), contribuent à 65\% du chiffre d'affaires, 61\% de la valeur ajoutée, 75\% des immobilisations corporelles et 86\% des exportations, soulignant ainsi une forte concentration de l'activité économique sur ces catégories. À l'inverse, les micro-entreprises et les PME hors micro-entreprises, représentant environ 99\% des entreprises, participent pour 35\% du chiffre d'affaires et 39\% de la valeur ajoutée, mais leur part dans les exportations reste marginale.

\section{Quel type d'entreprise ?}
	
Nous nous focaliserons sur les sociétés par action. Le capital de ces entreprises est divisé en actions, sans limite sur le nombre d'actionnaires. Chaque actionnaire (shareholder) détient une part de l'entreprise et a le droit de percevoir des dividendes. En 2021, ces sociétés ne représentaient que 7\% du nombre total d'entreprises, mais elles comptaient pour 38\% du nombre de salariés et 50\% de la valeur ajoutée, ce qui illustre leur importance en tant qu'entreprises de taille significative, notamment les entreprises de taille intermédiaire (ETI) et les grandes entreprises (GE).

Le capital de ces sociétés est divisé en actions, sans limite sur le nombre d'actionnaires. Cela permet une grande flexibilité dans leur gouvernance et leur financement. Chaque actionnaire détient une part de l'entreprise et a droit, en proportion de ses actions, à des dividendes, correspondant à une part des bénéfices.

\section{Qui prend les décisions ?}
	
Un conseil d'administration (CA) est élu par les actionnaires d'une société par actions lors des assemblées générales. Son rôle principal est triple : définir la politique générale de l'entreprise en fixant les grandes orientations stratégiques et en veillant à leur mise en œuvre ; contrôler les performances en surveillant les résultats financiers, la conformité aux lois et réglementations, ainsi que la bonne gestion des ressources ; et désigner et superviser le directeur général (CEO - \textit{Chief Executive Officer}), en nommant la personne responsable de la direction opérationnelle de l'entreprise et en évaluant son travail. Le CEO est en charge de la plupart des décisions impliquant la gestion de l'entreprise au quotidien. Ainsi, il existe une séparation entre la direction et la propriété de l'entreprise.
	
Les décisions des sociétés par actions cotées en bourse répondent à plusieurs objectifs stratégiques, qui peuvent parfois être en conflit. L'objectif principal des actionnaires est de maximiser la valeur boursière de l'entreprise, ce qui se traduit par une augmentation du cours de l'action et, potentiellement, des dividendes élevés. Cependant, il existe souvent un conflit d'intérêt entre les actionnaires et les dirigeants, ces derniers pouvant avoir des objectifs différents, comme la préservation de leur emploi ou des ambitions personnelles. De plus, les créanciers, qui fournissent également des capitaux, cherchent à minimiser le risque de défaut, ce qui peut entrer en contradiction avec les stratégies visant à maximiser la valeur boursière. Ainsi, les sociétés par actions doivent naviguer entre ces divers objectifs tout en gérant les tensions qui peuvent surgir entre les différentes parties prenantes.

L'asymétrie d'information entre les dirigeants, les actionnaires et les créanciers a des implications significatives pour la stratégie financière de l'entreprise. En raison de cette asymétrie, les dirigeants peuvent prendre des décisions qui ne sont pas toujours alignées avec les intérêts des actionnaires, entraînant des conflits d'intérêts. De plus, les créanciers, n'ayant pas accès aux mêmes informations que les dirigeants, peuvent percevoir un risque plus élevé, ce qui peut augmenter le coût du capital. Pour atténuer ces problèmes, les entreprises doivent adopter des pratiques de transparence, établir des mécanismes de contrôle efficaces et maintenir une communication ouverte avec toutes les parties prenantes, tout en choisissant des stratégies d'investissement prudentes.

\part{Les décisions d'investissement}

\chapter{Évaluation d'un projet}

\section{Principes de l'évaluation}

Un projet d'investissement peut être défini comme une séquence de flux financiers (cash flows). Les flux positifs, appelés inflows, sont les revenus générés par le projet, tandis que les flux négatifs, ou outflows, représentent les coûts associés. Le principe fondamental de l'évaluation d'un projet repose sur la comparaison entre coûts et bénéfices : un projet n'est rentable que si les bénéfices dépassent les coûts. De plus, il est important de noter que les coûts et les bénéfices sont échelonnés dans le temps, ce qui implique qu'une analyse actualisée des flux financiers doit être réalisée pour évaluer la viabilité du projet.

\begin{center}
	\includegraphics[scale = 0.5]{../../../Pictures/Screenshots/Capture d'écran 2025-01-20 150035}
\end{center}

Le principe de base en finance stipule que toute comparaison doit se faire dans la même unité monétaire et temporelle. Par exemple, 1 euro aujourd'hui n'a pas la même valeur qu'1 euro dans un an ou plus, en raison du taux de préférence pour le présent. En effet, 1 euro aujourd'hui peut être placé et vaudra davantage dans un an (ou plus). Pour comparer des flux financiers, il est nécessaire de les "faire voyager dans le temps", c'est-à-dire d'exprimer leurs valeurs à une seule et même date (et unité). On notera \( r \) le taux d'intérêt nominal, exprimé sur une base annuelle (1 période dans l'échéancier) et supposé constant.

\section{Voyage dans le futur}

L'opération visant à cumuler des flux dans le futur est appelée capitalisation. La valeur future d'un flux initial est définie comme la valeur, exprimée dans \( n \) périodes, du flux initial \( f \). Cette valeur future peut être calculée en tenant compte du taux d'intérêt et du temps écoulé.

\[ VF_n(f)=f\cdot(1+r)^n \]

L'opération visant à exprimer des flux futurs en valeur présente est appelée actualisation. La valeur actuelle d'un flux \( f \) qui sera reçu dans \( n \) périodes est déterminée en tenant compte du taux d'intérêt, permettant ainsi de comparer des flux à des moments différents dans le temps.

\[ VA_n(f)= \frac{f}{(1+r)^n} \]

Ce que vaut aujourd'hui le flux \( f \) obtenu dans \( n \) périodes.

\( r \) est aussi appelé le taux  d'actualisation (taux actuariel si les périodes considérées sont des années).
Ce que vaut aujourd'hui le flux \( f \) obtenu dans n périodes.

\( r \) est aussi appelé le taux d'actualisation (taux actuariel si les périodes considérées sont des années).

\[ \delta = \frac{1}{(1+r)^n} \]

\section{Valeur d'une séquence de flux}

Un projet \( P \) est défini comme une séquence de flux (positifs ou négatifs) représentés par l'échéancier. La valeur nette du projet correspond à la somme des valeurs des flux exprimées à une même date. On parle de Valeur Actuelle Nette (VAN) du projet si cette date est aujourd'hui, ce qui signifie qu'il s'agit de la somme actualisée des différents flux. Ainsi, la valeur actuelle nette de la séquence de flux \( P \) est un indicateur clé pour évaluer la rentabilité d'un projet.

\[ VAN_n(P)=f_0+\frac{f_1}{(1+r)}+\frac{f_2}{(1+r)^2}+\frac{f_3}{(1+r)^3}+\cdots+\frac{f_n}{(1+r)^n} = \sum_{t=0}^{n}\frac{f_t}{(1+r)^t}\]

Ce que vaut aujourd'hui la richesse du projet \( P \) réalisé sur \( n \) périodes.

La valeur nette du projet \( P \) peut également être exprimée dans le futur. Si on l'exprime à la date \( n \) (date de fin des flux associés au projet), on obtient la Valeur Finale Nette de la séquence de flux du projet \( P \). Cette approche permet d'évaluer la rentabilité du projet à l'échéance de ses flux financiers.

\[ VFN_n(P)=f_0(1+r)^n+f_1(1+r)^{n-1}+f_2(1+r)^{n-2} + \cdots + f_n = \sum_{t=0}^{n}f_t(1+r)^{n-t}\]

Du point de vue financier, la règle de décision basée sur la Valeur Actuelle Nette (VAN) est la suivante :

\begin{itemize}
	\item VAN > 0 : Le projet est rentable. Cela signifie que les revenus actualisés générés par l'investissement dépassent son coût initial. En d'autres termes, le projet crée de la valeur pour l'entreprise ou les investisseurs, et il peut donc être réalisé.
	\item VAN = 0 : Le projet est à l'équilibre. Les revenus actualisés couvrent exactement les coûts, mais ne génèrent pas de bénéfice net supplémentaire. Dans la plupart des cas, ce type de projet n'est pas retenu, sauf si d'autres critères stratégiques sont en jeu (comme une obligation légale ou un bénéfice indirect).
	\item VAN < 0 : Le projet n'est pas rentable. Les coûts dépassent les revenus actualisés. Ce type de projet doit être évité car il détruit de la valeur.
\end{itemize}


\subsection{Exercice}

Le Laboratoire AurelGap a développé une nouvelle molécule. Son brevet a une durée de vie de 17 ans. Les bénéfices attendus pour ce nouveau médicament s'élèvent à 2 millions d'euros la première année puis ils augmentent de 5\% par an durant la durée de vie du brevet. Au terme des 17 années, un générique pourra être mis en vente sur le marché et les bénéfices obtenus par le médicament original seront alors nuls.

\begin{enumerate}
	\item Déterminer la valeur actuelle du médicament si le taux d'actualisation est de 10\%.
	\item Même question mais pour un taux d'actualisation de 3\%.
\end{enumerate}

\begin{enumerate}
	\item Les gains espérés chaque année : 
	
\includegraphics[scale=1]{../../../Pictures/Screenshots/Capture d'écran 2025-01-21 195312}

\[ VA = \frac{2}{0,1-0,05}\left( 1 - \frac{1,05}{1,1}^{16}\right)=20,99  \]

\item \[ VA = \frac{2}{0,03-0,05}\left( 1 - \frac{1,05}{1,03}^{16}\right)=36,02  \]

\end{enumerate}

\section{Cas les plus usuels}

\subsection{Les annuités constantes}

Les annuités représentent une séquence de \( n \) flux versés à intervalles réguliers (souvent tous les ans). Lorsque les flux sont égaux, on parle d'annuités constantes. Un exemple courant d'annuités constantes est celui des emprunts à remboursements fixes à taux fixes, où le montant remboursé chaque année reste constant tout au long de la durée de l'emprunt.

\begin{center}
	\includegraphics[scale=0.5]{../../../Pictures/Screenshots/Capture d'écran 2025-01-21 200847}
\end{center}

\[ \text{Valeur actuelle nette de ces annuités =}\sum_{t=0}^{n}\frac{f}{(1+r)^t}
 \]

Soit la suite dont le \( n \)-ème terme est \( a_n \). La somme de ses \( n \) premiers termes, en partant du même terme (avec \( m > n \)), s'écrit :

\[
\sum_{i=m}^{n} U_i = U_m \frac{1-q^{n-m+1}}{1-q}
\]

On peut donc en déduire,

\[ VAN = \sum_{t=1}^{n} \frac{f}{(1+r)^t}=\frac{f}{r}\left( 1-\left( \frac{1}{1+r}\right) ^n\right)   \]

Attention ici : \( U_1=\frac{f}{(1+r)} \)

\subsection{Les rentes perpétuelles constantes}

La rente perpétuelle est un titre de dette qui prévoit le paiement régulier d'intérêts sans le remboursement du capital. Cela signifie qu'il s'agit d'une annuité constante qui, théoriquement, n'aurait pas de fin, semblable à une rente viagère.

\begin{center}
	\includegraphics[scale=0.5]{../../../Pictures/Screenshots/Capture d'écran 2025-01-21 211714}
\end{center}

\[ VAN =\sum_{t=1}^{\infty}\frac{f}{(1+r)^t}= \lim_{n\rightarrow \infty}\frac{f}{r}\left( 1-\left( \frac{1}{1+r}\right) ^n\right)=\frac{f}{r} \]

\subsection{Les annuités croissantes}

L'annuité croissante est une séquence de flux de trésorerie qui augmente de manière constante à chaque période. Ces flux sont versés à intervalles réguliers et le taux de croissance des flux, noté \( g \), reste constant sur toute la durée de l'annuité.

\begin{center}
\includegraphics[scale=0.5]{../../../Pictures/Screenshots/Capture d'écran 2025-01-21 213129}
\end{center}

\[ VAN = \frac{f}{1+r}\left( \frac{1+g}{1+r} \right)^t=\frac{f}{r-g}\left( 1-\left( \frac{1+g}{1+r} \right)^n \right) \]

\subsection{Les rentes perpétuelles croissantes}

La rente perpétuelle croissante consiste en des annuités croissantes versées à l'infini.

\begin{center}
\includegraphics[scale=0.5]{../../../Pictures/Screenshots/Capture d'écran 2025-01-21 213904}
\end{center}

\[
VAN = \sum_{t=1}^{n} \frac{f}{(1+g)} \left( \frac{1+g}{1+r} \right)^{t} = \lim_{n \to \infty} \frac{f}{r-g} \left( 1 - \left( \frac{1+g}{1+r} \right)^{n} \right) = \begin{cases} 
	\frac{f}{r-g} & \text{si } g < r \\ 
	+\infty & \text{si } g > r 
\end{cases}
\]

\section{Valeur d'une séquence de flux}

\subsection{Exemple}

Supposons qu'une entreprise envisage un projet nécessitant un investissement initial de 100 000 €. Les flux de trésorerie générés sur les 3 prochaines années sont estimés à 40 000\euro, 50 000\euro~et 60 000\euro~respectivement. Si le taux d'actualisation (reflétant le coût du capital ou le risque) est de 10\%, la VAN serait calculée ainsi :

\[
VAN = -100000 + \frac{40000}{(1+0,1)} + \frac{50000}{(1+0,1)^2} + \frac{60000}{(1+0,1)^3} = 22764,84\text{\euro}
\]

Ainsi, le projet est rentable.

\section{Calculer le taux de rentabilité interne (TRI)}

Jusqu'à présent, on a supposé un taux d'intérêt constant \( r \). À partir de l'échéancier associé à un projet, on peut également calculer le taux de rendement interne (TRI) associé à ce projet.

Le taux de rentabilité interne (TRI) est la valeur du taux d'intérêt tel que la somme actualisée des flux positifs soit égale à la valeur actualisée des flux négatifs, ce qui annule la VAN associée au projet.

Considérons un projet dont le coût initial est \( C \) et qui génère \( n \) flux de revenus égaux d'un montant \( f \). Le TRI associé est la solution de l'équation suivante :

\[
\frac{f}{TRI} \left( 1 - \left( \frac{1}{1 + TRI} \right)^{n} \right) = C
\]

Dans le cas d'une rente perpétuelle, la solution est évidente :

\[
TRI = \frac{f}{C}
\]

En revanche, dans le cas où \( n \) est fini, l'équation précédente n'a pas de solution analytique évidente. Le calcul du TRI nécessite donc l'aide d'un logiciel (comme la fonction TRI dans Excel ou l'utilisation de Mathematica) ou peut être approximé.

La règle du TRI stipule que tout investissement dont le taux de rentabilité interne dépasse le coût du capital (taux d'actualisation) doit être réalisé. 

Attention : pour comparer la rentabilité entre deux projets, il est toujours recommandé d'utiliser la VAN.

\section{Quel taux d'actualisation retenir ?}


Jusqu'à présent, nous avons supposé que le taux d'actualisation est un taux d'intérêt unique et constant \( r \).

Cependant, il existe de multiples taux d'intérêt, qui peuvent varier selon les banques, le type de placements ou les clients.


La valeur actuelle nette (VAN) d'un projet est sensible à une petite variation du taux d'actualisation.


La question se pose alors : lequel choisir pour le calcul de l'actualisation ?

Le taux d'actualisation à utiliser dépend essentiellement de plusieurs facteurs :

Il est crucial d'ajuster la cotation du taux d'intérêt (annuel) avec l'échéancier du projet (mensuel). Cela peut nécessiter le calcul de taux par période infra-annuelle à partir du Taux Annuel Effectif (TAE) sur la base du taux équivalent (sur un mois ou un semestre).

Soit un TAE de 5\%, le taux équivalent par mois est déterminé par :

\[
\frac{r_m}{100} = (1,05)^{\frac{1}{12}} - 1
\]

L'horizon temporel du projet est également un facteur important, comme l'indique la courbe des taux (ou structure par terme des taux d'intérêts), qui montre la différence entre les taux courts et les taux longs.

Enfin, le risque associé au projet doit être pris en compte dans le choix du taux d'actualisation.

\section{VAN  et structure par terme des taux}

Dans le calcul de la valeur actuelle nette (VAN), il est nécessaire de tenir compte de la structure par terme des taux d'intérêt, surtout si la courbe des taux n'est pas plate. En effet, il n'y a aucune raison pour que le taux d'intérêt à échéance d'un an soit égal à celui pratiqué pour une échéance de 2 ans (ou 3 ans, etc.). En général, la courbe des taux est croissante.

Ainsi, la VAN de flux sans risques avec \( k = 1, \ldots, n \) sera :

\[
VAN_i = \sum_{k=0}^{i} \frac{f_k}{(1 + r_i)^k}
\]

où \( r_i \) est le taux d'intérêt d'un placement arrivant à échéance dans \( i \) années (périodes).

\section{Valeur d' une séquence de flux}

\subsection{Exercice}

AurelDF envisage la construction d'une centrale nucléaire pour un coût de 120 millions d'\euro~à payer immédiatement. Le bénéfice obtenu devrait être de 20 millions d'\euro~par an pendant les 10 prochaines années. Ensuite, la centrale fermera et le site devra être nettoyé pour répondre aux normes environnementales, puis surveillé. Cette surveillance coûtera 2 millions d'\euro~par an sur une période infinie.

\begin{enumerate}
	\item En appliquant le critère du TRI, l'entreprise AurelDF a-t-elle intérêt à exploiter cette mine ?
	\item Si le coût du capital est de 8\% que conclure à partir du calcul de la VAN.
\end{enumerate} 

\begin{enumerate}
	\item Les échéanciers :
	
\begin{center}
\includegraphics[scale=0.6]{../../../Pictures/Screenshots/Capture d'écran 2025-01-21 232512}
\end{center}\[
	VAN = -120 + \frac{20}{TRI} \left( 1 - \left( \frac{1}{1 + TRI} \right)^{10} \right) - \frac{2}{TRI} = 0
	\]
	
La VAN s'annule pour un TRI de \( 0,02924 \) et un \( TRI_2 = 0,08723 \). Cependant, on ne peut pas conclure uniquement sur la base de ces valeurs.
	
Considérons l'équation suivante :
\item Si \(  r = 0,08 \) la VAN est de \( 2,621791 \) millions d'euros. Cela montre que la VAN n'est pas neutre par rapport au taux d'actualisation.	
\end{enumerate}

\section{Risque et taux d'intérêt}

Le financement de la grande majorité des projets d'investissement porte un risque, notamment le risque de non-paiement des intérêts ou de non-remboursement du capital. Plus ce risque est élevé, plus le créancier exige un rendement élevé, ce qui entraîne l'existence d'une prime de risque.

Ainsi, le taux d'actualisation choisi pour calculer la valeur actuelle nette (VAN) d'un projet doit tenir compte du risque associé à ce projet.

\section{Le coût du capital}

\subsection{Définition}

Le taux d'actualisation à utiliser pour déterminer la valeur actuelle ou future d'un projet est le coût du capital.

Le coût du capital d'un projet correspond au taux de rentabilité le plus élevé présenté par un placement alternatif de même horizon et de même risque.

L'idée est qu'un investisseur potentiel cherche à évaluer ce que rapporte le projet par rapport à un placement alternatif, ce qui représente le coût d'opportunité. 

Attention : la comparaison n'a de sens que si le placement et le projet sont de même risque et de même terme.

\subsection{Premier aperçu}

Considérons un projet (P) avec les caractéristiques suivantes :

- En \( t = 0 \) : coût = 0
- En \( t = 1 \) : flux = variable aléatoire \( J \) prenant les valeurs \( J_0 > 0 \) avec probabilité \( \pi \) et \( J_1 > J_0 \) avec probabilité \( 1 - \pi \)

On note \( r_f \) le taux d'intérêt certain (taux d'intérêt appliqué aux placements sans risque).

Si j'actualise au taux \( r_f \), la valeur actuelle (VA) du projet \( P \) est donnée par :

\[
VA(P) = \frac{E(J)}{1 + r_f} = \frac{\pi J_0 + (1 - \pi) J_1}{1 + r_f}
\]

Cependant, ce calcul pose un problème, car cela ne correspond pas au projet \( P \). En effet, en \( t = 1 \), on n'obtient pas \( E(J) \) avec certitude, alors que \( r_f \) est le taux d'intérêt au certain.

\subsection{Exercice}

Laurent a le choix entre deux projets, chacun d'eux devant l'occuper à plein temps. Le projet   P1 concerne la création d'une boulangerie. L'investissement initial est de 1000 euros, les bénéfices associés seront de 1100 euros la 1ère année, ils diminueront ensuite de 10\% chaque année. Le projet  P2  concerne la création d'une pizzeria équipée d'un unique four. Il n'est pas possible d'en installer davantage. L'investissement initial est de 1000 euros, les bénéfices associés seront de 400 euros la 1ère année, ils diminueront ensuite de 20\% chaque année du fait de l'usure du four. Le coût du capital est de 12\% pour les deux projets.
\begin{enumerate}
	\item Représentez l'échéancier associé à chacun des deux projets.
	\item Calculez la VAN et le TRI pour chacun des projets. Dans lequel de ces projets Laurent devrait-il investir?
	\item Laurent se rend compte que le local du pizzeria peut en fait contenir 20 fours avec le même montant d'investissement de 1000 €. L'échelle du projet P2 est donc multipliée par 20. Calculez la VAN et le TRI du projet P2 . Dans lequel de ces projets  Laurent devrait-il investir ? Commentez.
	\item Suivant vos conseils, Laurent se lance dans le projet  P2. Son fournisseur de fours lui propose un contrat de maintenance dont le coût est de 250 euros par an. La maintenance permet d'éliminer totalement l'usure des fours. On appelle  P2 le projet P2  avec contrat de maintenance. Ecrivez l'échéancier associé à ce nouveau projet. 
	\item Calculez la VAN et le TRI du projet  P2. Laurent devrait-il accepter le contrat de maintenance ? Commentez.
	\item Calculez le TRI différentiel associé à la comparaison entre   P2  et P1. Commentez.
	
\end{enumerate}

\begin{enumerate}
	\item Echéancier des projets P1 et P2 
	
\begin{center}
	\includegraphics[scale=0.6]{../../../Pictures/Screenshots/Capture d'écran 2025-01-22 001309}
\end{center}

\item Pour le projet \( P_1 \), la valeur actuelle nette (VAN) est donnée par :

\[
VAN_{P_1} = -1000 + \sum_{t=1}^{T} \frac{1100}{0,9} \left( \frac{0,9}{1,12} \right)^{t} = -1000 + \frac{1100}{0,22} \left( 1 - \left( \frac{0,9}{1,12} \right)^{T} \right)
\]

Lorsque \( T \to \infty \), nous avons :

\[
VAN_{P_1} = -1000 + \frac{1100}{0,22} = 4000 \text{ euros}
\]

Pour le taux de rendement interne (TRI) du projet \( P_1 \), nous avons :

\[
TRI_{P_1} \Rightarrow -1000 + \frac{1100}{TRI_{P_1} + 0,1} = 0 \Rightarrow TRI_{P_1} = 100\%
\]

Pour le projet \( P_2 \), la VAN est donnée par :

\[
VAN_{P_2} = -1000 + \sum_{t=1}^{T} \frac{400}{0,8} \left( \frac{0,8}{1,12} \right)^{t} = -1000 + \frac{400}{0,32} \left( 1 - \left( \frac{0,8}{1,12} \right)^{T} \right)
\]

Lorsque \( T \to \infty \), nous avons :

\[
VAN_{P_2} = -1000 + \frac{400}{0,32} = 250 \text{ euros}
\]

Pour le TRI du projet \( P_2 \), nous avons :

\[
TRI_{P_2} \Rightarrow -1000 + \frac{400}{TRI_{P_2} + 0,2} = 0 \Rightarrow TRI_{P_2} = 20\%
\]

Il convient donc de choisir \( P_1 \) plutôt que \( P_2 \) selon les critères de la VAN et du TRI.

\item Si à présent, le nombre de fours est multiplié par 20, nous avons, avec le même montant d'investissement :

\[
VAN_{P_2} = -1000 + \frac{(400 \times 20)}{0,32} = 24000 \text{ euros}
\]

Pour le TRI du projet \( P_2 \), nous avons :

\[
TRI_{P_2} \Rightarrow -1000 + \frac{(400 \times 20)}{TRI_{P_2} + 0,2} = 0 \Rightarrow TRI_{P_2} = 798\%
\]

Ainsi, Laurent doit choisir à présent le projet \( P_2 \).

\item Projet \( P_2 \) et contrat de maintenance

\begin{center}
	
	\includegraphics[scale=0.6]{../../../Pictures/Screenshots/Capture d'écran 2025-01-22 002943}

\end{center}

Lorsque \( T \to \infty \), la valeur actuelle nette (VAN) pour le projet \( P_2 \) est donnée par :

\[
VAN_{P_2} = -1000 + \frac{7750}{0,32} = 63583,33 \text{ euros}
\]

Pour le taux de rendement interne (TRI) du projet \( P_2 \) avec maintenance, nous avons :

\[
TRI_{P_2} \Rightarrow -1000 + \frac{7750}{TRI_{P_2} + 0,2} = 0 \Rightarrow TRI_{P_2} = 775\%
\]

Laurent devrait accepter le contrat de maintenance car la VAN est plus élevée. Toutefois, le TRI est plus faible que celui du projet \( P_2 \) sans maintenance.

\end{enumerate}

\subsection{Premier aperçu}

L'hypothèse d'aversion au risque stipule que la valeur d'une alternative risquée, notée \( V_A(P) \), est inférieure à la valeur actualisée des gains attendus, exprimée par la formule \[ \frac{\pi J_0 + (1 - \pi) J_1}{1 + r_f}  \]où \( r_f \) représente le taux sans risque. Pour qu'un projet soit jugé viable, le taux d'actualisation utilisé doit être supérieur à \( r_f \), ce qui implique un coût du capital. La prime de risque associée au projet, notée \( \pi_p \), est déterminée par l'évaluation des risques spécifiques liés à celui-ci, tels que la volatilité des flux de trésorerie et les incertitudes économiques. Ainsi, il existe un lien direct entre le risque associé au projet et le coût du capital : plus le risque est élevé, plus le coût du capital nécessaire pour financer le projet augmente, car les investisseurs exigent une compensation plus importante pour accepter une incertitude accrue.

\chapter{Coût du capital et risque}
\pagestyle{plain} 
\section{Risque et rentabilité}

Comme nous l'avons vu, le rendement d'un placement dépend non seulement de sa rentabilité, mais également du risque qui lui est associé, exprimé à travers la prime de risque. Le risque et la rentabilité d'un actif sont mesurés respectivement par la volatilité des rendements et le rendement attendu. La volatilité, souvent calculée à partir des écarts-types des rendements passés, quantifie l'incertitude liée aux fluctuations des prix de l'actif. La relation entre risque et rentabilité est généralement positive : les actifs présentant un risque plus élevé tendent à offrir des rendements attendus plus importants pour compenser les investisseurs pour l'incertitude accrue. En d'autres termes, les investisseurs s'attendent à être rémunérés par des rendements plus élevés lorsqu'ils prennent des risques supplémentaires.

Par nature, la rentabilité future d'un actif risqué est inconnue \textit{ex ante}. Pour pouvoir en dire quelque chose, nous nous plaçons dans un monde probabilisable. Nous faisons l'hypothèse qu'il est possible de définir l'ensemble des scénarios envisageables, ce qui nous permet d'établir une liste des états de la nature. De plus, nous supposons qu'il est possible d'associer une probabilité d'occurrence à chacun de ces scénarios, ce qui nous conduit à établir une distribution de probabilité sur les états de la nature possibles. Ces hypothèses sont fondamentales pour évaluer le risque et la rentabilité des actifs dans un cadre probabiliste.

La rentabilité espérée de l'actif \( R \) est donnée par la formule suivante :

\[
\mathbb{E}(R) = \sum_{i=1}^{n} p_i R_i
\]

où \( p_i \) représente la probabilité associée à chaque scénario et \( R_i \) la rentabilité dans ce scénario. La variance de l'actif \( R \) est calculée par :

\[
\mathbb{V}(R) = \sum_{i=1}^{n} p_i (R_i - \mathbb{E}(R))^2 = \mathbb{E}(R^2) - \mathbb{E}(R)^2
\]

L'écart-type de l'actif \( R \) est défini comme suit :

\[
\sigma_r = \sqrt{V(R)}
\]

Il est à noter que l'écart-type est la mesure traditionnelle du risque associé à un actif, quantifiant ainsi l'incertitude des rendements futurs.

Pour calculer l'espérance de rentabilité et le risque associé à un actif, il est nécessaire de disposer de la distribution de probabilité des rentabilités. Cependant, cette donnée est inobservable, ce qui conduit généralement à une estimation de cette distribution à partir de données historiques. Cette méthode présente des limites, notamment un problème de qualité de la prédiction (précision) qui est lié à la stabilité de l'environnement économique. En effet, les variations économiques peuvent affecter la fiabilité des estimations basées sur des données passées, rendant ainsi les prévisions moins précises.

La rentabilité historique désigne la rentabilité qui a été effectivement réalisée et constatée pour un actif donné au cours d'une période définie dans le passé. La rentabilité d'une action à la période \( t + 1 \) est calculée selon la formule suivante :

\[
R_{t+1} = \frac{P_{t+1} + \text{Div}_{t+1} - P_t}{P_t} = \frac{\text{Div}_{t+1}}{P_t} + \frac{P_{t+1} - P_t}{P_t}
\]

où \( \text{Div}_{t+1} \) représente les dividendes distribués en \( t + 1 \) et \( P_t \) le prix de l'action à la date \( t \). On peut estimer la distribution de probabilité des rentabilités en utilisant les rentabilités observées successivement durant un grand nombre d'années, ce qui permet de construire la densité de probabilité empirique.

\begin{center}
	\includegraphics[scale=0.5]{../../../Pictures/Screenshots/Capture d'écran 2025-01-26 145125}
\end{center}

À partir de la densité de probabilité empirique, il est facile de déduire une estimation (sans biais) de l'espérance de rentabilité et une estimation de la variance de la rentabilité d'un actif.


En notant \( R_t \) la rentabilité effective d'un actif en \( t \), observée entre les périodes 1 et \( T \), l'estimation sans biais de l'espérance de rentabilité de cet actif est donnée par :

\[
\bar{R} = \frac{1}{T} \sum_{t=1}^{T} R_t 
\]

\subsection*{Variance Empirique des Rentabilités Effectives}

De même, l'estimation de la variance des rentabilités de cet actif est donnée par :

\[
V(R) = \frac{1}{T-1} \sum_{t=1}^{T} (R_t - \bar{R})^2 
\]

Il faut garder en tête les limites associées à cette estimation de la rentabilité :

\begin{itemize}
	\item Elle repose sur une hypothèse forte : les rentabilités observées sont tirées de variables aléatoires indépendantes et identiquement distribuées (VA iid).
	\item Même si cette hypothèse est vérifiée, les erreurs de mesure peuvent être importantes et l'estimation peu précise avec un nombre d'observations limité.
\end{itemize}

L'aversion pour le risque des investisseurs implique qu'il devrait exister une relation croissante entre risque et rentabilité. 

\begin{itemize}
	\item On retrouve cette relation quand on s'intéresse à des portefeuilles diversifiés.
	\item En revanche, on ne la retrouve pas systématiquement quand on examine la rentabilité des titres individuels.
\end{itemize}

Comment expliquer ce phénomène ? Il est nécessaire de faire la différence entre risque spécifique et risque systématique.

\begin{center}
	\includegraphics[scale=0.5]{../../../Pictures/Screenshots/Capture d'écran 2025-01-26 151425}
	
	\includegraphics[scale=0.5]{../../../Pictures/Screenshots/Capture d'écran 2025-01-26 151713}
\end{center}

\section{Risque spécifique et risque systématique}

Le risque lié à la détention d'un actif se définit comme la possibilité que sa rentabilité effective soit inférieure à sa rentabilité espérée.

Les déterminants de la rentabilité d'une action incluent le cours de cette action et les dividendes distribués.

Les sources de variation dans les cours et les dividendes comprennent :
\begin{itemize}
	\item Informations spécifiques à l'entreprise.
	\item Informations relatives à l'ensemble du marché.
\end{itemize}

Il existe deux types de risque : le risque spécifique ou diversifiable, qui est l'incertitude associée aux informations spécifiques à l'entreprise, et le risque systématique ou non-diversifiable, qui est l'incertitude relative aux informations macroéconomiques.

Dans le cas d'un portefeuille d'actions, les risques indépendants se compensent entre eux (ils ne sont pas tous dans le même sens et ils n'arrivent pas tous en même temps), mais pas les risques systématiques.

La diversification permet de réduire le risque, mais ne l'élimine pas complètement. Le risque systématique demeure.

Nous avons vu qu'il existe une relation croissante entre risque systématique et rentabilité. Cependant, une telle relation n'existe pas entre risque spécifique et rentabilité. 

Comment expliquer ce second résultat ? Il est possible de s'assurer contre le risque spécifique en diversifiant son portefeuille d'actions. Dans ces conditions, il ne peut exister une prime de risque pour un actif porteur uniquement d'un risque spécifique.

\section{Prime de risque, risque spécifique et risque systématique}

En conclusion, la prime de risque offerte par un actif est déterminée uniquement par son risque systématique ; elle ne dépend pas de son risque diversifiable. L'écart-type constitue une information pertinente pour déterminer la prime de risque associée à un portefeuille d'actions, mais pas pour la prime de risque associée à un titre individuel. En effet, pour chaque titre, il n'y a pas de relation claire entre l'écart-type et la rentabilité.

Ainsi, pour estimer le coût du capital, il faut d'abord estimer la prime de risque, qui elle-même dépend de l'estimation du risque systématique.

\section{Mesurer le risque systématique}


Quelle part de la volatilité d'un actif incombe au risque systématique ? Cette part correspond à la sensibilité de la rentabilité de l'action aux chocs systémiques. 

Pour mesurer cette sensibilité, on compare la rentabilité de l'action à la rentabilité d'un portefeuille exclusivement exposé au risque de marché, appelé portefeuille efficient. Nous ne reviendrons pas sur l'identification d'un tel portefeuille, qui doit être suffisamment diversifié, c'est-à-dire contenir un grand nombre de titres, formant ainsi le portefeuille de marché. En pratique, on l'approxime par un indice boursier suffisamment large, comme le S\&P 500 ou le SBF 250.

\subsection{Exercice}

Parmi les risques suivants, lesquels sont systématiques et lesquels sont diversifiables ?

\begin{enumerate}
	\item Le PDG disparaît dans un accident d'avion. Ce risque est spécifique à l'entreprise et donc diversifiable.
	\item L'économie entre en récession, ce qui réduit la demande adressée à l'entreprise. Ce risque est systématique, car il est lié aux conditions macroéconomiques.
	\item L'ingénieur le plus créatif de la division R\&D part à la concurrence. Ce risque est spécifique à l'entreprise et donc diversifiable.
	\item Les recherches en cours dans la division R\&D ne débouchent pas sur des innovations. Ce risque est spécifique à l'entreprise et donc diversifiable.
\end{enumerate}

\subsection{Solution}

\begin{enumerate}
	\item Le PDG disparaît dans un accident d'avion. Type : Diversifiable. Explication : La disparition du PDG est un événement spécifique à l'entreprise. Ce type de risque peut être atténué en investissant dans plusieurs entreprises pour réduire l'impact d'un événement individuel.
	\item L'économie entre en récession, ce qui réduit la demande adressée à l'entreprise. Type : Systématique. Explication : La récession est un phénomène économique global qui affecte toutes les entreprises. Il est impossible de l'éliminer par diversification, car il touche tous les secteurs dans une certaine mesure.
	\item L'ingénieur le plus créatif de la division R\&D part à la concurrence. Type : Diversifiable
	\item Les recherches en cours dans la division R\&D ne débouchent pas sur des innovations. Type : Diversifiable
\end{enumerate}

\subsection{Exercice : Soit les rentabilités annuelles d'un titre sur la période 2011-2016}

\begin{center}
	\begin{tabular}{@{}lllllll@{}}
	\toprule
	Année     & 2011    & 2012 & 2013 & 2014  & 2015 & 2016 \\ \midrule
	Rendement & -12.5\% & -4\% & -8\% & 2.5\% & 10\% & 13\% \\ \bottomrule
\end{tabular}
\end{center}


\begin{enumerate}
	\item Quelle est la rentabilité annuelle moyenne sur la période ?
	\item Quelle est la variance des rentabilités sur la période ?
	\item Déterminer l'intervalle de confiance de la rentabilité annuelle espéré au seuil de 5\%. On supposera que les rentabilités annuelles sont des variables aléatoires indépendantes  qui suivent une loi de Student.
\end{enumerate}

Rappel : Soit \(  R_i, i =1,\cdots,n, \) \( n \) variable aléatoires indépendantes et de même loi et d'écart type  . On peut estimer l'écart type de la moyenne de ces variables aléatoires par : \( \sigma_R = \frac{\sigma_R }{\sqrt{N}} \)

\subsection{Solution}

\begin{enumerate}
	\item La rentabilité annuelle moyenne est donnée par : $R_n = \frac{1}{n} \cdot \sum R_i$, où $R_n = 0.0017$ (soit 0,17\%).
	\item La variance des rentabilités est donnée par : $\sigma^2 = \frac{1}{n} \cdot \sum (R_i - \bar{R})^2$, où $\sigma^2 = 0.0102$.
	\item L'intervalle de confiance est donné par : $\bar{R} \pm t_{n-1,\alpha/2} \cdot (\sigma / \sqrt{n})$. Au seuil de 5\%, l'intervalle de confiance est $[-0.1045, 0.1078]$ (soit $[-10,45\%, 10,78\%]$). Cela signifie que la rentabilité annuelle moyenne se situe avec 95\% de confiance dans cet intervalle.
\end{enumerate}


La mesure du risque systématique d'un actif consiste à calculer son bêta. 

Le bêta (\( \beta \)) d'un actif représente la variation en pourcentages de la rentabilité excédentaire d'un titre donné lorsque la rentabilité (excédentaire) du portefeuille de marché varie de 1\%. Il s'agit donc d'une élasticité.

La rentabilité excédentaire est définie comme l'écart entre la rentabilité de l'actif et le taux d'intérêt sans risque. Nous ne reviendrons pas sur les différentes façons d'estimer le bêta d'un actif, cela a été abordé dans le cours "Évaluation des actifs financiers".

\begin{center}
	\includegraphics[scale=0.5]{../../../Pictures/Screenshots/Capture d'écran 2025-01-26 173535}
\end{center}

\section{Estimer la prime de risque}

On appelle la prime de risque de marché ($\Pi_m$) la différence entre la rentabilité espérée du portefeuille de marché ($\mathbb{E}(R_m)$) et le taux d'intérêt sans risque ($r_f$) :
\[
\Pi_m = \mathbb{E}(R_m) - r_f
\]

Le risque systématique d'un titre sera proportionnel à son bêta ($\beta_i$), tout comme sa prime de risque. On a donc :
\[
\Pi_i =\beta_i(\mathbb{E}(R_m) - r_f) = \beta_i \Pi_m
\]

Il est alors possible de proposer une estimation de l'espérance de rentabilité d'un titre à partir de son bêta :
\[
\mathbb{E}(R_i) = r_f + \Pi_i = r_f + \beta_i (\mathbb{E}(R_m) - r_f)
\]

\section{Risque et coût du capital}

Rappel : le taux d'actualisation est égal au coût du capital, qui correspond à la rentabilité espérée d'actifs de même risque et de même rentabilité espérée.

On a donc : le coût du capital est égal à la rentabilité espérée d'actifs ayant le même bêta.

Le coût du capital $r_p$ d'un projet de bêta $\beta_p$ est donc :
\[
r_p = r_f + \beta_p (\mathbb{E}(R_m) - r_f)
\]

Comment déterminer $\beta_p$ ?

Généralement, on utilise le bêta de l'entreprise comme bêta (ou celui d'entreprises comparables) de ses projets d'investissement.

Dans le cas d'une entreprise non-endettée, le bêta des actions de l'entreprise peut être utilisé directement comme bêta de l'entreprise et donc comme bêta du projet.

Nous verrons dans la Partie II comment calculer le coût du capital pour une entreprise endettée.

\subsection{Exercice}

Soit les rentabilités annuelles d'un actif au cours de quatre années successives.

\begin{center}
	\begin{tabular}{@{}lllll@{}}
	\toprule
	Année     & 1    & 2    & 3    & 4    \\ \midrule
	Rendement & 10\% & 20\% & -5\% & 15\% \\ \bottomrule
\end{tabular}
\end{center}

\begin{enumerate}
	\item Quel est le taux de croissance annuel composé de cet actif sur les quatre années ?
	\item Quelle est la rentabilité annuelle moyenne de l'actif sur la période ?
	\item Quelle mesure de performance passée de l'actif doit-on choisir ?
	\item En supposant que les rentabilités sont des variables aléatoires indépendantes et identiquement distribuées, estimer la rentabilité espérée de  cet actif.
\end{enumerate}

\subsection{Solution}

\begin{enumerate}

\item Formule du taux de croissance annuel composé (\textit{Compound Annual Growth Rate - CAGR})

La formule du CAGR est :
\[
\text{CAGR} = \left[\left(1 + r_1\right) \times \left(1 + r_2\right) \times \dots \times \left(1 + r_n\right)\right]^{1/n} - 1
\]
Avec $r_i$ les taux de croissance annuels.

Dans l'exemple donné, on a : $r_1 = 10\%$, $r_2 = 20\%$, $r_3 = -5\%$ et $r_4 = 15\%$. Le CAGR est donc :
\[
\text{CAGR} = \left[(1.10 \times 1.20 \times 0.95 \times 1.15)\right]^{1/4} - 1 \approx 9.43\%
\]

Une seconde manière de calculer le CAGR est :
\[
\text{CAGR} = \left(\frac{A_T}{A_0}\right)^{1/n} - 1
\]
Où $A_T$ (resp. $A_0$) est la valeur de l'actif à la date terminale (resp. initiale). Ici, on ne connaît pas $A_0$, mais on sait que $A_T = A_0 \times (1.10 \times 1.20 \times 0.95 \times 1.15) \Rightarrow \frac{A_T}{A_0} = 1.44 \Rightarrow \text{CAGR} = 9.43\%$.

\item La rentabilité moyenne est calculée comme :
\[
\text{Rentabilité moyenne} = \frac{r_1 + r_2 + \dots + r_n}{n} = \frac{10\% + 20\% - 5\% + 15\%}{4} = 10\%
\]

\item Le CAGR est préféré pour mesurer la performance passée car :
\begin{itemize}
	\item Il reflète mieux la croissance réelle en tenant compte de la capitalisation.
	\item La moyenne arithmétique (rentabilité moyenne) surestime souvent la rentabilité réelle.
\end{itemize}
Donc, le CAGR est à privilégier pour analyser la performance passée.

\item La rentabilité espérée est égale à la moyenne des rendements, sous l'hypothèse que les rendements sont indépendants et identiquement distribués (iid) :
\[
\text{Rentabilité espérée} = \frac{10\% + 20\% - 5\% + 15\%}{4} = 10\%
\]
\end{enumerate}

\subsection{Exercice}
Il existe deux types d'entreprises S et I. La performance des actions des entreprises S dépend uniquement de la conjoncture. Lorsqu'elle est bonne leur rentabilité est de 40\%, lorsqu'elle est mauvaise, leur rentabilité est de -5\%. La conjoncture a une chance sur trois d'être bonne.

La performance des actions des entreprises I dépend uniquement de la qualité de leur dirigeant. Lorsque ce dernier est efficace leur rentabilité est de 35\%, lorsqu'il est inefficace, leur rentabilité est de -25\%. La moitié des dirigeants sont efficaces. De plus ces derniers sont répartis de façon aléatoire entre les entreprises I.

\begin{enumerate}
	\item Quelle est la volatilité des actions d'une entreprise de type  \( I \) ? Même question pour une entreprise de type \( S \) ?
\end{enumerate}

\subsection{Solution}

\begin{enumerate}
	\item \[
E(R_S) = \left(\frac{1}{3} \times 40\%\right) + \left(\frac{2}{3} \times (-5\%)\right)
\]

\[
E(R_S) = \frac{40}{3} - \frac{10}{3} = \frac{30}{3} = 10\%
\]



\[
E(R_I) = \left(\frac{1}{2} \times 35\%\right) + \left(\frac{1}{2} \times (-25\%)\right)
\]

\[
E(R_I) = \frac{35}{2} - \frac{25}{2} = \frac{10}{2} = 5\%
\]

\[
\text{Var}(R_S) = \left(\frac{1}{3} \times (40 - 10)^2\right) + \left(\frac{2}{3} \times (-5 - 10)^2\right)
\]

\[
\text{Var}(R_S) = \left(\frac{1}{3} \times 900\right) + \left(\frac{2}{3} \times 225\right)
\]

\[
\text{Var}(R_S) = 300 + 150 = 450
\]

\[
\sigma_S = \sqrt{450} \approx 21.21\%
\]

\[
\text{Var}(R_I) = \left(\frac{1}{2} \times (35 - 5)^2\right) + \left(\frac{1}{2} \times (-25 - 5)^2\right)
\]

\[
\text{Var}(R_I) = \left(\frac{1}{2} \times 900\right) + \left(\frac{1}{2} \times 900\right)
\]

\[
\text{Var}(R_I) = 900
\]

\[
\sigma_I = \sqrt{900} = 30\%
\]

\begin{itemize}
	\item Volatilité de l'entreprise de type S : \( \sigma_S \approx 21.21\% \)
	\item Volatilité de l'entreprise de type I : \( \sigma_I = 30\% \)
\end{itemize}
\end{enumerate}

\subsection{Exercice}

\begin{enumerate}
	\item Quelle est la volatilité d'un portefeuille composé de 100 actions d'entreprises de type I ? Idem  pour 100 actions d'entreprises de type S.
	\item On suppose que la rentabilité du portefeuille de marché augmente de 47\% en période d'expansion et baisse de 25\% en période de récession.  Quel type d'actifs aura le bêta le plus élevé ? Commentez.
\end{enumerate}

\subsection{Solution}
\begin{enumerate}
	\item \textbf{Volatilité d'un portefeuille de 100 actions d'entreprises de type I et S :}
	
	La volatilité d'un portefeuille composé d'actions identiques est proportionnelle à la volatilité d'une seule action. En effet, si l'on suppose que les rendements des actions sont parfaitement corrélés, la volatilité totale est donnée par :
	
	\[
	\sigma_{\text{portefeuille}} = \sigma_{\text{action}}
	\]
	
	Ainsi, pour un portefeuille de 100 actions d'entreprises de type I :
	
	\[
	\sigma_{I} = 30\%
	\]
	
	De même, pour un portefeuille de 100 actions d'entreprises de type S :
	
	\[
	\sigma_{S} \approx 21.21\%
	\]
	
	En conclusion, la volatilité reste la même, car la diversification ne réduit pas le risque dans le cas où toutes les actions sont identiques.
	
	\item \textbf{Comparaison du bêta des actifs de type I et S :}
	
	Le bêta (\(\beta\)) mesure la sensibilité de la rentabilité d'un actif par rapport au portefeuille de marché. Il est défini comme :
	
	\[
	\beta = \frac{\text{Cov}(R_{\text{actif}}, R_{\text{marché}})}{\text{Var}(R_{\text{marché}})}
	\]
	
	Les rendements du marché sont donnés par :
	
	\[
	R_{\text{marché}}^+ = 47\%, \quad R_{\text{marché}}^- = -25\%
	\]
	
	\textbf{Calcul de la rentabilité espérée du marché :}
	
	\[
	E(R_{\text{marché}}) = \frac{1}{2} \times 47\% + \frac{1}{2} \times (-25\%) = \frac{47 - 25}{2} = 11\%
	\]
	
	\textbf{Calcul de la variance du marché :}
	
	\[
	\text{Var}(R_{\text{marché}}) = \frac{1}{2} \times (47 - 11)^2 + \frac{1}{2} \times (-25 - 11)^2
	\]
	
	\[
	\text{Var}(R_{\text{marché}}) = \frac{1}{2} \times 1296 + \frac{1}{2} \times 1296 = 1296
	\]
	
	\[
	\sigma_{\text{marché}} = \sqrt{1296} = 36\%
	\]
	
	\textbf{Analyse des actifs I et S :}
	
	- Les entreprises de type S sont sensibles à la conjoncture économique, elles auront un bêta plus élevé si leur rentabilité varie fortement avec la conjoncture.
	- Les entreprises de type I sont influencées par des facteurs internes (efficacité du dirigeant), elles sont donc moins corrélées avec le marché.
	
	Par conséquent, les entreprises de type S devraient avoir un \(\beta\) plus élevé, car leur rentabilité est directement affectée par la conjoncture économique, qui est un facteur clé de la rentabilité du marché.
	
	\textbf{Conclusion :} Le bêta des entreprises de type S est plus élevé que celui des entreprises de type I, car leur performance est plus liée aux fluctuations du marché.
	
\end{enumerate}

\subsection{Exercice}


On suppose que le portefeuille de marché a autant de chance d'augmenter de 30\% que de diminuer de 10\%. 

\begin{enumerate}
	\item Quel est le béta d'un titre dont le cours augmente de 43\% en moyenne quand le marché est haussier et diminue de 17\% en moyenne quand le marché est baissier?
	\item Quel est le béta d'un titre dont le cours augmente en moyenne de 18\% lorsque le marché est baissier et diminue en moyenne de 22\% lorsque le marché est haussier?
	\item Exprimer le béta d'un titre dont la rentabilité espérée est de 4\% indépendamment de la rentabilité du marché?
\end{enumerate}

\subsection{Solution}

\begin{enumerate}
	\item \textbf{Calcul du bêta pour un titre augmentant de 43\% en marché haussier et diminuant de 17\% en marché baissier}
	
	\textbf{Données :}
	
	\begin{itemize}
		\item Rentabilités du marché : 
		\[
		R_M^+ = 30\%, \quad R_M^- = -10\%, \quad P(+) = P(-) = \frac{1}{2}
		\]
		\item Rentabilités du titre : 
		\[
		R_A^+ = 43\%, \quad R_A^- = -17\%
		\]
	\end{itemize}
	
	\textbf{Rentabilité espérée du marché :}
	\[
	E(R_M) = \frac{1}{2} \times 30\% + \frac{1}{2} \times (-10\%) = \frac{30 - 10}{2} = 10\%
	\]
	
	\textbf{Rentabilité espérée du titre :}
	\[
	E(R_A) = \frac{1}{2} \times 43\% + \frac{1}{2} \times (-17\%) = \frac{43 - 17}{2} = 13\%
	\]
	
	\textbf{Covariance entre le titre et le marché :}
	\[
	\text{Cov}(R_A, R_M) = \frac{1}{2} \left( (43 - 13)(30 - 10) + (-17 - 13)(-10 - 10) \right)
	\]
	\[
	\text{Cov}(R_A, R_M) = \frac{1}{2} \left( 30 \times 20 + (-30) \times (-20) \right)
	\]
	\[
	\text{Cov}(R_A, R_M) = \frac{1}{2} (600 + 600) = 600
	\]
	
	\textbf{Variance du marché :}
	\[
	\text{Var}(R_M) = \frac{1}{2} \left( (30 - 10)^2 + (-10 - 10)^2 \right)
	\]
	\[
	\text{Var}(R_M) = \frac{1}{2} \left( 20^2 + (-20)^2 \right) = 400
	\]
	
	\textbf{Bêta du titre :}
	\[
	\beta_A = \frac{\text{Cov}(R_A, R_M)}{\text{Var}(R_M)} = \frac{600}{400} = 1.5
	\]
	
	Le bêta du titre est donc \( \beta_A = 1.5 \).
	
	\item \textbf{Calcul du bêta pour un titre augmentant de 18\% en marché baissier et diminuant de 22\% en marché haussier}
	
	\textbf{Données :}
	
	\begin{itemize}
		\item Rentabilités du marché :
		\[
		R_M^+ = 30\%, \quad R_M^- = -10\%, \quad P(+) = P(-) = \frac{1}{2}
		\]
		\item Rentabilités du titre :
		\[
		R_B^+ = -22\%, \quad R_B^- = 18\%
		\]
	\end{itemize}
	
	\textbf{Rentabilité espérée du titre :}
	\[
	E(R_B) = \frac{1}{2} \times (-22\%) + \frac{1}{2} \times 18\% = \frac{-22 + 18}{2} = -2\%
	\]
	
	\textbf{Covariance entre le titre et le marché :}
	\[
	\text{Cov}(R_B, R_M) = \frac{1}{2} \left( (-22 + 2)(30 - 10) + (18 + 2)(-10 - 10) \right)
	\]
	\[
	\text{Cov}(R_B, R_M) = \frac{1}{2} \left( -24 \times 20 + 20 \times (-20) \right)
	\]
	\[
	\text{Cov}(R_B, R_M) = \frac{1}{2} (-480 - 400) = -440
	\]
	
	\textbf{Bêta du titre :}
	\[
	\beta_B = \frac{\text{Cov}(R_B, R_M)}{\text{Var}(R_M)} = \frac{-440}{400} = -1.1
	\]
	
	Le bêta du titre est donc \( \beta_B = -1.1 \).
	
	\item \textbf{Bêta d'un titre avec une rentabilité constante de 4\% indépendamment du marché}
	
	Lorsque la rentabilité d'un titre est constante et indépendante de la rentabilité du marché, la covariance entre le titre et le marché est nulle :
	
	\[
	\text{Cov}(R_C, R_M) = 0
	\]
	
	Par conséquent, le bêta du titre est donné par :
	
	\[
	\beta_C = \frac{\text{Cov}(R_C, R_M)}{\text{Var}(R_M)} = \frac{0}{400} = 0
	\]
	
	Le bêta de ce titre est donc \( \beta_C = 0 \), ce qui signifie que le titre est insensible aux fluctuations du marché.
	
\end{enumerate}

\subsection{Exercice}

On suppose que le portefeuille de marché a autant de chance d'augmenter de 30\% que de baisser de 10\% et que le taux d'intérêt sans risque est de 4\% (\( r_f \))

\begin{enumerate}
	\item A partir du béta calculé à  la question 1 de l'exercice précédent, calculez le coût du capital d'un projet de même béta.
	\item A partir du béta calculé à  la question 2 de l'exercice précédent, calculez le coût du capital d'un projet de même béta.
\end{enumerate}

\subsection{Correction de l'exercice}

On suppose que le portefeuille de marché a autant de chance d'augmenter de 30\% que de baisser de 10\%, et que le taux d'intérêt sans risque est de 4\% (\( r_f \)). 

\begin{enumerate}
	\item \textbf{Calcul du coût du capital pour un projet de même bêta que dans la question 1 de l'exercice précédent}
	
	\textbf{Données :}
	\begin{itemize}
		\item Taux d'intérêt sans risque : \( r_f = 4\% \)
		\item Rentabilité espérée du marché : 
		\[
		E(R_M) = \frac{1}{2} \times 30\% + \frac{1}{2} \times (-10\%) = \frac{30 - 10}{2} = 10\%
		\]
		\item Bêta du projet (issu de l'exercice précédent) : \( \beta_1 = 1.5 \)
	\end{itemize}
	
	\textbf{Formule du coût du capital selon le modèle d'évaluation des actifs financiers (MEDAF) :}
	\[
	E(R_A) = r_f + \beta (E(R_M) - r_f)
	\]
	
	\textbf{Application :}
	\[
	E(R_A) = 4\% + 1.5 \times (10\% - 4\%)
	\]
	
	\[
	E(R_A) = 4\% + 1.5 \times 6\% = 4\% + 9\% = 13\%
	\]
	
	Ainsi, le coût du capital du projet est de \( 13\% \).
	
	\item \textbf{Calcul du coût du capital pour un projet de même bêta que dans la question 2 de l'exercice précédent}
	
	\textbf{Données :}
	\begin{itemize}
		\item Taux d'intérêt sans risque : \( r_f = 4\% \)
		\item Rentabilité espérée du marché : 
		\[
		E(R_M) = 10\%
		\]
		\item Bêta du projet (issu de l'exercice précédent) : \( \beta_2 = -1.1 \)
	\end{itemize}
	
	\textbf{Formule du coût du capital :}
	\[
	E(R_B) = r_f + \beta (E(R_M) - r_f)
	\]
	
	\textbf{Application :}
	\[
	E(R_B) = 4\% + (-1.1) \times (10\% - 4\%)
	\]
	
	\[
	E(R_B) = 4\% - 1.1 \times 6\% = 4\% - 6.6\% = -2.6\%
	\]
	
	Ainsi, le coût du capital du projet est de \( -2.6\% \), ce qui signifie que ce projet est moins risqué que l'actif sans risque et pourrait générer une rentabilité inférieure au taux sans risque.
	
\end{enumerate}

\chapter{Critères de  choix d'investissement}


\section{VAN d'un projet d'investissement}

Le taux d'actualisation est fixé pour le calcul de la valeur actuelle nette (VAN) associée à un projet. Selon le principe établi, un projet dont la VAN est positive sera entrepris. En 2001, 75\% des entreprises américaines, contre seulement 35\% des entreprises françaises, utilisaient la VAN pour sélectionner leurs projets.

D'autres critères sont également pris en compte dans le processus de décision, tels que le délai de récupération et le taux de rendement interne (TRI).

\section{Le délai de récupération}

Le principe est qu'un projet sera entrepris s'il permet de récupérer les capitaux investis avant la fin de la dixième année. Ce critère est très utilisé, notamment en France, au Royaume-Uni et en Allemagne, car il est très simple à mettre en œuvre.

Cependant, une limite importante de ce critère est qu'il ne prend pas en compte la valeur temps de l'argent ni le coût du capital. Cela peut conduire à rejeter des projets ayant une valeur actuelle nette (VAN) positive.

\section{Le taux de rendement interne (TRI)}

Le principe est qu'un projet sera entrepris si le taux de rendement interne (TRI) dépasse le coût du capital. En général, les critères de la valeur actuelle nette (VAN) et du TRI conduisent à la même décision.

Cependant, ce résultat est vrai seulement si la VAN est une fonction décroissante du coût du capital. Il est donc déconseillé d'utiliser le TRI si ce n'est pas le cas, notamment lorsque les bénéfices précèdent les coûts. De plus, le TRI n'existe pas toujours ; dans ce cas, il est préférable d'utiliser la VAN. Enfin, lorsqu'il existe, le TRI n'est pas toujours unique. S'il ne l'est pas, il est également recommandé d'utiliser la VAN.


\section{Projets mutuellement exclusifs}

Jusqu'à présent, nous avons considéré des projets pouvant être entrepris indépendamment. Cependant, il arrive qu'une entreprise doive choisir un projet parmi plusieurs projets mutuellement exclusifs, également appelés projets incompatibles. Souvent, on suppose que ces projets sont envisagés pour répondre au même besoin, ce qui signifie qu'ils utilisent les mêmes ressources et le même montant d'investissement.

Dans ce contexte, le critère de la valeur actuelle nette (VAN) stipule qu'il faut choisir le ou les projets ayant la VAN la plus élevée. En revanche, le critère du taux de rendement interne (TRI) n'est pas un bon critère pour faire un choix parmi des projets mutuellement exclusifs.

\section{TRI et projets mutuellement exclusifs}

Le Taux de Rentabilité Interne (TRI) n'est pas sensible à l'échelle du projet, tandis que la Valeur Actuelle Nette (VAN) d'un projet (si elle est positive) doublera si l'échelle du projet double. Par exemple, pour le projet \( P_1 \) : coût = \( C \), rentes perpétuelles (flux = \( J > 0 \)) et coût du capital \( r \). Pour le projet \( P_2 \) : on multiplie l'échelle du projet \( P_1 \) par \( N \).

\[ VAN(P_1)=-C+\frac{J}{r}\Leftrightarrow TRI_1=\frac{J}{C} \]
\[ VAN(P_2)=N\left( -C+\frac{J}{r}\right) = NVAN(P_1)\Leftrightarrow TRI_2=\frac{J}{C} \]

Le Taux de Rentabilité Interne (TRI) peut également conduire à choisir un projet au détriment d'un autre de même Valeur Actuelle Nette (VAN) seulement parce que le calendrier des deux projets est différent.

Appliquer le Taux de Rentabilité Interne (TRI) à la comparaison de projets mutuellement exclusifs implique le calcul du TRI différentiel. 

Principe : Soit deux projets mutuellement exclusifs \( P_1 \) et \( P_2 \) ; on soustrait les flux du projet \( P_2 \) à ceux du projet \( P_1 \) pour calculer le TRI associé à la différence des flux, appelé TRI différentiel. Si celui-ci est supérieur au coût du capital, il faut choisir le projet \( P_1 \) ; sinon, il faut retenir le projet \( P_2 \). 

Cf. Exemple en utilisant le logiciel Python. Attention : cette méthode n'est valable que si les projets ont le même coût du capital.

\section{Choix d'investissement et contrainte sur les ressources}

Jusqu'à présent, nous avons abordé la notion de projets mutuellement exclusifs, qui se caractérisent par le fait que les deux projets nécessitent des investissements identiques et mobilisent les mêmes ressources. Cependant, il est également possible qu'une entreprise se trouve confrontée à des limitations concernant certaines ressources spécifiques. Par exemple, elle peut disposer d'un nombre limité d'employés, d'une taille restreinte pour son entrepôt de stockage, d'un nombre fixe de machines, ou encore faire face à des contraintes financières. Dans de telles situations, le critère de la Valeur Actuelle Nette (VAN) maximale ne constitue pas nécessairement le bon indicateur pour guider la prise de décision.

Un critère souvent utilisé en pratique pour évaluer des projets est l'indice de profitabilité. La méthode consiste d'abord à identifier la "ressource rare" sur laquelle pèse la contrainte, puis à calculer la Valeur Actuelle Nette (VAN) ainsi que les quantités consommées de cette ressource pour chacun des projets envisagés.

Le calcul de l'indice de profitabilité du projet \( P \) se formule comme suit :

\[
I P(P) = \frac{VAN(P)}{\text{Ressources consommées}}
\]

Il est recommandé d'accepter les projets pour lesquels l'indice de profitabilité est le plus élevé, et ce, jusqu'à épuisement de la "ressource rare".

\subsection{Exemple}
Considérons une contrainte de financement fixée à 10 millions d'euros d'investissement. Cette limitation budgétaire doit être prise en compte lors de l'évaluation des projets, car elle influence directement la capacité de l'entreprise à financer les initiatives envisagées.

\begin{center}
	\begin{tabular}{@{}cccc@{}}
	\toprule
	Projet & VAN & Investissement nécessaire & IP   \\ \midrule
	A      & 30  & 8                         & 3,75 \\
	B      & 20  & 4                         & 5    \\
	C      & 20  & 5                         & 4    \\
	D      & 1   & 1                         & 1    \\ \bottomrule
\end{tabular}

\end{center}
\begin{center}
	
\begin{tabular}{@{}ccc@{}}
	\toprule
	Projet & IP   & Investissement nécessaire cumulé \\ \midrule
	B      & 5    & 4                                \\
	C      & 4    & 9                                \\
	A      & 3,75 & 17                               \\
	D      & 1    & 18                               \\ \bottomrule
\end{tabular}

\end{center}

Dans cette situation, seuls les projets \( B \) et \( C \) sont entrepris, tandis que le projet ayant la Valeur Actuelle Nette (VAN) maximale, désigné comme projet \( A \), n'est pas retenu. 

Cependant, certaines limites doivent être prises en considération. Par exemple, le projet \( D \) aurait dû être entrepris, car cela aurait permis de saturer la contrainte de ressources disponible. De plus, il est important de noter que l'indice de profitabilité ne peut pas être utilisé de manière efficace lorsqu'il existe plusieurs contraintes à gérer simultanément.

\chapter{Évaluation des actifs d'une entreprise}

Jusqu'à présent, notre attention s'est principalement portée sur l'évaluation des projets d'investissement. Cependant, il est essentiel de reconnaître qu'une entreprise est souvent porteuse de nombreux projets. Dès lors, quel lien existe-t-il entre la valeur d'une entreprise et la Valeur Actuelle Nette (VAN) des projets qu'elle détient ?

La valeur d'une entreprise peut être définie comme la valeur de marché de son actif économique, c'est-à-dire le montant qu'un investisseur devrait débourser pour acquérir l'entreprise tout en remboursant intégralement ses dettes. Cette valeur se compose de deux éléments principaux : d'une part, la valeur de marché des capitaux propres, souvent représentée par la capitalisation boursière, et, d'autre part, la dette nette, qui est calculée comme la différence entre la dette totale et la trésorerie de l'entreprise.

Le principe de base de l'évaluation d'un actif repose sur l'absence d'opportunités d'arbitrage. En d'autres termes, le prix d'un actif financier est égal à la valeur actualisée des flux futurs auxquels il donne droit.

Avec ce principe en tête, on peut suivre trois étapes clés pour évaluer un actif :

\begin{enumerate}
	\item Prix d'une action : Déterminer le prix actuel d'une action sur le marché.
	\item Évaluation de la capitalisation boursière : Calculer la capitalisation boursière de l'entreprise, qui est le produit du prix de l'action par le nombre total d'actions en circulation.
	\item Valeur de marché de l'actif économique d'une entreprise : Évaluer la valeur totale de l'actif économique en tenant compte des capitaux propres et de la dette nette.
\end{enumerate}

Ces étapes permettent d'obtenir une évaluation précise et cohérente de la valeur d'une entreprise.

\section{Prix d'une action}

Considérons le placement à un an d'un investisseur. 

\begin{itemize}
	\item À \( t = 0 \) : l'investisseur effectue l'achat de l'action au prix \( P_0 \).
	\item À \( t = 1 \) : l'investisseur revend l'action au prix \( P_1 \) et reçoit également le versement d'un dividende à la fin de la période, noté \( Div_1 \).
\end{itemize}

\[ \text{Prix d\'\,équilibre de l\'\,action : }P_0 =\frac{Div_1 + P_1}{1+r_{cp}} \]

Le taux d'actualisation est défini comme le coût des capitaux propres, noté \( r_{cp} \). Ce coût représente la rentabilité espérée des placements alternatifs présentant le même niveau de risque que les actions de l'entreprise. Il est important de noter que les flux futurs de revenus ne sont pas connus avec certitude, ce qui rend impossible leur actualisation au taux d'intérêt sans risque.

Remarque : La rentabilité de l'action d'une entreprise est équivalente au coût des capitaux propres de cette même entreprise.

\[ r_{cp}=\underbrace{\underbrace{\frac{Div_1}{P_0}}_{\text{Redement}}+\underbrace{\frac{P_1-P_0}{P_0}}_{\text{Taux de plus-value}}}_{\text{Rentabilité}} \]

Placement à deux  ans :

\[ P_0=\frac{Div_1}{1 + r_{cp}}+\frac{Div_2+P_2}{(1 + r_{cp})^2} \]

Remarque: Ce résultat peut être retrouvé en réutilisant notre formule du placement à un an, soit :

\[
P_0 = \frac{Div_1 + P_1}{1 + r_{cp}} \hspace{0.5cm} \text{et}\hspace{0.5cm} P_1 = \frac{Div_2 + P_2}{1 + r_{cp}} 
\]

Conséquence : Le prix d'une action ne dépend pas explicitement de l'horizon du placement \( N \). Le gain obtenu sous forme de dividendes ou de plus-value est indifférent pour estimer la valeur d'une action.

Pour un placement à \( N \) années, la formule se présente comme suit :

\[
P_0 = \frac{Div_1}{1 + r_{cp}} + \frac{Div_2}{(1 + r_{cp})^2} + \ldots + \frac{Div_N + P_N}{(1 + r_{cp})^N}.
\]

Hypothèse : Tous les investisseurs partagent les mêmes anticipations (rationnelles) concernant les dividendes futurs. 

Ainsi, le prix d'une action est égal à la valeur actuelle de tous les dividendes futurs :

\[
P_0 = \sum_{n=1}^{N} \frac{Div_n}{(1 + r_{cp})^n}+\frac{P_n}{(1 + r_{cp})^n}
\]

Si l'on suppose que les dividendes croissent à un taux constant \( g \) et que l'action est détenue indéfiniment (rente perpétuelle croissante), on obtient la formule suivante :

\[
P_0 = \frac{Div_1}{r_{cp} - g},
\]

ce qui correspond au modèle de Gordon-Shapiro (1956).

\section{Évaluation de la capitalisation boursière d'une entreprise}

1ère approche : La capitalisation boursière est donnée par la formule suivante :

\[
\text{Capitalisation boursière} = P_0 \times N_0,
\]

où \( N_0 \) est le nombre d'actions en circulation à la date \( t = 0 \).

2ème approche : La valeur de marché des capitaux propres d'une entreprise est égale à la valeur actuelle des flux futurs que recevront les actionnaires.

Il existe deux façons alternatives de rémunérer les actionnaires :

\begin{enumerate}
	\item Verser des dividendes.
	\item Racheter des actions, ce qui diminue les dividendes tout en réduisant le nombre de titres en circulation, entraînant ainsi une augmentation du dividende par action
\end{enumerate}.

La capitalisation boursière peut être exprimée comme suit :

\[
\text{Capitalisation boursière} = VA(\text{Dividendes} + \text{rachats d'actions}).
\]

À partir de cette première approche, on peut retrouver le prix d'une action de l'entreprise :

\[
P_0 = \frac{VA(\text{Dividendes} + \text{rachats d'actions})}{N_0}.
\]

\section{Évaluation de l'actif économique d'une entreprise}

La valeur de marché d'une entreprise, notée \( V_0 \), est égale à la valeur actualisée des flux de trésorerie disponibles qui peuvent être versés à l'ensemble des investisseurs (actionnaires et créanciers) :

\[
V_0 = VA(\text{Flux de trésorerie disponibles}).
\]

Remarque : En notant \( D_0 \) la valeur de la dette actualisée à la date \( t = 0 \), on peut réécrire la valeur d'une action de l'entreprise de la façon suivante :

\[
P_0 = \frac{V_0 - D_0 + \text{Trésorerie}}{N_0}.
\]

Question : Dans le calcul de \( V_0 \), à quel taux doivent être actualisés les Flux de Trésorerie Disponibles (\( FTD \)) ?

Les deux types d'investisseurs (créanciers et actionnaires) ne prennent pas les mêmes risques, ce qui signifie que le taux d'actualisation doit tenir compte de cette différence. On utilise donc le Coût Moyen Pondéré du Capital, noté \( r_{cmpc} \), défini plus précisément dans la suite du cours.

La formule est la suivante :

\[
V_0 = \frac{FTD_1}{(1 + r_{cmpc})} + \frac{FTD_2}{(1 + r_{cmpc})^2} + \ldots + \frac{FTD_N + V_N}{(1 + r_{cmpc})^N},
\]

où \( V_N \) représente la valeur terminale (ou valeur de continuation) de l'entreprise.

Pour estimer \( V_N \), on fait généralement l'hypothèse d'un taux de croissance des Flux de Trésorerie Disponibles (\( FTD \)) constant à long terme, noté \( g_{FTD} \) (comme dans le cas des rentes perpétuelles croissantes).

Dans ce cas, on a :
\[
V_N = \frac{FTD_{N+1}}{r_{cmpc} - g_{FTD}}=\left(\frac{1+g_{FTD}}{r_{cmpc} - g_{FTD}} \right) FTD_N
\]

\section{Valeur de l'entreprise, prix des actions et choix d'investissement}

Les investissements présents et futurs génèrent des Flux de Trésorerie Disponibles (FTD) futurs.

La valeur de marché de l'entreprise, notée \( V_0 \), est égale à la valeur actualisée totale que l'entreprise obtiendra grâce à ses projets actuels et futurs.

La valeur actualisée (VA) d'un projet particulier représente la contribution de ce projet à la valeur de l'entreprise.

Il est conseillé de retenir tous les projets ayant une Valeur Actualisée Nette (VAN) positive (sous contrainte financière), ce qui permet de maximiser le prix de l'action de l'entreprise.

\part{La structure financière de l'entreprise}

\chapter{Structure financière dans le cadre de marchés parfaits}

Jusqu'à présent, nous avons exploré la question cruciale : "\textit{Quel projet d'investissement choisir ?}" Cette interrogation est fondamentale pour toute entreprise cherchant à optimiser ses ressources et à maximiser sa valeur.

Dans cette partie, nous nous pencherons sur une autre question tout aussi importante : "\textit{Comment financer le projet d'investissement choisi ?}" Il est pertinent de se demander si ces deux questions sont indépendantes. En effet, le mode de financement peut avoir des répercussions significatives sur la Valeur Actualisée Nette (VAN) d'un projet. Ainsi, nous chercherons à répondre à ces questions sous deux hypothèses :

\begin{itemize}
	\item Sous l'hypothèse de marchés parfaits : Dans ce cadre théorique, nous supposons qu'il n'existe ni impôts ni coûts de transaction, ce qui simplifie considérablement l'analyse.

	\item Sous l’hypothèse d’imperfections de marché : Ici, nous tiendrons compte des réalités économiques, telles que les taxes et les coûts associés aux transactions financières.
\end{itemize}

Pour une entreprise qui ne dispose pas de fonds propres, il existe principalement deux modes de financement à considérer :

\begin{itemize}
	\item L'emprunt : Ce mode de financement implique que l'investissement soit financé par la dette. Les entreprises peuvent recourir à différents types de prêts, qui doivent être remboursés selon des modalités définies, généralement avec des intérêts.
	\item L’émission d'actions : Dans ce cas, l'investissement est financé par les capitaux propres. Cela signifie que l'entreprise émet de nouvelles actions pour lever des fonds, ce qui dilue la propriété existante mais peut également apporter des ressources financières sans obligation de remboursement.
\end{itemize}

La composition du passif de l'entreprise, c'est-à-dire la proportion relative de dette et de capitaux propres, est connue sous le nom de structure financière de l'entreprise. Cette structure est cruciale, car elle influence non seulement la stabilité financière de l'entreprise, mais aussi sa capacité à réaliser des investissements futurs.
\noindent
Il existe des différences essentielles entre la dette et les actions :

\begin{itemize}
	\item Priorité de remboursement : Le remboursement de la dette a la priorité sur les paiements aux détenteurs d'actions. Cela signifie que, en cas de liquidation, les créanciers sont remboursés avant que les actionnaires ne reçoivent quoi que ce soit. Les détenteurs de capitaux propres sont donc considérés comme des ayant droits résiduels (\textit{residual claimants}).
	\item Risque associé : En général, la dette est perçue comme un actif moins risqué que les actions. Cela est dû au fait que les paiements d'intérêts sur la dette sont souvent fixes et que les créanciers ont une position privilégiée en cas de difficultés financières.
\end{itemize}

En ce qui concerne les rendements, il existe une relation inverse entre le risque et le rendement potentiel. Plus un investisseur prend de risques en investissant dans des actions, plus le rendement potentiel est élevé. En revanche, les investissements en dette, tels que les obligations, offrent généralement un rendement plus faible, mais aussi un risque plus faible. 

\begin{wrapfigure}{r}{0.6\textwidth}
	\centering
	\includegraphics{../../../Pictures/Screenshots/Capture d'écran 2025-02-15 222044}
\end{wrapfigure}

Il est important de noter qu'au-delà d'un certain seuil de risque, désigné par \( D \), le rendement des investissements en dette peut commencer à augmenter. Cependant, cela est souvent associé à des investissements plus risqués, ce qui nécessite une évaluation minutieuse des opportunités et des risques associés.

Ainsi, le choix du mode de financement ne doit pas être pris à la légère, car il peut influencer de manière significative la viabilité et la rentabilité des projets d'investissement.

\section{L'hypothèse de marchés parfaits}

Dans un marché parfait, il existe une absence d'opportunité d'arbitrage. Cela signifie que les prix des actifs reflètent intégralement leur valeur actualisée, qui est calculée à partir des flux monétaires futurs qu'ils génèrent. Ainsi, le prix d'un actif est égal à la valeur actualisée des flux monétaires offerts par cet actif.

La valeur boursière d'une entreprise non endettée est alors équivalente à la somme des valeurs actualisées des différents projets qu'elle porte. Cela implique que chaque projet contribue de manière additive à la valeur totale de l'entreprise.

Considérons un projet spécifique, qui présente les caractéristiques suivantes :

\begin{itemize}
	\item À \( t = 0 \) : le coût initial de l'investissement est noté \( C \).
	\item À \( t = 1 \) : le flux de trésorerie généré par le projet est une variable aléatoire \( f \) qui peut prendre deux valeurs :
\begin{itemize}
	\item 	\( f_0 > 0 \) avec une probabilité \( \theta \),
	\item 	\( f_1 > f_0 \) avec une probabilité \( 1 - \theta \), en fonction de la conjoncture économique.
\end{itemize}
	
\end{itemize}




La Valeur Actualisée Nette (VAN) du projet peut être exprimée par la formule suivante :

\[
\text{VAN} = -C + \frac{\mathbb{E}(f)}{(1 + r_{cp})} = -C + \frac{(\theta f_0 + (1 - \theta) f_1)}{(1 + r_{cp})}
\]

où \( \mathbb{E}(f) \) représente l'espérance mathématique des flux futurs.

Le coût du capital, noté \( r_{cp} \), est défini comme suit :

\[
r_{cp} = r_f + r_p
\]

où \( r_f \) est le taux d'intérêt sans risque et \( r_p \) est la prime de risque de marché. Ce coût du capital est essentiel pour évaluer la rentabilité des projets d'investissement.

Dans le cas où l'entrepreneur opte pour un financement total par capitaux propres, il doit faire face à une situation où il ne dispose pas de fonds propres. Par conséquent, pour financer les coûts \( C \) associés à son projet, il doit émettre des actions. Cette décision de financement a des implications sur la structure du capital de l'entreprise et sur sa capacité à attirer des investisseurs.

Dans un contexte de marchés parfaits, caractérisés par l'absence d'opportunités d'arbitrage, le prix de "l'actif entreprise", noté \( p_A \), est directement lié à l'investissement initial des actionnaires. Ce lien peut être exprimé par la relation suivante :

\[
p_A = VA = \frac{\mathbb{E}(f)}{(1 + r_{cp})}
\]

où \( VA \) représente la valeur actualisée des flux futurs attendus de l'entreprise.

Les profits de l'entreprise peuvent être calculés en soustrayant le coût initial de l'investissement \( C \) du prix de l'actif :

\[
\text{Profits} = p_A - C = \text{VAN}
\]

Cela indique que la Valeur Actualisée Nette (VAN) du projet revient aux propriétaires de l'entreprise, qu'il s'agisse des actionnaires ou de l'entrepreneur lui-même. En d'autres termes, les bénéfices générés par le projet sont directement redistribués aux investisseurs.

Pour évaluer l'espérance de rentabilité de l'actif, nous pouvons utiliser la formule suivante :

\[
\frac{\theta (f_0 - p_A)}{p_A} + \frac{(1 - \theta)(f_1 - p_A)}{p_A} = \frac{\mathbb{E}(f)}{p_A} - 1 = r_{cp}
\]

Cette équation montre que l'espérance de rentabilité de l'actif est égale au coût du capital de l'entreprise. En effet, les actionnaires sont rémunérés selon le coût du capital, ce qui reflète les risques qu'ils prennent en investissant dans l'entreprise.

Ainsi, nous pouvons conclure que la boucle est bouclée : si les actionnaires ne reçoivent pas une rémunération équivalente au coût du capital, cela créerait une opportunité d'arbitrage. Dans un marché parfait, les investisseurs chercheraient à profiter de cette opportunité, ce qui conduirait à un ajustement des prix jusqu'à ce que l'équilibre soit rétabli.

\section{Financement exclusif par capitaux propres : un exemple}

Reprenons l'exemple précédent avec les données suivantes :

\begin{itemize}
	\item Coût initial du projet : \( C = 800 \)
	\item Probabilité : \( \theta = \frac{1}{2} \)
	\item Flux \( f_0 = 900 \)
	\item Flux \( f_1 = 1400 \)
	\item Taux d'intérêt sans risque : \( r_f = 5\% \)
	\item Prime de risque : \( r_p = 10\% \)
\end{itemize}

\subsubsection{Calcul de la Valeur Actualisée Nette (VAN) du projet}

La VAN du projet est calculée comme suit :

\[
\text{VAN} = -C + \frac{\mathbb{E}(f)}{(1 + r_{cp})}
\]

où le coût du capital \( r_{cp} \) est donné par :

\[
r_{cp} = r_f + r_p = 5\% + 10\% = 15\%
\]

Pour calculer l'espérance des flux futurs \( \mathbb{E}(f) \) :

\[
\mathbb{E}(f) = \theta f_0 + (1 - \theta) f_1 = \frac{1}{2} \cdot 900 + \frac{1}{2} \cdot 1400 = 450 + 700 = 1150
\]

Ainsi, la VAN devient :

\[
\text{VAN} = -800 + \frac{1150}{1,15} \approx -800 + 1000 = 200
\]

\subsubsection{Valeur Actualisée des Capitaux Propres}

La valeur actualisée des capitaux propres levés par l'entrepreneur est :

\[
VA = \frac{\mathbb{E}(f)}{(1 + r_{cp})} = \frac{1150}{1,15} \approx 1000
\]

\subsubsection{Profit Actualisé de l'Entreprise}

En vendant toutes les actions de son entreprise, l'entrepreneur reçoit :

\[
\text{Profit (actualisé)} = VA - C = 1000 - 800 = 200
\]

Ce montant correspond à la VAN du projet :

\[
\text{VAN} = 200
\]

La valeur actualisée des capitaux propres est donc la somme des investissements initiaux et des profits futurs (actualisés). Cela illustre le lien entre le coût du capital, la VAN et la valeur des capitaux propres dans un cadre de marché parfait.

Quelle est la rentabilité espérée pour les actionnaires :

\begin{center}
\begin{tabular}{@{}ccccc@{}}
	\toprule
	$t=0$           & \multicolumn{2}{c}{$t=1-\text{flux}$} & \multicolumn{2}{c}{$t=1-\text{rentabilité}$} \\ \midrule
	Valeur initiale & Croissance         & Récession        & Croissance            & Récession            \\
	1000            & 1400               & 900              & 40 \%                 & -10\%                \\ \bottomrule
\end{tabular}
\end{center}

La rentabilité espérée de l'investissement est calculée comme suit :

\[
\text{Rentabilité espérée} = \theta \cdot (f_0 - p_A) + (1 - \theta) \cdot (f_1 - p_A)
\]

En utilisant les données fournies :

\begin{itemize}
	\item \( \theta = \frac{1}{2} \)
	\item \( f_0 = 900 \)
	\item \( f_1 = 1400 \)
	\item \( p_A = 1000 \)
\end{itemize}

Nous avons :

\[
\text{Rentabilité espérée} = \frac{1}{2} \cdot (900 - 1000) + \frac{1}{2} \cdot (1400 - 1000) = 150
\]

Pour obtenir le taux de rentabilité, nous divisons par le prix de l'actif :

\[
\text{Taux de rentabilité} = \frac{150}{1000} = 0,15 \quad \text{ou} \quad 15\%
\]

\subsubsection{Coût du Capital de l'entreprise}
\noindent
Ce taux de rentabilité correspond au coût du capital de l'entreprise. Cela signifie que :

\begin{itemize}
	\item Les actionnaires sont rémunérés à hauteur du risque qu'ils prennent.
	\item Le risque de l'entreprise non endettée correspond au risque du projet.
\end{itemize}

Ainsi, dans un cadre de marché parfait, la rentabilité espérée des actionnaires est alignée avec le coût du capital, illustrant l'importance de la gestion des risques dans les décisions d'investissement.

\section{Financement par capitaux propres et endettement : un exemple}

Supposons que l'entrepreneur finance une partie de l'investissement initial par endettement. 

\subsubsection{Détails de l'emprunt}

Montant de l'emprunt : 

\begin{itemize}
	\item L'entrepreneur emprunte 500 euros.
\end{itemize}

Caractéristiques de la Dette : 

\begin{itemize}
	\item La dette est considérée comme sans risque, ce qui signifie qu'elle devra toujours être remboursée.
	\item Taux d'actualisation : \( r_f = 5\% \).
\end{itemize}

Dans un an, l'entreprise devra rembourser un montant total de :
\[
\text{Montant total remboursé} = D \cdot (1 + r_f) = 500 \cdot (1 + 0,05) = 525 \, \text{€}
\]

Quelle est à présent la valeur des capitaux propres ?

\begin{center}
	\begin{tabular}{@{}cccc@{}}
		\toprule
		& $t=0$           & \multicolumn{2}{c}{$t=1-\text{flux}$} \\ \midrule
		& Valeur initiale & Croissance         & Recession        \\ \midrule
		Créanciers   & 500             & 525                & 525              \\
		Actionnaires & $V_{CP}$ = ?    & 875                & 375              \\
		Entreprise   & $V_{CP}+500$    & 1400               & 900              \\ \bottomrule
	\end{tabular}
\end{center}

Vision naïve:

\[ \frac{0,5\cdot(1400-525)+0,5\cdot(900-525)}{1,15}=\frac{0,5\cdot875+0,5\cdot375}{1,15} =543 \]

La valeur de l'entreprise endettée est calculée comme suit : \( 500 + 543 = 1043 \), ce qui est supérieur à \( 1000 \), la valeur de l'entreprise non endettée. Cependant, l'entreprise endettée porte exactement le même projet que l'entreprise non endettée. Par conséquent, il n'y a aucune raison que sa valeur initiale soit différente.


Nous devons avoir \( VA = V_{CP} + V_D = 1000 \), ce qui implique que \( V_{CP} = 500 \). Cependant, une erreur dans le raisonnement précédent est que le recours à l'endettement augmente le risque supporté par les actionnaires, ce qui signifie que le coût des capitaux propres doit évoluer en conséquence. Pour illustrer cela, calculons la prime de risque associée aux capitaux propres de l'entreprise endettée, afin de montrer le lien entre risque et rentabilité.

\subsubsection{Rentabilité des fonds propres selon la structure de financement :}

\begin{center}
	\begin{tabular}{@{}cccc@{}}
		\toprule
		& Croissance & Recession & $\mathbb{E}(R)$ \\ \midrule
		Dette                           & 5\%        & 5\%       & 5\%             \\
		Actions entreprise non-endettée & 40\%       & 10\%      & 15\%            \\
		Actions entreprise endettée     & 75\%       & -25\%     & 25\%            \\ \bottomrule
	\end{tabular}
\end{center}
\[ 75\% = \frac{875(375)-500}{500\cdot100} \]

\subsubsection{Prime de risque selon la structure de financement :}

\begin{center}
	\begin{tabular}{@{}ccc@{}}
	\toprule
	& Variation de la rentabilité & $r_p=\mathbb{E}(R)-r_f$ \\ \midrule
	Dette                   & 5\%-5\%=0\%                 & 5\%-5\%=0\%          \\
	Entreprise non-endettée & 40\%-(-10\%)=50\%           & 15\%-5\%=10\%        \\
	Entreprise endettée     & 75\% -(-25\%)=100\%         & 25\%-5\%=20\%        \\ \bottomrule
\end{tabular}
\end{center}

En raison de l'augmentation du risque qui pèse sur eux, les actionnaires exigent une rentabilité de 25\% pour financer le projet de l'entreprise endettée. La valeur actualisée des capitaux propres \( (V_{CP}) \) doit être calculée en prenant cette rentabilité comme coût du capital : 

\[
\frac{0,5 \times 875 + 0,5 \times 375}{1,25} = 500
\]

Ainsi, la valeur de l'entreprise reste inchangée à 1000, que l'entrepreneur ait recours à l'endettement ou qu'il se finance exclusivement par l'émission d'actions. Il est important de noter que l'endettement d'une entreprise augmente le risque des actions, même en supposant que le risque de faillite est nul.

\section{Propositions de Modigliani et Miller}

\subsection{Proposition 1}

Dans le cadre de marchés de capitaux parfaits, la valeur totale d'une entreprise est égale à la valeur de marché des flux de trésorerie de ses actifs, et cette valeur n'est pas influencée par la structure financière de l'entreprise. Ainsi, la structure financière est neutre et n'a pas d'effet sur la valeur de l'entreprise. En d'autres termes, la taille du gâteau ne dépend pas de la façon dont il est partagé. Par conséquent, un entrepreneur peut se concentrer sur le choix d'un projet sans avoir à se soucier de la manière dont il va le financer.

\subsection{Démonstration}

Considérons deux firmes (1 et 2) portant toutes les deux un projet dont les flux à la date \( t = 1 \) sont décrits par la même variable aléatoire \( X \). Notons la valeur de la dette de l'entreprise \( i \) en \( t = 0 \) par \( V_D^i \). L'entreprise 1 n'est pas endettée (\( V_D^1 = 0 \)), tandis que l'entreprise 2 est endettée (\( V_D^2 > 0 \)). Soit \( V_{CP}^1 \) et \( V_{CP}^2 \) la valeur actualisée (en \( t = 0 \)) des capitaux propres dans chacune des entreprises. Les valeurs actualisées (en \( t = 0 \)) de chacune des entreprises sont données par \( V^1 = V_{CP}^1 \) et \( V^2 = V_{CP}^2 + V_D^2 \). Selon le théorème de Modigliani-Miller, si les marchés financiers sont parfaits, alors nous avons \( V^1 = V^2 \).

Supposons que \( V^2 > V^1 \). La stratégie en \( t = 0 \) consiste à vendre des actions et des titres de dette de l'entreprise 2 pour acheter des actions de l'entreprise 1.

\begin{center}
	\begin{tabular}{@{}ccc@{}}
		\toprule
		& \multicolumn{2}{c}{Flux}                       \\ \midrule
		Transaction                  & $t=0$               & $t=1$                    \\
		Vente $\alpha$ de $V_D^2$    & $\alpha V_D^2$      & $-\alpha(1-r_f)V_D^2$    \\
		Vente $\alpha$ de $V_{CP}^2$ & $\alpha  V_{CP}^2$  & $-\alpha(X-(1+r_f)V_D^2)$ \\
		Achat $\alpha$ de $V_{CP}^1$ & $-\alpha V_{CP}^1$  & $\alpha X$               \\
		Somme                        & $\alpha(V^2-V^1)>0$ & 0                        \\ \bottomrule
	\end{tabular}
\end{center}

Cela crée une opportunité d'arbitrage, contrairement à l'hypothèse de départ. En suivant cette stratégie, l'investisseur réalise un profit sans risque à \( t = 0 \) (voir la somme des flux à \( t = 0 \)). Les conséquences sont significatives : si une telle opportunité d'arbitrage existe, les investisseurs vont massivement acheter les actions de l'entreprise 1 tout en vendant celles de l'entreprise 2. Cette demande accrue pour les actions de l'entreprise 1 fera augmenter leur prix, tandis que la vente massive des actions de l'entreprise 2 entraînera une baisse de leur prix. Ce processus se poursuivra jusqu'à ce que \( V^1 = V^2 \), éliminant ainsi l'opportunité d'arbitrage.

Supposons à présent que \( V^1 > V^2 \). La stratégie en \( t = 0 \) consiste à acheter des actions et des titres de dette de l'entreprise 2, financée par la vente d'actions de l'entreprise 1. 


\begin{center}
	\begin{tabular}{@{}ccc@{}}
		\toprule
		& \multicolumn{2}{c}{Flux}                        \\ \midrule
		Transaction                  & $t=0$               & $t=1$                     \\
		Achat $\alpha$ de $V_D^2$    & $-\alpha V_D^2$     & $\alpha(1+r_f)V_D^2$      \\
		Achat $\alpha$ de $V_{CP}^2$ & $-\alpha  V_{CP}^2$ & $-\alpha(X-(1+r_f)V_D^2)$ \\
		Vente $\alpha$ de $V_{CP}^1$ & $\alpha V_{CP}^1$   & $-\alpha X$               \\
		Somme                        & $\alpha(V^1-V^2)>0$ & 0                         \\ \bottomrule
	\end{tabular}
\end{center}
Cette approche crée également une opportunité d'arbitrage, car les investisseurs peuvent tirer profit de cette différence de valeur entre les deux entreprises

\section{Structure financière et effet de levier}

Selon le théorème de Modigliani-Miller, la valeur d'une entreprise est indépendante de sa structure financière, même si l'endettement augmente la rentabilité des capitaux propres. On dit que l'endettement exerce un effet de levier sur la rentabilité des capitaux propres. La question se pose alors : que se passe-t-il si des investisseurs préfèrent une structure financière différente de celle choisie par l'entreprise ? Par exemple, les investisseurs peuvent préférer une entreprise plus endettée, ce qui entraînerait un effet de levier plus fort.

\section{Le levier synthétique}

La réponse de Modigliani et Miller est que cela ne change rien. Les investisseurs peuvent, en ajustant leur portefeuille, répliquer la structure financière qui leur convient le mieux en créant une entreprise "\textit{fictive composite}". Ceux qui préfèrent une entreprise plus endettée peuvent s'endetter eux-mêmes, ce qui signifie qu'ils ont recours au levier synthétique, un parfait substitut au levier de l'entreprise endettée.

Reprenons l'exemple initial : \( C = 800 \), \( \theta = \frac{1}{2} \), \( f_0 = 900 \), \( f_1 = 1400 \), \( r_f = 5\% \), \( r_p = 10\% \), et considérons que l'entreprise est non endettée. Un investisseur ayant une préférence pour la structure financière de l'entreprise endettée peut emprunter 500 euros pour acheter des actions de l'entreprise non endettée. Il répliquera ainsi les flux de trésorerie de l'entreprise endettée, obtenant donc 875 euros (1400 - 525) en cas de croissance et 375 euros (900 - 525) en cas de récession. 

À l'inverse, un investisseur souhaitant supporter moins de risque que celui d'une entreprise endettée peut combiner l'achat d'actions avec la vente de titres de dette de l'entreprise endettée, ce qui constitue un levier synthétique négatif.

\section{Endettement et coût des capitaux propres}

Quel facteur d'actualisation choisir pour une entreprise endettée ? Le levier d'entreprise est équivalent au levier synthétique. Quelle que soit sa structure financière, une entreprise peut être analysée à partir d'une situation d'endettement nul. 

Pour répondre à cette question, considérons deux entreprises : E, qui est endettée, et N, qui ne l'est pas, portant un projet de même risque et de même rentabilité. On peut exprimer la rentabilité des actions de E en fonction de la rentabilité des actions de N. 

Notations : \( V_N \) est la valeur de marché de l'entreprise N, \( V_E \) est la valeur de marché de l'entreprise E, et \( V_{CP} \) et \( V_D \) sont les valeurs de marché respectives des capitaux propres et de la dette de l'entreprise E. 

Selon Modigliani-Miller, \( V_E = V_N \) implique que \( V_{CP} + V_D = V_N \). Ainsi, en détenant un portefeuille composé de dettes et de capitaux de l'entreprise E, un investisseur peut répliquer les flux de trésorerie dont bénéficie l'actionnaire de l'entreprise N. 

La rentabilité d'un portefeuille est donc la moyenne pondérée des rentabilités des titres qui le composent, avec comme pondération la part de chaque actif dans la valeur totale du portefeuille.

On note \( R_{CP} \), \( R_N \) et \( R_D \) les rentabilités respectives des capitaux propres de l'entreprise E, de l'entreprise N et de la dette. La relation suivante s'exprime comme suit :

\[
R_N = \left( \frac{V_{CP}}{V_{CP} + V_D} \right) R_{CP} + \left( \frac{V_D}{V_{CP} + V_D} \right) R_D
\]

Les deux firmes ont la même rentabilité en raison des possibilités d'arbitrage illimitées, ce qui donne :

\[
R_{CP} = R_N + \frac{V_D}{V_{CP}} (R_N - R_D)
\]

Le premier terme représente le rendement (risqué) de l'actif de l'entreprise N, tandis que le second terme indique le rendement (risqué) additionnel dû au levier.

\subsubsection{Conséquences de l'endettement sur la rentabilité des capitaux propres :}
La rentabilité des actions de l'entreprise E est égale à la rentabilité de l'entreprise N plus une prime liée au risque provoqué par l'endettement. Cette prime est positive si \( R_N > R_D \) (effet levier) et négative en général si \( R_N < R_D \) (effet massue). Le niveau du risque supplémentaire, et donc de la prime, dépend du niveau d'endettement, mesuré par le levier en valeur de marché \( \frac{V_D}{V_{CP}} \).

\subsection{Proposition 2}

Le coût des capitaux propres d'une entreprise endettée est égal au coût des capitaux propres d'une entreprise non endettée, plus une prime de risque proportionnelle au levier en valeur de marché de l'entreprise :

\[
r_{CP} = r_N + \frac{V_D}{V_{CP}} (r_N - r_D)
\]

En appliquant ce résultat à l'exemple précédent, on obtient :

\[
r_{CP} = 15\% + \frac{500}{500}(15\% - 5\%) = 25\%
\]

Si l'entreprise endettée l'était de seulement 200 euros, le rendement de ses capitaux propres serait :

\[
r_{CP} = 15\% + \frac{200}{800}(15\% - 5\%) = 17,5\%
\]

\section{Le coût moyen pondéré du capital et choix d'investissement}

\subsubsection{Choix d'investissement : actualisation des flux associés au projet}

Le taux d'actualisation, qui correspond au coût du capital, représente la rentabilité que l'entreprise doit offrir aux investisseurs pour les rémunérer du risque qu'ils prennent. Dans une entreprise endettée, il existe deux types d'investisseurs (créanciers et actionnaires) qui ne prennent pas le même risque. Le taux d'actualisation est donc le coût moyen pondéré du capital (CMCP), noté \( r_{CMCP} \), qui est la rentabilité moyenne espérée que l'entreprise doit offrir à ses investisseurs. Il se calcule comme suit :

\[
r_{CMCP} = \frac{V_{CP}}{V_{CP} + V_D} r_{CP} + \frac{V_D}{V_{CP} + V_D} r_D
\]

Selon la Proposition 2 de Modigliani-Miller, on a \( r_{CMCP} = r_N \). En conclusion, le taux d'actualisation à utiliser dépend seulement du projet entrepris et non de la structure financière de l'entreprise qui le porte, ce qui est une extension de la Proposition 1 de Modigliani-Miller.

\section{Endettement et Bêta}

Le risque d'une action d'une entreprise non endettée est équivalent au risque d'un portefeuille composé de titres de dette et d'actions d’une entreprise endettée. On observe une relation similaire pour le risque, mesuré par le bêta, que pour celui mesuré par les rendements :

\[
\beta_N = \frac{V_{CP}}{V_{CP} + V_D} \beta_{CP} + \frac{V_D}{V_{CP} + V_D} \beta_D
\]

Ici, \( \beta_N \) représente le bêta à endettement nul (ou bêta désendetté) de l'entreprise, qui mesure son risque de marché sans tenir compte de l'effet de sa dette. Lorsque la dette est sans risque (\( \beta_D = 0 \)), on obtient la relation suivante :

\[
\beta_{CP} = \left(1 + \frac{V_D}{V_{CP}}\right) \beta_N
\]

Le levier amplifie le risque de marché des actifs de l'entreprise E, et par conséquent, celui de ses actions. Cela permet de comprendre pourquoi des entreprises d'un même secteur présentent des bêtas différents, en raison de l'effet de leur structure financière.

\section{Qu'en est-il empriquement ?}

Le résultat de neutralité de la structure financière vis-à-vis de la valeur d’une firme est contredit par plusieurs observations. Les directeurs financiers consacrent du temps et des ressources à l'analyse et à l'optimisation de la structure financière. De plus, des différences systématiques de structure financière existent selon les secteurs d'activité. On constate également un lien entre la structure financière et la performance économique. Cela soulève un puzzle : l'hypothèse de marché parfait est-elle réaliste ?

\chapter{Structure financière en présence d'impôts}

\section{La déduction fiscale des intérêts}

\subsection{France}

En France, les entreprises sont assujetties à l'impôt sur les sociétés (IS). Les taux d'imposition ont évolué comme suit :
\begin{itemize}
	\item Taux historique : 33\% pour l'ensemble des bénéfices.
	\item Taux réduit : 28\% pour les profits inférieurs à 500 000 euros.
\end{itemize}


Baisse progressive** :
- 28\% sur l'ensemble des profits en 2020.
- 26,5\% en 2021.
- 25\% en 2022.
- Taux de 15\% pour les petites entreprises avec un résultat imposable (RE) inférieur à 38 000 euros.
- **Pour l'année 2024** : Le taux normal de l'IS est fixé à 25\% pour la majorité des entreprises.

2. **États-Unis**  
Aux États-Unis, le système d'imposition des sociétés est structuré comme suit :
- **Taux standard** : 40\% pour les firmes dont le profit dépasse 18,3 millions de dollars.
- **Taux réduit** : 15\% pour les entreprises dont le profit est inférieur à ce seuil.

3. **Allemagne**  
En Allemagne, le taux d'IS est :
- **Taux d'imposition** : 15\% sur les bénéfices des entreprises.

4. **Royaume-Uni**  
Au Royaume-Uni, le taux d'IS varie :
- **Taux standard** : 28\%.
- **Taux dégressif** : Pour les entreprises dont le bénéfice est inférieur à 1,5 million de livres, le taux diminue pour atteindre 21\% pour les profits inférieurs à 300 000 livres.

5. **Assiette de l'IS**  
L'assiette de l'impôt sur les sociétés correspond au résultat courant annuel de l'entreprise, soit les bénéfices après le paiement des intérêts d'emprunt.

6. **Impact de l'endettement**  
L'endettement influence l'assiette imposable de plusieurs façons :
- **Baisse de l'assiette imposable** : L'augmentation des intérêts d'emprunt réduit le bénéfice imposable, diminuant ainsi le montant de l'impôt dû.
- **Incitation à s'endetter** : Les entreprises peuvent être incitées à s'endetter pour minimiser leur charge fiscale, toutes choses égales par ailleurs.

7. **Dispositions fiscales pour 2024**  
- **Taux réduit pour les PME** : Les petites et moyennes entreprises (PME) peuvent bénéficier de taux réduits sous certaines conditions.
- **Autres dispositifs fiscaux** : Des crédits d'impôt et des abattements spécifiques à certains secteurs d'activité peuvent également impacter le montant de l'IS à payer.

Cette analyse met en évidence les différences de taux d'imposition des sociétés entre plusieurs pays, ainsi que l'importance de l'endettement et des dispositifs fiscaux dans la détermination de la charge fiscale des entreprises.




\end{document}
	
