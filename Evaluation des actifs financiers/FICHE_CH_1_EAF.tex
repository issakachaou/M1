\documentclass[a4paper, 12pt]{report}
\usepackage{graphicx}
\usepackage[utf8]{inputenc} 
\usepackage[french]{babel}
\usepackage[T1]{fontenc}
\usepackage{fancyhdr}
\usepackage{amsmath,amsfonts,amssymb, empheq}
\usepackage{eurosym}
\usepackage{booktabs}
\usepackage{wrapfig}
\pagestyle{fancy}
\fancyhead[R]{Université Paris-Est Créteil}
\fancyhead[L]{Évaluation des actifs financiers}
\usepackage{array,multirow,makecell}
\setcellgapes{1pt}
\usepackage{mathptmx} %times aves le mode math
\usepackage{tikz}
\usepackage{hyperref}
\usepackage{import}
\makegapedcells
\newcolumntype{R}[1]{>{\raggedleft\arraybackslash }b{#1}}
\newcolumntype{L}[1]{>{\raggedright\arraybackslash }b{#1}}
\newcolumntype{C}[1]{>{\centering\arraybackslash }b{#1}} 

\begin{document}
	
	\chapter{Choix optimal de portefeuille et modèle d’évaluation des actifs financiers}
	
\section{Le rendement attendu d’un portefeuille}

\subsection{Définir un portefeuille}
Un portefeuille est défini par ses pondérations $\omega_i$, qui indiquent la proportion de l’investissement total allouée à chaque actif :
\[
\omega_i = \frac{\text{Valeur de l'investissement } i}{\text{Valeur totale du portefeuille}}
\]
La somme des pondérations doit être égale à 1 : $\sum_{i=1}^N \omega_i = 1$.

\subsection{Rendement du portefeuille}
Le rendement d’un portefeuille est la moyenne pondérée des rendements des actifs qui le composent :
\[
r_p = \sum_{i=1}^N \omega_i \cdot r_i
\]

\paragraph{Exemple.} 
Un portefeuille comprend 200 actions Dolby à 30 \$ (pondération : $60\%$) et 100 actions Coca-Cola à 40 \$ (pondération : $40\%$). Si les prix montent respectivement à 36 \$ et 38 \$, la nouvelle valeur totale est $11\,000\$ $ avec un rendement global de $ 10\% $ 

\subsection{Rendement attendu}
Le rendement attendu d’un portefeuille est calculé comme la moyenne pondérée des rendements attendus des actifs :
\[
E(r_p) = \sum_{i=1}^N \omega_i \cdot E(r_i)
\]

\section{La volatilité et le risque}

Une position longue (\textit{long}) correspond à l'achat d'un actif dans l'anticipation que son prix augmentera, offrant ainsi un potentiel de gain illimité avec une perte maximale limitée au prix payé pour l'actif. À l'inverse, une position courte (\textit{short}) implique la vente à découvert d'un actif que l'investisseur ne possède pas, avec l'obligation de le racheter ultérieurement. Cette stratégie est utilisée lorsque l'on anticipe une baisse du prix de l'actif.

\subsection{Risque et diversification}
En combinant des actifs dans un portefeuille, il est possible de réduire le risque grâce à la diversification. Le risque total dépend de la corrélation entre les actifs.

\subsection{Variance d’un portefeuille à deux actifs}
Pour deux actifs, la variance du portefeuille est donnée par :
\[
\sigma_p^2 = \omega_1^2 \sigma_1^2 + \omega_2^2 \sigma_2^2 + 2 \omega_1 \omega_2 \rho_{1,2} \sigma_1 \sigma_2
\]
où $\rho_{1,2}$ est la corrélation entre les deux actifs.

\paragraph{Exemple.} 
Deux actifs avec $\sigma_1 = 13,4\%$, $\sigma_2 = 13,4\%$, et $\rho = -0,12$ :
\[
\sigma_p^2 = 0.5^2 \cdot 0.134^2 + 0.5^2 \cdot 0.134^2 + 2 \cdot 0.5 \cdot 0.5 \cdot (-0.12) \cdot 0.134 \cdot 0.134
\]

\subsection{Covariance et corrélation}
La covariance mesure comment les rendements de deux actifs évoluent ensemble :
\[
\text{cov}(r_i, r_j) = \frac{1}{T} \sum_{t=1}^T (r_{i,t} - \mu_i)(r_{j,t} - \mu_j)
\]
La corrélation est la covariance normalisée :
\[
\rho_{i,j} = \frac{\text{cov}(r_i, r_j)}{\sigma_i \sigma_j}, \quad -1 \leq \rho_{i,j} \leq 1
\]

\section{Le portefeuille efficient}

\subsection{Définition}
Un portefeuille est efficient s’il maximise le rendement attendu pour un niveau de risque donné ou minimise le risque pour un rendement attendu donné.

\subsection{Frontière efficiente}
La frontière efficiente est l’ensemble des portefeuilles qui offrent les meilleurs compromis entre rendement et risque.

\paragraph{Exemple.} 
Un portefeuille composé de 40\% d’Intel ($E(r) = 26\%$, $\sigma = 50\%$) et de 60\% de Coca-Cola ($E(r) = 6\%$, $\sigma = 25\%$) a :
\[
E(r_p) = 0.4 \cdot 26 + 0.6 \cdot 6 = 14\%
\]

\subsection{Impact de la corrélation}
La diversification est plus efficace lorsque la corrélation entre les actifs est faible ou négative. Si $\rho = -1$, il est possible de construire un portefeuille sans risque.

\section{Le ratio de Sharpe et le portefeuille tangent}

\subsection{Ratio de Sharpe}
Le ratio de Sharpe mesure la performance ajustée au risque d’un portefeuille :
\[
S = \frac{E(r_p) - r_f}{\sigma_p}
\]
où $r_f$ est le taux sans risque.

\subsection{Portefeuille tangent}
Le portefeuille tangent a le ratio de Sharpe le plus élevé et représente le meilleur portefeuille à combiner avec un actif sans risque.

\section{Le modèle d’évaluation des actifs financiers (CAPM)}

\subsection{Hypothèses clés}
\begin{itemize}
	\item Les investisseurs peuvent emprunter ou prêter au taux sans risque.
	\item Tous les investisseurs détiennent le portefeuille de marché, qui est efficient.
	\item Les investisseurs ont des attentes homogènes sur les rendements et les risques.
\end{itemize}

\subsection{Équation du CAPM}
Le rendement attendu d’un actif est donné par :
\[
E(r_i) = r_f + \beta_i (E(r_M) - r_f)
\]
où $\beta_i$ mesure la sensibilité de l’actif $i$ aux variations du marché :
\[
\beta_i = \frac{\text{cov}(r_i, r_M)}{\sigma_M^2}
\]

\subsection{Interprétation du $\beta$}
\begin{itemize}
	\item $\beta > 1$ : L’actif est plus risqué que le marché.
	\item $\beta < 1$ : L’actif est moins risqué que le marché.
	\item $\beta = 1$ : L’actif suit exactement les variations du marché.
\end{itemize}

\end{document}
