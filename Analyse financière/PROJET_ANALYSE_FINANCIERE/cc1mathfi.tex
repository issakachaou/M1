\documentclass[a4paper, 12pt]{article}
\usepackage{amsmath, amssymb}
\usepackage{booktabs}
\usepackage{eurosym}
\usepackage[utf8]{inputenc}
\usepackage[french]{babel}
\begin{document}
	
	\title{Correction détaillée du Contrôle Continu N\textsuperscript{\textdegree}1}
	\author{}
	\date{}
	\maketitle
	
	\section*{Exercice 1 : Calculs financiers}
	
	\subsection*{1.1 Calcul de la somme acquise en banque}
	Chaque ann'ee, une personne verse 50 000 \euro\ avec un taux d'intérêt annuel de 2\% pendant 3 ans. La somme acquise est donn'ee par la formule de la valeur acquise d'une suite d'annuit'es :
	
	\begin{equation}
		C_3 = A \times \frac{(1+i)^n -1}{i}
	\end{equation}
	
	Avec :
	\begin{itemize}
		\item $ A = 50 000 $
		\item $ i = 0.02 $
		\item $ n = 3 $
	\end{itemize}
	
	\begin{equation}
		C_3 = 50 000 \times \frac{(1.02)^3 -1}{0.02} = 50 000 \times \frac{1.0612 -1}{0.02} = 50 000 \times 3.06 = 153 000 \euro.
	\end{equation}
	
	\subsection*{1.2 Valeur acquise après 7 ans de placement à 4 \%}
	On utilise la capitalisation sur 7 ans avec un taux de 4 \% :
	
	\begin{equation}
		C_{10} = C_3 \times (1.04)^7
	\end{equation}
	
	\begin{equation}
		C_{10} = 153 000 \times (1.04)^7 = 153 000 \times 1.3161 = 201 385.30 \euro.
	\end{equation}
	
	\subsection*{1.3 Valeur acquise apr`es 10 ann'ees de retraits}
	L’'epargnant effectue 10 retraits de 5 000 \euro. Le capital se d'ecroit suivant une suite de retraits d’une rente.
	La formule est :
	
	\begin{equation}
		S = R \times \frac{1 - (1+i)^{-n}}{i}
	\end{equation}
	
	Avec :
	\begin{itemize}
		\item $ R = 5 000 $
		\item $ i = 0.02 $
		\item $ n = 10 $
	\end{itemize}
	
	\begin{equation}
		S = 5 000 \times \frac{1 - (1.02)^{-10}}{0.02} = 5 000 \times 8.98 = 44 900 \euro.
	\end{equation}
	
	\subsection*{1.4 Solde total apr`es le dernier retrait}
	
	\begin{equation}
		Solde = C_{10} - S = 201 385.30 - 44 900 = 156 485.30 \euro.
	\end{equation}
	
	\subsection*{1.5 Somme constante pour solde nul}
	On cherche un capital initial $ X $ tel que :
	
	\begin{equation}
		X \times (1.02)^{10} = S
	\end{equation}
	
	En r'esolvant :
	
	\begin{equation}
		X = \frac{S}{(1.02)^{10}} = \frac{44 900}{1.2190} = 36 826.07 \euro.
	\end{equation}
	
	\section*{Exercice 2 : Emprunt et amortissement}
	
	\subsection*{2.1 Calcul du taux mensuel et de la mensualit'e}
	Le taux mensuel est :
	
	\begin{equation}
		i_m = \frac{5.65}{12} = 0.0047083 = 0.4708%.
	\end{equation}
	
	La mensualit'e est :
	
	\begin{equation}
		M = \frac{C_0 \times i}{1 - (1 + i)^{-n}}
	\end{equation}
	
	Avec :
	\begin{itemize}
		\item $ C_0 = 20 000 $
		\item $ n = 120 $
		\item $ i = 0.0047083 $
	\end{itemize}
	
	\begin{equation}
		M = \frac{20 000 \times 0.0047083}{1 - (1.0047083)^{-120}}
	\end{equation}
	
	\begin{equation}
		M = 218.05 \euro.
	\end{equation}
	
	\section*{Exercice 3 : Constitution d'un capital}
	
	\subsection*{3.1 Calcul des taux proportionnels et '{e}quivalents}
	
	Les taux proportionnels sont :
	
	\begin{equation}
		i_m = \frac{8}{12} = 0.6667\%, \quad i_t = \frac{8}{4} = 2\%, \quad i_s = \frac{8}{2} = 4\%
	\end{equation}
	
	Les taux '{e}quivalents sont obtenus par :
	
	\begin{equation}
		1 + i_m = (1.08)^{1/12}, \quad 1 + i_t = (1.08)^{1/4}, \quad 1 + i_s = (1.08)^{1/2}
	\end{equation}
	
	\subsection*{3.2 Valeur acquise et actuelle de la s'erie}
	
	On divise les versements en 3 s'eries et on utilise la capitalisation.
	
	\begin{equation}
		VA = 1000 \times \frac{(1.08)^6 -1}{0.08} \times (1.08)^9 + 2000 \times \frac{(1.08)^4 -1}{0.08} \times (1.08)^5 + 3000 \times \frac{(1.08)^5 -1}{0.08}
	\end{equation}
	
	La valeur actuelle est obtenue en actualisant la valeur acquise `{a} $ t = 0 $.
	
\end{document}

