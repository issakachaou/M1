\documentclass[a4paper, 12pt]{report}
\usepackage{graphicx}
\usepackage[utf8]{inputenc} 
\usepackage[french]{babel}
\usepackage[T1]{fontenc}
\usepackage{fancyhdr}
\usepackage{amsmath,amsfonts,amssymb, empheq}
\usepackage{eurosym}
\usepackage{booktabs}
\usepackage{cancel}
\usepackage{wrapfig}
\usepackage{hyperref}
\pagestyle{fancy}
\usepackage{mathptmx} %times aves le mode math
\fancyhead[R]{Université Paris-Est Créteil}
\fancyhead[L]{Analyse financière}
\usepackage{array,multirow,makecell}
\setcellgapes{1pt}
\makegapedcells
\newcolumntype{R}[1]{>{\raggedleft\arraybackslash }b{#1}}
\newcolumntype{L}[1]{>{\raggedright\arraybackslash }b{#1}}
\newcolumntype{C}[1]{>{\centering\arraybackslash }b{#1}} 
%\renewcommand{\thechapter}{\Roman{chapter}}
%\setcounter{chapter}{1} % pour numéroter le chapitre 

\begin{document}

\chapter*{Introduction}
	
\section{Quelques éléments sur l'entreprise}

\subsection{Entreprise}

L'entreprise est un noeud de contrats entre les ayants droit, portant sur le contrôle des ressources et la répartition de la richesse créée. 

\subsection{Parties prenantes}

Les parties prenantes considérées, telles que les actionnaires et les prêteurs, s'intéressent au capital ainsi qu'à la richesse économique actuelle et future de l'entreprise.

\subsection{Analyse}

L'analyse se concentre sur la production et la répartition de la richesse économique de l'entreprise, en tenant compte des cycles de l'entreprise.

\subsection{Rentabilité et solvabilité}

Enfin, il est essentiel d'évaluer la rentabilité et la solvabilité pour comprendre la santé économique de l'entreprise.
	
\subsection{Les cycles de l'entreprise}

Nous commencerons par examiner quelques éléments fondamentaux sur l'entreprise, suivis d'une analyse des cycles de l'entreprise, qui incluent le cycle d'exploitation, le cycle d'investissement et le cycle de financement.

\subsection{Cycle d'exploitation}

Le cycle d'exploitation est crucial pour l'activité de l'entreprise. Il repose sur deux logiques principales. La première est la logique marchande et commerciale, qui englobe les transactions avec les clients ainsi que la gestion des flux monétaires. La seconde est la logique répétitive, qui se concentre sur la recherche d'économies d'échelle, souvent associée à une approche industrielle, et sur la création d'une réputation commerciale solide.

\subsection{Phases du cycle d'exploitation}

Le cycle d'exploitation se divise en trois phases distinctes. La première phase est celle de l'approvisionnement, où l'entreprise acquiert les ressources nécessaires à son fonctionnement. La deuxième phase est la phase de production, durant laquelle les inputs sont mobilisés dans un processus technologique. Enfin, la troisième phase est celle de la commercialisation, où les produits ou services sont offerts aux clients.
	
Représentation du cycle d'exploitation

\begin{wrapfigure}{r}{0.6\textwidth}
	\centering
\includegraphics[scale=0.5]{../../../Pictures/Screenshots/Capture d'écran 2025-01-08 221742}
\end{wrapfigure}

\subsection{Contrepartie}

La notion de contrepartie dans le contexte de l'entreprise se réfère à l'enchaînement de dettes et de créances. Cela inclut les dettes envers les fournisseurs, ainsi que les charges et les coûts intermédiaires. Ce processus commence par un décaissement, qui est suivi d'un encaissement, illustrant ainsi le besoin de financement nécessaire pour soutenir les opérations de l'entreprise.
\newpage
\subsection{Financement du cycle d'exploitation}
	
\begin{wrapfigure}{l}{0.4\textwidth}
	\centering
\includegraphics[scale=0.5]{../../../Pictures/Screenshots/Capture d'écran 2025-01-08 222413}
\end{wrapfigure}	
	
Les durées spécifiques varient selon chaque secteur, branche ou produit. Par exemple, dans le cas d'une entreprise de prestations de services qui est payée au comptant, la durée du cycle d'exploitation est nulle, ce qui signifie qu'il n'y a pas de délai entre le décaissement et l'encaissement.

\subsection{Cycle d'investissement}

Le cycle d'investissement concerne la création du capital économique nécessaire à la production, qui sera ensuite utilisé dans le cadre du cycle d'exploitation. Cet investissement implique une immobilisation de monnaie, ce qui signifie que des fonds sont engagés et ne sont pas immédiatement disponibles. 

De plus, l'amortissement des investissements physiques permet un retour à la liquidité, en récupérant progressivement les fonds investis. Il est également important de noter que l'investissement peut être de nature financière, comme dans le cas de la prise de contrôle d'une autre entreprise. Enfin, la durée du cycle d'investissement peut être plus ou moins longue, variant en fonction de la nature de l'investissement réalisé.

\begin{wrapfigure}{r}{0.4\textwidth}
	\centering
\includegraphics[scale=0.5]{../../../Pictures/Screenshots/Capture d'écran 2025-01-08 222859}
\end{wrapfigure}

\subsection{Cycle de financement}

Le cycle de financement constitue la contrepartie des cycles d'exploitation et d'investissement. Il implique la mise à disposition de liquidités par des apporteurs externes, tels que les actionnaires et les prêteurs. 
		
La durée de la ressource financière peut être courte, longue ou infinie, ce qui influence le rythme de la trésorerie de l'entreprise. Ainsi, une gestion efficace de ce cycle est essentielle pour assurer la liquidité nécessaire au bon fonctionnement des opérations.

\subsection{Rentabilité et solvabilité}

\subsubsection{Rentabilité}

La rentabilité est un moyen de rémunérer les apporteurs de ressources, comme les actionnaires. Elle constitue également un indicateur de rendement et d'efficacité dans l'allocation des ressources, permettant d'évaluer la performance économique de l'entreprise.
\[ \text{Rentabilité} = \frac{\text{Résultat obtenu}}{\text{Moyens mis en œuvre}} \]

La rentabilité est spécifique à chacune des parties prenantes.

Les moyens mis en oeuvre pour obtenir un capital économique sont essentiels à la création de valeur au sein de l'entreprise. Ce capital économique permet de financer les opérations et d'assurer la pérennité de l'activité. Il est constitué des ressources financières, matérielles et humaines nécessaires pour soutenir la production et le développement de l'entreprise.

\subsubsection{Solvabilité}

La solvabilité est la capacité d'une entreprise à assurer durablement le paiement de ses dettes exigibles. En cas de cessation des paiements, l'entreprise doit faire face à ses obligations envers les prêteurs et les fournisseurs, ce qui peut entraîner des procédures amiables ou judiciaires.

Dans une perspective de court terme, la solvabilité est liée à la liquidité de l'entreprise, qui peut être exprimée par la formule suivante :
\[
\text{Décaissements}(t) \leq \text{Encaissements}(t) + \text{Stock de monnaie}(t-1)
\]

En revanche, dans une perspective de long terme, il est essentiel que les encaissements soient structurellement supérieurs aux dépenses.

L'analyse financière menée par les créanciers se concentre sur le risque majeur de défaut de paiement généralisé, également connu sous le nom de défaillance. Ainsi, la notion de solvabilité se trouve au cœur de cette analyse.

\section{Information comptable}

L'information comptable est une obligation légale qui repose sur une logique d'évaluation par un tiers, tels que les actionnaires et les prêteurs. Elle est régie par des principes et des règles spécifiques à la comptabilité.

L'exploitation de cette information permet de mener une analyse financière approfondie. L'objectif principal est de produire une image fidèle et sincère du patrimoine, de la situation financière et du résultat de l'entreprise. Cela se traduit par la production de documents comptables conformes aux normes établies.

Les principes comptables sont essentiels pour produire ces documents, qui incluent à la fois les comptes individuels et les comptes consolidés.

La comptabilité des entreprises non-financières est régie par une réglementation élaborée par l'Autorité des Normes Comptables (ANC), accessible sur leur site internet : \url{http://www.anc.gouv.fr/}. 

Le Plan Comptable Général (PCG) constitue le cadre de référence pour cette comptabilité. Il est important de noter qu'il existe également des comptabilités spécifiques pour les entreprises financières, telles que les banques et les assurances.

Au fil du temps, la comptabilité a connu de nombreuses évolutions, s'adaptant aux changements économiques et réglementaires.

Les référentiels comptables utilisés incluent le référentiel national, connu sous le nom de French GAAP (\textit{Generally Accepted Accounting Principles}), qui régit la comptabilité en France. En outre, le référentiel IFRS (\textit{International Financial Reporting Standards}) a été développé par l'IASB (\textit{International Accounting Standards Board}), un organisme privé de normalisation comptable.

La réglementation européenne stipule que toutes les sociétés cotées, régies par le droit national d'un État européen, doivent appliquer le référentiel IFRS dans leurs comptes consolidés à partir du 1er janvier 2005. Il est également important de noter qu'il existe des référentiels comptables hors de l'Espace Économique Européen, tels que les US GAAP.

\subsection{Principes comptables}

Les principes comptables reposent sur la primauté du droit sur le fait, ce qui signifie que l'enregistrement comptable est associé à un acte juridique. Cela entraîne la création d'une nouvelle créance ou d'une nouvelle dette pour l'entreprise. La date et la méthode d'enregistrement ne sont pas nécessairement liées à la réalité économique.

Un autre principe fondamental est celui de l'évaluation au coût historique. Selon ce principe, les biens entrent dans le patrimoine de l'entreprise sur la base de leur valeur historique, c'est-à-dire à l'acquisition. Cette approche repose sur une valeur objective et constante, tandis que la valeur économique ou d'usage n'est pas retenue, car elle est considérée comme subjective et fluctuante.

En outre, seuls l'amortissement ou le provisionnement affectent l'évaluation comptable des actifs. Ce cadre comptable a une dimension backward-looking, c'est-à-dire qu'il est tourné vers le passé.

Les principes comptables incluent le principe de prudence, qui impose un traitement comptable dissymétrique entre les charges et les produits. Les charges sont prises en compte dès qu'elles sont probables, ce qui inclut la constitution de provisions. En revanche, les produits ne sont comptabilisés que lorsqu'ils sont réalisés, ce qui signifie que les plus-values potentielles ne sont pas prises en compte.

De plus, il n'y a pas de compensations entre les moins-values latentes et les plus-values latentes. Cette approche peut conduire à une sous-évaluation de l'entreprise dans sa valeur comptable.

Les comptes individuels sont régis par le référentiel national, connu sous le nom de French GAAP. En revanche, pour les comptes consolidés des groupes cotés, les autres référentiels, tels que les IFRS (\textit{International Financial Reporting Standards}), sont appliqués.

Il existe des divergences dans les principes comptables selon les référentiels utilisés. Par exemple, le coût historique et la primauté du droit sur le fait sont remis en cause dans le référentiel IFRS. Dans ce cadre, la primauté de la réalité économique, la comptabilité d'intention et l'évaluation à la "juste valeur" sont des principes privilégiés selon les normes IFRS.

\subsection{Documents comptables}

Le livre-journal enregistre chronologiquement les opérations affectant le patrimoine de l'entreprise. Le grand livre, quant à lui, regroupe les opérations du livre-journal en fonction du plan de compte de l'entreprise, qui est défini par la nomenclature du Plan comptable général.

L'inventaire est également un document essentiel dans la comptabilité. Les documents de synthèse, qui sont reportés sur l'inventaire, comprennent trois documents principaux : le bilan, le compte de résultat et l'annexe. Ces documents correspondent aux comptes annuels, qui doivent être déposés au greffe du tribunal de commerce dans le mois suivant l'approbation des comptes.

Les détails des comptes dépendent de critères liés à la taille des entreprises. Il existe trois niveaux de présentation, comme indiqué en annexe. Ces niveaux sont le système abrégé, le système de base et le système développé. 

Les différences entre ces trois systèmes résident principalement dans le niveau de détails fournis. 

Les documents comptables incluent la certification des comptes par un commissaire aux comptes, qui est une obligation légale si deux des trois critères suivants sont vérifiés : un chiffre d'affaires supérieur à 3,1 millions d'euros, un total de bilan supérieur à 1,55 million d'euros, ou un nombre moyen de salariés supérieur à 50.

La liasse fiscale est l'ensemble des imprimés fiscaux renseignés par l'entreprise, permettant de déterminer l'impôt sur les sociétés. Cette information peut être plus riche que celle contenue dans les documents comptables, notamment en ce qui concerne les amortissements, les provisions, ainsi que les échéances des créances et des dettes.

\section{Annexe}

Les détails des comptes dépendent de critères liés à la taille des entreprises. Il existe trois niveaux de présentation : le système abrégé, le système de base et le système développé.

Le système abrégé est destiné aux "petites" entreprises. Il permet la production d'un bilan et d'un compte de résultat simplifiés, à condition de respecter au moins deux des trois critères suivants : un total du bilan inférieur à 267 000 euros, un chiffre d'affaires net inférieur à 534 000 euros, ou un nombre moyen de salariés inférieur à 10.

Il est important de noter que les seuils mentionnés peuvent être amenés à changer dans le temps en raison de l'évolution du niveau général des prix.

Les documents comptables incluent le système de base, qui s'applique aux moyennes et grandes entreprises. Dans ce système, le bilan et le compte de résultat sont plus complets, accompagnés d'une annexe détaillée.

Cependant, il est possible de présenter une annexe simplifiée si au moins deux des trois critères suivants sont respectés : un total du bilan inférieur à 3,65 millions d'euros, un chiffre d'affaires net inférieur à 7,3 millions d'euros, ou un nombre moyen de salariés inférieur à 50.

Les documents comptables incluent le système développé, qui comporte des documents supplémentaires éclairant la gestion de l'entreprise. Ce système est facultatif et peut inclure des exemples de documents tels que le tableau de capacité d'auto-financement, le tableau de financement et le tableau de variation des capitaux propres.

\chapter{Le bilan comptable}

\section*{Introduction}

Les référentiels comptables jouent un rôle central dans la préparation et l'analyse des états financiers. Parmi les principaux référentiels, on trouve le référentiel national, également connu sous le nom de \textit{French GAAP}, qui constitue le cadre de droit commun applicable aux comptes sociaux individuels des entreprises en France. On trouve également le référentiel IFRS, qui est le cadre européen utilisé pour les comptes consolidés des entreprises cotées. 

Il existe une tendance marquée vers la convergence entre ces référentiels, avec pour objectif une harmonisation des pratiques comptables à l'échelle internationale. 

Les éléments du bilan permettent de décrire la situation patrimoniale de l'entreprise. Le patrimoine varie dans le temps donc le bilan est daté le plus souvent du \( 31/12/N \). L'actif correspond à ce que possède l'entreprise, tandis que le passif représente ce qu'elle doit. Dans ce chapitre on utilisera le référentiel national soit le \textit{French GAAP}.

Des retraitements peuvent être nécessaires afin de transformer l'information comptable en une information exploitable par l'analyste financier, aboutissant à un bilan financier mieux adapté à la prise de décision.

\section{Analyse de l'actif}

L'analyse de l'actif consiste à examiner les moyens utilisés par l'entreprise pour exercer son activité. L'actif récapitule, à une date donnée, les droits de propriété et les créances de l'entreprise. 

La logique de construction de l'actif repose sur une approche fonctionnelle. On distingue plusieurs catégories principales. L'actif immobilisé regroupe les utilisations durables, c'est-à-dire les biens destinés à rester durablement dans l'entreprise. L'actif circulant correspond aux éléments dont le renouvellement est régulier, comme les stocks ou les créances à court terme. Enfin, les comptes de régularisation permettent d'ajuster les charges et les produits à la période comptable concernée.

Les principaux postes de l'actif reflètent les moyens financiers et matériels de l'entreprise. On trouve d'abord le capital souscrit non appelé, qui représente la part du capital social encore non versée par les actionnaires. Les immobilisations se divisent en trois catégories : les immobilisations incorporelles, qui comprennent les éléments immatériels tels que les brevets ou les logiciels, les immobilisations corporelles, qui incluent les biens matériels comme les bâtiments et les machines, et les immobilisations financières, qui regroupent les participations et autres investissements à long terme. 

Les stocks correspondent aux biens destinés à être vendus ou transformés. Les créances et avances représentent les montants dus à l'entreprise par ses clients ou partenaires. La trésorerie englobe les liquidités disponibles, qu'elles soient sous forme d'espèces ou de dépôts bancaires. Enfin, les comptes de régularisation permettent de répartir les charges et produits sur les périodes comptables appropriées.

\subsection{Capital souscrit non appelé}

Le capital souscrit non appelé correspond à la contrepartie à l'actif d'un engagement des actionnaires, qui est comptabilisé dans le capital social de l'entreprise. Il s'agit d'une créance de la société sur ses actionnaires, dont l'appel des fonds est décidé par le conseil d'administration ou le directoire de l'entreprise. 

Comptablement, ce poste est initialement classé avec les immobilisations. Toutefois, un reclassement en tant qu'actif de trésorerie peut être envisagé, notamment lorsqu'il représente des ressources monétaires liquides pouvant être mobilisées rapidement.

\textit{La mention "(dont versé…)" de la ligne DA concerne les sociétés dotées, à leur
création, d’un capital dont une partie seulement a été effectivement versée dans la
caisse sociale. Cette partie est aujourd’hui fixée à 50 \%, le solde devant être versé
ensuite dans les 5 ans. Même dans le cas où il n’est pas intégralement versé (on dira
"libéré", c’est-à-dire "libre… disponible"), le capital de 37 000 € sera représenté
par des actions qui auront trouvé souscripteur. On dira que le capital est souscrit,
une partie seulement étant libérée. La partie non libérée pourra être "appelée" par
la société à tout moment sur décision de son Conseil d’Administration (au plus
tard dans les 5 ans). Tant que cette part non libérée n’est pas appelée, la société ne
dispose évidemment pas des fonds correspondants.}

\subsection{Immobilisations incorporelles}


Les immobilisations incorporelles représentent des emplois durables de fonds qui ne sont ni des actifs physiques ni des actifs financiers. Elles correspondent à des droits obtenus en contrepartie de dépenses spécifiques.

\begin{itemize}
	\item Les frais de recherche et de développement, qui correspondent aux dépenses engagées pour créer ou améliorer des produits, procédés ou services.

	\item Les brevets, licences, marques et autres droits, qui représentent des actifs intangibles protégés par des droits légaux.

	\item Le fonds commercial, incluant des éléments tels que la clientèle et le droit de bail.

	\item Les frais d’établissement, qui couvrent les dépenses engagées lors de la constitution de l’entreprise, telles que les honoraires ou les droits d’enregistrement, ainsi que les coûts liés à son développement.

\end{itemize}
Ces immobilisations doivent être amorties sur une durée maximale de cinq ans, car elles correspondent à des biens intangibles ou immatériels. Toutefois, certaines dépenses, comme les frais de recherche, peuvent être directement passées en charge lorsqu'elles ne répondent pas aux critères de capitalisation.

\subsection{Immobilisations corporelles}

Les immobilisations corporelles représentent des actifs physiques durables dont l'entreprise est propriétaire. Elles constituent un élément clé dans le fonctionnement de l'activité économique. Par exemple, pour une entreprise industrielle, le capital de production comprend des éléments comme les terrains, les constructions, les installations techniques ou encore le matériel industriel.

Ces actifs sont soumis à un amortissement destiné à refléter leur dépréciation due à l'usure ou à l'obsolescence. L'amortissement peut être calculé de manière linéaire ou dégressive, selon les règles comptables applicables et les besoins de l'entreprise. 

La comptabilisation des immobilisations corporelles s'effectue généralement au coût historique. Toutefois, certaines entreprises peuvent opter pour une évaluation à la "\textit{fair value}", reflétant la valeur de marché des actifs. Cette méthode peut inclure la possibilité d’une réévaluation des actifs, qu'il s'agisse d’une appréciation ou d’une dépréciation.

\subsection{Immobilisation financière}

Les immobilisations financières comprennent les créances et les titres détenus dans une perspective de long terme, en lien avec la stratégie de développement de l'entreprise. 

Il existe plusieurs types d'immobilisations financières. Tout d'abord, les participations, qui correspondent à l'acquisition de plus de 10\% du capital d'une autre entreprise, permettent d'influencer sa gestion. Ensuite, les titres immobilisés de l'activité de portefeuille, qui sont des actions détenues sur le long terme sans intervention dans la gestion de l'entreprise concernée. Dans cette situation, on possède des actions de l'entreprise mais en faible proportion de sorte qu'on ne peut pas influencer la gestion de l'entreprise. Enfin, les prêts, qui sont des créances d'une durée supérieure à un an, incluent également les prêts accordés à la société mère ou aux associés.

\subsection{Stock}

Les stocks sont associés à l'actif circulant et comprennent différents types. Parmi eux, on trouve les matières premières et les approvisionnements, ainsi que les en-cours de production et les produits intermédiaires ou finis.

Il existe plusieurs méthodes d'évaluation ou de valorisation des stocks, qui soulèvent certaines problématiques, notamment pour les unités interchangeables : quel "prix" appliquer pour les sorties ?

\begin{itemize}
	\item La première méthode consiste à évaluer les sorties au coût moyen pondéré des entrées.
	\item La méthode FIFO (\textit{First In, First Out}) valorise les sorties au coût de l'élément le plus ancien.
	\item La méthode LIFO (\textit{Last In, First Out}) valorise les sorties au coût de l'élément le plus récent.
	\item Enfin, le coût de remplacement prend en compte le cours du marché pour évaluer les sorties.
\end{itemize}

En France, seules les méthodes 1 (coût moyen pondéré) et 2 (FIFO) sont autorisées. Ces méthodes peuvent entraîner des plus-values latentes en période d'inflation, ce qui affecte le résultat de l'entreprise. 

Les implications du choix de la méthode d'évaluation des stocks peuvent être particulièrement importantes si le délai de rotation des stocks est faible. En effet, dans un contexte de fluctuations des prix, le choix de la méthode peut influencer significativement les états financiers et la perception de la performance de l'entreprise.

\subsection{Créances et avances}

Les avances et acomptes versés sur commandes sont des montants versés à des fournisseurs. Ces avances constituent une créance sur un tiers. 

Les créances clients et les comptes rattachés représentent les comptes débiteurs de tous les clients. Les mouvements réels correspondants sont liés à des biens livrés ou à des prestations de services effectuées. 

Il est également important de mentionner la provision pour dépréciation, qui est constituée pour les clients douteux ou litigieux afin de couvrir les risques de non-recouvrement.

Enfin, les autres créances incluent les avances et acomptes versés au personnel, ainsi que les créances sur l'État et sur le "Groupe et associés".

\subsection{Trésorerie}

La trésorerie regroupe les encaisses disponibles ou quasi-disponibles. Elle comprend plusieurs types de rubriques ou de comptes.

Tout d'abord, les valeurs mobilières de placement, qui incluent des actions, des obligations, des bons du trésor, des titres de créance négociables (TCN) et des parts de fonds communs de placement (FCP) monétaires. 

Ensuite, les instruments de trésorerie, qui représentent les variations de valeurs des opérations en cours sur les marchés de produits dérivés, tels que les contrats à terme et les options.

Enfin, les disponibilités incluent les comptes bancaires et la caisse de l'entreprise.

Il est important de noter les différences entre les normes \textit{French GAAP} et IFRS. Par exemple, selon l'IFRS 7, la définition de la trésorerie est plus restrictive, excluant certaines obligations d'État ou OPCVM obligataires qui ne sont pas considérées comme des placements de trésorerie.

\subsection{Comptes de régularisation}

Les comptes de régularisation comprennent plusieurs types de charges. 

Tout d'abord, les charges constatées d'avance, qui concernent l'achat de biens et de services dont la fourniture ou la prestation sera ultérieure. Par exemple, cela inclut des factures d'achat reçues ou des primes d'assurance payées en avance.

Ensuite, il y a les charges à répartir sur plusieurs exercices. Ce sont des charges importantes et non répétitives dont les effets s'étalent dans le temps. L'imputation de ces charges se fait par le débit d'un compte de dotation aux amortissements, ce qui les enregistre négativement à l'actif.

Les écarts de conversion représentent la contrepartie comptable au bilan des pertes de change latentes. Cela inclut la diminution de valeur des créances ou l'augmentation de valeur des dettes, ce qui peut également conduire à la constitution d'une provision pour risque financier.


\subsection{Les principaux postes de l'actif}

Les principaux postes de l'actif sont les suivants :

\begin{enumerate}
	\item Capital souscrit non appelé : Il représente le montant des actions souscrites par les actionnaires mais qui n'ont pas encore été appelées par la société.
	\item Immobilisations: Ce poste se divise en trois catégories :
\begin{itemize}
	\item  Incorporelles : Comprend les actifs non physiques tels que les brevets, les marques et les droits d'auteur.
	\item  Corporelles : Inclut les actifs physiques comme les terrains, les bâtiments et les équipements.	
	\item Financières : Englobe les participations dans d'autres entreprises et les prêts à long terme.
\end{itemize}
	\item Stocks : Représente les biens destinés à la vente ou à la production.
	\item Créances et avances : Comprend les montants dus par les clients et les avances versées à des tiers.
	\item Trésorerie : Regroupe les encaisses disponibles ou quasi-disponibles.
	\item Comptes de régularisation : Inclut les charges constatées d'avance et les écarts de conversion, entre autres.
\end{enumerate}

\section{Analyse du passif}

Les éléments du passif représentent les dettes réelles de l'entreprise envers des tiers. Ils récapitulent, à une date donnée, les engagements de l'entreprise, tant vis-à-vis des tiers qu'à l'égard de ses propriétaires. 

La logique de construction du passif repose sur une distinction selon la nature juridique et financière des éléments qui le composent. Les dettes correspondent à des engagements qui doivent être remboursés à leur échéance, selon les conditions contractuelles établies. En revanche, les capitaux propres (Fond Propre), qui représentent les ressources apportées par les propriétaires ou générées par l'activité, ont un horizon temporel théoriquement infini, car ils ne sont pas soumis à une obligation de remboursement.

Les principaux postes du passif permettent de structurer les engagements de l'entreprise en fonction de leur nature et de leur horizon temporel. Ils se décomposent comme suit : 

Les capitaux propres représentent les ressources apportées par les actionnaires ou générées par l'activité de l'entreprise. Ils constituent un financement à long terme, sans obligation de remboursement.

Les provisions pour risques et charges sont des passifs potentiels ou certains, liés à des événements passés, dont l'échéance ou le montant restent incertains.

Les dettes se divisent en plusieurs catégories. 

Les dettes financières correspondent aux emprunts contractés 
auprès des institutions financières. 

Les dettes d'exploitation regroupent les montants dus dans le cadre des activités courantes, tels que les dettes fournisseurs. 

Les dettes diverses incluent des engagements spécifiques, comme les dettes fiscales ou sociales.

Les comptes de régularisation permettent d'ajuster les charges et produits aux périodes comptables correspondantes.

\subsection{Capitaux propres}

Les capitaux propres représentent les ressources permanentes mises à disposition de l'entreprise. Ils incluent à la fois les apports initiaux des actionnaires et les surplus monétaires générés au fil du temps. 

Le capital social et les primes qui y sont liées constituent une partie essentielle des capitaux propres. Le capital social correspond à la valeur nominale des actions émises par l'entreprise, c'est la valeur apportée par les actionnaires lors de la création de l'entreprise. Les primes liées, quant à elles, constatent la différence entre la valeur des apports "initiaux" et les accroissements du capital social. Cela inclut, par exemple, l'excédent du prix d'émission des actions par rapport à leur valeur nominale.

Les bénéfices mis en réserves (les réserves) représentent le cumul historique de la fraction des bénéfices réalisés par l'entreprise et conservés en interne, plutôt que redistribués sous forme de dividendes.

Enfin, les subventions d'équipement ou d'investissement, octroyées par des collectivités publiques, viennent également compléter les capitaux propres. Ces aides visent à soutenir des projets spécifiques ou le développement de l'entreprise.

Les capitaux propres incluent également des provisions réglementées, qui bénéficient d'un traitement fiscal particulier. Ces provisions, non imposées, comprennent notamment :

\begin{itemize}
	\item La provision pour investissement, qui résulte d’un avantage fiscal accordé à l’entreprise lorsque celle-ci distribue une partie de son résultat aux salariés, dans le cadre du régime obligatoire. Cette provision permet à l’entreprise de mettre en réserve une somme non imposable.
	\item La provision pour hausse de prix, utilisée lorsque les prix des stocks de produits ou de matières premières augmentent de plus de 10\% sur les deux dernières années.
	\item D'autres provisions réglementées, comme celles destinées à l’implantation à l’étranger ou au risque de crédit à moyen terme.
\end{itemize}

Par ailleurs, les autres fonds propres regroupent des éléments hétérogènes caractérisés par des statuts juridiques complexes, souvent hybrides ou confus. Cela inclut, par exemple, les émissions de titres participatifs ou les avances conditionnées par l'État.

Ces éléments complètent la structure des capitaux propres, apportant à l’entreprise une flexibilité supplémentaire dans la gestion de ses ressources à long terme.

\subsection{Provisions pour risques et charges}

Les provisions pour risques et charges sont destinées à couvrir un risque ou une charge prévisible, mais qui ne sont pas directement affectés à un élément spécifique de l'actif. Elles correspondent à une dette probable, dont le montant et l'échéance restent incertains.

À titre d'exemple, ces provisions peuvent inclure : 
\begin{itemize}
	\item Les provisions pour litiges ;
	\item Les provisions pour pertes de change ;
	\item Les provisions pour pertes sur contrats ;
	\item Les provisions pour restructurations ;
	\item Les provisions pour grosses réparations, etc.
\end{itemize}

Comptablement, la constitution d'une provision entraîne une baisse du résultat, mais les fonds restent dans l’entreprise jusqu’à ce que le risque ou la charge se réalisent. Cela constitue un moyen pour l’entreprise de lisser ses résultats sur plusieurs exercices.

Toutefois, dans le cadre des normes IFRS, le traitement des provisions est plus restrictif, limitant leur utilisation aux cas répondant à des critères précis.

\subsection{Dettes}

Les dettes représentent des engagements financiers de l’entreprise envers des tiers. Elles se répartissent en plusieurs catégories :

\subsubsection{Dettes financières}

Les dettes financières constituent un moyen de financement durable, généralement à long terme (au delà d'un an). Elles correspondent à des dettes arrivant à échéance mais souvent renouvelées. Parmi les prêteurs figurent :  
\begin{itemize}
	\item Les banques ;  
	\item Les marchés financiers ;  
	\item Les autres entreprises, notamment celles du même groupe ;  
	\item Les associés.
\end{itemize}  

\subsubsection{Dettes d'exploitation}

Les dettes d’exploitation sont directement liées à l’activité courante de l’entreprise. Elles incluent :  
\begin{itemize}
	\item Les avances et acomptes reçus sur commandes en cours, correspondant aux sommes versées par les clients avant la livraison des biens ou services ;  
	\item Les dettes fournisseurs et comptes rattachés, représentant les sommes restant dues aux fournisseurs pour des biens ou services livrés ;  
	\item Les dettes fiscales et sociales, regroupant les montants dus envers le personnel, la Sécurité Sociale ou l'État, notamment pour la collecte de la TVA.
\end{itemize}

\subsubsection{Dettes diverses (hors exploitation)}

Ces dettes ne sont pas directement liées à l’exploitation courante. Elles incluent notamment :  

\begin{itemize}
	\item Les dettes fiscales, comme l’impôt sur les bénéfices ;  
	\item Les autres dettes, telles que la réserve de participation des salariés ou les dividendes à payer aux associés.   
\end{itemize}

\subsection{Comptes de régularisation}

Les comptes de régularisation regroupent notamment les produits constatés d'avance. Ils représentent une dette de l'exercice en cours envers les exercices suivants. Ces produits correspondent à un engagement de l'entreprise à fournir un bien ou une prestation ultérieurement, en contrepartie de montants déjà perçus.

\subsection{Trésorerie}

La trésorerie inclut les dettes à court terme, telles que :  
\begin{itemize}
	\item Les soldes bancaires créditeurs ;  
	\item Les comptes courants des sociétés apparentées ou des sociétés mères.  
\end{itemize}
Ces éléments reflètent des engagements financiers immédiats ou de très court terme.

\subsection{Les principaux postes du passif}

Les principaux postes du passif permettent de structurer les engagements financiers de l'entreprise. Ils se décomposent comme suit : 

\begin{enumerate}
	\item Capitaux propres
	\item Provisions pour risques et charges
	\item Dettes
	\begin{itemize}
		\item Dettes financières
		\item Dettes d'exploitation
		\item Dettes diverses
	\end{itemize}
	\item Comptes de régularisation
	\item Trésorerie
\end{enumerate}

\section{Les comptes consolidés}

L'analyse de l'actif et du passif est essentielle pour la compréhension des comptes consolidés. Ces comptes ont pour objectif de fournir une image fidèle de la réalité économique et financière d'un ensemble coordonné d'entreprises, c'est-à-dire un groupe. Dans ce contexte, les comptes individuels de la société mère présentent un portefeuille de titres à l'actif, ainsi que des droits sur d'autres entreprises. L'opération de consolidation consiste à substituer à la quote-part des titres de participation tout ou partie des éléments d'actif et de passif de l'entreprise concernée ou contrôlée.

Les entreprises cotées sur un marché européen doivent produire leurs comptes consolidés selon les normes IFRS. La réalisation de comptes consolidés est obligatoire pour les groupes, qu'ils soient cotés ou non, qui réunissent 2 des 3 critères suivants : un total du bilan supérieur à 15 millions d'euros, un chiffre d'affaires dépassant 30 millions d'euros, ou un nombre de salariés supérieur à 250. En ce qui concerne les normes, les groupes peuvent choisir entre les \textit{French GAAP} ou les IFRS. En pratique, seuls les petits groupes non cotés utilisent généralement les \textit{French GAAP}.

\subsection{Principe}

Le principe des comptes consolidés repose sur des méthodes de consolidation qui varient selon la nature des relations entre la société mère et sa filiale. Trois méthodes possibles existent. 

\subsubsection{L'intégration globale (IAS 27)}

La première est l'intégration globale (IAS 27), qui s'applique lorsque la société mère exerce un contrôle exclusif sur la filiale, ce qui est généralement le cas lorsque le groupe possède 50\% des droits de vote de la filiale. Toutefois, le critère utilisé est plus large : selon la norme IFRS, il suffit de disposer de la majorité des sièges au conseil d'administration, tandis que selon la norme française, le groupe peut détenir 40\% des droits de vote de la filiale. Dans ce cas, les comptes de la société mère reprennent l'intégralité des actifs et passifs. De plus, la prise en compte des actionnaires minoritaires se traduit par l'apparition, au passif de la société mère, du poste "intérêts minoritaires ou non-contrôlant".


\subsubsection{La mise en équivalence (IAS 28)}

La mise en équivalence (IAS 28) s'applique dans les situations où le groupe exerce une influence notable sur une entreprise, sans pour autant contrôler sa politique financière et opérationnelle. Dans ce cas, la société mère n'est pas déterminante dans les prises de décision. Selon les normes IFRS et la pratique des commissaires aux comptes, un seuil de 20\% des droits de vote est requis pour établir cette influence. La mise en équivalence implique une méthode de réévaluation, où la valeur comptable des titres de participation est substituée par la valeur de la quote-part des capitaux propres de la filiale. Dans les comptes de la société mère, cela se traduit par un poste "Titres mis en équivalence" à l'actif et un écart d'évaluation dans les capitaux propres au passif.

\subsubsection{Intégration proportionnelle}

L'intégration proportionnelle (IFRS 31) concerne les situations de contrôle conjoint, où un groupe partage de manière stable le contrôle avec une ou plusieurs entreprises, comme dans le cas des co-entreprises (joint-venture). Dans ce cadre, les comptes de la société mère reprennent une proportion des actifs et passifs de la filiale. Il est important de noter qu'il n'y a pas d'apparition "d'intérêts minoritaires" dans ce type de consolidation. La norme IFRS recommande également d'appliquer une mise en équivalence dans ces situations.

\subsection{Plan comptable}

Le plan comptable et les méthodes de comptabilisation des comptes consolidés diffèrent de ceux des comptes individuels (i.e., French GAAP), notamment par les noms des items à l'actif et au passif. Un exemple d'item spécifique aux comptes consolidés est l'écart d'acquisition (survaleur, ou \textit{goodwill}) à l'actif consolidé. Cet écart provient de la prise en compte des actifs réévalués de la filiale, et non de la valeur comptable du bilan individuel de la filiale, ainsi que de la prise en compte des plus-values. Il est important de noter que le prix d'acquisition de la filiale par la société mère peut différer de la valeur comptable des fonds propres de la filiale. Cela implique un écart entre les éléments d'actif et de passif qui sont remontés lors de la consolidation.

\subsubsection{Exemple de différence dans les méthodes de valorisation : immobilisations financières}

Un exemple de différence dans les méthodes de valorisation concerne les immobilisations financières. La classification des instruments financiers, initialement régie par IAS 39, a été remplacée par IFRS 9 en 2018. Cette norme définit trois catégories ou modèles de valorisation : 

\begin{itemize}
	\item Modèle Juste Valeur par Résultat : catégorie par défaut.
	\item Modèle Coût Amorti : applicable pour un business model de type HTC (held to collect), consistant à percevoir les flux de trésorerie contractuels et à conserver l'instrument financier jusqu'à son échéance.
	\item Modèle Juste Valeur par OCI (Other Comprehensive Income) : recyclable (i.e., par Fonds Propres), destiné aux business models de type HTCS (held to collect and sell), qui consistent à percevoir les flux contractuels et à vendre l'actif.
\end{itemize}

Ces classifications influencent significativement la manière dont les immobilisations financières sont présentées dans les comptes consolidés.

\subsection{Représentation shématique pour l'IFRS 9}

\begin{wrapfigure}{r}{0.4\textwidth}
	\centering
\includegraphics[scale=0.5]{../../../Pictures/Screenshots/Capture d'écran 2025-01-21 111643}
\end{wrapfigure}

\section{Conclusion}

Le bilan comptable reflète la situation patrimoniale de l'entreprise à une date donnée, indiquant les moyens mobilisés par l'entreprise pour réaliser son activité. L'étape suivante consiste à analyser les résultats de l'activité. Pour ce faire, il est essentiel d'exploiter les informations du bilan comptable, ce qui inclut le calcul de ratios financiers. Des retraitements peuvent être nécessaires afin de construire un bilan financier plus adapté pour l'analyste financier, facilitant ainsi une meilleure évaluation de la performance et de la santé financière de l'entreprise.

\chapter{Le compte de résultat et autres documents}

\section*{Introduction}

Les documents de synthèse comprennent le bilan, le compte de résultat et l'annexe. 

\subsection{Compte de résultat}
Le compte de résultat présente l'ensemble des flux de produits et de charges imputables à l'exercice comptable. Il se divise en plusieurs catégories :
\begin{itemize}
	\item Produits et charges d'exploitation
	\item Produits et charges financières
	\item Produits et charges exceptionnels
\end{itemize}

L'objectif final est de calculer le résultat net de l'exercice.

\subsection{Annexe}
L'annexe fournit des précisions sur les méthodes comptables utilisées ainsi que des informations ponctuelles, permettant une meilleure compréhension des états financiers présentés.

\section{Compte de résultat}

\subsection{Éléments du compte de résultat}
Le compte de résultat se compose des éléments suivants :

\begin{itemize}
	\item Opérations d'exploitation
	\item Opérations financières
	\item Opérations exceptionnelles
	\item Participation des salariés
	\item Impôts sur les bénéfices
\end{itemize}

Chacun de ces éléments joue un rôle crucial dans le calcul du résultat net de l'exercice, offrant une vue d'ensemble de la performance financière de l'entreprise.

\subsubsection{Opérations d'exploitation}

Les opérations d'exploitation se composent des produits d'exploitation suivants :

1. **Ventes de marchandises** : il s'agit d'une absence de transformations par l'entreprise, englobant des activités purement commerciales telles que les opérations de négoce et de distribution.

2. **Production vendue** : ce terme fait référence au processus de création, fabrication ou transformation. 
\[
\text{Chiffre d'affaires net} = \text{ventes de marchandises} + \text{production vendue}
\]

3. **Production stockée** : cela représente la variation de stock au cours de l'exercice, calculée comme suit :
\[
\text{Variation de stock} = \text{stock final} - \text{stock initial}
\]
Cela inclut :
- Stocks de produits intermédiaires et finis
- Encours de production de biens (biens en cours de fabrication)
- Encours de production de services (études ou prestations en cours, évaluées au coût de production)

4. **Production immobilisée** : cela concerne les investissements corporels ou incorporels réalisés par l'entreprise pour elle-même.

Ces éléments sont essentiels pour déterminer la performance économique de l'entreprise au cours de l'exercice.












\section{Compte consolidé}



\section{L'annexe et autres documents}


































\end{document}
