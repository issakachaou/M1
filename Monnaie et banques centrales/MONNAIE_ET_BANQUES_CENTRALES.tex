\documentclass[a4paper, 12pt]{report}
\usepackage{graphicx}
\usepackage[utf8]{inputenc} 
\usepackage[french]{babel}
\usepackage[T1]{fontenc}
\usepackage{fancyhdr}
\usepackage{amsmath,amsfonts,amssymb, empheq}
\usepackage{eurosym}
\usepackage{booktabs}
\usepackage{cancel}
\usepackage{wrapfig}
%\usepackage{tikz}
\usepackage{hyperref}
\pagestyle{fancy}
\usepackage{mathptmx} %times aves le mode math
\fancyhead[R]{Université Paris-Est Créteil}
\fancyhead[L]{Monnaie et banques centrales}
\usepackage{array,multirow,makecell}
\setcellgapes{1pt}
\makegapedcells
\newcolumntype{R}[1]{>{\raggedleft\arraybackslash }b{#1}}
\newcolumntype{L}[1]{>{\raggedright\arraybackslash }b{#1}}
\newcolumntype{C}[1]{>{\centering\arraybackslash }b{#1}} 
%\renewcommand{\thechapter}{\Roman{chapter}}
%\setcounter{chapter}{1} % pour numéroter le chapitre 

\begin{document}
	
\chapter{Analyse de la création de	monnaie et de l'offre et de la demande de monnaie}
	
\section*{Introduction}


La monnaie est un moyen de libération général, indéterminé et immédiat. Elle permet d'acheter des biens et services (B\&S), de rembourser des dettes et d'épargner. De plus, elle représente un droit de créance sur un établissement émetteur de monnaie. Mais qui émet de la monnaie ? La monnaie a une utilité indirecte et rend des services monétaires, ce qui correspond aux fonctions traditionnelles de la monnaie. C'est pour cette raison qu'il existe une demande de monnaie.

La monnaie facilite les échanges en réduisant les coûts de transaction, ce qui permet d'éviter l'économie de troc. Elle a cours légal d'après le code monétaire et financier et est imposée comme moyen de paiement par la loi pour un pays ou une zone monétaire. Les agents économiques (AE) doivent avoir confiance dans la valeur de la monnaie.

La monnaie sert d'étalon de valeur, permettant d'évaluer le prix des biens et services (B\&S) dans une même unité de mesure. Par exemple, dans une économie de troc comparée à une économie monétaire, en France au XVIIe siècle, il existait plusieurs unités de compte comme la livre parisis (Paris) ou la livre tournois (Tours), tandis que la monnaie en circulation comprenait le Louis d'or, les écus (en argent) et les billets. Dans la zone euro, l'unité de compte et la monnaie sont l'euro (\euro).

L'unité de compte est rattachée par convention, selon les régimes monétaires, à un étalon métallique (poids stable de métal) ou à une devise étrangère (taux de change stable ou valeur externe). Elle doit également vérifier la stabilité du pouvoir d'achat de la monnaie (valeur interne). Cela implique un ancrage nominal de la monnaie.

La monnaie est conservée dans le temps, permettant le transfert du pouvoir d'achat (PA) du présent vers le futur pour une consommation différée ou la constitution d'une épargne (patrimoine), selon Keynes. Elle est considérée comme un actif sans risque en valeur nominale. Cependant, l'inflation peut réduire sa valeur réelle.

Des exemples d'hyperinflation incluent l'Allemagne en 1922-1923, ainsi que certains pays d'Amérique latine et l'Ukraine dans les années 1980-1990, et plus récemment, le Zimbabwe, le Venezuela, l'Argentine et la Turquie. Il est à noter que de nombreux pays ont également connu des cas d'inflation à partir de 2021.

Environ 90\% de la monnaie est créée par les banques (établissements de crédit). Cela leur confère un pouvoir exorbitant, car elles transforment des reconnaissances de dettes individuelles (prêts) en reconnaissances de dettes collectivement acceptées par tous (monnaie). Si l'épargne n'est pas suffisante pour financer de nouveaux projets, il y a recours au crédit, entraînant ainsi la création de monnaie. 

Les agents économiques (AE) doivent être capables de rembourser leur crédit, ce qui implique une anticipation de revenus futurs suffisants. Cependant, on observe une augmentation des situations de surendettement, selon la Banque de France (BdF).


- Ce pouvoir de création est-il illimité ? 

- Qu'est-ce que l'offre et la demande de monnaie ?

- Le mécanisme de création de monnaie dans le cadre de la politique monétaire non conventionnelle s'est-il transformé ?

\section{Processus de la création monétaire}

La création monétaire est principalement assurée par les institutions financières monétaires (IFM) résidentes, telles que les établissements de crédit (EC), les banques centrales (BC), les établissements de monnaie électronique (EME) et les organismes de placement collectif (OPC) monétaires.

\subsection{Création de monnaie par une banque unique}
	
Une seule banque peut créer de la monnaie scripturale. Cette monnaie circule grâce aux instruments (ou moyens) de paiement.
	
1ère opération : achat d'un bien entre deux agents privés non financiers.  

- Pas de création de monnaie.

2ème opération : achat de devises étrangères par une banque à une entreprise non financière (ENF) résidente exportatrice.

3ème opération : achat de titres financiers par une banque auprès d'une ENF résidente.

4ème opération : achat de produits dérivés de la banque auprès d'une société d'assurance.

5ème opération : crédit accordé par une banque à un agent privé non financier.  

- Dépôt augmente de +100 dans chaque opération.  

- Agrégat de monnaie M1 et M3 augmentent.

\subsection*{Rappel : Les agrégats monétaires M1, M2 et M3}

Les agrégats monétaires M1, M2 et M3 sont des indicateurs utilisés pour mesurer la masse monétaire en circulation dans une économie. Voici un résumé :

\subsection*{M1 : Monnaie étroite}
\begin{itemize}
	\item Comprend les formes les plus liquides de monnaie, immédiatement utilisables pour les transactions.
	\item \textbf{Inclut :} 
	\begin{itemize}
		\item Les pièces et billets en circulation (monnaie fiduciaire).
		\item Les dépôts à vue (comptes courants) disponibles immédiatement.
	\end{itemize}
\end{itemize}

\subsection*{M2 : Monnaie intermédiaire}
\begin{itemize}
	\item Inclut \textbf{M1} ainsi que certaines formes de dépôts légèrement moins liquides.
	\item \textbf{Ajoute :}
	\begin{itemize}
		\item Les dépôts à terme (placements à court terme).
		\item Les comptes d'épargne (disponibles rapidement mais pas utilisables directement comme moyen de paiement).
	\end{itemize}
\end{itemize}

\subsection*{M3 : Monnaie large}
\begin{itemize}
	\item Inclut \textbf{M2} ainsi que des instruments financiers encore moins liquides.
	\item \textbf{Ajoute :}
	\begin{itemize}
		\item Les dépôts à long terme.
		\item Les titres de créance négociables (comme les certificats de dépôt et certains titres de marché monétaire).
	\end{itemize}
\end{itemize}

Chaque agrégat est une mesure progressive, \textbf{M1} étant le plus liquide et \textbf{M3} englobant des actifs plus larges, moins immédiatement accessibles. Ces agrégats sont suivis par les banques centrales pour évaluer et gérer la politique monétaire.
	
	
\subsection{Création de monnaie par un système bancaire diversifié et hiérarchisé}

La monnaie centrale permet aux établissements de crédit, c'est-à-dire aux banques, de se procurer des billets. Elle leur permet également d'acheter des devises étrangères. 

De plus, la monnaie centrale facilite le règlement des soldes de créances et de dettes entre les institutions financières monétaires (IFM) résultant des opérations effectuées par la clientèle à la fin de la journée. 

Les banques peuvent également effectuer des opérations avec l'administration publique (AP). Enfin, elles réalisent des opérations de politique monétaire, telles que les "\textit{Open Market}" et les facilités permanentes, afin d'approvisionner leur compte courant auprès de la banque centrale, que ce soit pour les réserves obligatoires ou les réserves excédentaires.

Ainsi, les établissements de crédit et l'administration publique disposent d'un compte courant auprès de la banque centrale en monnaie centrale.
	
\section{Offre de monnaie}

L'offre de monnaie peut être analysée selon une approche conventionnelle, qui établit une relation entre la base monétaire et la masse monétaire. Cette relation peut être interprétée en termes de multiplicateur ou de diviseur monétaires. 

Dans cette approche, la création de la masse monétaire (M) est proportionnelle à la base monétaire (H), ce qui implique un multiplicateur où la base est considérée comme exogène. En revanche, le diviseur représente une relation inverse, où la base dépend de la masse monétaire, ce qui en fait une base endogène.

Cependant, une approche plus récente et non conventionnelle remet en question cette relation explicative entre H et M. En effet, l'excédent de liquidité (base), résultant des opérations de politique monétaire non conventionnelles, empêche d'établir un lien stable entre la masse monétaire et la base monétaire.

\subsection{Facteurs et conditions de la liquidité bancaire}

Bilan simplifié consolidé de l'Eurosystème

\begin{center}
	\includegraphics[scale=0.6]{../../../Downloads/Screenshot 2025-01-11 at 15-25-03 PARTIE 3 INTERMEDIAIRES FINANCIERS ET CREATION MONETAIRE - C1 MBC 2025.pdf}
\end{center}

\subsubsection{Facteurs autonomes}

Le besoin net de liquidité est affecté par quatre types d'opérations. 

Tout d'abord, les billets en circulation influencent la liquidité de manière inverse : lorsque le montant de billets en circulation augmente, la liquidité diminue, entraînant ainsi un retrait de liquidité. 

Ensuite, les dépôts des administrations publiques, qui représentent la monnaie centrale détenue en compte courant auprès de la banque centrale, ont également un impact inverse sur la liquidité. En effet, une augmentation du montant des dépôts des administrations publiques entraîne un retrait de liquidité.

De plus, d'autres facteurs nets contribuent également à un retrait de liquidité. 

Enfin, les avoirs nets de change, incluant l'or et les devises, influencent la liquidité dans le même sens que le montant des avoirs nets de change. Ainsi, une augmentation de ces avoirs se traduit par un apport de liquidité.

\subsubsection{Réserves}

Les réserves obligatoires (RO) sont un instrument de politique monétaire permettant d'accroître la liquidité bancaire. Les RO correspondent aux avoirs des établissements de crédit (EC) en compte courant auprès de la Banque centrale.

Avant septembre 2023, les RO étaient rémunérées à un taux de 1\%, qui était en vigueur depuis 2012 (auparavant à 2\%).

L'assiette des RO comprend les dépôts à vue, les dépôts à terme inférieurs ou égaux à 2 ans, les dépôts remboursables avec un préavis inférieur ou égal à 3 mois, les pensions, les titres d'organismes de placement collectif (OPC) monétaires, ainsi que les titres de créances d'une durée inférieure ou égale à 2 ans.

Les réserves totales sont composées des RO et des réserves excédentaires (RE), qui correspondent à la facilité de dépôt.

Les Réserves excédentaires étaient rémunérées de manière négative de 2014 à juillet 2022. Depuis décembre 2022, elles sont rémunérées au taux de la facilité de dépôt.

De octobre 2019 à juillet 2022, un système de rémunération à deux paliers était en place : les RE jusqu'à 6 fois le montant des RO étaient rémunérées à 0\%, le reste étant rémunéré de manière négative à -0,5\%. Ce système de tiering a été suspendu depuis.

\begin{wrapfigure}{r}{0.7\textwidth}
	\centering
\includegraphics[scale=0.5]{../../../Pictures/Screenshots/Capture d'écran 2025-01-26 213812}
\end{wrapfigure}
Ce graphique, issu de la BCE, illustre l'évolution de la liquidité fournie par les opérations d'open market et de l'excédent de liquidité dans la zone euro entre 2016 et 2023.  

Les portefeuilles d'achats fermes de titres (en jaune) ont fortement augmenté à partir de 2020, reflétant les mesures de politique monétaire expansionniste mises en œuvre en réponse à la pandémie de COVID-19. Cette composante a atteint un pic en 2022 avant de diminuer progressivement.  

Les opérations de crédit (en bleu) ont suivi une dynamique croissante jusqu'en 2021, puis une baisse marquée, indiquant un reflux des opérations de refinancement ciblé.  

L'excédent de liquidité (courbe rouge) a suivi une tendance haussière jusqu'en 2022, avant de se contracter en 2023, suggérant une absorption progressive des liquidités excédentaires, probablement en lien avec le resserrement monétaire.  

Cette évolution reflète les ajustements de la politique monétaire de la BCE, marquée par une phase d'expansion pour soutenir l'économie, suivie d'une normalisation face aux pressions inflationnistes

\subsection{Approches conventionnelle et non conventionnelle de la création de monnaie}
	
\subsubsection{Approche conventionnelle}



La base monétaire (H) correspond à la monnaie émise par la Banque centrale. Elle se compose des billets en circulation et de la monnaie centrale.

La monnaie centrale comprend les réserves, qui sont elles-mêmes composées des réserves obligatoires (RO) et des réserves excédentaires (RE). Les réserves correspondent aux avoirs des établissements de crédit (EC) en compte courant auprès de la Banque centrale, ainsi qu'à leurs avoirs à la facilité de dépôt.

La base monétaire figure au passif du bilan de la Banque centrale, tandis que ses contreparties se trouvent à l'actif du bilan.
	
\subsubsection{Base monétaire}
	
\begin{wrapfigure}{r}{0.5\textwidth}
	\centering
\includegraphics[scale=0.5]{../../../Pictures/Screenshots/Capture d'écran 2025-01-26 215458}
\end{wrapfigure}

Le graphique de la BCE illustre l'évolution de la base monétaire de la zone euro entre 1999 et 2017. On observe une tendance générale à la hausse, avec une accélération marquée après la crise financière de 2008. Cette augmentation reflète les politiques monétaires non conventionnelles mises en œuvre par la BCE pour stimuler l'économie. La composition de la base monétaire montre une prédominance des comptes courants des banques, soulignant le rôle central du système bancaire dans la création monétaire. Ces évolutions ont eu des implications significatives pour la stabilité des prix, l'activité économique et la stabilité financière de la zone euro.

\begin{center}
	\includegraphics[scale=0.55]{../../../Pictures/Screenshots/Capture d'écran 2025-01-26 215959}
\end{center}
	
\subsubsection{Mécanisme de l'expansion monétaire}
	
Soit une banque disposant ex-ante de réserves excédentaires (RE) auprès de la Banque centrale pour un montant de 1000.

Le coefficient de réserves obligatoires (b = B/D) est de 20\% et le taux de rémunération des réserves obligatoires (r) est de 1\%.

Dans ce cadre, la création de monnaie par le multiplicateur monétaire se déroule de la manière suivante :
\begin{center}
	
\includegraphics[scale=0.8]{../../../Pictures/Screenshots/Capture d'écran 2025-01-26 220820}
	
\end{center}
	
\paragraph{Modèle de création monétaire}

On considère un système bancaire où :
\begin{itemize}
	\item $RE$: réserves excédentaires
	\item $b$: coefficient de réserve
	\item $r$: taux de réserve obligatoire
	\item $D$: dépôts
	\item $C$: création de monnaie (équivalent à $\Delta C$)
	\item $\Delta M$: variation de la masse monétaire totale
	\item $k$: multiplicateur de crédit
\end{itemize}

\paragraph{Processus de création monétaire}

Soit $C_0$ un dépôt initial. La banque conserve une fraction $b$ en réserve et prête le reste $(1-b)C_0$, créant ainsi un nouveau dépôt $C_1$. Ce processus se répète.

\paragraph{Calcul de la variation totale de la masse monétaire:}

$$\Delta M = \Delta C = C_0 + (1-b)C_0 + (1-b)^2C_0 + ... = C_0 \sum_{i=0}^{\infty} (1-b)^i$$

En utilisant la formule de la somme d'une série géométrique, on obtient :

$$\Delta M = \frac{C_0}{1-(1-b)} = \frac{C_0}{b}$$

\paragraph{Multiplicateur de crédit:}

On peut généraliser cette formule en fonction des réserves excédentaires initiales $RE$:

$$\Delta M = RE \cdot \frac{1}{b + r(1-b)} = RE \cdot k$$

où $k = \frac{1}{b + r(1-b)}$ est le multiplicateur de crédit.
	
\subsubsection{Relation macroéconomique entre M et H}
	
On suppose que la masse monétaire (M) est composée de billets de banque (B) et de dépôts à vue (DAV). Les billets représentent une proportion b de la masse monétaire. Le montant des réserves obligatoires des banques auprès de la Banque centrale est une proportion r des dépôts à vue. Les banques détiennent aussi des réserves excédentaires pour une proportion e des dépôts à vue.

\begin{align*}
	M &= B + DAV \\
	B &= b \times M \\
	DAV &= (1 - b) \times M \\
	R &= (r + e) \times DAV \\
	R &= (r + e) \times (1 - b) \times M
\end{align*}

La base monétaire (H) comprend les billets de banque et les réserves :
\begin{align*}
	H &= B + R \\
	&= b \times M + (r + e) \times (1 - b) \times M \\
	&= [b + (r + e) \times (1 - b)] \times M
\end{align*}

Si on pose $d = [b + (r + e) \times (1 - b)]$, alors : $H = d \times M$ où $d$ représente le diviseur, $d < 1$. Le lien de causalité va de la masse vers la base dans le cas du diviseur.

La relation peut s'écrire dans l'autre sens : $M = 1/[b + (r + e) \times (1 - b)] \times H$. Si on pose $k = 1/[b + (r + e) \times (1 - b)]$, alors : $M = k \times H$ où $k$ est le multiplicateur, $k > 1$. Le lien de causalité va de la base vers la masse dans le cas du multiplicateur.
	
Cependant, le multiplicateur $k = 1/d$ n'est pas toujours acceptable. En effet, il n'est acceptable que s'il est stable ou une fonction stable des coefficients caractéristiques (b, r, e), et si la base monétaire H est exogène. Or, dans le cas présent, le multiplicateur est variable car il dépend de b, r et e, et la base monétaire H est endogène, car elle dépend de la masse monétaire M.

Mais il existe des cas où le multiplicateur est acceptable. En effet, si les coefficients caractéristiques (b, r, e) sont stables ou forment une fonction stable, et si la base monétaire H est exogène, alors le multiplicateur $k = 1/d$ sera également stable et acceptable.

\subsubsection{Approche non conventionnelle}

\begin{center}
	\includegraphics[scale=0.7]{../../../Pictures/Screenshots/Capture d'écran 2025-01-26 222812}
\end{center}


Le graphique ci-dessus, issu de la Banque Centrale Européenne (BCE), illustre l'évolution de la base monétaire, de la monnaie au sens large (M3) et du multiplicateur monétaire au sein de la zone euro entre 1999 et 2017. On observe une croissance continue de la base monétaire et de M3 sur cette période, avec une accélération marquée après la crise financière de 2008, reflétant les mesures de politique monétaire non conventionnelles mises en œuvre par la BCE. Le multiplicateur monétaire, qui mesure le rapport entre M3 et la base monétaire, a quant à lui connu des fluctuations plus importantes, notamment en lien avec les différentes opérations de refinancement à long terme (LTRO) menées par la BCE. Ces évolutions soulignent l'impact des politiques monétaires sur la création de monnaie et la transmission de ces politiques à l'économie réelle.
	
\begin{center}
	\includegraphics[scale=0.9]{../../../Pictures/Screenshots/Capture d'écran 2025-01-26 223134}
\end{center}
	
Le graphique ci-dessus présente l'évolution du multiplicateur monétaire dans la zone euro et aux États-Unis depuis 1999. Le multiplicateur monétaire, calculé comme le rapport entre la masse monétaire au sens large (M3 en zone euro, M2 aux États-Unis) et la base monétaire, reflète l'efficacité avec laquelle les banques commerciales créent de la monnaie à partir des réserves bancaires.

On observe que le multiplicateur monétaire a connu des fluctuations importantes dans les deux zones monétaires. Dans l'ensemble, il a tendance à diminuer au cours de la période considérée, suggérant une baisse de la capacité des banques à créer de la monnaie. Cette diminution pourrait être liée à plusieurs facteurs, tels qu'une augmentation des exigences de fonds propres, une aversion accrue au risque et des changements dans le comportement des agents économiques.

Les politiques monétaires mises en œuvre par les banques centrales, notamment les opérations de refinancement à long terme (LTRO) et les achats d'actifs, ont également pu influencer l'évolution du multiplicateur monétaire.	
	
\subsection{Marché du crédit et marché interbancaire}
	
\begin{center}
	\includegraphics[scale=0.7]{../../../Pictures/Screenshots/Capture d'écran 2025-01-26 224225}
\end{center}
	
À partir du bilan simplifié de la banque, on obtient :  
\[
RF = C + r(1-b) M - (1-b) M \quad (1)
\]
Si on pose que les contreparties de la masse monétaire sont les crédits (C) plus les contreparties externes exogènes (X) :  
\[
M = C + X
\]
En remplaçant dans (1) \( M \) par ses contreparties, on obtient :  
\[
RF = C (b+r-rb) - X (1-b-r+rb)
\]
Le résultat financier (RF) de la banque centrale dépend positivement du taux de refinancement \( i_{RF} \), qui correspond au taux des opérations principales de refinancement (OPR). L'offre de monnaie centrale est ainsi contrôlée par la banque centrale.  

Les dépôts de la clientèle sont rémunérés au taux créditeur \( i_C \). Pour la banque, le coût du crédit comprend le coût de collecte des dépôts ainsi que le coût du refinancement auprès de la banque centrale, ce qui se traduit par l'expression suivante :  
\[
i_C (1-b) M + i_{RF} RF \quad (2)
\]
De plus, le taux de refinancement est défini comme suit :  
\[
i_{RF} = i_{RF}(RF), \quad \frac{\partial i_{RF}}{\partial RF} \geq 0
\]

En remplaçant \( M \) et \( RF \) par leur valeur dans l'équation (2), on obtient :  
\[
i_C (1-b) (C+X) + i_{RF} \left[ C (b+r-rb) - X (1-b-r+rb) \right]
\]
La fonction de profit bancaire s'écrit :  
\[
\pi = i_D C - \left[ i_C (1-b) (C+X) + i_{RF} \left[ C (b+r-rb) - X (1-b-r+rb) \right] \right]
\]
où \( i_D \) désigne le taux d'intérêt débiteur.  

En supposant un comportement concurrentiel du système bancaire, la fonction d'offre de crédit se déduit de la maximisation de la fonction de profit par rapport au volume de crédit.  

Les conditions du premier ordre déterminent la relation liant l'offre de crédit aux différentes variables du modèle.

La fonction d'offre \( C_s \) dépend positivement :  
\begin{itemize}
	\item du taux débiteur et des contreparties externes.
\end{itemize}

et négativement :  
\begin{itemize}
	\item du taux créditeur, du taux de réserves obligatoires, du taux de refinancement et de la part des billets dans la masse monétaire.
\end{itemize}

La fonction de demande de crédit \( C_d \) est une fonction croissante de l'activité économique \( y \) et décroissante du taux débiteur \( i_D \).  

À l'équilibre sur ce marché, l'offre de crédit est égale à la demande de crédit.  

Imaginons une hausse de la demande de crédit :  
\begin{itemize}
	\item Déplacement de la courbe de demande \( C_d' \).
	\item Cela conduit à un nouvel équilibre sur le marché du crédit avec un taux débiteur \( i_D \) plus élevé.
\end{itemize}

Cette hausse du crédit provoque une hausse de la demande de refinancement des banques auprès de la banque centrale, ce qui entraîne un déplacement de la courbe \( RF_d' \).  

L'incidence de cette hausse dépend de la fonction d'offre de refinancement de la banque centrale \( RF_s \).  

La banque centrale augmente son taux de refinancement \( i_{RF} \), ce qui correspond à un nouvel équilibre.  

La création de monnaie se déduit de l'interaction entre les fonctions d'offre et de demande de refinancement et les fonctions d'offre et de demande de crédit.

	
\begin{center}
	\includegraphics[scale=0.7]{../../../Pictures/Screenshots/Capture d'écran 2025-01-26 224840}
\end{center}

Ciblage du taux \( i_{RF} \) : La banque centrale contrôle le taux de refinancement \( i_{RF} \), ce qui l'oblige à satisfaire toute la demande de refinancement des banques au taux fixé.  

Le volume de refinancement \( RF \) est insensible aux variations du taux de refinancement.  

Tout déplacement de la demande de crédit entraîne mécaniquement un déplacement de la demande de refinancement, soit \( \Delta RF = (b+r-rb) \Delta C \).  

Toute la demande de crédit des agents est satisfaite au taux débiteur \( i_D \) affiché.  

Le diviseur : le sens de la causalité va de la masse monétaire \( M \) vers la base monétaire \( H \).  

Cela entraîne une création de monnaie endogène.

\begin{center}
	\includegraphics[scale=0.7]{../../../Pictures/Screenshots/Capture d'écran 2025-01-26 225233}
\end{center}
	
La banque centrale (BC) : il n'y a pas une politique monétaire exclusive de taux ou de base, mais un arbitrage entre les deux.  

Depuis 2009, l'excédent de liquidité (\( H \)) ne détermine plus la masse monétaire (\( M \)).  

Il n'y a plus d'effet de multiplicateur.  

La création de \( M \) est contrainte par la demande de monnaie et la réponse apportée par les banques à l'octroi de crédit.
	
\section{Demande de monnaie}	
	
\subsection{Premières approches sur la demande de monnaie}
	
\subsubsection{Équation des échanges}
	
I. Fisher : Première reformulation de la théorie quantitative de la monnaie des classiques (Ricardo, Say, Walras).  

L'équation des échanges, parue dans « The Purchasing Power of Money » en 1911 :  
« \textit{Dans chaque vente ou achat, la monnaie et les biens échangés sont ipso facto équivalents… }»  
\begin{center}
	Le membre de la monnaie est le total de la monnaie payée et peut être considéré comme le produit de la monnaie par sa vitesse de circulation. Le membre des biens se compose du produit de la quantité des biens échangés par leurs prix respectifs.
\end{center}  

Où \( M \) est la quantité de monnaie matérielle (pièces, billets), \( M' \) est la quantité de monnaie scripturale, \( V \) est la vitesse de circulation de \( M \) (nombre moyen de fois qu'une unité monétaire a circulé pour effectuer une transaction), \( V' \) est la vitesse de circulation de \( M' \), \( P \) est le niveau général des prix (NGP) et \( T \) est le volume global des transactions pendant la période donnée.  

Identité comptable : D'après la théorie quantitative, toute augmentation de \( M \) entraîne une augmentation proportionnelle de \( P \).  

Cela est vérifié si \( V \) et \( T \) sont constants.  

Hypothèses :  
\begin{itemize}
	\item Neutralité de la monnaie : \( T \) ne varie pas quand \( M \) augmente. La monnaie n'influence pas les quantités produites ni l'emploi. \( M \), introduite après dans l'équation, ne modifie pas l'équilibre.
	\item \( V \) est supposée constante à court terme (CT) ou converger vers une valeur d'équilibre \( V^\star \). À long terme (LT), elle dépend de facteurs structurels : habitudes de paiement, densité de la population, etc.
\end{itemize}
	
\subsubsection{Impact de l'équation}

L'équation de Fisher :  
\[
M \times V = P \times Y
\]
où \( T \) est remplacé par \( Y \) (production ou PIB).  

En termes de taux de croissance, on a :  
\[
\frac{\Delta M}{M} + \frac{\Delta V}{V} = \frac{\Delta P}{P} + \frac{\Delta Y}{Y}
\]

\begin{center}
	\includegraphics[scale=0.7]{../../../Pictures/Screenshots/Capture d'écran 2025-01-26 230716}
\end{center}
	
Le lien entre la masse monétaire (M) et le niveau général des prix (NGP) est plus complexe que ne le suggère la théorie quantitative de la monnaie. En effet, la vitesse de circulation de la monnaie (V) n'est pas constante et peut varier en fonction de nombreux facteurs tels que l'incertitude économique, l'évolution des primes de risque, les innovations financières ou les changements réglementaires. Par conséquent, une augmentation de la masse monétaire ne se traduit pas nécessairement par une hausse proportionnelle du niveau général des prix. De plus, une partie de la liquidité injectée dans l'économie peut être utilisée pour acquérir des actifs financiers, ce qui entraîne une hausse des prix de ces actifs plutôt qu'une inflation généralisée. Des études empiriques, comme celle de Bruggeman (2007), ont d'ailleurs montré que dans de nombreux cas, un excès de liquidité peut conduire à une hausse des prix d'actifs plutôt qu'à une inflation généralisée, soulignant ainsi la transmission non automatique de la politique monétaire. Néanmoins, des recherches plus récentes, telles que celles de Nicolini et Weber (2020), suggèrent que la masse monétaire conserve un impact sur le niveau général des prix à long terme, ce qui incite les banques centrales à continuer de surveiller cet agrégat monétaire parmi d'autres indicateurs pour atteindre leurs objectifs de stabilité des prix.
	
\begin{center}
	\includegraphics[scale=0.7]{../../../Pictures/Screenshots/Capture d'écran 2025-01-26 230833}
\end{center}	


\paragraph{Résumé}
\begin{itemize}
	\item Il y a à court terme (CT) et moyen terme (MT) un relâchement de la relation entre \(M\) et \(NGP\).
	\item Les effets des politiques monétaires non conventionnelles (PMNC) sur l'activité et les prix transitent par d'autres canaux que celui de \(M\).
	\item Les programmes d'achats d'actifs affectent la courbe des taux et les conditions de financement.
\end{itemize}

\subsection{Demande de monnaie transactionnelle}

\subsubsection{École de Cambridge}


A. C. Pigou dans "\textit{The Value of Money}" (1917) et A. Marshall dans "\textit{Money, Credit and Commerce}" (1923) ont analysé les motifs de détention d'une encaisse monétaire.
	
Théorie quantitative deviendra théorie de la demande de monnaie

Fonction de demande (ou d'encaisses) pour le motif de transaction déduite de l'équation des échanges :

$M \times V = P \times T \Leftrightarrow M = \frac{P \times T}{V} \Leftrightarrow P = \frac{M \times V}{T}$

Offre et demande de monnaie : $M^s = M^d$

$M^d$ remplace $M$ et $\frac{1}{V}$ par $k$

$M^d = k \times P \times Y$

où $M^d$ = demande de monnaie, $P$ = niveau général des prix, $Y$ = revenu national réel, $k$ = constante (dépend de facteurs structurels)

Monnaie détenue pour son rôle intermédiaire des échanges

Les agents demandent de la monnaie (détention de stock de monnaie) car existence d'1 décalage entre les recettes (revenus) et les dépenses

Pouvoir achat du stock de monnaie : $M^d/P$

Monnaie neutre n'agit pas sur les équilibres réels

Si $M^o$ augmente comme $M^o = M^d$, la demande de monnaie nominale (encaisse effective nominale) augmente, mais pouvoir d'achat $M^d / P$ reste identique car si $M^o$ augmente $P$ augmente aussi

Agents pas victime de l'illusion monétaire

Effet d'encaisses réelles

2.2 La demande de monnaie en termes de gestion de stock

M. Allais (1947) dans son ouvrage "\textit{Economie et Intérêt}" et W. Baumol (1952) dans "\textit{The transaction demand for cash: an inventory theoretic approach}"

Intérêt de la théorie de la gestion des stocks :
\begin{itemize}
	\item Demande de monnaie pour motif de transaction
	\item Titres, coûts de transaction, temps pour liquider titres, commissions
	\item Monnaie : économiser des coûts de transaction
	\item Agents subissent coût d'opportunité en détenant de la monnaie (perte en termes d'intérêt)
\end{itemize}

Problème : trouver le nombre optimal de conversion de titres en monnaie de façon à minimiser coûts d'opportunité et coûts de transaction

Dans le modèle :
\begin{itemize}
	\item $Y$: revenu annuel de l'agent en début d'année, $r$ : taux d'intérêt annuel constant perdu sur titres qui sont achetés pour répondre à ses dépenses
	\item Agent vendent régulièrement 1 fraction constante de titres pour répondre à ses dépenses qui sont échelonnées dans le temps.
	\item $b$ représente le coût de conversion de titres en monnaie et $n$ le nombre de conversion de titres en monnaie par an.
	\item L'agent place $(\frac{Y}{n})$ en titres et conserve en monnaie pour ses dépenses immédiates en début d'année.
\end{itemize}

Son revenu moyen $R$ se calcule de la façon suivante : $R = r \times (\frac{Y - \frac{Y}{n}}{2})$.

Modèle de gestion de stock de monnaie (Allais, Baumol)

Revenu et coûts
Revenu moyen:
$$R = r \cdot \frac{Y - \frac{Y}{n}}{2}$$
Revenu net:
$$R - bn = r \cdot \frac{Y - \frac{Y}{n}}{2} - bn$$

Optimisation du nombre de conversions
\begin{itemize}
	\item Problème: Trouver le nombre optimal de conversions ($n^*$) qui maximise le revenu net.
	\item Résolution:
	\begin{enumerate}
		\item Calcul de la dérivée: $\frac{d}{dn} \left( r \cdot \frac{Y - \frac{Y}{n}}{2} - bn \right) = 0$
		\item Simplification: $\frac{rY}{2n^2} - b = 0$
		\item Solution: $n^* = \sqrt{\frac{rY}{2b}}$
	\end{enumerate}
\end{itemize}

 Demande de monnaie

Encaisse moyenne: $\frac{Y}{2n^*}$
Demande de monnaie: $M^d = \frac{Y}{2n^*} = \sqrt{\frac{bY}{2r}}$

Interprétation

Ce modèle montre que la demande de monnaie est une fonction croissante du revenu et du coût de transaction, et décroissante du taux d'intérêt. Les agents économiques cherchent à optimiser leur gestion de trésorerie en minimisant les coûts de transaction liés à la conversion des titres en monnaie tout en profitant des intérêts offerts par les placements.

Explications des notations:
\begin{itemize}
	\item $Y$: Revenu annuel de l'agent
	\item $r$: Taux d'intérêt
	\item $n$: Nombre de conversions de titres en monnaie par an
	\item $b$: Coût par conversion
	\item $M^d$: Demande de monnaie
\end{itemize}

Ce modèle est un classique en économie monétaire et permet de comprendre comment les individus gèrent leur liquidité en fonction des coûts et des opportunités de placement.

	
\subsection{Demande de monnaie patrimoniale}

\subsubsection{La fonction de demande de monnaie chez Keynes}

J. Maynard Keynes, dans son "\textit{Traité sur la monnaie}" (1930) et sa "\textit{Théorie générale de l'emploi, de l'intérêt et de la monnaie}" (1936), définit la monnaie comme :
\begin{itemize}
	\item Une unité de compte, un intermédiaire des échanges et une réserve de valeur.
	\item Le revenu se partage entre consommation (C) et épargne (S) ; l'épargne peut prendre deux formes : soit sous forme de monnaie, soit sous forme de titres. Les titres, tels que les rentes perpétuelles, sont rémunérés.
	\item Les titres représentent un droit de consommation différée, tandis que la monnaie représente un droit de consommation immédiate.
	\item L'intérêt versé sur les titres est considéré comme une " \textit{récompense pour la renonciation à la liquidité}".
	\item Les titres rapportent un taux de rentabilité, qui peut fluctuer en fonction de la valeur en capital.
\end{itemize}

En situation d'incertitude, les agents évaluent le taux de rentabilité anticipé.

La détention de liquidité pour motif de spéculation est influencée par l'incertitude liée à l'évolution future des taux d'intérêt. Les agents économiques prennent des décisions basées sur leurs anticipations concernant ces taux.

Si les agents anticipent une baisse des taux futurs, cela entraîne une hausse du prix des titres. Dans ce cas, ils préfèrent acheter des titres maintenant. À l'inverse, lorsqu'ils anticipent une hausse future des taux, ils peuvent choisir de détenir plus de liquidités.

Pour déterminer son encaisse spéculative optimale, l'agent compare le taux de rentabilité anticipé des titres, noté \( r_e \), à celui de la monnaie, qui est \( r_e = 0 \). 

\begin{itemize}
	\item Si \( r_e > 0 \), l'agent détient uniquement des titres.
	\item Si \( r_e < 0 \), il détient uniquement de la monnaie.
\end{itemize}

Il est important de noter que l'agent ne peut pas détenir simultanément de la monnaie et des titres

Lorsque \( r_e = 0 \), on atteint le taux d'intérêt critique noté \( i_c \). À ce stade, l'agent est indifférent entre la détention de titres et celle de la monnaie. 

La comparaison entre le taux de marché \( i \) et le taux d'intérêt critique \( i_c \) détermine le choix de l'agent :
\begin{itemize}
	\item Si \( i > i_c \), l'agent détient des titres à 100 \% et pas de monnaie.
	\item Si \( i < i_c \), l'agent détient de la monnaie à 100 \%.
	\item Si \( i = i_c \), l'agent est indifférent entre les deux options.
\end{itemize}

Concernant la demande de monnaie spéculative, il est important de noter que la fonction macro de demande de monnaie spéculative ne résulte pas simplement de l'agrégation des encaisses spéculatives individuelles. Cela s'explique par le fait que les agents n'ont pas les mêmes anticipations sur l'évolution future des taux d'intérêt. 

Ainsi, il existe autant de taux d'intérêt critiques \( i_c \) que d'agents.

La demande agrégée d'encaisse spéculative est une fonction continue et décroissante du taux d'intérêt. Elle comprend deux situations extrêmes :

\begin{itemize}
	\item Lorsque le taux d'intérêt sur le marché \( i \) est élevé, les agents détiennent des titres et pas d'encaisses spéculatives. Dans ce cas, la demande de monnaie spéculative est inélastique par rapport au taux d'intérêt.
	\item À l'inverse, lorsque le taux d'intérêt atteint un niveau minimal, les agents estiment qu'il ne descendra jamais en dessous de ce seuil. Ils choisissent alors de détenir leur richesse (réserve de valeur) sous forme d'encaisses spéculatives. Dans cette situation, la demande de monnaie spéculative est infiniment élastique au taux.
\end{itemize}

Il existe trois motifs principaux pour la détention de liquidité (monnaie) : 
\begin{itemize}
	\item La transaction,
	\item La précaution,
	\item La spéculation.
\end{itemize}

Les encaisses peuvent être classées selon différents motifs :

\begin{itemize}
	\item Motif de transaction : Ce motif est lié au décalage temporel entre les recettes et les dépenses.
	\item Motif de précaution: Il s'agit de répondre à des transactions imprévues.
\end{itemize}

Les encaisses de transaction et de précaution sont regroupées dans une même fonction de liquidité, notée \( L_1(Y) \), qui dépend positivement du revenu courant \( Y \). En revanche, les encaisses spéculatives, notées \( L_2(i) \), dépendent négativement du taux d'intérêt \( i \).

La demande totale d'encaisses peut être exprimée sous forme additive :
\[
M_d = L_1(Y) + L_2(i), \quad M_d = f(Y, i)
\]
L'équilibre sur le marché de la monnaie est donné par \( M_o = M_d \), où \( M_o \) est exogène, soit \( M_o = M \).

La vitesse de circulation du revenu \( V \) est définie comme :
\[
V = \frac{P \cdot Y}{M_d}
\]
Il est important de noter que \( V \) est variable et instable, dépendant du taux d'intérêt \( i \). Par conséquent, la politique monétaire (PM) n'est pas un bon instrument de politique économique.

\subsection{La prise en compte de la diversification du portefeuille dans la demande de monnaie}
	
Tobin (1958) propose que la préférence pour la liquidité est liée au comportement face au risque. Selon lui :

L'agent peut diversifier son portefeuille en détenant simultanément de la monnaie (encaisse liquide) non rémunérée et des titres rémunérés (avec une rentabilité aléatoire).

Ce choix implique une décision entre un actif sans risque (la monnaie) et des actifs risqués (les titres). L'objectif est de déterminer la composition optimale du portefeuille afin que l'agent maximise l'utilité qu'il en retire tout en minimisant le risque associé.

Sous l'hypothèse de maximisation de l'utilité espérée, le portefeuille optimal dépend de plusieurs facteurs :
\begin{itemize}
	\item Le degré d'aversion de l'individu envers le risque (averse ou en amour du risque),
	\item Sa richesse,
	\item Les caractéristiques de la distribution des rentabilités des actifs, notamment la moyenne et la variance.
\end{itemize}

Un agent qui détient plus de titres dans son portefeuille :

- Présente un risque total du portefeuille plus élevé, mais bénéficie d'une rentabilité supérieure (individu risquophile).

En revanche, s'il détient davantage de monnaie :

- Le risque total est moins élevé, mais sa rentabilité l'est aussi (individu risquophobe). 

La monnaie est considérée comme un actif sans risque qui permet de diminuer le risque total du portefeuille. En ce sens, la monnaie agit comme une réserve de valeur.

Une augmentation du taux d'intérêt exerce deux effets contradictoires :

\begin{itemize}
	\item Effet de substitution : Lorsque les titres sont mieux rémunérés, l'individu est incité à demander moins de monnaie et à en détenir davantage de titres. Pour conserver un niveau d'utilité donné, l'individu accepte un risque plus élevé en échange d'une rentabilité accrue, remplaçant ainsi la monnaie par des titres.
	\item Effet revenu : En desserrant la contrainte patrimoniale, la rentabilité plus élevée des titres peut réduire la demande de monnaie, amenant l'individu à en détenir davantage.
\end{itemize}

Pour un individu risquophobe (averse au risque), la demande de monnaie décroît avec le taux d'intérêt si l'effet de substitution l'emporte sur l'effet revenu. En effet, la monnaie sert de couverture face au risque et constitue également un moyen de diversifier son portefeuille d'actifs dans un environnement risqué.

Les variables déterminantes pour expliquer la demande de monnaie sont le taux d'intérêt et la taille du portefeuille (ou le niveau de richesse). Il est important de noter que la monnaie au sens strict ne rapporte pas d'intérêt, alors qu'il existe d'autres formes de monnaie qui donnent droit à des intérêts (comme les comptes sur livrets).

La monnaie est dominée par ces autres actifs monétaires. Il est intéressant de se demander quelles sont les différences entre ces autres formes de monnaie (quasi-monnaie) et les titres. Enfin, la théorie de Tobin explique la demande d'actifs risqués, mais ne rend pas compte de la demande de monnaie.
	
\subsection{La demande de monnaie chez Friedman}

Friedman (1956) propose une reformulation de la théorie quantitative de Fisher dans son ouvrage \textit{"The Quantity Theory of Money}". 

Il introduit une théorie unitaire de la demande de monnaie, s'opposant à l'approche de Keynes qui considère la demande de monnaie comme une fonction additive.

Cette théorie prolonge la théorie du choix de portefeuille de Tobin, mais dans un cadre plus large de gestion de patrimoine. Elle s'inscrit dans une branche particulière de la théorie du capital.

Dans cette perspective, la monnaie est considérée comme un actif parmi d'autres actifs patrimoniaux, représentant une forme de détention de la richesse. La richesse \( W \) comprend tous les actifs : 

\begin{itemize}
	\item Monnaie,
	\item Actifs financiers (obligations, actions),
	\item Actifs réels (immobiliers, consommation, production),
	\item Actifs humains (capacité à travailler).
\end{itemize}

Les agents choisissent simultanément entre tous ces actifs, qui sont parfaitement substituables entre eux. L'agent cherche ainsi à déterminer la composition optimale de son patrimoine \( W \).

La fonction de demande de monnaie est déduite d'un programme de maximisation d'une fonction d'utilité espérée $E(u)$ sous une contrainte patrimoniale. Une forme simplifiée de cette fonction, correspondant à une gestion optimale du patrimoine, est donnée par :

$$\frac{M^d}{P} = f(Y^p, i_b - i_m, i_s - i_m, \pi^e - i_m)$$

où :
\begin{itemize}
	\item $M^d/P$ représente la demande d'encaisses réelles
	\item $Y^p$ est le revenu permanent (valeur actualisée de tous les revenus futurs anticipés)
	\item $i_b$, $i_s$, et $i_m$ sont respectivement les taux de rentabilité anticipés des obligations, des actions, et de la monnaie
	\item $\pi^e$ est le taux d'inflation anticipé
\end{itemize}

Cette équation montre que la demande de monnaie réelle varie positivement avec le revenu permanent $Y^p$ et négativement avec les différentiels de taux de rentabilité anticipés.

L'action directe de la quantité de monnaie \( M \) a un impact sur la demande de tous les actifs et influence tous les marchés, qu'ils soient réels ou financiers. Une variation de \( M_0 \) agit directement sur la variation du produit \( \Delta Y \) à court terme, tandis qu'à long terme, elle affecte le niveau général des prix (NGP).

Friedman minimise le rôle des taux d'intérêt à court terme dans la demande de monnaie, arguant que les différentiels de taux de rentabilité ont peu d'effet sur cette demande. Dans sa théorie, un rôle déterminant est attribué à \( Y_p \) dans la fonction de demande. De plus, contrairement à la théorie de Fisher, la vitesse de circulation de la monnaie n'est plus constante, mais stable.

Il simplifie l'équation de demande de monnaie : 

\[\frac{M^d}{P} = f(Y^p) \]

et l'offre de monnaie est exogène : $M^o = M$

$Y^p$ stable dans le cycle économique, donc $M^d$ varie mais est relativement stable car fonction de variable stable.

$M^d$ prévisible et politique monétaire utile

Équilibre marché de la monnaie $M^o = M^d \Rightarrow M = P f(Y^p)$

Vitesse-revenu s'écrit : \( V = \frac{PY}{M} \)
\[ V = \frac{PY}{Pf(Y^p)} \Rightarrow V = \frac{Y}{Y^p}\]

\chapter{Banques centrales}

Une banque centrale (BC) est une institution de droit public ou privé, chargée par délégation des pouvoirs publics d'accomplir diverses missions d'intérêt général. Parmi ces missions, on trouve :

\begin{itemize}
	\item La gestion de la monnaie sur le plan interne, notamment en ce qui concerne le taux d'inflation, et sur le plan externe, avec le taux de change.
	\item La gestion du système de paiement et de règlement.
	\item La surveillance macro et micro prudentielle du système financier.
	\item Le service et la diffusion d'informations monétaires et financières.
\end{itemize}

La plupart des pays disposent d'une banque centrale indépendante, bien que certaines soient contrôlées par l'État. La majorité des banques centrales ont pour objectif d'assurer la stabilité des prix, ce qui implique de maintenir la valeur interne de la monnaie (ancrage nominal). Cependant, certaines banques centrales peuvent avoir deux objectifs, ce qui peut engendrer des conflits potentiels dans leur réalisation.

Dans les mandats des banques centrales, les objectifs sont précisés et inscrits dans la loi, que ce soit dans la constitution, un traité ou un statut, ou encore en dehors du cadre législatif. 

Depuis les crises des dernières années, les banques centrales doivent coordonner leur politique monétaire avec d'autres politiques économiques, telles que la politique budgétaire. Cela remet en question l'indépendance des banques centrales.

\section{Fonctions d'une banque centrale}


\subsection{Émergence et évolution}

L'évolution des banques centrales peut être divisée en quatre étapes principales :

\begin{itemize}
	\item \textbf{XVII et XVIIIème siècles :} Création des premiers instituts d'émission, tels que la Banque de Suède en 1668, la Banque d'Angleterre en 1694, et la Banque de France en 1800. Ces banques émettent des billets contre l'escompte de lettres de change, devenant ainsi une ressource pour l'État et assurant le financement des guerres. Elles offrent également du crédit à tous types d'agents économiques.
	
	\item \textbf{XIXème siècle :} Les besoins de financement augmentent en raison de l'industrialisation des économies. Les banques centrales se voient accorder le privilège d'émission des billets, et la Banque de France obtient ce privilège sur l'ensemble du territoire en 1848.
	
	\item \textbf{XXème siècle :} Après la Seconde Guerre mondiale, les banques centrales sont nationalisées pour aider à la reconstruction des pays et pour exercer le caractère régalien de battre monnaie. Elles mettent en œuvre la politique monétaire décidée par les États, et la plupart des banques centrales acquièrent leur indépendance au cours des années 1990.
	
	\item \textbf{XXIème siècle :} Les missions des banques centrales évoluent vers la surveillance micro et macro prudentielle, où la stabilité financière devient un enjeu majeur en raison des crises. Cette stabilité prend également en compte le changement climatique. Les débats sur la libre concurrence ou le monopole d'émission de la monnaie se posent, avec des concepts comme le "\textit{Free Banking}" ou le "\textit{Central Banking}".
\end{itemize}

Dans la lignée du \textit{Free Banking}, Hayek (1976) a proposé de dénationaliser la monnaie pour éviter sa manipulation par l'État, arguant que l'émission de monnaie pour financer les dépenses de l'État peut créer des déséquilibres majeurs. Il envisage une concurrence entre monnaies privées émises par des banques de second rang, permettant l'émergence de meilleures monnaies et un oubli des crises à répétition.

Pour les partisans du \textit{Central Banking}, notamment dans les années 1920, les défaillances de marché nécessitent l'intervention des banques centrales, qui injectent de la monnaie centrale pour sauver les banques.

Le \textit{Free Banking} a existé dans les faits en France de 1796 à 1803, en Écosse de 1716 à 1848, ainsi qu'en Angleterre et aux États-Unis avant 1913. Ce long processus d'évolution des banques centrales a conduit à l'imposition du \textit{Central Banking}.

Enfin, les cryptoactifs questionnent le monopole de l'émission monétaire.
\subsection{Missions}

Les missions d'une banque centrale incluent plusieurs aspects clés :

\textbf{Définition et mise en œuvre de la politique monétaire :} La banque centrale est indépendante du gouvernement et des intérêts privés, et elle est responsable de la stabilité monétaire.

L'objectif peut être unique, comme la stabilité des prix (valeur interne de la monnaie) pour la Banque Centrale Européenne (BCE), ou multiple, comme pour la Réserve Fédérale (FED), qui inclut également l'emploi et nécessite une hiérarchisation des objectifs. De plus, la banque centrale gère la stabilité externe de la monnaie en veillant à la relation et à la parité de sa monnaie par rapport aux autres devises sur le marché des changes.

\textbf{Émission de la monnaie de banque centrale (MBC) et gestion des systèmes de paiement et de règlement :} La banque centrale émet de la monnaie centrale, qui est la monnaie ultime circulant entre les institutions financières monétaires (IFM) et l'administration centrale. Elle gère également des systèmes de transfert de fonds de gros montants entre les IFM, contribuant ainsi à la stabilité financière.

Concernant la politique microprudentielle, la banque centrale surveille la stabilité des grands établissements de crédit, dits systémiques. Pour la politique macroprudentielle, elle surveille le système financier dans son ensemble et les risques systémiques, ainsi que l'évolution du secteur non bancaire (IFNB).

\textbf{Service et diffusion de l'information monétaire et financière :} La banque centrale évalue la viabilité des débiteurs à travers des fiches entreprises, assure la sécurité des paiements, réalise des travaux économiques et gère les comptes des administrations publiques.

Enfin, les banques centrales des pays émergents ont davantage de missions en raison de leur rôle dans le développement du système financier.

\subsection{Approfondissement des missions}

L'approfondissement des missions d'une banque centrale se manifeste par plusieurs aspects essentiels :

\subsubsection{Fonction de prêteur en dernier ressort} 

Cette fonction est la raison d'être des banques centrales, surtout en période de crises à répétition. Elles fournissent des liquidités immédiates aux banques pour éviter des runs bancaires. Cela agit comme une assurance collective, incitant les banques à prendre plus de risques. La banque centrale doit être capable de discriminer entre les banques illiquides mais solvables et celles qui sont insolvables. Elle peut procéder à des renflouements (bail out) de banques qualifiées de "\textit{too big to fail}" ou "\textit{too interconnected to fail}". Pour ce faire, elle prend en garantie des actifs de moindre qualité, ce qui peut avoir un impact sur la politique financière de la banque centrale et entraîner une perte de réputation et de crédibilité. Dans ce rôle, la banque centrale agit également en tant que "\textit{market maker}".

\subsubsection{Indépendance}

Dans les années 1970-1980, l'indépendance des banques centrales est devenue cruciale pour lutter contre le biais inflationniste. Ce modèle s'est imposé dans les années 1990, car il s'est révélé plus performant pour atteindre les objectifs finaux. L'indépendance se décline en deux dimensions : l'indépendance politique et l'indépendance économique.

\subsubsection{Indépendance politique}

Les dirigeants des banques centrales (BC) sont protégés des pressions du pouvoir politique et des cycles électoraux. 

Les membres des instances dirigeantes sont nommés pour une durée longue et prédéterminée : par exemple, les membres du Board of Governors de la Fed sont nommés pour 14 ans non renouvelables, tandis que le président et le vice-président le sont pour 4 ans (renouvelables). De même, les membres du directoire de la BCE ont un mandat de 8 ans non renouvelable. L'indépendance est plus forte si le gouvernement n'intervient pas dans la nomination des gouverneurs et autres membres. 

Il est important de noter qu'un membre du gouvernement peut participer aux réunions des organes de décision des BC et voter. Cependant, il n'y a pas de représentant du gouvernement aux réunions du \textit{Federal Open Market Committee} (FOMC) de la Fed. Le président du conseil EcoFin de l'UE-27 et de la CE peuvent assister aux réunions du conseil des gouverneurs de la BCE, mais sans droit de vote. 

Enfin, la BC n'est pas indépendante du gouvernement en ce qui concerne le choix des objectifs, car l'objectif final est fixé par la loi.

\subsubsection{Indépendance économique}

La banque centrale (BC) doit disposer des moyens nécessaires pour remplir ses missions. 

Il est impossible pour le gouvernement (État) d'obtenir un financement direct par le biais de la BC, que ce soit par l'émission de monnaie centrale (MBC) ou par crédit au trésor public. De plus, le gouvernement ne peut pas obtenir un financement indirect par l'achat de titres de dette publique sur le marché primaire. 

La BC choisit la nature des instruments (taux d'intérêt, base monétaire) sous son propre contrôle et a une absence de responsabilité ou une responsabilité partagée avec le gouvernement dans le contrôle bancaire (comme dans l'Eurosystème ou la Banque d'Angleterre). 

De plus, la BC bénéficie d'une indépendance financière pour mener ses opérations de politique monétaire, disposant de ses propres ressources financières et établissant un compte de résultat ainsi qu'un bilan. Elle doit avoir suffisamment de fonds propres (FP) pour ne pas dépendre d'une recapitalisation par l'État. Le gouvernement peut exercer un contrôle budgétaire en fonction du degré d'indépendance de la BC.

Indicateurs d’indépendance des banques centrales
Résultats synthétiques sur l’Indépendance des BC
Les 3 BC les + indépendantes sont BCE, BC de Suisse, Fed

\begin{center}
	\includegraphics[scale=0.7]{../../../Pictures/Screenshots/Capture d'écran 2025-02-16 203042}
\end{center}

Source : Weber C S et Forschner B, 2014, \textit{ECB:
Independence at risk?} Intereconomics, vol. 49.










\section{Système Européen de Banques Centrales}
	
\subsection{Architecture du dispositif}
	
\subsection{Banque Centrale Européenne}
	
\subsection{Banque de France}
	
	
	
	
	
	
	
	
	
	
\end{document}