\documentclass[a4paper, 12pt]{report}
\usepackage{graphicx}
\usepackage[utf8]{inputenc} 
\usepackage[french]{babel}
\usepackage[T1]{fontenc}
\usepackage{fancyhdr}
\usepackage{amsmath,amsfonts,amssymb, empheq}
\usepackage{eurosym}
\usepackage{booktabs}
\usepackage{cancel}
\usepackage{wrapfig}
%\usepackage{tikz}
\usepackage{hyperref}
\pagestyle{fancy}
\usepackage{mathptmx} %times aves le mode math
\fancyhead[R]{Université Paris-Est Créteil}
\fancyhead[L]{Monnaie et banques centrales}
\usepackage{array,multirow,makecell}
\setcellgapes{1pt}
\makegapedcells
\newcolumntype{R}[1]{>{\raggedleft\arraybackslash }b{#1}}
\newcolumntype{L}[1]{>{\raggedright\arraybackslash }b{#1}}
\newcolumntype{C}[1]{>{\centering\arraybackslash }b{#1}} 
%\renewcommand{\thechapter}{\Roman{chapter}}
%\setcounter{chapter}{1} % pour numéroter le chapitre 

\begin{document}
	
\chapter{Analyse de la création de	monnaie et de l'offre et de la demande de monnaie}
	
\section*{Introduction}

\section{Processus de la création monétaire}
	
\subsection{Création de monnaie par une banque unique}
	
\subsection{Création de monnaie par un système bancaire diversifié et hiérarchisé}
	
\section{Offre de monnaie}

\subsection{Facteurs et conditions de la liquidité bancaire}

\subsection{Approches conventionnelle et non conventionnelle de la création de monnaie}
	
\subsection{Marché du crédit et marché interbancaire}
	
\section{Demande de monnaie}	
	
\subsection{Premières approches sur la demande de monnaie}
	
\subsection{Demande de monnaie transactionnelle}
	
\subsection{Demande de monnaie patrimoniale}
	
\subsection{Approches empiriques}
	
	
	
	
	
\end{document}