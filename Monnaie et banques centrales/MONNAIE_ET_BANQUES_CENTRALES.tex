\documentclass[a4paper, 12pt]{report}
\usepackage{graphicx}
\usepackage[utf8]{inputenc} 
\usepackage[french]{babel}
\usepackage[T1]{fontenc}
\usepackage{fancyhdr}
\usepackage{amsmath,amsfonts,amssymb, empheq}
\usepackage{eurosym}
\usepackage{booktabs}
\usepackage{cancel}
\usepackage{wrapfig}
%\usepackage{tikz}
\usepackage{hyperref}
\pagestyle{fancy}
\usepackage{mathptmx} %times aves le mode math
\fancyhead[R]{Université Paris-Est Créteil}
\fancyhead[L]{Monnaie et banques centrales}
\usepackage{array,multirow,makecell}
\setcellgapes{1pt}
\makegapedcells
\newcolumntype{R}[1]{>{\raggedleft\arraybackslash }b{#1}}
\newcolumntype{L}[1]{>{\raggedright\arraybackslash }b{#1}}
\newcolumntype{C}[1]{>{\centering\arraybackslash }b{#1}} 
%\renewcommand{\thechapter}{\Roman{chapter}}
%\setcounter{chapter}{1} % pour numéroter le chapitre 

\begin{document}
	
\chapter{Analyse de la création de	monnaie et de l'offre et de la demande de monnaie}
	
\section*{Introduction}


La monnaie est un moyen de libération général, indéterminé et immédiat. Elle permet d'acheter des biens et services (B\&S), de rembourser des dettes et d'épargner. De plus, elle représente un droit de créance sur un établissement émetteur de monnaie. Mais qui émet de la monnaie ? La monnaie a une utilité indirecte et rend des services monétaires, ce qui correspond aux fonctions traditionnelles de la monnaie. C'est pour cette raison qu'il existe une demande de monnaie.

La monnaie facilite les échanges en réduisant les coûts de transaction, ce qui permet d'éviter l'économie de troc. Elle a cours légal d'après le code monétaire et financier et est imposée comme moyen de paiement par la loi pour un pays ou une zone monétaire. Les agents économiques (AE) doivent avoir confiance dans la valeur de la monnaie.

La monnaie sert d'étalon de valeur, permettant d'évaluer le prix des biens et services (B\&S) dans une même unité de mesure. Par exemple, dans une économie de troc comparée à une économie monétaire, en France au XVIIe siècle, il existait plusieurs unités de compte comme la livre parisis (Paris) ou la livre tournois (Tours), tandis que la monnaie en circulation comprenait le Louis d’or, les écus (en argent) et les billets. Dans la zone euro, l'unité de compte et la monnaie sont l'euro (\euro).

L'unité de compte est rattachée par convention, selon les régimes monétaires, à un étalon métallique (poids stable de métal) ou à une devise étrangère (taux de change stable ou valeur externe). Elle doit également vérifier la stabilité du pouvoir d'achat de la monnaie (valeur interne). Cela implique un ancrage nominal de la monnaie.

La monnaie est conservée dans le temps, permettant le transfert du pouvoir d'achat (PA) du présent vers le futur pour une consommation différée ou la constitution d'une épargne (patrimoine), selon Keynes. Elle est considérée comme un actif sans risque en valeur nominale. Cependant, l'inflation peut réduire sa valeur réelle.

Des exemples d'hyperinflation incluent l'Allemagne en 1922-1923, ainsi que certains pays d'Amérique latine et l'Ukraine dans les années 1980-1990, et plus récemment, le Zimbabwe, le Venezuela, l'Argentine et la Turquie. Il est à noter que de nombreux pays ont également connu des cas d'inflation à partir de 2021.

Environ 90\% de la monnaie est créée par les banques (établissements de crédit). Cela leur confère un pouvoir exorbitant, car elles transforment des reconnaissances de dettes individuelles (prêts) en reconnaissances de dettes collectivement acceptées par tous (monnaie). Si l'épargne n'est pas suffisante pour financer de nouveaux projets, il y a recours au crédit, entraînant ainsi la création de monnaie. 

Les agents économiques (AE) doivent être capables de rembourser leur crédit, ce qui implique une anticipation de revenus futurs suffisants. Cependant, on observe une augmentation des situations de surendettement, selon la Banque de France (BdF).


- Ce pouvoir de création est-il illimité ? 

- Qu'est-ce que l'offre et la demande de monnaie ?

- Le mécanisme de création de monnaie dans le cadre de la politique monétaire non conventionnelle s'est-il transformé ?

\section{Processus de la création monétaire}

La création monétaire est principalement assurée par les institutions financières monétaires (IFM) résidentes, telles que les établissements de crédit (EC), les banques centrales (BC), les établissements de monnaie électronique (EME) et les organismes de placement collectif (OPC) monétaires.

\subsection{Création de monnaie par une banque unique}
	
Une seule banque peut créer de la monnaie scripturale. Cette monnaie circule grâce aux instruments (ou moyens) de paiement.
	
1ère opération : achat d'un bien entre deux agents privés non financiers.  

- Pas de création de monnaie.

2ème opération : achat de devises étrangères par une banque à une entreprise non financière (ENF) résidente exportatrice.

3ème opération : achat de titres financiers par une banque auprès d'une ENF résidente.

4ème opération : achat de produits dérivés de la banque auprès d'une société d'assurance.

5ème opération : crédit accordé par une banque à un agent privé non financier.  

- Dépôt augmente de +100 dans chaque opération.  

- Agrégat de monnaie M1 et M3 augmentent.

\subsection*{Rappel : Les agrégats monétaires M1, M2 et M3}

Les agrégats monétaires M1, M2 et M3 sont des indicateurs utilisés pour mesurer la masse monétaire en circulation dans une économie. Voici un résumé :

\subsection*{M1 : Monnaie étroite}
\begin{itemize}
	\item Comprend les formes les plus liquides de monnaie, immédiatement utilisables pour les transactions.
	\item \textbf{Inclut :} 
	\begin{itemize}
		\item Les pièces et billets en circulation (monnaie fiduciaire).
		\item Les dépôts à vue (comptes courants) disponibles immédiatement.
	\end{itemize}
\end{itemize}

\subsection*{M2 : Monnaie intermédiaire}
\begin{itemize}
	\item Inclut \textbf{M1} ainsi que certaines formes de dépôts légèrement moins liquides.
	\item \textbf{Ajoute :}
	\begin{itemize}
		\item Les dépôts à terme (placements à court terme).
		\item Les comptes d'épargne (disponibles rapidement mais pas utilisables directement comme moyen de paiement).
	\end{itemize}
\end{itemize}

\subsection*{M3 : Monnaie large}
\begin{itemize}
	\item Inclut \textbf{M2} ainsi que des instruments financiers encore moins liquides.
	\item \textbf{Ajoute :}
	\begin{itemize}
		\item Les dépôts à long terme.
		\item Les titres de créance négociables (comme les certificats de dépôt et certains titres de marché monétaire).
	\end{itemize}
\end{itemize}

Chaque agrégat est une mesure progressive, \textbf{M1} étant le plus liquide et \textbf{M3} englobant des actifs plus larges, moins immédiatement accessibles. Ces agrégats sont suivis par les banques centrales pour évaluer et gérer la politique monétaire.
	
	
\subsection{Création de monnaie par un système bancaire diversifié et hiérarchisé}

La monnaie centrale permet aux établissements de crédit, c'est-à-dire aux banques, de se procurer des billets. Elle leur permet également d'acheter des devises étrangères. 

De plus, la monnaie centrale facilite le règlement des soldes de créances et de dettes entre les institutions financières monétaires (IFM) résultant des opérations effectuées par la clientèle à la fin de la journée. 

Les banques peuvent également effectuer des opérations avec l'administration publique (AP). Enfin, elles réalisent des opérations de politique monétaire, telles que les "\textit{Open Market}" et les facilités permanentes, afin d'approvisionner leur compte courant auprès de la banque centrale, que ce soit pour les réserves obligatoires ou les réserves excédentaires.

Ainsi, les établissements de crédit et l'administration publique disposent d'un compte courant auprès de la banque centrale en monnaie centrale.
	
\section{Offre de monnaie}

L'offre de monnaie peut être analysée selon une approche conventionnelle, qui établit une relation entre la base monétaire et la masse monétaire. Cette relation peut être interprétée en termes de multiplicateur ou de diviseur monétaires. 

Dans cette approche, la création de la masse monétaire (M) est proportionnelle à la base monétaire (H), ce qui implique un multiplicateur où la base est considérée comme exogène. En revanche, le diviseur représente une relation inverse, où la base dépend de la masse monétaire, ce qui en fait une base endogène.

Cependant, une approche plus récente et non conventionnelle remet en question cette relation explicative entre H et M. En effet, l'excédent de liquidité (base), résultant des opérations de politique monétaire non conventionnelles, empêche d'établir un lien stable entre la masse monétaire et la base monétaire.

\subsection{Facteurs et conditions de la liquidité bancaire}

Bilan simplifié consolidé de l'Eurosystème

\begin{center}
	\includegraphics[scale=0.6]{../../../Downloads/Screenshot 2025-01-11 at 15-25-03 PARTIE 3 INTERMEDIAIRES FINANCIERS ET CREATION MONETAIRE - C1 MBC 2025.pdf}
\end{center}

\subsubsection{Facteurs autonomes}

Le besoin net de liquidité est affecté par quatre types d'opérations. 

Tout d'abord, les billets en circulation influencent la liquidité de manière inverse : lorsque le montant de billets en circulation augmente, la liquidité diminue, entraînant ainsi un retrait de liquidité. 

Ensuite, les dépôts des administrations publiques, qui représentent la monnaie centrale détenue en compte courant auprès de la banque centrale, ont également un impact inverse sur la liquidité. En effet, une augmentation du montant des dépôts des administrations publiques entraîne un retrait de liquidité.

De plus, d'autres facteurs nets contribuent également à un retrait de liquidité. 

Enfin, les avoirs nets de change, incluant l'or et les devises, influencent la liquidité dans le même sens que le montant des avoirs nets de change. Ainsi, une augmentation de ces avoirs se traduit par un apport de liquidité.


\subsection{Approches conventionnelle et non conventionnelle de la création de monnaie}
	
\subsection{Marché du crédit et marché interbancaire}
	
\section{Demande de monnaie}	
	
\subsection{Premières approches sur la demande de monnaie}
	
\subsection{Demande de monnaie transactionnelle}
	
\subsection{Demande de monnaie patrimoniale}
	
\subsection{Approches empiriques}
	
	
	
	
	
\end{document}