\documentclass[a4paper, 12pt]{report}
\usepackage{graphicx}
\usepackage[utf8]{inputenc} 
\usepackage[french]{babel}
\usepackage[T1]{fontenc}
\usepackage{fancyhdr}
\usepackage{amsmath,amsfonts,amssymb, empheq}
\usepackage{eurosym}
\usepackage{booktabs}
\usepackage{cancel}
\usepackage{wrapfig}
\usepackage{hyperref}
\pagestyle{fancy}
\usepackage{mathptmx} %times aves le mode math
\fancyhead[R]{Université Paris-Est Créteil}
\fancyhead[L]{Méthodes de valorisation}
\usepackage{array,multirow,makecell}
\setcellgapes{1pt}
\makegapedcells
\newcolumntype{R}[1]{>{\raggedleft\arraybackslash }b{#1}}
\newcolumntype{L}[1]{>{\raggedright\arraybackslash }b{#1}}
\newcolumntype{C}[1]{>{\centering\arraybackslash }b{#1}} %times aves le mode math

\begin{document}
	
	\chapter{Les méthodes de valorisation}
	
\section{Valorisation par le rendement}
	\textbf{Principe} : Cette méthode repose sur la relation entre le revenu généré par un actif et le taux de rendement attendu par les investisseurs.  
	
	\textbf{Formule} :  
	\[
	V = \frac{R}{T}
	\]  
	où :  
	\begin{itemize}
		\item \(V\) = Valeur de l’actif,
		\item \(R\) = Revenu net généré par l'actif (loyers nets, bénéfices),
		\item \(T\) = Taux de rendement attendu ou capitalisation.
	\end{itemize}
	
\subsection{Exemple d'application}
  
	Si un bien immobilier génère 20 000 \euro~de loyer net annuel et que le taux de rendement est de 5\%, alors :  
	\[
	V = \frac{20\,000}{0.05} = 400\,000
	\]
	
\section{ Valorisation par les flux de trésorerie actualisés (DCF)}
	\textbf{Principe} : Cette méthode consiste à actualiser les flux de trésorerie futurs que l’actif générera pour obtenir sa valeur actuelle.  
	
	\textbf{Formule} :  
	\[
	V = \sum_{t=1}^{n} \frac{FCF_t}{(1 + r)^t} + \frac{TV}{(1 + r)^n}
	\]  
	où :  
	\begin{itemize}
		\item \(FCF_t\) = Flux de trésorerie libre à l’année \(t\),
		\item \(r\) = Taux d’actualisation,
		\item \(TV\) = Valeur terminale (valeur résiduelle de l’actif après \(n\) années),
		\item \(n\) = Durée de projection.
	\end{itemize}
	
\subsection{Exemple d'application}

Utilisée pour valoriser une entreprise ou un projet en calculant les flux futurs actualisés.
	
\section{Valorisation par la comparaison}

	\textbf{Principe} : Cette méthode compare l’actif à d’autres actifs similaires ayant été récemment vendus pour déterminer sa valeur.  
	
	\textbf{Formules possibles} :  
	\[
	\text{Prix/m}^2 = \frac{\text{Prix de vente du bien}}{\text{Surface}}
	\]  
	\[
	\text{PER} = \frac{\text{Prix de l’actif}}{\text{Résultat net}}
	\]
	
	\textbf{Étapes} :  
	\begin{enumerate}
		\item Identifier des actifs comparables (même secteur, caractéristiques proches).
		\item Analyser les ratios observés sur ces transactions (par ex., prix/m\( ^2 \)).
		\item Appliquer ces ratios à l’actif à valoriser.
	\end{enumerate}
	
\subsection{Exemple d'application} 

	Un appartement similaire de 50 m\( ^2 \) a été vendu à 300 000\euro~ dans le même quartier. Le prix/m\( ^2 \) est donc 6 000\euro, et un appartement de 60 m\( ^2 \) serait valorisé à :  
	\[
	6\,000 \times 60 = 360\,000\,€
	\]
	
\end{document}
