\documentclass[a4paper, 12pt]{report}
\usepackage{graphicx}
\usepackage[utf8]{inputenc} 
\usepackage[french]{babel}
\usepackage[T1]{fontenc}
\usepackage{fancyhdr}
\usepackage{amsmath,amsfonts,amssymb, empheq}
\usepackage{eurosym}
\usepackage{booktabs}
\usepackage{cancel}
\usepackage{wrapfig}
%\usepackage{tikz}
\usepackage{hyperref}
\pagestyle{fancy}
\usepackage{mathptmx} %times aves le mode math
\fancyhead[R]{Université Paris-Est Créteil}
\fancyhead[L]{Analyse financière}
\usepackage{array,multirow,makecell}
\setcellgapes{1pt}
\makegapedcells
\newcolumntype{R}[1]{>{\raggedleft\arraybackslash }b{#1}}
\newcolumntype{L}[1]{>{\raggedright\arraybackslash }b{#1}}
\newcolumntype{C}[1]{>{\centering\arraybackslash }b{#1}} 
%\renewcommand{\thechapter}{\Roman{chapter}}
%\setcounter{chapter}{1} % pour numéroter le chapitre 

\begin{document}

\chapter*{Introduction}
	
\section{Quelques éléments sur l'entreprise}

\subsection{Entreprise}

L'entreprise est un noeud de contrats entre les ayants droit, portant sur le contrôle des ressources et la répartition de la richesse créée. 

\subsection{Parties prenantes}

Les parties prenantes considérées, telles que les actionnaires et les prêteurs, s'intéressent au capital ainsi qu'à la richesse économique actuelle et future de l'entreprise.

\subsection{Analyse}

L'analyse se concentre sur la production et la répartition de la richesse économique de l'entreprise, en tenant compte des cycles de l'entreprise.

\subsection{Rentabilité et solvabilité}

Enfin, il est essentiel d'évaluer la rentabilité et la solvabilité pour comprendre la santé économique de l'entreprise.
	
\subsection{Les cycles de l'entreprise}

Nous commencerons par examiner quelques éléments fondamentaux sur l'entreprise, suivis d'une analyse des cycles de l'entreprise, qui incluent le cycle d'exploitation, le cycle d'investissement et le cycle de financement.

\subsection{Cycle d'exploitation}

Le cycle d'exploitation est crucial pour l'activité de l'entreprise. Il repose sur deux logiques principales. La première est la logique marchande et commerciale, qui englobe les transactions avec les clients ainsi que la gestion des flux monétaires. La seconde est la logique répétitive, qui se concentre sur la recherche d'économies d'échelle, souvent associée à une approche industrielle, et sur la création d'une réputation commerciale solide.

\subsection{Phases du cycle d'exploitation}

Le cycle d'exploitation se divise en trois phases distinctes. La première phase est celle de l'approvisionnement, où l'entreprise acquiert les ressources nécessaires à son fonctionnement. La deuxième phase est la phase de production, durant laquelle les inputs sont mobilisés dans un processus technologique. Enfin, la troisième phase est celle de la commercialisation, où les produits ou services sont offerts aux clients.
	
Représentation du cycle d'exploitation

\begin{wrapfigure}{r}{0.6\textwidth}
	\centering
\includegraphics[scale=0.5]{../../../Pictures/Screenshots/Capture d'écran 2025-01-08 221742}
\end{wrapfigure}

\subsection{Contrepartie}

La notion de contrepartie dans le contexte de l'entreprise se réfère à l'enchaînement de dettes et de créances. Cela inclut les dettes envers les fournisseurs, ainsi que les charges et les coûts intermédiaires. Ce processus commence par un décaissement, qui est suivi d'un encaissement, illustrant ainsi le besoin de financement nécessaire pour soutenir les opérations de l'entreprise.
\newpage
\subsection{Financement du cycle d'exploitation}
	
\begin{wrapfigure}{l}{0.4\textwidth}
	\centering
\includegraphics[scale=0.5]{../../../Pictures/Screenshots/Capture d'écran 2025-01-08 222413}
\end{wrapfigure}	
	
Les durées spécifiques varient selon chaque secteur, branche ou produit. Par exemple, dans le cas d'une entreprise de prestations de services qui est payée au comptant, la durée du cycle d'exploitation est nulle, ce qui signifie qu'il n'y a pas de délai entre le décaissement et l'encaissement.

\subsection{Cycle d'investissement}

Le cycle d'investissement concerne la création du capital économique nécessaire à la production, qui sera ensuite utilisé dans le cadre du cycle d'exploitation. Cet investissement implique une immobilisation de monnaie, ce qui signifie que des fonds sont engagés et ne sont pas immédiatement disponibles. 

De plus, l'amortissement des investissements physiques permet un retour à la liquidité, en récupérant progressivement les fonds investis. Il est également important de noter que l'investissement peut être de nature financière, comme dans le cas de la prise de contrôle d'une autre entreprise. Enfin, la durée du cycle d'investissement peut être plus ou moins longue, variant en fonction de la nature de l'investissement réalisé.

\begin{wrapfigure}{r}{0.4\textwidth}
	\centering
\includegraphics[scale=0.5]{../../../Pictures/Screenshots/Capture d'écran 2025-01-08 222859}
\end{wrapfigure}

\subsection{Cycle de financement}

Le cycle de financement constitue la contrepartie des cycles d'exploitation et d'investissement. Il implique la mise à disposition de liquidités par des apporteurs externes, tels que les actionnaires et les prêteurs. 
		
La durée de la ressource financière peut être courte, longue ou infinie, ce qui influence le rythme de la trésorerie de l'entreprise. Ainsi, une gestion efficace de ce cycle est essentielle pour assurer la liquidité nécessaire au bon fonctionnement des opérations.

\subsection{Rentabilité et solvabilité}

\subsubsection{Rentabilité}

La rentabilité est un moyen de rémunérer les apporteurs de ressources, comme les actionnaires. Elle constitue également un indicateur de rendement et d'efficacité dans l'allocation des ressources, permettant d'évaluer la performance économique de l'entreprise.
\[ \text{Rentabilité} = \frac{\text{Résultat obtenu}}{\text{Moyens mis en œuvre}} \]

La rentabilité est spécifique à chacune des parties prenantes.

Les moyens mis en oeuvre pour obtenir un capital économique sont essentiels à la création de valeur au sein de l'entreprise. Ce capital économique permet de financer les opérations et d'assurer la pérennité de l'activité. Il est constitué des ressources financières, matérielles et humaines nécessaires pour soutenir la production et le développement de l'entreprise.

\subsubsection{Solvabilité}

La solvabilité est la capacité d'une entreprise à assurer durablement le paiement de ses dettes exigibles. En cas de cessation des paiements, l'entreprise doit faire face à ses obligations envers les prêteurs et les fournisseurs, ce qui peut entraîner des procédures amiables ou judiciaires.

Dans une perspective de court terme, la solvabilité est liée à la liquidité de l'entreprise, qui peut être exprimée par la formule suivante :
\[
\text{Décaissements}(t) \leq \text{Encaissements}(t) + \text{Stock de monnaie}(t-1)
\]

En revanche, dans une perspective de long terme, il est essentiel que les encaissements soient structurellement supérieurs aux dépenses.

L'analyse financière menée par les créanciers se concentre sur le risque majeur de défaut de paiement généralisé, également connu sous le nom de défaillance. Ainsi, la notion de solvabilité se trouve au cœur de cette analyse.

\section{Information comptable}

L'information comptable est une obligation légale qui repose sur une logique d'évaluation par un tiers, tels que les actionnaires et les prêteurs. Elle est régie par des principes et des règles spécifiques à la comptabilité.

L'exploitation de cette information permet de mener une analyse financière approfondie. L'objectif principal est de produire une image fidèle et sincère du patrimoine, de la situation financière et du résultat de l'entreprise. Cela se traduit par la production de documents comptables conformes aux normes établies.

Les principes comptables sont essentiels pour produire ces documents, qui incluent à la fois les comptes individuels et les comptes consolidés.

La comptabilité des entreprises non-financières est régie par une réglementation élaborée par l'Autorité des Normes Comptables (ANC), accessible sur leur site internet : \url{http://www.anc.gouv.fr/}. 

Le Plan Comptable Général (PCG) constitue le cadre de référence pour cette comptabilité. Il est important de noter qu'il existe également des comptabilités spécifiques pour les entreprises financières, telles que les banques et les assurances.

Au fil du temps, la comptabilité a connu de nombreuses évolutions, s'adaptant aux changements économiques et réglementaires.

Les référentiels comptables utilisés incluent le référentiel national, connu sous le nom de French GAAP (\textit{Generally Accepted Accounting Principles}), qui régit la comptabilité en France. En outre, le référentiel IFRS (\textit{International Financial Reporting Standards}) a été développé par l'IASB (\textit{International Accounting Standards Board}), un organisme privé de normalisation comptable.

La réglementation européenne stipule que toutes les sociétés cotées, régies par le droit national d'un État européen, doivent appliquer le référentiel IFRS dans leurs comptes consolidés à partir du 1er janvier 2005. Il est également important de noter qu'il existe des référentiels comptables hors de l'Espace Économique Européen, tels que les US GAAP.

\subsection{Principes comptables}

Les principes comptables reposent sur la primauté du droit sur le fait, ce qui signifie que l'enregistrement comptable est associé à un acte juridique. Cela entraîne la création d'une nouvelle créance ou d'une nouvelle dette pour l'entreprise. La date et la méthode d'enregistrement ne sont pas nécessairement liées à la réalité économique.

Un autre principe fondamental est celui de l'évaluation au coût historique. Selon ce principe, les biens entrent dans le patrimoine de l'entreprise sur la base de leur valeur historique, c'est-à-dire à l'acquisition. Cette approche repose sur une valeur objective et constante, tandis que la valeur économique ou d'usage n'est pas retenue, car elle est considérée comme subjective et fluctuante.

En outre, seuls l'amortissement ou le provisionnement affectent l'évaluation comptable des actifs. Ce cadre comptable a une dimension backward-looking, c'est-à-dire qu'il est tourné vers le passé.

Les principes comptables incluent le principe de prudence, qui impose un traitement comptable dissymétrique entre les charges et les produits. Les charges sont prises en compte dès qu'elles sont probables, ce qui inclut la constitution de provisions. En revanche, les produits ne sont comptabilisés que lorsqu'ils sont réalisés, ce qui signifie que les plus-values potentielles ne sont pas prises en compte.

De plus, il n'y a pas de compensations entre les moins-values latentes et les plus-values latentes. Cette approche peut conduire à une sous-évaluation de l'entreprise dans sa valeur comptable.

Les comptes individuels sont régis par le référentiel national, connu sous le nom de French GAAP. En revanche, pour les comptes consolidés des groupes cotés, les autres référentiels, tels que les IFRS (\textit{International Financial Reporting Standards}), sont appliqués.

Il existe des divergences dans les principes comptables selon les référentiels utilisés. Par exemple, le coût historique et la primauté du droit sur le fait sont remis en cause dans le référentiel IFRS. Dans ce cadre, la primauté de la réalité économique, la comptabilité d'intention et l'évaluation à la "juste valeur" sont des principes privilégiés selon les normes IFRS.

\subsection{Documents comptables}

Le livre-journal enregistre chronologiquement les opérations affectant le patrimoine de l'entreprise. Le grand livre, quant à lui, regroupe les opérations du livre-journal en fonction du plan de compte de l'entreprise, qui est défini par la nomenclature du Plan comptable général.

L'inventaire est également un document essentiel dans la comptabilité. Les documents de synthèse, qui sont reportés sur l'inventaire, comprennent trois documents principaux : le bilan, le compte de résultat et l'annexe. Ces documents correspondent aux comptes annuels, qui doivent être déposés au greffe du tribunal de commerce dans le mois suivant l'approbation des comptes.

Les détails des comptes dépendent de critères liés à la taille des entreprises. Il existe trois niveaux de présentation, comme indiqué en annexe. Ces niveaux sont le système abrégé, le système de base et le système développé. 

Les différences entre ces trois systèmes résident principalement dans le niveau de détails fournis. 

Les documents comptables incluent la certification des comptes par un commissaire aux comptes, qui est une obligation légale si deux des trois critères suivants sont vérifiés : un chiffre d'affaires supérieur à 3,1 millions d'euros, un total de bilan supérieur à 1,55 million d'euros, ou un nombre moyen de salariés supérieur à 50.

La liasse fiscale est l'ensemble des imprimés fiscaux renseignés par l'entreprise, permettant de déterminer l'impôt sur les sociétés. Cette information peut être plus riche que celle contenue dans les documents comptables, notamment en ce qui concerne les amortissements, les provisions, ainsi que les échéances des créances et des dettes.

\section{Annexe}

Les détails des comptes dépendent de critères liés à la taille des entreprises. Il existe trois niveaux de présentation : le système abrégé, le système de base et le système développé.

Le système abrégé est destiné aux "petites" entreprises. Il permet la production d'un bilan et d'un compte de résultat simplifiés, à condition de respecter au moins deux des trois critères suivants : un total du bilan inférieur à 267 000 euros, un chiffre d'affaires net inférieur à 534 000 euros, ou un nombre moyen de salariés inférieur à 10.

Il est important de noter que les seuils mentionnés peuvent être amenés à changer dans le temps en raison de l'évolution du niveau général des prix.

Les documents comptables incluent le système de base, qui s'applique aux moyennes et grandes entreprises. Dans ce système, le bilan et le compte de résultat sont plus complets, accompagnés d'une annexe détaillée.

Cependant, il est possible de présenter une annexe simplifiée si au moins deux des trois critères suivants sont respectés : un total du bilan inférieur à 3,65 millions d'euros, un chiffre d'affaires net inférieur à 7,3 millions d'euros, ou un nombre moyen de salariés inférieur à 50.

Les documents comptables incluent le système développé, qui comporte des documents supplémentaires éclairant la gestion de l'entreprise. Ce système est facultatif et peut inclure des exemples de documents tels que le tableau de capacité d'auto-financement, le tableau de financement et le tableau de variation des capitaux propres.

\chapter{Le bilan comptable}


\end{document}
