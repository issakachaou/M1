\documentclass[a4paper, 12pt]{report}
\usepackage{graphicx}
\usepackage[utf8]{inputenc} 
\usepackage[french]{babel}
\usepackage[T1]{fontenc}
\usepackage{fancyhdr}
\usepackage{amsmath,amsfonts,amssymb, empheq}
\usepackage{eurosym}
\usepackage{booktabs}
\usepackage{cancel}
\usepackage{wrapfig}
\usepackage{hyperref}
\pagestyle{fancy}
\usepackage{mathptmx} %times aves le mode math
\fancyhead[R]{Université Paris-Est Créteil}
\fancyhead[L]{Analyse financière}
\usepackage{array,multirow,makecell}
\setcellgapes{1pt}
\makegapedcells
\newcolumntype{R}[1]{>{\raggedleft\arraybackslash }b{#1}}
\newcolumntype{L}[1]{>{\raggedright\arraybackslash }b{#1}}
\newcolumntype{C}[1]{>{\centering\arraybackslash }b{#1}} 
%\renewcommand{\thechapter}{\Roman{chapter}}
%\setcounter{chapter}{1} % pour numéroter le chapitre 

\begin{document}
\begin{titlepage}
	\centering
\begin{center}
	\includegraphics[scale=0.3]{../../../Pictures/FAC_SEG_rvb}
\end{center}
\vspace*{2cm}

\Huge

\textbf{Analyse financière}
\vspace{1.5cm}

\Large
Cours intégral

\vspace{2cm}

\textbf{Issa KACHAOU} \\
{\normalsize Délégué général du Master 1 MBFA}


\vfill

\Large

\textsc{\textbf{Université Paris Est-Créteil}}	 \\
\textbf{Département d'\'Economie} \\
\textbf{2025}

\end{titlepage}
\thispagestyle{empty}
\newpage
\clearpage
\mbox{}
\thispagestyle{empty}

\tableofcontents

\thispagestyle{empty}
\newpage
\mbox{}
\thispagestyle{empty} %Dernière page vide
%\backmatter	

\pagestyle{plain} 
\chapter*{Introduction}
	
\section{Quelques éléments sur l'entreprise}

\subsection{Entreprise}

L'entreprise est un noeud de contrats entre les ayants droit, portant sur le contrôle des ressources et la répartition de la richesse créée. 

\subsection{Parties prenantes}

Les parties prenantes considérées, telles que les actionnaires et les prêteurs, s'intéressent au capital ainsi qu'à la richesse économique actuelle et future de l'entreprise.

\subsection{Analyse}

L'analyse se concentre sur la production et la répartition de la richesse économique de l'entreprise, en tenant compte des cycles de l'entreprise.

\subsection{Rentabilité et solvabilité}

Enfin, il est essentiel d'évaluer la rentabilité et la solvabilité pour comprendre la santé économique de l'entreprise.
	
\subsection{Les cycles de l'entreprise}

Nous commencerons par examiner quelques éléments fondamentaux sur l'entreprise, suivis d'une analyse des cycles de l'entreprise, qui incluent le cycle d'exploitation, le cycle d'investissement et le cycle de financement.

\subsection{Cycle d'exploitation}

Le cycle d'exploitation est crucial pour l'activité de l'entreprise. Il repose sur deux logiques principales. La première est la logique marchande et commerciale, qui englobe les transactions avec les clients ainsi que la gestion des flux monétaires. La seconde est la logique répétitive, qui se concentre sur la recherche d'économies d'échelle, souvent associée à une approche industrielle, et sur la création d'une réputation commerciale solide.

\subsection{Phases du cycle d'exploitation}

Le cycle d'exploitation se divise en trois phases distinctes. La première phase est celle de l'approvisionnement, où l'entreprise acquiert les ressources nécessaires à son fonctionnement. La deuxième phase est la phase de production, durant laquelle les inputs sont mobilisés dans un processus technologique. Enfin, la troisième phase est celle de la commercialisation, où les produits ou services sont offerts aux clients.
	
Représentation du cycle d'exploitation

\begin{wrapfigure}{r}{0.6\textwidth}
	\centering
\includegraphics[scale=0.5]{../../../Pictures/Screenshots/Capture d'écran 2025-01-08 221742}
\end{wrapfigure}

\subsection{Contrepartie}

La notion de contrepartie dans le contexte de l'entreprise se réfère à l'enchaînement de dettes et de créances. Cela inclut les dettes envers les fournisseurs, ainsi que les charges et les coûts intermédiaires. Ce processus commence par un décaissement, qui est suivi d'un encaissement, illustrant ainsi le besoin de financement nécessaire pour soutenir les opérations de l'entreprise.
\newpage
\subsection{Financement du cycle d'exploitation}
	
\begin{wrapfigure}{l}{0.4\textwidth}
	\centering
\includegraphics[scale=0.5]{../../../Pictures/Screenshots/Capture d'écran 2025-01-08 222413}
\end{wrapfigure}	
	
Les durées spécifiques varient selon chaque secteur, branche ou produit. Par exemple, dans le cas d'une entreprise de prestations de services qui est payée au comptant, la durée du cycle d'exploitation est nulle, ce qui signifie qu'il n'y a pas de délai entre le décaissement et l'encaissement.

\subsection{Cycle d'investissement}

Le cycle d'investissement concerne la création du capital économique nécessaire à la production, qui sera ensuite utilisé dans le cadre du cycle d'exploitation. Cet investissement implique une immobilisation de monnaie, ce qui signifie que des fonds sont engagés et ne sont pas immédiatement disponibles. 

De plus, l'amortissement des investissements physiques permet un retour à la liquidité, en récupérant progressivement les fonds investis. Il est également important de noter que l'investissement peut être de nature financière, comme dans le cas de la prise de contrôle d'une autre entreprise. Enfin, la durée du cycle d'investissement peut être plus ou moins longue, variant en fonction de la nature de l'investissement réalisé.

\begin{wrapfigure}{r}{0.4\textwidth}
	\centering
\includegraphics[scale=0.5]{../../../Pictures/Screenshots/Capture d'écran 2025-01-08 222859}
\end{wrapfigure}

\subsection{Cycle de financement}

Le cycle de financement constitue la contrepartie des cycles d'exploitation et d'investissement. Il implique la mise à disposition de liquidités par des apporteurs externes, tels que les actionnaires et les prêteurs. 
		
La durée de la ressource financière peut être courte, longue ou infinie, ce qui influence le rythme de la trésorerie de l'entreprise. Ainsi, une gestion efficace de ce cycle est essentielle pour assurer la liquidité nécessaire au bon fonctionnement des opérations.

\subsection{Rentabilité et solvabilité}

\subsubsection{Rentabilité}

La rentabilité est un moyen de rémunérer les apporteurs de ressources, comme les actionnaires. Elle constitue également un indicateur de rendement et d'efficacité dans l'allocation des ressources, permettant d'évaluer la performance économique de l'entreprise.
\[ \text{Rentabilité} = \frac{\text{Résultat obtenu}}{\text{Moyens mis en œuvre}} \]

La rentabilité est spécifique à chacune des parties prenantes.

Les moyens mis en oeuvre pour obtenir un capital économique sont essentiels à la création de valeur au sein de l'entreprise. Ce capital économique permet de financer les opérations et d'assurer la pérennité de l'activité. Il est constitué des ressources financières, matérielles et humaines nécessaires pour soutenir la production et le développement de l'entreprise.

\subsubsection{Solvabilité}

La solvabilité est la capacité d'une entreprise à assurer durablement le paiement de ses dettes exigibles. En cas de cessation des paiements, l'entreprise doit faire face à ses obligations envers les prêteurs et les fournisseurs, ce qui peut entraîner des procédures amiables ou judiciaires.

Dans une perspective de court terme, la solvabilité est liée à la liquidité de l'entreprise, qui peut être exprimée par la formule suivante :
\[
\text{Décaissements}(t) \leq \text{Encaissements}(t) + \text{Stock de monnaie}(t-1)
\]

En revanche, dans une perspective de long terme, il est essentiel que les encaissements soient structurellement supérieurs aux dépenses.

L'analyse financière menée par les créanciers se concentre sur le risque majeur de défaut de paiement généralisé, également connu sous le nom de défaillance. Ainsi, la notion de solvabilité se trouve au cœur de cette analyse.

\section{Information comptable}

L'information comptable est une obligation légale qui repose sur une logique d'évaluation par un tiers, tels que les actionnaires et les prêteurs. Elle est régie par des principes et des règles spécifiques à la comptabilité.

L'exploitation de cette information permet de mener une analyse financière approfondie. L'objectif principal est de produire une image fidèle et sincère du patrimoine, de la situation financière et du résultat de l'entreprise. Cela se traduit par la production de documents comptables conformes aux normes établies.

Les principes comptables sont essentiels pour produire ces documents, qui incluent à la fois les comptes individuels et les comptes consolidés.

La comptabilité des entreprises non-financières est régie par une réglementation élaborée par l'Autorité des Normes Comptables (ANC), accessible sur leur site internet : \url{http://www.anc.gouv.fr/}. 

Le Plan Comptable Général (PCG) constitue le cadre de référence pour cette comptabilité. Il est important de noter qu'il existe également des comptabilités spécifiques pour les entreprises financières, telles que les banques et les assurances.

Au fil du temps, la comptabilité a connu de nombreuses évolutions, s'adaptant aux changements économiques et réglementaires.

Les référentiels comptables utilisés incluent le référentiel national, connu sous le nom de French GAAP (\textit{Generally Accepted Accounting Principles}), qui régit la comptabilité en France. En outre, le référentiel IFRS (\textit{International Financial Reporting Standards}) a été développé par l'IASB (\textit{International Accounting Standards Board}), un organisme privé de normalisation comptable.

La réglementation européenne stipule que toutes les sociétés cotées, régies par le droit national d'un État européen, doivent appliquer le référentiel IFRS dans leurs comptes consolidés à partir du 1er janvier 2005. Il est également important de noter qu'il existe des référentiels comptables hors de l'Espace Économique Européen, tels que les US GAAP.

\subsection{Principes comptables}

Les principes comptables reposent sur la primauté du droit sur le fait, ce qui signifie que l'enregistrement comptable est associé à un acte juridique. Cela entraîne la création d'une nouvelle créance ou d'une nouvelle dette pour l'entreprise. La date et la méthode d'enregistrement ne sont pas nécessairement liées à la réalité économique.

Un autre principe fondamental est celui de l'évaluation au coût historique. Selon ce principe, les biens entrent dans le patrimoine de l'entreprise sur la base de leur valeur historique, c'est-à-dire à l'acquisition. Cette approche repose sur une valeur objective et constante, tandis que la valeur économique ou d'usage n'est pas retenue, car elle est considérée comme subjective et fluctuante.

En outre, seuls l'amortissement ou le provisionnement affectent l'évaluation comptable des actifs. Ce cadre comptable a une dimension backward-looking, c'est-à-dire qu'il est tourné vers le passé.

Les principes comptables incluent le principe de prudence, qui impose un traitement comptable dissymétrique entre les charges et les produits. Les charges sont prises en compte dès qu'elles sont probables, ce qui inclut la constitution de provisions. En revanche, les produits ne sont comptabilisés que lorsqu'ils sont réalisés, ce qui signifie que les plus-values potentielles ne sont pas prises en compte.

De plus, il n'y a pas de compensations entre les moins-values latentes et les plus-values latentes. Cette approche peut conduire à une sous-évaluation de l'entreprise dans sa valeur comptable.

Les comptes individuels sont régis par le référentiel national, connu sous le nom de French GAAP. En revanche, pour les comptes consolidés des groupes cotés, les autres référentiels, tels que les IFRS (\textit{International Financial Reporting Standards}), sont appliqués.

Il existe des divergences dans les principes comptables selon les référentiels utilisés. Par exemple, le coût historique et la primauté du droit sur le fait sont remis en cause dans le référentiel IFRS. Dans ce cadre, la primauté de la réalité économique, la comptabilité d'intention et l'évaluation à la "juste valeur" sont des principes privilégiés selon les normes IFRS.

\subsection{Documents comptables}

Le livre-journal enregistre chronologiquement les opérations affectant le patrimoine de l'entreprise. Le grand livre, quant à lui, regroupe les opérations du livre-journal en fonction du plan de compte de l'entreprise, qui est défini par la nomenclature du Plan comptable général.

L'inventaire est également un document essentiel dans la comptabilité. Les documents de synthèse, qui sont reportés sur l'inventaire, comprennent trois documents principaux : le bilan, le compte de résultat et l'annexe. Ces documents correspondent aux comptes annuels, qui doivent être déposés au greffe du tribunal de commerce dans le mois suivant l'approbation des comptes.

Les détails des comptes dépendent de critères liés à la taille des entreprises. Il existe trois niveaux de présentation, comme indiqué en annexe. Ces niveaux sont le système abrégé, le système de base et le système développé. 

Les différences entre ces trois systèmes résident principalement dans le niveau de détails fournis. 

Les documents comptables incluent la certification des comptes par un commissaire aux comptes, qui est une obligation légale si deux des trois critères suivants sont vérifiés : un chiffre d'affaires supérieur à 3,1 millions d'euros, un total de bilan supérieur à 1,55 million d'euros, ou un nombre moyen de salariés supérieur à 50.

La liasse fiscale est l'ensemble des imprimés fiscaux renseignés par l'entreprise, permettant de déterminer l'impôt sur les sociétés. Cette information peut être plus riche que celle contenue dans les documents comptables, notamment en ce qui concerne les amortissements, les provisions, ainsi que les échéances des créances et des dettes.

\section{Annexe}

Les détails des comptes dépendent de critères liés à la taille des entreprises. Il existe trois niveaux de présentation : le système abrégé, le système de base et le système développé.

Le système abrégé est destiné aux "petites" entreprises. Il permet la production d'un bilan et d'un compte de résultat simplifiés, à condition de respecter au moins deux des trois critères suivants : un total du bilan inférieur à 267 000 euros, un chiffre d'affaires net inférieur à 534 000 euros, ou un nombre moyen de salariés inférieur à 10.

Il est important de noter que les seuils mentionnés peuvent être amenés à changer dans le temps en raison de l'évolution du niveau général des prix.

Les documents comptables incluent le système de base, qui s'applique aux moyennes et grandes entreprises. Dans ce système, le bilan et le compte de résultat sont plus complets, accompagnés d'une annexe détaillée.

Cependant, il est possible de présenter une annexe simplifiée si au moins deux des trois critères suivants sont respectés : un total du bilan inférieur à 3,65 millions d'euros, un chiffre d'affaires net inférieur à 7,3 millions d'euros, ou un nombre moyen de salariés inférieur à 50.

Les documents comptables incluent le système développé, qui comporte des documents supplémentaires éclairant la gestion de l'entreprise. Ce système est facultatif et peut inclure des exemples de documents tels que le tableau de capacité d'auto-financement, le tableau de financement et le tableau de variation des capitaux propres.

\chapter{Le bilan comptable}

\section*{Introduction}

Les référentiels comptables jouent un rôle central dans la préparation et l'analyse des états financiers. Parmi les principaux référentiels, on trouve le référentiel national, également connu sous le nom de \textit{French GAAP}, qui constitue le cadre de droit commun applicable aux comptes sociaux individuels des entreprises en France. On trouve également le référentiel IFRS, qui est le cadre européen utilisé pour les comptes consolidés des entreprises cotées. 

Il existe une tendance marquée vers la convergence entre ces référentiels, avec pour objectif une harmonisation des pratiques comptables à l'échelle internationale. 

Les éléments du bilan permettent de décrire la situation patrimoniale de l'entreprise. Le patrimoine varie dans le temps donc le bilan est daté le plus souvent du \( 31/12/N \). L'actif correspond à ce que possède l'entreprise, tandis que le passif représente ce qu'elle doit. Dans ce chapitre on utilisera le référentiel national soit le \textit{French GAAP}.

Des retraitements peuvent être nécessaires afin de transformer l'information comptable en une information exploitable par l'analyste financier, aboutissant à un bilan financier mieux adapté à la prise de décision.

\section{Analyse de l'actif}

L'analyse de l'actif consiste à examiner les moyens utilisés par l'entreprise pour exercer son activité. L'actif récapitule, à une date donnée, les droits de propriété et les créances de l'entreprise. 

La logique de construction de l'actif repose sur une approche fonctionnelle. On distingue plusieurs catégories principales. L'actif immobilisé regroupe les utilisations durables, c'est-à-dire les biens destinés à rester durablement dans l'entreprise. L'actif circulant correspond aux éléments dont le renouvellement est régulier, comme les stocks ou les créances à court terme. Enfin, les comptes de régularisation permettent d'ajuster les charges et les produits à la période comptable concernée.

Les principaux postes de l'actif reflètent les moyens financiers et matériels de l'entreprise. On trouve d'abord le capital souscrit non appelé, qui représente la part du capital social encore non versée par les actionnaires. Les immobilisations se divisent en trois catégories : les immobilisations incorporelles, qui comprennent les éléments immatériels tels que les brevets ou les logiciels, les immobilisations corporelles, qui incluent les biens matériels comme les bâtiments et les machines, et les immobilisations financières, qui regroupent les participations et autres investissements à long terme. 

Les stocks correspondent aux biens destinés à être vendus ou transformés. Les créances et avances représentent les montants dus à l'entreprise par ses clients ou partenaires. La trésorerie englobe les liquidités disponibles, qu'elles soient sous forme d'espèces ou de dépôts bancaires. Enfin, les comptes de régularisation permettent de répartir les charges et produits sur les périodes comptables appropriées.

\subsection{Capital souscrit non appelé}

Le capital souscrit non appelé correspond à la contrepartie à l'actif d'un engagement des actionnaires, qui est comptabilisé dans le capital social de l'entreprise. Il s'agit d'une créance de la société sur ses actionnaires, dont l'appel des fonds est décidé par le conseil d'administration ou le directoire de l'entreprise. 

Comptablement, ce poste est initialement classé avec les immobilisations. Toutefois, un reclassement en tant qu'actif de trésorerie peut être envisagé, notamment lorsqu'il représente des ressources monétaires liquides pouvant être mobilisées rapidement.

\textit{La mention "(dont versé…)" de la ligne DA concerne les sociétés dotées, à leur
création, d’un capital dont une partie seulement a été effectivement versée dans la
caisse sociale. Cette partie est aujourd’hui fixée à 50 \%, le solde devant être versé
ensuite dans les 5 ans. Même dans le cas où il n’est pas intégralement versé (on dira
"libéré", c’est-à-dire "libre… disponible"), le capital de 37 000 € sera représenté
par des actions qui auront trouvé souscripteur. On dira que le capital est souscrit,
une partie seulement étant libérée. La partie non libérée pourra être "appelée" par
la société à tout moment sur décision de son Conseil d’Administration (au plus
tard dans les 5 ans). Tant que cette part non libérée n’est pas appelée, la société ne
dispose évidemment pas des fonds correspondants.}

\subsection{Immobilisations incorporelles}


Les immobilisations incorporelles représentent des emplois durables de fonds qui ne sont ni des actifs physiques ni des actifs financiers. Elles correspondent à des droits obtenus en contrepartie de dépenses spécifiques.

\begin{itemize}
	\item Les frais de recherche et de développement, qui correspondent aux dépenses engagées pour créer ou améliorer des produits, procédés ou services.

	\item Les brevets, licences, marques et autres droits, qui représentent des actifs intangibles protégés par des droits légaux.

	\item Le fonds commercial, incluant des éléments tels que la clientèle et le droit de bail.

	\item Les frais d’établissement, qui couvrent les dépenses engagées lors de la constitution de l’entreprise, telles que les honoraires ou les droits d’enregistrement, ainsi que les coûts liés à son développement.

\end{itemize}
Ces immobilisations doivent être amorties sur une durée maximale de cinq ans, car elles correspondent à des biens intangibles ou immatériels. Toutefois, certaines dépenses, comme les frais de recherche, peuvent être directement passées en charge lorsqu'elles ne répondent pas aux critères de capitalisation.

\subsection{Immobilisations corporelles}

Les immobilisations corporelles représentent des actifs physiques durables dont l'entreprise est propriétaire. Elles constituent un élément clé dans le fonctionnement de l'activité économique. Par exemple, pour une entreprise industrielle, le capital de production comprend des éléments comme les terrains, les constructions, les installations techniques ou encore le matériel industriel.

Ces actifs sont soumis à un amortissement destiné à refléter leur dépréciation due à l'usure ou à l'obsolescence. L'amortissement peut être calculé de manière linéaire ou dégressive, selon les règles comptables applicables et les besoins de l'entreprise. 

La comptabilisation des immobilisations corporelles s'effectue généralement au coût historique. Toutefois, certaines entreprises peuvent opter pour une évaluation à la "\textit{fair value}", reflétant la valeur de marché des actifs. Cette méthode peut inclure la possibilité d’une réévaluation des actifs, qu'il s'agisse d’une appréciation ou d’une dépréciation.

\subsection{Immobilisation financière}

Les immobilisations financières comprennent les créances et les titres détenus dans une perspective de long terme, en lien avec la stratégie de développement de l'entreprise. 

Il existe plusieurs types d'immobilisations financières. Tout d'abord, les participations, qui correspondent à l'acquisition de plus de 10\% du capital d'une autre entreprise, permettent d'influencer sa gestion. Ensuite, les titres immobilisés de l'activité de portefeuille, qui sont des actions détenues sur le long terme sans intervention dans la gestion de l'entreprise concernée. Dans cette situation, on possède des actions de l'entreprise mais en faible proportion de sorte qu'on ne peut pas influencer la gestion de l'entreprise. Enfin, les prêts, qui sont des créances d'une durée supérieure à un an, incluent également les prêts accordés à la société mère ou aux associés.

\subsection{Stock}

Les stocks sont associés à l'actif circulant et comprennent différents types. Parmi eux, on trouve les matières premières et les approvisionnements, ainsi que les en-cours de production et les produits intermédiaires ou finis.

Il existe plusieurs méthodes d'évaluation ou de valorisation des stocks, qui soulèvent certaines problématiques, notamment pour les unités interchangeables : quel "prix" appliquer pour les sorties ?

\begin{itemize}
	\item La première méthode consiste à évaluer les sorties au coût moyen pondéré des entrées.
	\item La méthode FIFO (\textit{First In, First Out}) valorise les sorties au coût de l'élément le plus ancien.
	\item La méthode LIFO (\textit{Last In, First Out}) valorise les sorties au coût de l'élément le plus récent.
	\item Enfin, le coût de remplacement prend en compte le cours du marché pour évaluer les sorties.
\end{itemize}

En France, seules les méthodes 1 (coût moyen pondéré) et 2 (FIFO) sont autorisées. Ces méthodes peuvent entraîner des plus-values latentes en période d'inflation, ce qui affecte le résultat de l'entreprise. 

Les implications du choix de la méthode d'évaluation des stocks peuvent être particulièrement importantes si le délai de rotation des stocks est faible. En effet, dans un contexte de fluctuations des prix, le choix de la méthode peut influencer significativement les états financiers et la perception de la performance de l'entreprise.

\subsection{Créances et avances}

Les avances et acomptes versés sur commandes sont des montants versés à des fournisseurs. Ces avances constituent une créance sur un tiers. 

Les créances clients et les comptes rattachés représentent les comptes débiteurs de tous les clients. Les mouvements réels correspondants sont liés à des biens livrés ou à des prestations de services effectuées. 

Il est également important de mentionner la provision pour dépréciation, qui est constituée pour les clients douteux ou litigieux afin de couvrir les risques de non-recouvrement.

Enfin, les autres créances incluent les avances et acomptes versés au personnel, ainsi que les créances sur l'État et sur le "Groupe et associés".

\subsection{Trésorerie}

La trésorerie regroupe les encaisses disponibles ou quasi-disponibles. Elle comprend plusieurs types de rubriques ou de comptes.

Tout d'abord, les valeurs mobilières de placement, qui incluent des actions, des obligations, des bons du trésor, des titres de créance négociables (TCN) et des parts de fonds communs de placement (FCP) monétaires. 

Ensuite, les instruments de trésorerie, qui représentent les variations de valeurs des opérations en cours sur les marchés de produits dérivés, tels que les contrats à terme et les options.

Enfin, les disponibilités incluent les comptes bancaires et la caisse de l'entreprise.

Il est important de noter les différences entre les normes \textit{French GAAP} et IFRS. Par exemple, selon l'IFRS 7, la définition de la trésorerie est plus restrictive, excluant certaines obligations d'État ou OPCVM obligataires qui ne sont pas considérées comme des placements de trésorerie.

\subsection{Comptes de régularisation}

Les comptes de régularisation comprennent plusieurs types de charges. 

Tout d'abord, les charges constatées d'avance, qui concernent l'achat de biens et de services dont la fourniture ou la prestation sera ultérieure. Par exemple, cela inclut des factures d'achat reçues ou des primes d'assurance payées en avance.

Ensuite, il y a les charges à répartir sur plusieurs exercices. Ce sont des charges importantes et non répétitives dont les effets s'étalent dans le temps. L'imputation de ces charges se fait par le débit d'un compte de dotation aux amortissements, ce qui les enregistre négativement à l'actif.

Les écarts de conversion représentent la contrepartie comptable au bilan des pertes de change latentes. Cela inclut la diminution de valeur des créances ou l'augmentation de valeur des dettes, ce qui peut également conduire à la constitution d'une provision pour risque financier.


\subsection{Les principaux postes de l'actif}

Les principaux postes de l'actif sont les suivants :

\begin{enumerate}
	\item Capital souscrit non appelé : Il représente le montant des actions souscrites par les actionnaires mais qui n'ont pas encore été appelées par la société.
	\item Immobilisations: Ce poste se divise en trois catégories :
\begin{itemize}
	\item  Incorporelles : Comprend les actifs non physiques tels que les brevets, les marques et les droits d'auteur.
	\item  Corporelles : Inclut les actifs physiques comme les terrains, les bâtiments et les équipements.	
	\item Financières : Englobe les participations dans d'autres entreprises et les prêts à long terme.
\end{itemize}
	\item Stocks : Représente les biens destinés à la vente ou à la production.
	\item Créances et avances : Comprend les montants dus par les clients et les avances versées à des tiers.
	\item Trésorerie : Regroupe les encaisses disponibles ou quasi-disponibles.
	\item Comptes de régularisation : Inclut les charges constatées d'avance et les écarts de conversion, entre autres.
\end{enumerate}

\section{Analyse du passif}

Les éléments du passif représentent les dettes réelles de l'entreprise envers des tiers. Ils récapitulent, à une date donnée, les engagements de l'entreprise, tant vis-à-vis des tiers qu'à l'égard de ses propriétaires. 

La logique de construction du passif repose sur une distinction selon la nature juridique et financière des éléments qui le composent. Les dettes correspondent à des engagements qui doivent être remboursés à leur échéance, selon les conditions contractuelles établies. En revanche, les capitaux propres (Fond Propre), qui représentent les ressources apportées par les propriétaires ou générées par l'activité, ont un horizon temporel théoriquement infini, car ils ne sont pas soumis à une obligation de remboursement.

Les principaux postes du passif permettent de structurer les engagements de l'entreprise en fonction de leur nature et de leur horizon temporel. Ils se décomposent comme suit : 

Les capitaux propres représentent les ressources apportées par les actionnaires ou générées par l'activité de l'entreprise. Ils constituent un financement à long terme, sans obligation de remboursement.

Les provisions pour risques et charges sont des passifs potentiels ou certains, liés à des événements passés, dont l'échéance ou le montant restent incertains.

Les dettes se divisent en plusieurs catégories. 

Les dettes financières correspondent aux emprunts contractés 
auprès des institutions financières. 

Les dettes d'exploitation regroupent les montants dus dans le cadre des activités courantes, tels que les dettes fournisseurs. 

Les dettes diverses incluent des engagements spécifiques, comme les dettes fiscales ou sociales.

Les comptes de régularisation permettent d'ajuster les charges et produits aux périodes comptables correspondantes.

\subsection{Capitaux propres}

Les capitaux propres représentent les ressources permanentes mises à disposition de l'entreprise. Ils incluent à la fois les apports initiaux des actionnaires et les surplus monétaires générés au fil du temps. 

Le capital social et les primes qui y sont liées constituent une partie essentielle des capitaux propres. Le capital social correspond à la valeur nominale des actions émises par l'entreprise, c'est la valeur apportée par les actionnaires lors de la création de l'entreprise. Les primes liées, quant à elles, constatent la différence entre la valeur des apports "initiaux" et les accroissements du capital social. Cela inclut, par exemple, l'excédent du prix d'émission des actions par rapport à leur valeur nominale.

Les bénéfices mis en réserves (les réserves) représentent le cumul historique de la fraction des bénéfices réalisés par l'entreprise et conservés en interne, plutôt que redistribués sous forme de dividendes.

Enfin, les subventions d'équipement ou d'investissement, octroyées par des collectivités publiques, viennent également compléter les capitaux propres. Ces aides visent à soutenir des projets spécifiques ou le développement de l'entreprise.

Les capitaux propres incluent également des provisions réglementées, qui bénéficient d'un traitement fiscal particulier. Ces provisions, non imposées, comprennent notamment :

\begin{itemize}
	\item La provision pour investissement, qui résulte d’un avantage fiscal accordé à l’entreprise lorsque celle-ci distribue une partie de son résultat aux salariés, dans le cadre du régime obligatoire. Cette provision permet à l’entreprise de mettre en réserve une somme non imposable.
	\item La provision pour hausse de prix, utilisée lorsque les prix des stocks de produits ou de matières premières augmentent de plus de 10\% sur les deux dernières années.
	\item D'autres provisions réglementées, comme celles destinées à l’implantation à l’étranger ou au risque de crédit à moyen terme.
\end{itemize}

Par ailleurs, les autres fonds propres regroupent des éléments hétérogènes caractérisés par des statuts juridiques complexes, souvent hybrides ou confus. Cela inclut, par exemple, les émissions de titres participatifs ou les avances conditionnées par l'État.

Ces éléments complètent la structure des capitaux propres, apportant à l’entreprise une flexibilité supplémentaire dans la gestion de ses ressources à long terme.

\subsection{Provisions pour risques et charges}

Les provisions pour risques et charges sont destinées à couvrir un risque ou une charge prévisible, mais qui ne sont pas directement affectés à un élément spécifique de l'actif. Elles correspondent à une dette probable, dont le montant et l'échéance restent incertains.

À titre d'exemple, ces provisions peuvent inclure : 
\begin{itemize}
	\item Les provisions pour litiges ;
	\item Les provisions pour pertes de change ;
	\item Les provisions pour pertes sur contrats ;
	\item Les provisions pour restructurations ;
	\item Les provisions pour grosses réparations, etc.
\end{itemize}

Comptablement, la constitution d'une provision entraîne une baisse du résultat, mais les fonds restent dans l’entreprise jusqu’à ce que le risque ou la charge se réalisent. Cela constitue un moyen pour l’entreprise de lisser ses résultats sur plusieurs exercices.

Toutefois, dans le cadre des normes IFRS, le traitement des provisions est plus restrictif, limitant leur utilisation aux cas répondant à des critères précis.

\subsection{Dettes}

Les dettes représentent des engagements financiers de l’entreprise envers des tiers. Elles se répartissent en plusieurs catégories :

\subsubsection{Dettes financières}

Les dettes financières constituent un moyen de financement durable, généralement à long terme (au delà d'un an). Elles correspondent à des dettes arrivant à échéance mais souvent renouvelées. Parmi les prêteurs figurent :  
\begin{itemize}
	\item Les banques ;  
	\item Les marchés financiers ;  
	\item Les autres entreprises, notamment celles du même groupe ;  
	\item Les associés.
\end{itemize}  

\subsubsection{Dettes d'exploitation}

Les dettes d’exploitation sont directement liées à l’activité courante de l’entreprise. Elles incluent :  
\begin{itemize}
	\item Les avances et acomptes reçus sur commandes en cours, correspondant aux sommes versées par les clients avant la livraison des biens ou services ;  
	\item Les dettes fournisseurs et comptes rattachés, représentant les sommes restant dues aux fournisseurs pour des biens ou services livrés ;  
	\item Les dettes fiscales et sociales, regroupant les montants dus envers le personnel, la Sécurité Sociale ou l'État, notamment pour la collecte de la TVA.
\end{itemize}

\subsubsection{Dettes diverses (hors exploitation)}

Ces dettes ne sont pas directement liées à l’exploitation courante. Elles incluent notamment :  

\begin{itemize}
	\item Les dettes fiscales, comme l’impôt sur les bénéfices ;  
	\item Les autres dettes, telles que la réserve de participation des salariés ou les dividendes à payer aux associés.   
\end{itemize}

\subsection{Comptes de régularisation}

Les comptes de régularisation regroupent notamment les produits constatés d'avance. Ils représentent une dette de l'exercice en cours envers les exercices suivants. Ces produits correspondent à un engagement de l'entreprise à fournir un bien ou une prestation ultérieurement, en contrepartie de montants déjà perçus.

\subsection{Trésorerie}

La trésorerie inclut les dettes à court terme, telles que :  
\begin{itemize}
	\item Les soldes bancaires créditeurs ;  
	\item Les comptes courants des sociétés apparentées ou des sociétés mères.  
\end{itemize}
Ces éléments reflètent des engagements financiers immédiats ou de très court terme.

\subsection{Les principaux postes du passif}

Les principaux postes du passif permettent de structurer les engagements financiers de l'entreprise. Ils se décomposent comme suit : 

\begin{enumerate}
	\item Capitaux propres
	\item Provisions pour risques et charges
	\item Dettes
	\begin{itemize}
		\item Dettes financières
		\item Dettes d'exploitation
		\item Dettes diverses
	\end{itemize}
	\item Comptes de régularisation
	\item Trésorerie
\end{enumerate}

\section{Les comptes consolidés}

L'analyse de l'actif et du passif est essentielle pour la compréhension des comptes consolidés. Ces comptes ont pour objectif de fournir une image fidèle de la réalité économique et financière d'un ensemble coordonné d'entreprises, c'est-à-dire un groupe. Dans ce contexte, les comptes individuels de la société mère présentent un portefeuille de titres à l'actif, ainsi que des droits sur d'autres entreprises. L'opération de consolidation consiste à substituer à la quote-part des titres de participation tout ou partie des éléments d'actif et de passif de l'entreprise concernée ou contrôlée.

Les entreprises cotées sur un marché européen doivent produire leurs comptes consolidés selon les normes IFRS. La réalisation de comptes consolidés est obligatoire pour les groupes, qu'ils soient cotés ou non, qui réunissent 2 des 3 critères suivants : un total du bilan supérieur à 15 millions d'euros, un chiffre d'affaires dépassant 30 millions d'euros, ou un nombre de salariés supérieur à 250. En ce qui concerne les normes, les groupes peuvent choisir entre les \textit{French GAAP} ou les IFRS. En pratique, seuls les petits groupes non cotés utilisent généralement les \textit{French GAAP}.

\subsection{Principe}

Le principe des comptes consolidés repose sur des méthodes de consolidation qui varient selon la nature des relations entre la société mère et sa filiale. Trois méthodes possibles existent. 

\subsubsection{L'intégration globale (IAS 27)}

La première est l'intégration globale (IAS 27), qui s'applique lorsque la société mère exerce un contrôle exclusif sur la filiale, ce qui est généralement le cas lorsque le groupe possède 50\% des droits de vote de la filiale. Toutefois, le critère utilisé est plus large : selon la norme IFRS, il suffit de disposer de la majorité des sièges au conseil d'administration, tandis que selon la norme française, le groupe peut détenir 40\% des droits de vote de la filiale. Dans ce cas, les comptes de la société mère reprennent l'intégralité des actifs et passifs. De plus, la prise en compte des actionnaires minoritaires se traduit par l'apparition, au passif de la société mère, du poste "intérêts minoritaires ou non-contrôlant".


\subsubsection{La mise en équivalence (IAS 28)}

La mise en équivalence (IAS 28) s'applique dans les situations où le groupe exerce une influence notable sur une entreprise, sans pour autant contrôler sa politique financière et opérationnelle. Dans ce cas, la société mère n'est pas déterminante dans les prises de décision. Selon les normes IFRS et la pratique des commissaires aux comptes, un seuil de 20\% des droits de vote est requis pour établir cette influence. La mise en équivalence implique une méthode de réévaluation, où la valeur comptable des titres de participation est substituée par la valeur de la quote-part des capitaux propres de la filiale. Dans les comptes de la société mère, cela se traduit par un poste "Titres mis en équivalence" à l'actif et un écart d'évaluation dans les capitaux propres au passif.

\subsubsection{Intégration proportionnelle}

L'intégration proportionnelle (IFRS 31) concerne les situations de contrôle conjoint, où un groupe partage de manière stable le contrôle avec une ou plusieurs entreprises, comme dans le cas des co-entreprises (joint-venture). Dans ce cadre, les comptes de la société mère reprennent une proportion des actifs et passifs de la filiale. Il est important de noter qu'il n'y a pas d'apparition "d'intérêts minoritaires" dans ce type de consolidation. La norme IFRS recommande également d'appliquer une mise en équivalence dans ces situations.

\subsection{Plan comptable}

Le plan comptable et les méthodes de comptabilisation des comptes consolidés diffèrent de ceux des comptes individuels (i.e., French GAAP), notamment par les noms des items à l'actif et au passif. Un exemple d'item spécifique aux comptes consolidés est l'écart d'acquisition (survaleur, ou \textit{goodwill}) à l'actif consolidé. Cet écart provient de la prise en compte des actifs réévalués de la filiale, et non de la valeur comptable du bilan individuel de la filiale, ainsi que de la prise en compte des plus-values. Il est important de noter que le prix d'acquisition de la filiale par la société mère peut différer de la valeur comptable des fonds propres de la filiale. Cela implique un écart entre les éléments d'actif et de passif qui sont remontés lors de la consolidation.

\subsubsection{Exemple de différence dans les méthodes de valorisation : immobilisations financières}

Un exemple de différence dans les méthodes de valorisation concerne les immobilisations financières. La classification des instruments financiers, initialement régie par IAS 39, a été remplacée par IFRS 9 en 2018. Cette norme définit trois catégories ou modèles de valorisation : 

\begin{itemize}
	\item Modèle Juste Valeur par Résultat : catégorie par défaut.
	\item Modèle Coût Amorti : applicable pour un business model de type HTC (held to collect), consistant à percevoir les flux de trésorerie contractuels et à conserver l'instrument financier jusqu'à son échéance.
	\item Modèle Juste Valeur par OCI (Other Comprehensive Income) : recyclable (i.e., par Fonds Propres), destiné aux business models de type HTCS (held to collect and sell), qui consistent à percevoir les flux contractuels et à vendre l'actif.
\end{itemize}

Ces classifications influencent significativement la manière dont les immobilisations financières sont présentées dans les comptes consolidés.

\subsection{Représentation shématique pour l'IFRS 9}

\begin{wrapfigure}{r}{0.4\textwidth}
	\centering
\includegraphics[scale=0.5]{../../../Pictures/Screenshots/Capture d'écran 2025-01-21 111643}
\end{wrapfigure}

\section{Conclusion}

Le bilan comptable reflète la situation patrimoniale de l'entreprise à une date donnée, indiquant les moyens mobilisés par l'entreprise pour réaliser son activité. L'étape suivante consiste à analyser les résultats de l'activité. Pour ce faire, il est essentiel d'exploiter les informations du bilan comptable, ce qui inclut le calcul de ratios financiers. Des retraitements peuvent être nécessaires afin de construire un bilan financier plus adapté pour l'analyste financier, facilitant ainsi une meilleure évaluation de la performance et de la santé financière de l'entreprise.

\chapter{Le compte de résultat et autres documents}

\section*{Introduction}

Les documents de synthèse comprennent le bilan, le compte de résultat et l'annexe. 

\subsection{Compte de résultat}
Le compte de résultat présente l'ensemble des flux de produits et de charges imputables à l'exercice comptable. Il se divise en plusieurs catégories :
\begin{itemize}
	\item Produits et charges d'exploitation
	\item Produits et charges financières
	\item Produits et charges exceptionnels
\end{itemize}

L'objectif final est de calculer le résultat net de l'exercice.

\subsection{Annexe}
L'annexe fournit des précisions sur les méthodes comptables utilisées ainsi que des informations ponctuelles, permettant une meilleure compréhension des états financiers présentés.

\section{Compte de résultat}

\subsection{Éléments du compte de résultat}
Le compte de résultat se compose des éléments suivants :

\begin{itemize}
	\item Opérations d'exploitation
	\item Opérations financières
	\item Opérations exceptionnelles
	\item Participation des salariés
	\item Impôts sur les bénéfices
\end{itemize}

Chacun de ces éléments joue un rôle crucial dans le calcul du résultat net de l'exercice, offrant une vue d'ensemble de la performance financière de l'entreprise.

\subsubsection{Opérations d'exploitation}

Les opérations d'exploitation se composent des produits d'exploitation suivants :

\begin{enumerate}
	\item Ventes de marchandises : il s'agit d'une absence de transformations par l'entreprise, englobant des activités purement commerciales telles que les opérations de négoce et de distribution.
	\item Production vendue : ce terme fait référence au processus de création, fabrication ou transformation. 

\[
\text{Chiffre d'affaires net} = \text{ventes de marchandises} + \text{production vendue}
\]

\item Production stockée : cela représente la variation de stock au cours de l'exercice, calculée comme suit :
\[
\text{Variation de stock} = \text{stock final} - \text{stock initial}
\]
Cela inclut :
\begin{itemize}
	\item Stocks de produits intermédiaires et finis
	\item Encours de production de biens (biens en cours de fabrication)
	\item Encours de production de services (études ou prestations en cours, évaluées au coût de production)
\end{itemize}

\item Production immobilisée : cela concerne les investissements corporels ou incorporels réalisés par l'entreprise pour elle-même.

Ces éléments sont essentiels pour déterminer la performance économique de l'entreprise au cours de l'exercice.

   \item \textbf{Subventions d'exploitation} : compensent l'insuffisance du prix de vente (ex : secteur agricole).\\
\textit{Définition} : Aides financières accordées par l'État ou d'autres organismes pour compenser la différence entre le prix de vente et le coût de production.\\
\textit{Importance} : Aident les producteurs à maintenir leur activité malgré des prix de vente bas.

\item \textbf{Reprises de provisions} : provisions préalablement constituées pour des dépréciations ou des risques.\\
\textit{Définition} : Montants mis de côté pour faire face à des pertes potentielles.\\
\textit{Importance} : Permettent de récupérer des fonds lorsque les pertes sont inférieures à ce qui avait été prévu.

\item \textbf{Transferts de charges} :
\begin{itemize}
	\item L'entreprise a supporté des charges pour le compte d'un tiers, créance encaissable.\\
	\textit{Définition} : Frais payés pour une autre entité, créant une créance.\\
	\textit{Importance} : Aide à la gestion des flux de trésorerie.
	
	\item L'entreprise étale une charge sur plusieurs années (implique amortissement sur plusieurs années).\\
	\textit{Définition} : Répartition des coûts sur plusieurs exercices.\\
	\textit{Importance} : Reflète mieux l'utilisation des ressources sur le long terme.
	
	\item Permet une correction si une dépense a été enregistrée à tort dans un compte de charges déterminé.\\
	\textit{Définition} : Ajustement des erreurs comptables.\\
	\textit{Importance} : Garantit l'exactitude des états financiers.
\end{itemize}

\item \textbf{Autres produits} : produits et brevets ou de licences, redevances pour concessions.\\
\textit{Définition} : Revenus générés par la vente de produits, brevets, licences, ou redevances.\\
\textit{Importance} : Diversifient les sources de revenus et contribuent à la rentabilité de l'entreprise.
\end{enumerate}

\subsection{Remarques sur le chiffre d'affaires (CA) :}
\begin{itemize}
	\item \textbf{Évaluation au prix du marché} : Le chiffre d'affaires est évalué au prix du marché, hors taxes et net de remises commerciales. Cela reflète la valeur réelle des ventes réalisées.
	\item \textbf{Indicateur clé de performance} : Le CA mesure la capacité de l'entreprise à matérialiser son capital économique et commercial à travers des contrats de vente.
	\item \textbf{Enregistrement comptable} : Le chiffre d'affaires inclut les prises de commande et la signature de contrats de vente, ce qui garantit une traçabilité des engagements commerciaux.
	\item \textbf{Rubriques spécifiques} : Certaines rubriques permettent une meilleure analyse du CA :
	\begin{itemize}
		\item \textit{Commandes à livrer} : Correspond aux ventes enregistrées mais non encore livrées.
		\item \textit{Ventes à facturer} : Désigne les ventes réalisées mais pour lesquelles la facturation est en attente.
	\end{itemize}
	\item \textbf{Manipulations potentielles} : Il est possible de gonfler artificiellement le chiffre d'affaires, par exemple :
	\begin{itemize}
		\item En enregistrant des ventes fictives qui seront annulées au début de l'exercice suivant.
		\item Si le ratio \((\text{CA à facturer}) / (\text{CA total}) > 10\%\), cela peut indiquer un risque de manipulation.
	\end{itemize}
	\item \textbf{Impact sur la fiabilité des résultats} : Une part importante de CA non encore facturée ou liée à des commandes en attente peut entraîner des incertitudes sur la stabilité des résultats et le niveau réel de performance.
\end{itemize}


\subsection{Remarques sur la production stockée}

\begin{itemize}
	\item \textbf{Un produit d'exploitation} : La production stockée est considérée comme un produit d'exploitation. Cela signifie qu'elle contribue directement au résultat de l'entreprise en augmentant les revenus déclarés, même si ces stocks ne sont pas encore vendus.
	\item \textbf{Méthode de valorisation des stocks} : Le choix de la méthode utilisée pour valoriser les stocks (par exemple, FIFO, LIFO ou le coût moyen pondéré) a un impact direct sur le résultat de l'entreprise. Une méthode peut augmenter artificiellement les bénéfices ou minimiser les pertes, selon la situation économique.
	\item \textbf{Analyse du résultat} : Une part importante du résultat peut être influencée par des variations dans les stocks, ce qui peut refléter des problèmes ou des opportunités spécifiques :
	\begin{itemize}
		\item \textit{Un échec commercial} : Une augmentation des stocks peut indiquer que les produits ne se vendent pas comme prévu, entraînant une accumulation non souhaitée.
		\item \textit{Un pourcentage élevé d'invendus} : Cela peut signaler une inadéquation entre l'offre et la demande ou des erreurs dans la stratégie commerciale.
	\end{itemize}
	\item \textbf{Stratégies pour masquer les pertes} : Certaines entreprises peuvent manipuler la production stockée pour embellir leurs résultats financiers :
	\begin{itemize}
		\item \textit{Augmenter artificiellement les stocks} : En produisant davantage, même si la demande ne suit pas, les entreprises peuvent reporter leurs pertes sur les exercices futurs.
		\item \textit{Réduction apparente des coûts unitaires} : En augmentant la production, le coût unitaire des produits peut sembler diminuer, ce qui améliore artificiellement la marge brute.
	\end{itemize}
	\item \textbf{Impact sur la rentabilité} : Une augmentation des stocks peut donner une illusion de rentabilité, mais cela peut cacher des faiblesses structurelles telles qu'un modèle économique non viable ou des problèmes de gestion des ventes.
\end{itemize}

\subsection{Les Charges de l'Entreprise}
\begin{enumerate}
	\item Achats
\begin{itemize}
	\item \textbf{Achats de marchandises} : Correspondent aux achats de biens destinés à être revendus en l'état, sans transformation, dans le cadre d'une activité commerciale.
	\item \textbf{Achats de matières premières et autres approvisionnements} : Ces achats concernent les ressources nécessaires à une activité de transformation (par exemple, la fabrication de produits finis).
	\item \textbf{Correction par les variations de stocks} : Les achats sont corrigés par les variations de stocks. Ainsi, un déstockage apparaît comme une charge supplémentaire, tandis qu'un restockage réduit la charge nette.
\end{itemize}

\item Autres charges externes
\begin{itemize}
	\item Ces charges comprennent des frais liés à des prestations de sous-traitance, des consommations diverses (eau, énergie), des loyers, des primes d'assurances, des frais postaux, et des frais sur services bancaires, entre autres.
	\item Ces éléments sont regroupés sous l'appellation de \textbf{charges de gestion courante}.
\end{itemize}

\item Charges de personnel
\begin{itemize}
	\item \textbf{Salaires et charges sociales} : Incluent les rémunérations des employés et les cotisations sociales, y compris celles liées au personnel intérimaire.
	\item \textbf{Participation et intéressement} : Intègrent les dispositifs de participation des salariés aux résultats de l'entreprise ainsi que les primes d'intéressement.
\end{itemize}

\item Dotations aux amortissements et aux provisions
	\begin{itemize}
		\item \textbf{Nature des dotations} : Les dotations aux amortissements et aux provisions sont des charges calculées qui ne se traduisent pas nécessairement par un décaissement immédiat.
		\item \textbf{Amortissements} :
		\begin{itemize}
			\item Concernent les immobilisations corporelles (par exemple, les machines, les bâtiments).
			\item Permettent de répartir le coût d'une immobilisation sur sa durée de vie utile.
		\end{itemize}
		\item \textbf{Provisions} :
		\begin{itemize}
			\item \textit{Actifs immobilisés non amortissables} : Par exemple, les terrains ou les actifs incorporels comme les fonds de commerce.
			\item \textit{Dépréciation des actifs circulants} : Inclut la baisse de valeur des stocks ou des comptes clients.
			\item \textit{Risques et charges d'exploitation} : Couvre les grosses réparations, les procès en cours ou les indemnités de licenciement potentielles.
		\end{itemize}
	\end{itemize}
	
\item Autres charges (de gestion courante)
	\begin{itemize}
		\item Ces charges incluent des dépenses liées à l'utilisation de concessions, de brevets ou de licences.
		\item Elles sont généralement considérées comme des charges d'exploitation récurrentes.
	\end{itemize}
\end{enumerate}

\subsubsection{Opération financières}

\paragraph{Objectif des Opérations Financières}
L'objectif des opérations financières est de faire ressortir un résultat net provenant des activités de financement et de placement. Ces opérations incluent les produits financiers générés par différents types d'actifs ainsi que les variations de leur valeur.

\paragraph*{Les Produits Financiers}
\begin{enumerate}
	\item \textbf{Produits financiers de participations} :
	\begin{itemize}
		\item Revenus générés par les participations, principalement sous forme de dividendes.
		\item Ces produits reflètent la rentabilité des investissements stratégiques dans d'autres entreprises.
	\end{itemize}
	\item \textbf{Produits des autres valeurs mobilières et créances de l'actif immobilisé} :
	\begin{itemize}
		\item Revenus issus de prêts et de titres immobilisés (autres que les participations).
		\item Ces produits incluent les intérêts et les gains liés à des placements à long terme.
	\end{itemize}
	\item \textbf{Autres intérêts et produits assimilés} :
	\begin{itemize}
		\item Revenus provenant des valeurs mobilières de placement et des actifs de trésorerie.
		\item Ces produits sont souvent liés à des placements à court terme ou à la gestion de la trésorerie.
	\end{itemize}
	\item \textbf{Reprises sur provisions} :
	\begin{itemize}
		\item Correspondent à la matérialisation d'une dépréciation antérieure concernant un actif financier.
		\item Par exemple, une moins-value potentielle peut être annulée si l'actif recouvre sa valeur.
	\end{itemize}
	\item \textbf{Différences positives de change} :
	\begin{itemize}
		\item Gains réalisés en raison des variations favorables des taux de change sur des actifs libellés en devises étrangères.
	\end{itemize}
	\item \textbf{Produits nets sur cession de valeurs mobilières de placement} :
	\begin{itemize}
		\item Plus-values réalisées lors de la vente de titres de placement.
		\item Ces gains reflètent la gestion efficace du portefeuille de placement de l'entreprise.
	\end{itemize}
\end{enumerate}

\paragraph{Les charges financières}

\begin{enumerate}
	\item \textbf{Dotations financières aux amortissements et provisions} :\\
	Les dotations aux amortissements et provisions sont des charges comptables permettant de comptabiliser la perte de valeur des actifs ou de prévoir des risques à venir. Les amortissements concernent la perte de valeur des immobilisations (comme les équipements, les bâtiments, etc.) sur une période donnée. Les provisions, quant à elles, sont constituées pour anticiper des risques financiers, tels que des créances douteuses, ou des moins-values potentielles, qui pourraient survenir dans le futur.
	
	\item \textbf{Intérêts et charges assimilées} :\\
	Les intérêts et charges assimilées regroupent les coûts financiers associés aux emprunts et aux dettes d'une entreprise. Cela inclut les intérêts dus sur les dettes bancaires, les obligations ou toute autre forme de financement. Ces charges sont cruciales pour mesurer le coût de l'endettement et l'impact financier des emprunts sur la rentabilité de l'entreprise.
	
	\item \textbf{Différences négatives de changes} :\\
	Les différences de changes résultent des variations des taux de change entre différentes monnaies. Lorsqu’une entreprise réalise des transactions internationales, elle peut être exposée à des fluctuations de devises qui affectent la valeur de ses actifs et passifs libellés en devises étrangères. Une différence négative de change survient lorsque la devise dans laquelle l'entreprise est investie se déprécie par rapport à la devise de référence, entraînant une perte comptabilisée dans les charges financières.
	
	\item \textbf{Charges nettes sur cession de valeurs mobilières de placement} :\\
	Les charges nettes sur cession de valeurs mobilières de placement font référence aux moins-values réalisées lors de la vente d'actifs financiers tels que des actions ou des obligations. Lorsqu'une entreprise cède des valeurs mobilières à un prix inférieur à leur coût d'acquisition, la différence entre le prix de vente et le coût d'acquisition est une perte, comptabilisée comme une charge. Cela permet de refléter la baisse de valeur des investissements et l'impact négatif sur la performance financière de l'entreprise.
\end{enumerate}

\subsubsection{Opérations exceptionnelles}

\textbf{Objectifs :} 
\begin{itemize}
	\item Comptabiliser les résultats d’éléments non-récurrents : Les opérations exceptionnelles visent à enregistrer des événements ou des transactions qui ne sont pas liés à l’activité normale et récurrente de l’entreprise. Ces éléments peuvent avoir un impact significatif sur le résultat net, mais ne se reproduisent pas régulièrement.
	
	\item Comptabiliser les résultats de ce qui est inclassable ailleurs (exploitation ou opérations financières) : Certaines opérations ne peuvent pas être classées dans les catégories classiques de l’exploitation ou des charges financières. Celles-ci peuvent inclure des événements tels que la vente d'actifs non-stratégiques, des indemnités exceptionnelles, ou des résultats liés à des changements importants dans la structure de l'entreprise.
\end{itemize}

\textbf{Critères pour définir des opérations exceptionnelles :} 
\begin{itemize}
	\item \textbf{Montant ou nature de l'opération} : Un des critères déterminants pour classer une opération comme exceptionnelle est son montant ou sa nature. Si l'opération est d'une taille exceptionnelle ou présente un caractère non-récurrent dans le contexte des activités habituelles de l'entreprise, elle peut être considérée comme exceptionnelle. Par exemple, la vente d'un bien immobilier appartenant à l'entreprise qui n'est pas destiné à être récurrente dans le cadre de son activité pourrait être qualifiée d’opération exceptionnelle.
	
	\item \textbf{Critère de montant et subjectivité} : L'introduction du critère de montant peut parfois rendre la définition d'une opération exceptionnelle subjective. Par exemple, les frais de licenciement liés à une restructuration peuvent être considérés soit comme des charges de personnel normales, soit comme des charges exceptionnelles, en fonction de l'ampleur de l'événement et de son caractère inhabituel dans le cadre de l’activité courante de l'entreprise.
\end{itemize}

\paragraph{Les opérations (produits ou charges)}

\begin{enumerate}
	\item \textbf{Revenus exceptionnels sur opérations de gestion} :
	\begin{itemize}
		\item \textbf{Charges} :\\
		Les charges exceptionnelles sur opérations de gestion comprennent les amendes, les pénalités et les frais liés aux activités non-récurrentes de l’entreprise. Elles incluent également les créances irrécouvrables, c'est-à-dire des créances qui ne peuvent plus être récupérées, ainsi que les frais de restructuration, qui sont engagés pour réorganiser l'entreprise dans des situations particulières.
		
		\item \textbf{Produits} :\\
		Les produits exceptionnels sur opérations de gestion incluent des dons ou des subventions d’équilibre. Ces sommes sont généralement perçues par l’entreprise pour des raisons spécifiques et ne relèvent pas de l'activité courante. Par exemple, des dons peuvent être accordés à une entreprise pour soutenir une action particulière, et des subventions d'équilibre peuvent être attribuées pour combler un déficit ponctuel.
	\end{itemize}
	
	\item \textbf{Revenus exceptionnels sur opérations en capital} :
	\begin{itemize}
		\item \textbf{Charge} :\\
		La charge exceptionnelle sur opérations en capital correspond à la valeur comptable des éléments d'actifs cédés. Cela inclut la Valeur Comptable des Éléments d'Actifs Cédés (VCEAC), qui est la valeur des actifs avant leur cession. Elle peut inclure des amortissements et des provisions qui ont été comptabilisés sur ces éléments.
		
		\item \textbf{Produit} :\\
		Le produit exceptionnel sur opérations en capital se réfère aux produits des cessions d'éléments d'actifs (PCEA). Il s'agit du produit réalisé lors de la vente d'actifs, comme des terrains, des équipements ou d'autres biens non-stratégiques pour l'entreprise. La différence entre la VCEAC et le PCEA représente la plus-value ou la moins-value de la cession, qui est comptabilisée comme un produit ou une charge exceptionnelle.
	\end{itemize}
	
	\item \textbf{Dotations et reprises} :\\
	Les dotations et reprises sont liées aux dotations aux amortissements dérogatoires et aux provisions réglementées. Les dotations aux amortissements dérogatoires permettent de comptabiliser des amortissements qui ne suivent pas les règles comptables classiques, souvent pour des raisons fiscales. Les provisions réglementées, quant à elles, sont des provisions constituées selon des règles spécifiques définies par les autorités fiscales, et peuvent être reprises sous certaines conditions.
\end{enumerate}

\paragraph{Autres Éléments}

\begin{enumerate}
	\item \textbf{Participation des salariés} :\\
	La participation des salariés est obligatoire pour les entreprises de plus de 50 salariés. Elle est calculée selon la formule suivante : 
	\[
	\text{Participation} = \left( \frac{1}{2} \right) \times B \times 5\% \times C \times \left( \frac{S}{VA} \right)
	\]
	où :
	\begin{itemize}
		\item $B$ représente le bénéfice net fiscal,
		\item $C$ désigne les capitaux propres de l'entreprise,
		\item $S$ est le montant des salaires versés aux employés,
		\item $VA$ est la valeur ajoutée de l'entreprise.
	\end{itemize}
	La participation est une forme d'intéressement qui permet aux salariés de recevoir une part des bénéfices de l'entreprise. Elle représente un élément accessoire de la rémunération du personnel et est souvent perçue comme un outil de motivation et de fidélisation.
	
	\item \textbf{Impôt sur les bénéfices} :\\
	L'impôt sur les sociétés (IS) est un impôt direct sur les bénéfices réalisés par les entreprises. Avant 2019, le taux de l'impôt était de 33,33\%, mais il a progressivement diminué pour atteindre 25\% en 2022. L'impôt comporte également un minimum d'imposition forfaitaire annuelle, qui dépend du chiffre d'affaires (CA) et du type d'entreprise. Ce minimum est une imposition minimale, même si l'entreprise réalise un bénéfice faible ou nul.
	
	\item \textbf{Calcul complexe et optimisation fiscale} :\\
	Le calcul de l'impôt sur les sociétés peut être complexe, car il inclut des possibilités d'optimisation fiscale, des reports en avant ou en arrière des déficits fiscaux, et d'autres mécanismes de réduction d'impôt. Il n'existe pas de relation simple et directe entre l'impôt et le résultat net de l'entreprise, car plusieurs facteurs peuvent influencer le montant final de l'impôt à payer.
	
	\item \textbf{Interprétation de l'impôt sur les bénéfices} :\\
	L'impôt sur les sociétés est un indicateur souvent interprété comme un signe de la performance de l'entreprise. Un impôt régulièrement élevé peut refléter une entreprise profitable, mais il doit aussi être mis en perspective avec l'optimisation fiscale réalisée par l'entreprise. Un impôt trop faible pourrait également susciter des questions sur la rentabilité réelle de l'entreprise ou la gestion fiscale.
\end{enumerate}


\section{Compte consolidé et normes IFRS}

\begin{enumerate}
	\item \textbf{Compte consolidé et normes IFRS} :\\
	Le compte consolidé est un état financier qui regroupe l'ensemble des comptes des sociétés d'un groupe, traitées comme une seule entité économique. Cela permet de présenter une vision globale et cohérente de la situation financière du groupe dans son ensemble, en éliminant les transactions internes entre les sociétés du groupe. Les normes IFRS (International Financial Reporting Standards) sont un ensemble de normes comptables internationales qui régissent la préparation des comptes consolidés. Elles visent à harmoniser la présentation des états financiers à l'échelle internationale, permettant une comparaison plus facile des performances des entreprises à travers les pays.
	
	\item \textbf{Compte de résultat consolidé} :\\
	Le compte de résultat consolidé regroupe les produits et charges des sociétés intégrées dans le groupe, en tenant compte des ajustements nécessaires. Il présente la performance financière globale du groupe, en agrégeant les résultats des différentes entités, tout en supprimant les transactions internes qui ne doivent pas apparaître dans les comptes consolidés. Par exemple, les ventes entre sociétés du groupe sont éliminées afin d’éviter une double comptabilisation.
	
	\item \textbf{Déduction des flux internes (achats/ventes)} :\\
	Lors de la consolidation des comptes, les flux internes, tels que les achats et ventes entre entités du groupe, sont déduits. Ces transactions internes, appelées "transferts internes", ne doivent pas être comptabilisés dans les résultats consolidés, car elles ne représentent pas de véritables flux économiques pour le groupe dans son ensemble. Par exemple, si une société du groupe vend des biens à une autre société du groupe, cette vente ne sera pas incluse dans le compte consolidé, car elle ne représente pas une transaction externe au groupe. Cela permet de donner une image plus précise de la rentabilité du groupe en excluant les effets des transactions internes.
\end{enumerate}

\subsubsection{Quelques éléments spécifiques avec la consolidation}

\paragraph{Quote-part de résultat des sociétés mise en équivalence}

\begin{itemize}
	\item \textbf{Principe}: La consolidation par mise en équivalence est utilisée lorsque l'entreprise consolidante exerce une influence notable sur une autre entreprise (dite "entreprise associée"), sans pour autant en détenir le contrôle exclusif.
	\item \textbf{Méthode}: Au lieu d'intégrer l'ensemble des actifs, passifs et résultats de l'entreprise associée, on se contente de comptabiliser, dans les états financiers consolidés de l'entreprise consolidante, la quote-part du résultat net de l'entreprise associée, proportionnelle au pourcentage du capital détenu.
	\item \textbf{Justification}: Cette méthode est justifiée par le fait que l'entreprise consolidante exerce une influence significative sur l'entreprise associée, mais sans en avoir le contrôle total. Il est donc logique de ne retenir que la part du résultat qui lui revient.
	\item \textbf{Distinction avec l'intégration globale}: Dans le cas de l'intégration globale (où l'entreprise consolidante contrôle l'entreprise consolidée), on intègre l'ensemble des flux de charges et de produits de l'entreprise consolidée, et non pas seulement une quote-part du résultat net.
\end{itemize}

\paragraph{Part des minoritaires / Part du groupe}

\begin{itemize}
	\item \textbf{Contexte}: Ces notions sont pertinentes dans le cadre de la consolidation par intégration globale, où l'ensemble des actifs, passifs et résultats des entreprises consolidées sont intégrés dans les états financiers consolidés.
	\item \textbf{Part des minoritaires}: Elle représente la fraction du capital des entreprises consolidées qui est détenue par des actionnaires autres que la société mère. Cette part est comptabilisée dans les capitaux propres consolidés, mais de manière distincte de la part du groupe.
	\item \textbf{Part du groupe}: Elle représente la fraction du résultat consolidé qui revient à la société mère, en fonction de ses participations directes et indirectes dans les entreprises consolidées. Cette part est également comptabilisée dans les capitaux propres consolidés.
	\item \textbf{Importance}: La distinction entre la part des minoritaires et la part du groupe est importante car elle permet de mieux appréhender la répartition du résultat consolidé entre les différents actionnaires.
\end{itemize}

\subsubsection{Quelques éléments spécifiques en IFRS}

\paragraph{Disparition des produits et charges exceptionnels et de la participation des salariés}

\begin{itemize}
	\item \textbf{Constat}: Les normes IFRS ne font plus de distinction entre les éléments exceptionnels et les éléments courants dans le compte de résultat.
	\item \textbf{Justification}: Les IFRS considèrent que cette distinction est subjective et peut être source de manipulations. De plus, elles estiment que les investisseurs sont capables d'analyser eux-mêmes la nature des différents éléments du compte de résultat, sans qu'il soit nécessaire de les regrouper dans une catégorie spécifique.
	\item \textbf{Traitement de la participation des salariés}: La participation des salariés, qui était traditionnellement considérée comme une charge exceptionnelle, est désormais traitée comme une charge d'exploitation.
	\item \textbf{Justification}: Les IFRS considèrent que la participation des salariés est un élément récurrent de la rémunération du personnel et qu'elle doit donc être comptabilisée comme une charge d'exploitation.
	\item \textbf{Traitement des plus ou moins-values de cession d'immobilisations}: Les plus ou moins-values de cession d'immobilisations, qui étaient parfois considérées comme des éléments exceptionnels, sont désormais comptabilisées dans le résultat courant, sauf si elles sont liées à une activité non poursuivie.
	\item \textbf{Justification}: Les IFRS considèrent que les cessions d'immobilisations sont des opérations courantes pour de nombreuses entreprises et qu'elles doivent donc être comptabilisées dans le résultat courant.
\end{itemize}

\paragraph{Exemples}

\begin{itemize}
	\item \textbf{Participation des salariés}: La participation des salariés est désormais comptabilisée comme une charge d'exploitation, au même titre que les salaires et les charges sociales.
	\item \textbf{Plus ou moins-values de cession d'immobilisations}: Les plus ou moins-values de cession d'immobilisations sont comptabilisées dans le résultat courant, sauf si elles sont liées à une activité non poursuivie. Par exemple, si une entreprise vend un immeuble qu'elle utilisait pour son activité principale, la plus ou moins-value sera comptabilisée dans le résultat courant. En revanche, si une entreprise vend une filiale qu'elle avait acquise dans le cadre d'une opération exceptionnelle, la plus ou moins-value pourra être présentée comme un élément non courant.
\end{itemize}

\subsubsection{Modèle de compte de résultat (income statement) en IFRS}

Le compte de résultat en normes IFRS présente généralement les informations suivantes :

\paragraph{Présentation}

Le compte de résultat peut être présenté de deux manières :

\begin{itemize}
	\item \textbf{Présentation par nature}: Les charges sont regroupées par nature (par exemple, salaires, charges de personnel, matières premières, etc.).
	\item \textbf{Présentation par fonction}: Les charges sont regroupées par fonction (par exemple, coût des ventes, charges administratives, charges de distribution, etc.).
\end{itemize}

Les normes IFRS autorisent les deux présentations, mais elles recommandent la présentation par fonction pour les entreprises qui ont des activités diversifiées.

\paragraph{Structure du compte de résultat}

\begin{itemize}
	\item \textbf{Chiffre d'affaires (Revenue)}
	\item \textbf{Coût des ventes (Cost of sales)}
	\item \textbf{Bénéfice brut (Gross profit)}
	\item \textbf{Charges d'exploitation (Operating expenses)}
	\item \textbf{Résultat d'exploitation (Operating profit)}
	\item \textbf{Produits et charges financiers (Finance income and expenses)}
	\item \textbf{Résultat avant impôts (Profit before tax)}
	\item \textbf{Impôt sur le résultat (Income tax expense)}
	\item \textbf{Résultat net (Profit for the period)}
	\item \textbf{Autres éléments du résultat global (Other comprehensive income)}
	\item \textbf{Résultat global (Total comprehensive income)}
\end{itemize}

\begin{center}
	\includegraphics[scale=0.5]{../../../Pictures/Screenshots/Capture d'écran 2025-01-28 183546}
\end{center}

\section{L'annexe et autres documents}

\subsubsection{Annexe : 3ème Document Comptable de Synthèse}

L'annexe accompagne les états financiers et a pour objectif de fournir des informations complémentaires. Elle est essentielle pour mieux juger la situation financière et les résultats de l'entreprise.


\subsubsection{Objectif de l'annexe}

\begin{itemize}
	\item Compléter les informations chiffrées : Elle enrichit les données présentées dans le bilan et le compte de résultat.
	\item Explications et commentaires : Fournit des détails qui permettent une meilleure compréhension des informations chiffrées.
\end{itemize}

\subsubsection{Importance des Autres Documents}

D'autres documents, tels que les rapports de gestion et les états financiers prévisionnels, jouent également un rôle crucial dans l'évaluation de la performance et de la santé financière de l'entreprise.

Cette structure permet de mieux appréhender les différents aspects de la comptabilité et de l'analyse financière, en soulignant l'importance de chaque document dans le processus décisionnel.

\subsubsection{Exemples d'Informations Reportées dans l'Annexe}

L'annexe contient des informations complémentaires essentielles pour une compréhension approfondie des états financiers. Voici quelques exemples d'informations souvent reportées :

Détails sur les entrées et sorties des immobilisations :

\begin{itemize}
	\item Informations sur l'acquisition et la cession des actifs.
	\item Impact sur la valeur nette des immobilisations.
\end{itemize}

Informations sur l'acquisition et la cession des actifs.
Impact sur la valeur nette des immobilisations.

Détails sur le pourcentage ou le montant des amortissements :
\begin{itemize}
	\item Méthodes d'amortissement utilisées (linéaire, dégressif, etc.).
	\item Montants amortis durant la période.
\end{itemize}

Détails sur le pourcentage ou le montant des encours de dettes:

\begin{itemize}
	\item Répartition des dettes à court et long terme.
	\item Taux d'intérêt appliqués et échéances.
\end{itemize}

- **Rappel sur les méthodes d'évaluation retenues** :
- Méthodes comptables appliquées pour évaluer les actifs et passifs.
- Justifications des choix méthodologiques.

\subsubsection{Dimension de l'Annexe}

L'annexe a une dimension importante, tant sur le plan obligatoire que discrétionnaire :

Informations obligatoires minimales :
\begin{itemize}
	\item La réglementation impose un ensemble d'informations minimales à fournir dans l'annexe.
\end{itemize}

Informations mentionnées par la réglementation :
\begin{itemize}
	\item Certaines informations doivent être fournies de manière systématique, tandis que d'autres peuvent être ajoutées selon la pertinence.
\end{itemize}

Fourniture discrétionnaire :
\begin{itemize}
	\item L'entreprise peut choisir d'inclure des informations supplémentaires si elle considère qu'elles ont une importance significative pour les utilisateurs des états financiers.
\end{itemize}

\subsubsection{Géométrie Variable Selon les Entreprises}

La présentation de l'annexe peut varier d'une entreprise à l'autre, en fonction de la taille, de la complexité des opérations et des exigences spécifiques du secteur. Chaque entreprise doit adapter l'annexe pour refléter fidèlement sa situation financière tout en respectant les normes comptables en vigueur.

\subsubsection{Les Documents d'Information}

Le rapport de gestion est réalisé par le Conseil d'Administration ou le Directoire et porte sur l'activité de la société et de ses filiales, les perspectives futures, la répartition de l'actionnariat et les dividendes. Le tableau de résultat sur 5 ans inclut des éléments tels que le capital social, les résultats de l'exercice, les résultats par action et des informations sur le personnel. La plaquette annuelle, bien que non obligatoire, sert d'outil de communication pour les sociétés faisant appel public à l'épargne. Enfin, les documents semestriels, qui comprennent les comptes semestriels, sont exigés pour les entreprises cotées.

\subsubsection{Les Documents d'Information en IFRS}

Le "\textit{Statement of Comprehensive Income}" est un tableau de résultat global qui intègre des éléments ne constituant pas des produits ou charges, mais qui affectent directement les capitaux propres. Il résulte de l'application de la juste valeur en fonds propres, incluant par exemple les écarts de réévaluation et les plus ou moins-values. Sa finalité est de permettre une meilleure appréciation de la variation du stock de capitaux propres. 

Le "\textit{Statement of Change in Equity}" explique quant à lui la variation globale des capitaux propres, en détaillant la répartition du résultat, les émissions et les rachats.

\chapter{Les soldes intermdiaires de gestion}

Le compte de résultat présente les charges et les produits durant un exercice comptable. Les charges se divisent en décaissables et non décaissables. Les charges décaissables impliquent un décaissement immédiat ou différé, comme le loyer. Les charges non décaissables incluent la dépréciation définitive par amortissement ou la dépréciation potentielle sous forme de provisions. 

Les produits, quant à eux, sont classés en encaissables et non encaissables. Les produits encaissables correspondent à des encaissements immédiats ou différés, comme les ventes, tandis que les produits non encaissables peuvent diminuer ou annuler des amortissements, dépréciations ou provisions.

Les soldes intermédiaires de gestion (SIG) permettent d'exploiter l'information du compte de résultat et sont des outils d'analyse financière. Ils se rapportent à une période donnée, généralement l'horizon annuel de l'exercice comptable. La finalité du compte de résultat est de calculer le résultat net, qui peut être considéré comme un solde global. Pour une analyse financière approfondie, il est nécessaire d'identifier les différents éléments contribuant au résultat net.

Les SIG incluent : la marge commerciale, la valeur ajoutée, l'excédent brut d'exploitation, le résultat d'exploitation, le résultat courant, le résultat net et la capacité d'autofinancement (CAF).

\section{Chiffre d'affaire, production et marges}

Le chiffre d'affaires (CA) n'est pas un solde intermédiaire de gestion, mais constitue le point de départ de l'analyse de l'activité de l'entreprise. Il se calcule comme suit : 
\[ \text{CA = Vente de marchandises (négoces) + Production vendue} \]
Ce montant représente les sommes (hors taxes) effectivement facturées au titre de l'activité courante "normale". 

Il est essentiel d'analyser l'évolution du chiffre d'affaires de manière globale et par segment, en tenant compte de la nature de l'activité et de la zone géographique. De plus, il est important de distinguer l'effet prix de l'effet quantité dans cette analyse.

\subsection{Calcul des marges}

\begin{center}
	\includegraphics[scale=0.8]{../../../Pictures/Screenshots/Capture d'écran 2025-02-17 212624}
\end{center}

\subsection{La marge commerciale (MC)}

La marge commerciale (MC) est essentielle pour les entreprises de négoce. Elle se calcule comme suit : 
\[ \text{MC = Ventes de marchandises - Consommation de marchandises} \]

où la consommation de marchandises est déterminée par les achats moins la variation de stocks. La variation de stocks se calcule en soustrayant le stock final du stock initial. 

Le taux de marge est donné par la formule : 

\[ \text{Taux de marge} = \frac{\text{MC}}{\text{Prix d'achat des marchandises (HT)}}   \]

Ce taux représente un point d'ancrage de la rentabilité, car il sert à couvrir d'autres charges. Il est pertinent de comparer ce taux à la moyenne du secteur, disponible sur le site de l'INSEE : \url{https://www.insee.fr/fr/statistiques/2015453}. 

Enfin, le taux de marque se calcule comme suit : Taux de marque = MC / Prix de vente des marchandises (HT).

\subsection{Marge Brute (MB)}

La marge brute (MB) est un indicateur clé dans le cadre d'une activité de transformation. Elle se calcule comme suit : 

\[ 
MB = MC + \text{Production vendue} - \text{Production stockée} 
\]

où la production stockée est déterminée par la différence entre le stock final de produits et le stock initial de produits. La consommation de matières premières se calcule par :

\[ 
\text{Consommation de matières premières} = \text{Achats} - \Delta \text{de stocks} 
\]

La marge brute mesure à la fois les marges de négoce et de transformation. Cependant, elle est peu comparable d'une entreprise à l'autre, car elle dépend fortement du mode de fabrication, notamment si celui-ci est intégré. Il est donc essentiel d'analyser l'évolution de la marge brute au fil du temps.

\section{La valeur ajoutée}

La valeur ajoutée (VA) est un concept clé en économie, représentant la création de richesse réalisée par une entreprise. 
Sur le plan comptable, elle se définit comme la différence entre tout ce qu'une entreprise produit et vend, et tout ce qu'elle consomme en provenance de tiers. La valeur ajoutée prend en compte toutes les valeurs produites, qu'il s'agisse de ventes, de stockage ou d'immobilisations.

Elle offre une dimension économique réelle de l'entreprise, mettant en lumière sa capacité à créer de la valeur. De plus, le degré d'intégration, mesurant le pourcentage du cycle de production assuré par l'entreprise, est également un aspect important à considérer. 

Enfin, la structure interne de l'entreprise peut être analysée à travers la répartition de la valeur ajoutée entre les différents facteurs de production. 

Il convient de noter que la valeur ajoutée, bien qu'indicative de la performance économique, n'est pas un indicateur de rentabilité.

La valeur ajoutée (VA) se calcule comme suit :

\[
VA = \text{Production de l'exercice} + \text{Marge Commerciale (MC)}
\]

\subsubsection{Calcul de la VA dans le cadre des SIG}

\begin{center}
	\includegraphics[scale=0.6]{../../../Pictures/Screenshots/Capture d'écran 2025-02-17 214315}
\end{center}

Le taux de valeur ajoutée (VA) est exprimé en pourcentage de la production ou du chiffre d'affaires (CA) de l'entreprise. Il est essentiel de comparer ce taux à la moyenne du secteur pour évaluer la performance relative de l'entreprise. Des données sur le taux de VA par secteur sont disponibles sur le site de l'INSEE : \url{https://www.insee.fr/fr/statistiques/2015453}.

L'analyse de l'évolution du taux de VA peut s'avérer complexe. Par exemple, une variation du taux peut résulter d'une augmentation de la sous-traitance. Cela peut avoir des implications positives ou négatives sur la valeur ajoutée. Une hausse de la sous-traitance pourrait indiquer une optimisation des coûts, mais elle pourrait également signaler une dépendance accrue à des tiers pour des activités essentielles, ce qui pourrait nuire à la création de valeur interne.

\section{L'Excédent Brut d'Exploitation (EBE) et le résultat d'exploitation (RE)}

\subsubsection{Calcul de l'Excédent Brut d'Exploitation (EBE)}

\begin{center}
	\includegraphics[scale=0.7]{../../../Pictures/Screenshots/Capture d'écran 2025-02-17 215438}
\end{center}

\subsubsection{L'excédent brut d'exploitation (EBE)}

L'excédent brut d'exploitation (EBE) correspond au résultat purement économique dégagé par l'entreprise dans le cadre de ses activités courantes "\textit{normales}". Il représente la performance opérationnelle de l'entreprise avant la prise en compte des choix politiques en matière de structure financière, d'investissement, de financement et de fiscalité. Ainsi, l'EBE permet d'évaluer la rentabilité des opérations de base de l'entreprise sans être influencé par des décisions stratégiques ou des éléments financiers externes.

L'excédent brut d'exploitation (EBE) est un indicateur essentiel qui reflète la performance économique d'une entreprise. Il est utilisé par l'entreprise pour :

\begin{itemize}
	\item Faire face aux aléas des activités courantes et exceptionnelles.
	\item Maintenir et développer l'outil de production.
	\item Rémunérer les prêteurs et les associés.
	\item S'acquitter de ses obligations envers l'État.
\end{itemize}

Si l'EBE est inférieur à zéro, on parle d'insuffisance brute d'exploitation.

\subsubsection{Information Apportée par l'EBE}

L'EBE fournit des informations cruciales sur :

\begin{itemize}
	\item La maîtrise des charges de personnel.
	\item La comparaison de la progression de l'EBE par rapport à la valeur ajoutée (VA).
\end{itemize}

Il est à noter que le nombre de salariés est souvent reporté dans l'annexe des documents de synthèse, ce qui déterminera la capacité d'auto-financement de l'entreprise. De plus, l'EBE est un indicateur de la productivité de l'entreprise. Une cible courante est de maintenir la charge de personnel en dessous de 70\% à 75\% de la valeur ajoutée.

\subsubsection{Cas des Entreprises Individuelles}

Dans le contexte des entreprises individuelles, la situation diffère de celle des sociétés en ce qui concerne la rémunération du dirigeant. En effet, la rémunération du dirigeant n'est pas intégrée dans les charges de personnel, mais est assurée directement par le résultat de l'entreprise.

Ainsi, pour une entreprise individuelle, l'excédent brut d'exploitation (EBE) doit être plus élevé afin de couvrir non seulement les charges d'exploitation, mais aussi la rémunération du dirigeant. Cela souligne l'importance d'optimiser l'EBE pour garantir la viabilité économique de l'entreprise individuelle.

\subsubsection{Calcul du résultat d'exploitation (RE)}

\begin{center}
	\includegraphics[scale=0.6]{../../../Pictures/Screenshots/Capture d'écran 2025-02-17 221038}
\end{center}

\subsubsection{Résultat d'Exploitation (RE)}

Le résultat d'exploitation (RE) mesure la rentabilité industrielle et commerciale de l'entreprise. Il est un indicateur clé pour évaluer la performance opérationnelle.

\subsubsection{Informations Apportées par le RE}

Le RE prend en compte plusieurs éléments importants :

\begin{itemize}
	\item Politique d'Investissement : Il intègre les amortissements liés aux investissements réalisés par l'entreprise.
	\item Activités Courantes "Anormales" : Le RE inclut également d'autres charges et produits qui ne relèvent pas des activités normales.
	\item Aléas "Normaux" : Les dépréciations et provisions sont considérées comme des aléas normaux et sont donc intégrées dans le calcul du RE.
\end{itemize}

\subsubsection{Problèmes pour l'Analyse du RE}

Un des problèmes majeurs pour l'analyse du RE est le recours au crédit-bail, qui permet de financer l'investissement. Les charges liées au crédit-bail n'entrent pas dans le RE, car elles sont intégrées dans les frais financiers.

\subsubsection{Manipulations Comptables}

Certaines manipulations comptables peuvent affecter le RE :

\begin{itemize}
	\item Sous-estimation des Dépréciations et Provisions : Cette pratique peut "améliorer" artificiellement le résultat en réduisant les charges comptabilisées.
	\item Ajustement dans le Temps des Provisions : En lissant les provisions sur plusieurs périodes, les entreprises peuvent également influencer la présentation du résultat, rendant ainsi le RE moins représentatif de la réalité économique.
\end{itemize}

Ces éléments soulignent l'importance d'une analyse rigoureuse du RE pour comprendre la véritable performance de l'entreprise.


\section{Le résultat courant (RC) et le résultat net (RN)}

\subsubsection{Calcul du résultat courant (RC)}

\begin{center}
	\includegraphics[scale=0.6]{../../../Pictures/Screenshots/Capture d'écran 2025-02-17 223057}
\end{center}

Définition du Résultat sur Opérations en Commun

Le résultat sur opérations en commun fait référence aux bénéfices ou pertes réalisés sur les opérations effectuées par le biais de structures collaboratives telles que :

\begin{itemize}
	\item Groupement d'Intérêt Économique (GIE) : Un GIE permet à plusieurs entreprises de se regrouper pour réaliser des projets communs tout en conservant leur autonomie.
	\item Sociétés en Participation : Ce type de société permet à plusieurs partenaires de collaborer sur un projet spécifique sans créer une entité juridique distincte.
	\item Société de Moyens : Une société de moyens est constituée pour mettre en commun certains moyens matériels ou humains, facilitant ainsi la coopération entre les entreprises participantes.
\end{itemize}

Ce résultat est crucial pour évaluer l'impact économique des collaborations entre entreprises et pour mesurer l'efficacité de ces structures communes dans la réalisation d'objectifs partagés.

\subsubsection{Calcul du résultat net (RN)}

\begin{center}
\includegraphics[scale=0.6]{../../../Pictures/Screenshots/Capture d'écran 2025-02-17 224626}
\end{center}

\subsubsection{Remarque sur les Éléments Exceptionnels}

Les éléments exceptionnels sont des événements :

\begin{itemize}
	\item Aléatoires : Ils ne peuvent pas être prévus par définition.
	\item Non Intégrés dans les SIG Précédents : Cela est fait pour éviter de "perturber" l'analyse des résultats.
\end{itemize}

\subsubsection{Information Apportée par les Éléments Exceptionnels}

Les éléments exceptionnels affectent le résultat net (RN) et doivent donc être intégrés dans l'analyse. Ils peuvent révéler des insuffisances graves, par exemple :

\begin{itemize}
	\item Exemple 1 : Amendes fiscales, rappels d'impôts, cotisations sociales suite à des contrôles.
	\item Exemple 2 : Pertes récurrentes sur cessions d'immobilisations, amortissements insuffisants et mauvaise politique d'investissement.
\end{itemize}

\subsubsection{Remarques}

\begin{itemize}
	\item Report du RN : Le résultat net est reporté au passif du bilan, considéré comme une ressource.
	\item Répartition du RN : Le RN peut être réparti sous forme de dividendes ou être mis en réserve.
\end{itemize}

\subsubsection{Information du RN}

Le RN mesure la rentabilité finale de l'entreprise, particulièrement pour les associés. Il est important de le comparer :

\begin{itemize}
	\item Aux Capitaux Propres : Pour évaluer la rentabilité des investissements.
	\item Au Chiffre d'Affaires : Pour analyser la performance opérationnelle.
\end{itemize}

Ces éléments soulignent l'importance d'une analyse complète des éléments exceptionnels pour comprendre la santé financière de l'entreprise.

\section{La capacité d'autofinancement (CAF)}

\subsubsection{Calcul de la CAF}

\begin{center}
	\includegraphics[scale=0.6]{../../../Pictures/Screenshots/Capture d'écran 2025-02-17 225232}
\end{center}

\subsubsection{Logique du Calcul : Retraitement du Résultat Net (RN)}

Le retraitement du RN vise à "neutraliser" :

\begin{enumerate}
	\item Produits et Charges Non Encaissables : Éléments qui n'entraînent pas d'encaissements ou de décaissements effectifs.
	\item Opérations en Capital : Transactions liées à la cession d'actifs.
\end{enumerate}

\subsubsection{Information Apportée}

Le retraitement permet d'obtenir le résiduel de trésorerie (effective ou potentielle) générée par l'ensemble des opérations de gestion de l'entreprise, correspondant à :

\begin{itemize}
	\item Écart entre Produits Encaissables et Charges Décaissables.
\end{itemize}

\subsubsection{Remarques}

\begin{itemize}
	\item Potentielle : Les calculs sont basés sur les enregistrements comptables et non sur les règlements effectifs (ex : chiffre d'affaires à facturer).
	\item Opérations de Gestion : Seules les opérations d'exploitation, financières ou exceptionnelles sont prises en compte. Les opérations en capital (ex : cessions d'actifs) ne le sont pas.
\end{itemize}

\subsubsection{Utilisations Possibles de la Capacité d'Autofinancement (CAF)}

La CAF peut être utilisée pour :

\begin{itemize}
	\item Faire face aux Aléas de Gestion : Constitution de provisions.
	\item Maintenir la Capacité de Production : Dotation aux amortissements.
	\item Contribuer au Financement de la Croissanc : Investissements en immobilisations.
	\item Rémunérer les Associés : Distribution de dividendes.
\end{itemize}

\subsubsection{Auto-financement}

L'auto-financement est calculé comme suit :

\[
\text{Auto-financement} = \text{CAF} - \text{Distribution (dividendes)}
\]

\subsubsection{Manipulations Comptables}

Certaines manipulations peuvent influencer la CAF :

\begin{itemize}
	\item Immobilisation de Charges : Cas des immobilisations incorporelles dans le cadre des frais de recherche et développement.
	\item Gonflement du Chiffre d'Affaires : Exemple du chiffre d'affaires à facturer.
\end{itemize}

\subsubsection{Calcul des Soldes Intermédiaires de Gestion (SIG)}

Les SIG fournissent des indicateurs sur différentes dimensions de la rentabilité de l'entreprise et des informations sur la "construction" du résultat. 

\subsubsection{Limite}

Les SIG sont mesurés en unité monétaire (i.e., en euros), ce qui peut limiter leur interprétation. 

\subsubsection{Définition de Ratio}

Les ratios sont des outils d'analyse qui permettent d'évaluer la performance financière et opérationnelle de l'entreprise.

\chapter{Le bilan financier}


Le bilan comptable, tel qu'exposé au Chapitre 1, présente certaines limites qui le rendent inadapté pour une analyse financière approfondie. En effet, il nécessite des retraitements afin d'obtenir un bilan financier plus pertinent. Ce dernier permet de calculer les grandes masses financières du bilan, offrant ainsi une structure mieux adaptée à l'analyse financière. Grâce à cette approche, il devient possible de faire ressortir les points forts et faibles du bilan, facilitant ainsi une compréhension plus claire de la santé financière de l'entreprise. L'objectif principal est de mettre en lumière les liens entre les ressources, représentées par le passif, et les besoins ou emplois, qui sont quant à eux reflétés dans l'actif.

\section{Le reclassement du bilan}

\subsection{Masses financières (tant à l'actif qu'au passif)}

Le reclassement du bilan est une étape cruciale dans l'analyse des valeurs structurelles d'une entreprise. Il permet de réorganiser les masses financières tant à l'actif qu'au passif, facilitant ainsi une compréhension approfondie de la structure financière. En examinant le haut du bilan, on y trouve les emplois stables et les ressources stables, qui sont directement liés aux investissements à long terme de l'entreprise. À l'inverse, le bas du bilan regroupe les emplois variables et les ressources variables, en relation avec l'activité courante de l'entreprise. Enfin, la trésorerie représente à la fois des emplois et des ressources en liquidités, soulignant l'importance de la gestion de la trésorerie dans le fonctionnement quotidien de l'entreprise. Ce reclassement permet ainsi d'obtenir une vision claire des différentes composantes financières, essentielles pour une analyse pertinente.

\subsection{Traitements à l'actif}

Les traitements à l'actif sont essentiels pour obtenir une vision précise de la situation financière d'une entreprise. Concernant les immobilisations, il est crucial de prendre en compte les valeurs nettes, c'est-à-dire après déduction des amortissements et des dépréciations. L'actif se divise en deux catégories : l'actif d'exploitation, qui comprend les éléments nécessaires au fonctionnement quotidien de l'entreprise, et l'actif hors exploitation. En ce qui concerne les stocks et les encours, il est important de les évaluer à leur valeur nette au bilan. De même, les avances et acomptes versés doivent être présentés sous leur valeur nette. Les créances clients et les comptes rattachés figurent également au bilan à leur valeur nette, augmentée des effets escomptés non échus (EENE), afin de refléter la réalité des créances. Les autres créances doivent également être évaluées à leur valeur nette. Enfin, la trésorerie est constituée des valeurs mobilières nettes et des disponibilités, qui sont des éléments clés pour la gestion de la liquidité de l'entreprise.

\subsection{Traitements au passif}

Les traitements au passif sont cruciaux pour une évaluation complète de la structure financière d'une entreprise. Les ressources stables comprennent principalement les ressources propres, qui englobent les capitaux propres ainsi que les comptes courants d'associés, tout en tenant compte des non-valeurs de l'actif, telles que les écarts de conversion. Les provisions pour risques et charges sont également intégrées dans cette catégorie, car elles représentent des engagements futurs potentiels. En outre, les emprunts à moyen et long terme, qui peuvent inclure les comptes courants d'associés, constituent une source de financement stable pour l'entreprise. 

Les dettes d'exploitation et hors exploitation sont également des éléments importants, comprenant les avances et acomptes reçus, les dettes fournisseurs et les comptes rattachés, ainsi que les dettes fiscales et sociales. Les autres dettes, excluant la trésorerie, doivent également être prises en compte pour une évaluation globale. 

Enfin, la trésorerie au passif comprend les escomptes, les soldes créditeurs de banque, tels que les découverts et les facilités de caisse, ainsi que d'autres concours de trésorerie, comme la mobilisation Dailly, qui sont essentiels pour la gestion de la liquidité et le bon fonctionnement de l'entreprise.

Le bilan financier regroupe les postes selon leur fonction économique, ce qui permet une analyse plus pertinente de la structure financière de l'entreprise. Dans cette optique, les emprunts et les dettes sont classés dans différents postes, facilitant ainsi leur identification et leur évaluation. Les ressources d'emprunt stables sont généralement distinguées des dettes bancaires, qui sont souvent considérées comme des engagements à court terme. En pratique, il est possible d'isoler la part des emprunts à moyen et long terme qui échue dans moins d'un an. Cette distinction est particulièrement utile pour comparer ces montants avec la capacité d'autofinancement (CAF) de l'entreprise, permettant ainsi d'évaluer la solvabilité et la liquidité à court terme.

\begin{center}
	\includegraphics[scale=0.6]{../../../Pictures/Screenshots/Capture d'écran 2025-02-25 154219}
\end{center}

\section{Les valeurs structurelle}

Les valeurs structurelles permettent de caractériser la structure du bilan d'une entreprise, offrant ainsi une vision claire de sa santé financière. Quatre éléments principaux sont calculés pour cette analyse :

\begin{enumerate}
	\item \textbf{Les ressources propres} : Elles reflètent la part des capitaux qui appartient aux actionnaires et sont essentielles pour évaluer la solvabilité de l'entreprise.
	\item \textbf{Le fonds de roulement (FR)} : Il représente la différence entre les ressources stables et les emplois stables, indiquant la capacité de l'entreprise à financer son exploitation courante.
	\item \textbf{Le besoin en fonds de roulement (BFR)} : Ce montant représente les ressources nécessaires pour financer le cycle d'exploitation, incluant les stocks et les créances clients.
	\item \textbf{La trésorerie nette (TN)} : Elle est calculée en soustrayant les dettes de trésorerie des disponibilités, fournissant une indication de la liquidité immédiate de l'entreprise.
\end{enumerate}

Ces quatre éléments sont cruciaux pour une évaluation complète de la structure financière et de la gestion des ressources d'une entreprise.

\subsection{Les ressources propres}

Les ressources propres représentent les capitaux apportés par les associés ou laissés à la disposition de l'entreprise. Elles supportent le risque de faillite, ce qui en fait un indicateur clé de la solidité financière d'une entreprise. On peut les définir par la formule suivante :
\[
\text{Ressources propres} = 
\]
\[ 
\text{Capitaux propres} + \text{Comptes courants d\'\,associés bloqués} - \text{Actif sans valeur}
\]
En pratique, il est courant de soustraire également le bénéfice net (RN) à distribuer aux associés, bien que cette distribution de dividendes soit souvent inconnue au 31/12/N. Cette approche se réfère généralement à la politique appliquée pour l'affectation des résultats.

Les informations apportées par les ressources propres permettent d'évaluer le risque de non-remboursement encouru par les créanciers. Selon la norme, il est recommandé que les ressources propres représentent au moins 20\% du total net du bilan. Cela doit permettre de couvrir les moins-values potentielles sur les cessions des actifs en cas de liquidation, illustrant l'importance de la valeur liquidative des installations techniques par rapport à leur valeur de bilan.

\subsection{Le fonds de roulement (FR)}

Le fonds de roulement (FR) représente la part des ressources stables disponibles après le financement des actifs immobilisés (et des actifs sans valeur) pour contribuer à couvrir les besoins de financement liés à l'exploitation, c'est-à-dire l'actif circulant. 

Il est défini comme l'excédent des ressources stables sur les emplois stables, et peut être calculé selon la formule suivante :

\[
FR = \text{Ressources stables} - \text{Actif immobilisé net}
\]

Un fonds de roulement positif indique que l'entreprise dispose de ressources suffisantes pour financer ses opérations courantes, tandis qu'un fonds de roulement négatif peut signaler des difficultés potentielles dans la gestion de la liquidité.

\begin{center}
	\includegraphics[scale=0.5]{../../../Pictures/Screenshots/Screenshot 2025-02-25 at 17-37-28 Chap4.pdf}
\end{center}

Le fonds de roulement (FR) fournit des informations cruciales sur la continuité de l'exploitation d'une entreprise, agissant comme un indicateur de sa vulnérabilité ou de son autonomie vis-à-vis des prêteurs à court terme, tels que les fournisseurs et les banquiers. 

Un fonds de roulement positif est essentiel, car il doit permettre de financer certains éléments de l'actif circulant, notamment les stocks et les créances clients. L'objectif est que le fonds de roulement soit supérieur à zéro, ce qui indiquerait que l'entreprise est capable de "gérer" une remise en cause de ses financements à court terme.

Cependant, il existe des cas particuliers dans certains secteurs, comme la grande distribution ou certains services, où un fonds de roulement négatif peut être acceptable. Cela s'explique par des pratiques telles que :

\begin{itemize}
	\item Des stocks de faible valeur ou à rotation rapide.
	\item Des clients qui paient comptant.
	\item Des volumes d'achat importants par rapport au chiffre d'affaires, permettant aux fournisseurs d'accorder des délais de paiement longs.
\end{itemize}

La norme pour le fonds de roulement est spécifique à chaque secteur ; par exemple, dans les entreprises industrielles classiques, un fonds de roulement supérieur à 20\% de l'actif circulant est souvent recommandé.

\subsection{Le besoin en fonds de roulement (BFR)}

Le besoin en fonds de roulement (BFR) représente la part des besoins de financement liés à l'activité (qu'elle soit d'exploitation ou hors exploitation) qui n'est pas couverte par les ressources provenant de cette même activité. 

Il est défini comme l'excédent des emplois variables sur les ressources variables, et peut être calculé selon la formule suivante :

\[
BFR = \text{Actifs d\'\,exploitation et hors exploitation} - \text{Dettes d\'\,exploitation et hors exploitation}
\]

Il est également possible de distinguer et d'utiliser un BFR spécifique à l'exploitation, ce qui permet une analyse plus fine des besoins de financement liés aux opérations courantes de l'entreprise.

\begin{center}
	\includegraphics[scale=0.5]{../../../Pictures/Screenshots/Screenshot 2025-02-25 at 17-37-28 Chap4.pdf}
\end{center}

Le besoin en fonds de roulement (BFR) est étroitement lié au financement de l'exploitation, incluant les stocks et les créances clients. Les ressources de l'exploitation, quant à elles, comprennent les dettes fournisseurs ainsi que les dettes fiscales et sociales, telles que la TVA à décaisser et les salaires et charges sociales dus. 

L'importance de ces postes dépend des contraintes spécifiques au secteur et à l'entreprise, telles que la durée du cycle de production et les usages en matière de crédit. Cela rend l'interprétation du BFR délicate.

Le BFR mesure un besoin de financement structurel et devrait être couvert, en partie, par des ressources stables, notamment à travers le fonds de roulement (FR). Les évolutions du BFR dans le temps peuvent être volatiles, reflétant les variations du cycle d'exploitation. 

Il est important de noter que le BFR est spécifique à chaque secteur et à chaque entreprise, rendant son analyse contextuelle et nuancée.

\subsection{La trésorerie nette (TN)}

La trésorerie nette (TN) est définie comme la différence entre les liquidités (qu'elles soient placées ou non) et les financements bancaires courants. Elle peut être formulée comme suit :

\[
\text{Trésorerie nette} = \text{Trésorerie active} - \text{Trésorerie passive}
\]

De plus, la relation entre la trésorerie nette, le fonds de roulement (FR) et le besoin en fonds de roulement (BFR) peut s'exprimer par l'équation suivante :

\[
\text{TN} = FR - BFR
\]

Un indicateur important est que lorsque la trésorerie nette est inférieure à zéro (TN < 0), cela représente une situation standard pour de nombreuses petites et moyennes entreprises (PME). Dans ce contexte, la contribution des banques est essentielle au financement des besoins courants de l'entreprise. 

Cependant, un risque majeur réside dans le fait qu'une part importante du BFR financée par la trésorerie nette peut compromettre la continuité de l'exploitation. En effet, la suppression ou la diminution des financements bancaires de trésorerie peut entraîner des difficultés financières significatives.

\subsection{Normes sur les valeurs structurelles}

Une entreprise est considérée en "bonne" situation si elle respecte certaines normes relatives à ses valeurs structurelles. Parmi ces normes, on peut citer :

\begin{itemize}
	\item Un besoin en fonds de roulement (BFR) raisonnable par rapport à l'activité ou au marché de l'entreprise.
	\item Un BFR financé de manière significative par le fonds de roulement (FR), idéalement supérieur à 40\%.
	\item Un fonds de roulement composé à plus de 50\% par des ressources propres.
\end{itemize}

Ces critères permettent d'évaluer la solidité financière de l'entreprise et sa capacité à faire face à ses obligations à court terme.

\section{Situations types}

\subsection{Situation soutenable (de référence)}

Une situation soutenable pour une entreprise peut être caractérisée par les critères suivants :

\begin{itemize}
	\item Fonds de roulement (FR) > 0
	\item Besoin en fonds de roulement (BFR) > 0
	\item BFR > FR, ce qui indique un besoin de crédit de trésorerie.
\end{itemize}

Cette situation implique que le fonds de roulement, c'est-à-dire les ressources de moyen à long terme de l'entreprise, finance une partie du BFR, qui est lié à l'activité de l'entreprise. De plus, les banques jouent également un rôle crucial en finançant une partie du BFR, permettant ainsi à l'entreprise de maintenir ses opérations courantes.

\subsection{FR négatif et BFR négatif}

\begin{center}
	\includegraphics[scale=0.5]{../../../Pictures/Screenshots/Screenshot 2025-02-25 at 17-37-28 Chap4.pdf}
\end{center}

Lorsque le fonds de roulement (FR) et le besoin en fonds de roulement (BFR) sont tous deux négatifs, cela indique une situation financière particulière. L'analyse de cette situation peut être résumée comme suit :

\begin{itemize}
	\item Il y a une insuffisance des ressources stables par rapport aux emplois stables, ce qui se traduit par un FR < 0.
	\item Cette insuffisance est compensée par un excédent des ressources variables sur les emplois variables, ce qui signifie que le BFR < 0.
\end{itemize}

Cette configuration est courante dans certains secteurs, notamment :

\begin{itemize}
	\item La grande distribution, où les délais de paiement des fournisseurs peuvent être plus longs que les délais de paiement des clients.
	\item Les entreprises faisant appel à la sous-traitance, avec des délais de paiement des sous-traitants supérieurs aux délais de paiement des clients.
\end{itemize}

Ces situations peuvent permettre à ces entreprises de fonctionner malgré des indicateurs financiers négatifs.

\subsection{FR négatif et BFR positif}

\begin{center}
	\includegraphics[scale=0.5]{../../../Pictures/Screenshots/Capture d'écran 2025-02-25 192145}
\end{center}

Lorsque le fonds de roulement (FR) est négatif et le besoin en fonds de roulement (BFR) est positif, cela indique une situation financière préoccupante. L'analyse de cette situation peut être résumée comme suit :

\begin{itemize}
	\item Il y a une insuffisance des ressources stables par rapport aux emplois stables, ce qui se traduit par un FR < 0.
	\item Cette situation est accompagnée d'une insuffisance des ressources variables sur les emplois variables, ce qui signifie que le BFR > 0.
	\item Ainsi, il existe une insuffisance des ressources aussi bien dans le haut que dans le bas du bilan.
\end{itemize}

Cette configuration implique un financement par des crédits de trésorerie. Cependant, il est important de noter que cette situation ne peut être que temporaire, car les banques n'acceptent généralement pas de financer des entreprises présentant ce type de structure déséquilibrée. Une telle situation doit donc être rapidement rectifiée pour assurer la pérennité de l'entreprise.

\subsection{FR positif et BFR négatif}

\begin{center}
	\includegraphics[scale=0.5]{../../../Pictures/Screenshots/Capture d'écran 2025-02-25 192648}
\end{center}

La situation où le fonds de roulement (FR) est positif et le besoin en fonds de roulement (BFR) est négatif mérite une attention particulière. Elle soulève des questions sur la gestion financière et l'utilisation des ressources.

Il est essentiel de reclasser les valeurs structurelles pour mieux comprendre l'impact de cette configuration sur l'entreprise.

Cette situation se caractérise par :

\begin{itemize}
	\item Un excédent des ressources stables sur les emplois stables, ce qui se traduit par un FR > 0.
	\item Un excédent des ressources variables sur les emplois variables, ce qui signifie que le BFR < 0.
	\item Ainsi, il y a un excédent de ressources tant dans le haut que dans le bas du bilan.
\end{itemize}

Cette configuration entraîne des excédents de trésorerie. Cependant, il est pertinent de se demander : quelle est l'utilité d'avoir un FR > 0 lorsque le BFR est < 0 ?

Cette situation est généralement temporaire, car un FR positif devient inutile si le BFR est négatif. Cela peut indiquer un excès de fonds propres ou d'emprunts à moyen-long terme qui sont à la fois inutiles et coûteux pour l'entreprise.

\subsection{FR nul et BFR positif}
	
\begin{center}
	\includegraphics{../../../Pictures/Screenshots/Capture d'écran 2025-02-25 193225}
\end{center}

Dans une situation où le fonds de roulement (FR) est nul et le besoin en fonds de roulement (BFR) est positif, on observe les caractéristiques suivantes :
	
	\begin{itemize}
		\item Les ressources stables de l'entreprise sont équivalentes à l'actif immobilisé net, ce qui implique que FR = 0.
		\item Le BFR est entièrement financé par du crédit bancaire de trésorerie, créant ainsi une dépendance significative de l'entreprise vis-à-vis du secteur bancaire pour financer son activité.
	\end{itemize}
	
Cette situation est considérée comme temporaire, car les banques ne vont généralement pas accepter de financer uniquement le BFR de l'entreprise sur le long terme. Une telle dépendance peut poser des risques financiers importants et nécessite une attention particulière pour assurer la pérennité de l'entreprise.

\section{Conclusion}

Le bilan financier est un outil essentiel qui permet de calculer les valeurs structurelles d'une entreprise. Il offre une évaluation de la structure du bilan, permettant de répondre à des questions cruciales telles que :

\begin{itemize}
	\item Le fonctionnement et le financement de l'entreprise sont-ils soutenables ?
	\item Quelles sont les marges d'endettement disponibles ?
	\item Y a-t-il un besoin accru de fonds propres ?
\end{itemize}

Cependant, il est important de noter certaines limites. Les valeurs structurelles peuvent varier au cours d'une année, ce qui rend nécessaire l'établissement de normes claires. De plus, il est essentiel de définir et d'utiliser des ratios appropriés pour évaluer efficacement la structure de l'entreprise. Ces éléments permettent d'obtenir une vision plus précise et plus fiable de la santé financière de l'entreprise.

\section{Annexe : définitions}

\subsection{Définition des EENE}

Les EENE, ou effets escomptés non échus, désignent un poste comptable représentant les actifs escomptés auprès des banques, mais dont le terme n'est pas encore arrivé. 

Une entreprise entretient des relations avec des fournisseurs et des clients. Lorsqu'elle accorde des délais de paiement à ses clients, elle détient des créances à leur égard. Ces créances peuvent être considérées comme des effets de commerce. Un effet de commerce est un titre négociable basé sur la créance. Le détenteur de cet effet peut demander le remboursement de la créance auprès du client de l'entreprise. 

Ainsi, l'effet de commerce peut être cédé à un tiers. Dans ce cas, l'entreprise cède ses effets de commerce et reçoit en contrepartie une somme d'argent correspondant au montant de la créance, déduction faite de la commission. Cette commission s'explique par l'avance en trésorerie, mais aussi par les risques encourus par l'acquéreur de l'effet de commerce en cas de défaillance du client.

L'effet escompté non échue (EENE) est donc un effet de commerce dont la créance n'est pas encore arrivée à son terme. L'échéance de l'effet de commerce est précisée dans ses caractéristiques dès l'origine.

\subsection{Définition de mobilisation "Dailly"}

La mobilisation "Dailly" est une technique de financement qui permet à une entreprise de bénéficier de crédit en contrepartie de la production de factures ou d'autres documents représentant des créances sur ses clients ou sur une collectivité publique (État, région, département, commune). 

Il est donc possible de mobiliser une créance sur un client, sur une subvention obtenue mais non versée, sur un crédit de TVA, etc. Cette technique met en relation un établissement de crédit, un emprunteur et un débiteur de la créance.

La cession ou le nantissement d'une créance ne peut être consenti qu'au profit d'un établissement de crédit, tel qu'une banque ou une société financière. L'emprunteur doit être une entreprise, qu'il s'agisse d'une société, d'une association, d'un commerçant, d'un artisan, d'un professionnel libéral, ou d'un agriculteur. En revanche, le débiteur de la créance cédée ou nantie doit être une entreprise ou une collectivité publique (État, région, département ou commune). Il est important de noter qu'il est impossible de mobiliser des créances sur des particuliers ; par exemple, un avocat ne peut pas utiliser cette technique s'il travaille avec des clients particuliers.





































\end{document}
