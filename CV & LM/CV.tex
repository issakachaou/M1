\documentclass[11pt,a4paper,french]{moderncv}        % autres options possibles : taille de fonte ('10pt', '11pt' et '12pt'), format de papier ('a4paper', 'letterpaper', 'a5paper', 'legalpaper', 'executivepaper' and 'landscape') et famille de fonte ('sans' and 'roman')
\moderncvstyle{classic}                             % autres styles : 'casual' (défaut), 'classic', 'oldstyle' and 'banking'
\moderncvcolor{red}                               % autres couleurs : 'blue' (défaut), 'orange', 'green', 'red', 'purple', 'grey' and 'black'
%\nopagenumbers{}                                  % décommenter pour supprimer la numérotation automatique des pages pour les CVs de plus d'une page
\usepackage[utf8]{inputenc}                       % si vous n'utilisez pas xelatex ou lualatex, remplacer par le codage d'entrée que vous utilisez
\usepackage[scale=0.75,a4paper]{geometry}
\usepackage{babel}
\usepackage{libertine}
\usepackage{ragged2e}
\usepackage{newunicodechar}
\newunicodechar{»}{{\fontencoding{T1}\selectfont\guillemotright}}
\newunicodechar{«}{{\fontencoding{T1}\selectfont\guillemotleft}}
\nopagenumbers
%----------------------------------------------------------------------------------
%            informations personnelles
%----------------------------------------------------------------------------------
\firstname{Issa}
\familyname{Kachaou}
%\title{Admission en Master MASERATI}                               % optionnel : supprimer ou commenter si non souhaité
\address{1 rue François Mansart}{94000, Créteil}{} % optionnel : supprimer ou commenter si non souhaité; l'argument « Pays » peut être omis ou vide
\mobile{06 95 32 45 02}                          % optionnel : supprimer ou commenter si non souhaité
%\phone{Numéro de téléphone}                           % optionnel : supprimer ou commenter si non souhaité
%\fax{Numéro de fax}                             % optionnel : supprimer ou commenter si non souhaité
\email{issa.kachaou@etu.u-pec.fr}                               % optionnel : supprimer ou commenter si non souhaité
%\homepage{Page Web personnelle}                         % optionnel : supprimer ou commenter si non souhaité
\extrainfo{22 ans}                 % optionnel : supprimer ou commenter si non souhaité
% \photo[64pt][0.4pt]{Image} % optionnel : décommenter si souhaité ; '64pt' est un exemple de hauteur que doit avoir la photo, 0.4pt est un exemple d'épaisseur que doit avoir le cadre qui l'entoure (à mettre à 0pt pour supprimer le cadre) et « Image » est le nom du fichier de la photo
%\quote{Citation}                                 % optionnel : supprimer ou commenter si non souhaité
%
\begin{document}
\makecvtitle
\section{Formation}
\cventry{2024}{Licence Économie et Gestion}{Université Paris-Est Créteil}{}{Mention Expertise économique et financière}{Microéconométrie, Mathématiques des systèmes dynamiques, Macroéconomie}                      % les arguments 3 à 6 peuvent être laissés vides
\cventry{2019}{Baccalauréat série ES (Économique et Social)}{Lycée Léon Blum}{Créteil}{Mention Assez bien}{Spécialité Mathématiques}
\section{Expérience}
\subsection{Principale}
\cventry{2021-2023}{Moniteur étudiant}{Université Paris-Est Créteil}{}{}{}
%\cventry{Année--Année}{Emploi}{Employeur}{Ville}{}{Description générale d'au plus 1 ou 2 lignes}
%\subsection{Divers}
%\cventry{Année--Année}{Emploi}{Employeur}{Ville}{}{Description générale d'au plus 1 ou 2 lignes}
\section{Langues}
\cvitemwithcomment{Français}{Langue maternelle}{}
\cvitemwithcomment{Anglais}{Niveau B2}{}
\cvitemwithcomment{Espagnol}{Notions}{}
\section{Compétences informatiques}
\cvdoubleitem{Logiciel SAS}{Notions de bases}{\LaTeX}{Maîtrise}
\cvdoubleitem{R}{Notions de bases}{Suite Microsoft Office}{Maîtrise experte}
\cvdoubleitem{Python}{Notions de bases}{Zotero}{Maîtrise}
%\section{Centres d'intérêt}
%\cvitem{Loisir 1}{Description}
%\cvitem{Loisir 2}{Description}
%\cvitem{Loisir 3}{Description}
%\section{Extra 1}
%\cvlistitem{Item 1}
%\cvlistitem{Item 2}
%\cvlistitem{Item 3}
%\section{Extra 2}
%\cvlistdoubleitem{Item 1}{Item 4}
%\cvlistdoubleitem{Item 2}{Item 5}
%\cvlistdoubleitem{Item 3}{Item 6}
%\section{References}
%\begin{cvcolumns}
  %\cvcolumn{Catégorie 1}{Commentaire}
  %\cvcolumn{Catégorie 2}{Commentaire}
  %\cvcolumn{Catégorie 3}{Commentaire}
%\end{cvcolumns}
\clearpage
%-----       letter       ---------------------------------------------------------
% recipient data
\recipient{Université Paris Sorbonne Nord}{Faculté de Sciences Economiques et de Gestion\\99 avenue Jean-Baptiste Clément\\93430, Villetaneuse}
\date{A Créteil, le 18 mars 2024}
\opening{Madame, Monsieur,}
\closing{Je vous prie d’agréer, Madame, Monsieur, l’expression de mes sentiments distingués.}
%\enclosure{Pièces jointes}          %  utiliser l'argument optionnel pour spécifier un autre mot que "Enclosure", ou redéfinir \enclname
\makelettertitle
\justifying

\hspace*{1cm}
Actuellement étudiant en 3 \ieme ~année de licence d’économie en parcours expertise économique et financière à l’Université Paris-Est Créteil, je me permets de vous adresser ma candidature afin d’intégrer le master ingénierie financière et modélisation (IFIM) au sein de l'Université Paris Sorbonne Nord.

Durant mon parcours universitaire, j'ai pu me passionner pour la programmation et l’économétrie, et je suis convaincu que le master IFIM représente l'opportunité idéale pour approfondir mes connaissances et développer mes compétences dans ces domaines. Mon parcours académique m’a permis d'acquérir de solides bases en mathématiques, en économétrie et en économie, mais également en programmation et en analyse de données. J'ai pu notamment lors de mon mémoire de fin de licence, intitulé « De l’impact de la santé mentale sur l’économie », mettre en application mes connaissances en économétrie et en analyse de données et je souhaiterais poursuivre ma formation dans ce domaine.

Le master IFIM et ses enseignements de « Introduction à python et à la gestion des bases de données  », « Microéconomie de l’incertain et de l’information  » et de « Microéconomie de l’incertain et de l’information  » ont tout particulièrement attiré mon attention lors de mes recherches de formations. Sa renommée dans le secteur ainsi que le témoignage de nombreux anciens élèves de votre master m'ont confirmé de l'excellence de votre programme.

En plus de mon intérêt académique pour le domaine, je suis par ailleurs attiré par les débouchés professionnels offerts par le master IFIM. En effet, je souhaiterais, au terme de mes études, effectuer un doctorat et me spécialiser en microéconométrie de l’évaluation des politiques publiques. Je suis donc déterminé à exploiter toutes les opportunités offertes par ce master pour développer mes compétences techniques grâce à des stages et à la production d’un mémoire.

Enfin, je suis convaincu que ma motivation, ma rigueur et ma capacité à travailler en équipe sont autant d'atouts qui me permettront de m'épanouir pleinement au sein du master IFIM.

Je me tiens à votre entière disposition pour tout complément d'information et vous remercie par avance de l'attention que vous porterez à ma candidature.














\makeletterclosing
\end{document}
