\documentclass[a4paper, 12pt]{report}
\usepackage[utf8]{inputenc} 
\usepackage{graphicx}
\usepackage[utf8]{inputenc} 
\usepackage[french]{babel}
\usepackage[T1]{fontenc}
\usepackage{amsmath,amsfonts,amssymb, empheq}
\usepackage{afterpage}
\begin{document}
\thispagestyle{empty}	
\begin{center}
	\vspace*{5cm}\Huge{{\sc\textbf{Projet recherche en master}}} \\ \vspace*{5cm}
	\textbf{Issa KACHAOU}
\end{center}


\newpage

\chapter*{Analyse de l'économie des ressources naturelles au Moyen-Orient: Impacts, défis et voies vers la durabilité}

\section*{Introduction}

Le Moyen-Orient est une région riche en ressources naturelles telles que le pétrole, le gaz naturel et les minéraux. Ces ressources jouent un rôle crucial dans l'économie de la région, mais elles présentent également des défis et des risques. Ce projet de recherche vise à analyser l'économie des ressources naturelles au Moyen-Orient en mettant l'accent sur les impacts économiques, les défis rencontrés et les voies possibles vers la durabilité.

\section*{Objectifs de recherche}

L'analyse des effets économiques des ressources naturelles sur la croissance et le développement au Moyen-Orient révèle des défis inhérents à une dépendance excessive, tels que la volatilité des prix et la vulnérabilité aux chocs externes, soulignant ainsi la nécessité d'une diversification économique. Les politiques et stratégies actuelles de gestion durable des ressources naturelles dans la région sont examinées, tandis que les opportunités de développement durable au-delà de ces ressources, comme dans les secteurs de l'énergie renouvelable, du tourisme, de l'agriculture et de l'industrie manufacturière, sont identifiées. En vue d'une gestion plus équilibrée, des recommandations politiques sont proposées, mettant en avant la promotion de la diversification économique, l'investissement dans des secteurs non pétroliers et la stimulation de l'innovation pour assurer la durabilité à long terme de la région.

\section*{Méthodologie}

La collecte de données économiques et de ressources naturelles auprès d'organismes nationaux et internationaux dans les pays du Moyen-Orient est entreprise en vue d'analyser les tendances économiques et d'évaluer l'impact des ressources naturelles sur la croissance économique. Cette analyse est complétée par l'étude des politiques de gestion des ressources naturelles dans la région, en examinant les approches réglementaires, fiscales et de gouvernance adoptées par les différents pays. Parallèlement, une analyse des opportunités de diversification économique au-delà des ressources naturelles est menée, en prenant en compte les avantages comparatifs, les ressources humaines et les infrastructures existantes. Sur cette base, des recommandations politiques sont formulées pour une gestion plus durable des ressources naturelles au Moyen-Orient, en mettant particulièrement l'accent sur la promotion de la diversification économique, l'investissement dans les secteurs non pétroliers et la stimulation de l'innovation.

\section*{Résultats attendus}

Ce projet vise à approfondir notre compréhension des impacts économiques des ressources naturelles sur la croissance et le développement au Moyen-Orient, tout en évaluant les défis associés à une dépendance excessive à ces ressources et à une faible diversification économique. Nous entreprendrons une analyse approfondie des politiques et des stratégies actuelles visant à gérer de manière durable les ressources naturelles dans la région. En parallèle, nous identifierons les opportunités de diversification économique et de développement durable au-delà des ressources naturelles, en mettant en lumière les secteurs prometteurs tels que les énergies renouvelables, le tourisme, l'agriculture et l'industrie manufacturière. Enfin, nous formulerons des recommandations politiques pour une gestion plus équilibrée des ressources naturelles au Moyen-Orient, en mettant l'accent sur la promotion de la diversification économique, l'investissement dans les secteurs non pétroliers et la stimulation de l'innovation pour soutenir une croissance durable dans la région.

\section*{Conclusion}

Ce projet de recherche permettra d'analyser l'économie des ressources naturelles au Moyen-Orient en mettant l'accent sur les impacts, les défis et les voies vers la durabilité. En identifiant les impacts économiques, les défis et les opportunités de diversification économique, ce projet fournira des recommandations politiques pour une gestion plus durable des ressources naturelles et un développement économique équilibré au Moyen-Orient.

\section*{Bibliographie}

Khouri, R. G. (2018). The Gulf Economies: From Economic Miracle to Future Challenges. Saqi Books. \\

Chalmin, P. (2015). La soif du monde: Water Wars: Comprendre les crises de l'eau du XXIe siècle. Éditions Odile Jacob.\\

Faucon, B. (2017). Les dessous du pétrole: La face cachée de la manne pétrolière. La Découverte.\\

Ianchovichina, E., & Devarajan, S. (2017). Natural Resources and Diversification in the Middle East and North Africa: The Role of Policies and Institutions. World Bank Publications.\\

Diwan, I., & Gelb, A. H. (Eds.). (2016). Oil, Urbanization, and Political Change in the Middle East. Oxford University Press.\\

Karl, T. L. (1997). The Paradox of Plenty: Oil Booms and Petro-States. University of California Press.\\

Luciani, G. (1990). The Arab Petro-States in the World Economy: An Inquiry into the Nature of Dependency. Routledge.\\

World Bank. (2015). Arab Republic of Egypt: Managing Water Resources to Maximize Sustainable Growth. World Bank Publications.\\

Stiglitz, J. E. (2016). The Great Divide: Unequal Societies and What We Can Do About Them. W. W. Norton & Company.\\

Gelb, A. H., & Grasmann, S. (2019). Oil Prices, the Middle East, and the World Economy. Cambridge University Press.\\






\end{document}
