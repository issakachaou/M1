\documentclass{beamer}
\usetheme{Boadilla}
\usecolortheme{seahorse}
\usepackage[french]{babel}

\title{Chapitre I. Élément d'introduction}
\subtitle{Introduction à la microéconomie}

\author{Issa KACHAOU}


\begin{document}
	\begin{frame}[plain]
		\maketitle
	\end{frame}


\begin{frame}{Table des matières}
\tableofcontents	
\end{frame}

\section{Qu'est ce que l'économie ?}
	\begin{frame}{Qu'est ce que l'économie ?}
		\begin{alertblock}{Définition de l'économie} 
		L'économie est la science qui s'intéresse à l'allocation optimale de ressources limité pour répondre au besoin illimité des hommes.
		\end{alertblock}

	\end{frame}
	
\section{Qu'est ce que la microéconomie ?}
	\begin{frame}{Qu'est ce que la microéconomie ?}

\begin{alertblock}{Définition de la microéconomie} 
	L'économie est la science qui s'intéresse à l'allocation optimale de ressources limité pour répondre au besoin illimité des hommes.
\end{alertblock}

	\end{frame}
	
\section{Microéconomie et Macroéconomie : quel approche privilégier ?}
	\begin{frame}{Microéconomie et Macroéconomie : quel approche privilégier ?}
				\begin{block}{yes}{Remarque}
			contenu...
		\end{block}
	\end{frame}
	
	
\section{Les pré-requis de ce cours}
	\begin{frame}{Les pré-requis de ce cours}
		\begin{block}{yes}{Remarque}
			contenu...
		\end{block}
	\end{frame}
	
\section{Bibliographie}
\begin{frame}{Bibliographie}
	
	\begin{itemize}
		\item Hachon, Christophe, Reynald-Alexandre Laurent, et Arnaud Mayeur. 2013. Microéconomie - Cours et application. Paris: Nathan.
		\item Jullien Bruno économiste et Picard Pierre. 2011. Éléments de microéconomie . 2. Exercices et corrigés. 4e édition. Éco. Paris: Montchrestien Lextenso.
		\item Lecointre, Jérôme. 2018. Microéconomie. 1er édition. Louvain-la-Neuve (Belgique): DE BOECK SUP.
		\item Médan Pierre. 2020. Microéconomie: QCM et exercices corrigés, 16 sujets d’examen corrigés, avec rappels de cours. 6e édition. TD. Malakoff: Dunod.
	\end{itemize}


\end{frame}	
	
	
	\begin{frame}
		\begin{block}{yes}{Remarque}
			contenu...
		\end{block}
		
		\begin{alertblock}{Définition}{Important theorem} 
			
		\end{alertblock}
		
		\begin{exampleblock}{Exemple numérique}
			\[ 
			\sum_{i=1}^{t}(x-\bar{x})^2
			\]
		\end{exampleblock}
	\end{frame}
	
	
	
\end{document}
