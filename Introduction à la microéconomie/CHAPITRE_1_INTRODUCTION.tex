\documentclass{beamer}
\usetheme{Boadilla}
\usecolortheme{seahorse}
\usepackage[french]{babel}

\title{Chapitre I. Élément d'introduction}
\subtitle{Introduction à la microéconomie}

\author{Issa KACHAOU}


\begin{document}
	\begin{frame}[plain]
		\maketitle
	\end{frame}


\begin{frame}{Table des matières}
\tableofcontents	
\end{frame}

\section{Qu'est ce que l'économie ?}
	\begin{frame}{Qu'est ce que l'économie ?}
		\begin{alertblock}{Définition de l'économie} 
		L'économie est la science qui s'intéresse à l'allocation optimale de ressources limité pour répondre au besoin illimité des hommes.
		\end{alertblock}

	\end{frame}
	
\section{Qu'est ce que la microéconomie ?}
	\begin{frame}{Qu'est ce que la microéconomie ?}

\begin{alertblock}{Définition de la microéconomie} 

La microéconomie est une branche de l'économie qui étudie les comportements des agents économiques individuels, tels que les consommateurs, les entreprises, les travailleurs, et les marchés spécifiques. Elle analyse la manière dont ces acteurs prennent leurs décisions concernant la production, la consommation, les échanges, et la fixation des prix. La microéconomie se concentre sur des aspects comme l'offre et la demande, la concurrence, l'optimisation des ressources, et l'équilibre des marchés à petite échelle.
\end{alertblock}

	\end{frame}
	
\section{Microéconomie et Macroéconomie : quel approche privilégier ?}
	\begin{frame}{Microéconomie et Macroéconomie : quel approche privilégier ?}
				\begin{block}{Un faux débat}
		La microéconomie analyse les décisions individuelles et les mécanismes des marchés, tandis que la macroéconomie étudie l'économie dans son ensemble, en se concentrant sur des phénomènes globaux comme la croissance, l'inflation et le chômage. L'approche à privilégier dépend du contexte : la microéconomie est utile pour des décisions spécifiques ou l'analyse de marchés, tandis que la macroéconomie permet de comprendre les tendances économiques globales et les politiques économiques.
		\end{block}
	\end{frame}
	
	
\section{Les pré-requis de ce cours}
	\begin{frame}{Les pré-requis de ce cours}
		\begin{block}{C'est un cours d'introduction abordable}
		es prérequis pour un cours d'introduction à la microéconomie incluent généralement une connaissance de base des concepts économiques fondamentaux. Cela comprend une compréhension des notions de demande et offre, des marchés, des prix et de l'allocation des ressources. Une certaine aisance avec les mathématiques de niveau lycée, en particulier l'algèbre et les graphiques, est souvent recommandée pour comprendre les courbes et les équilibres. En outre, une familiarité avec les principes de base des décisions rationnelles et de l'optimisation est également utile.
		\end{block}
	\end{frame}
	
\section{Bibliographie}
\begin{frame}{Bibliographie}
	
	\begin{itemize}
		
		\item Hachon, Christophe, Reynald-Alexandre Laurent, et Arnaud Mayeur. 2013.\textit{Microéconomie - Cours et application}. Paris: Nathan.
		
		\item Jullien Bruno et Picard Pierre. 2011. \textit{Éléments de microéconomie . 2. Exercices et corrigés}. 4e édition. Éco. Paris: Montchrestien Lextenso.
		
		\item Lecointre, Jérôme. 2018.\textit{ Microéconomie}. 1er édition. Louvain-la-Neuve (Belgique): DE BOECK SUP.
		
		\item Médan Pierre. 2020.\textit{ Microéconomie: QCM et exercices corrigés, 16 sujets d’examen corrigés, avec rappels de cours}. 6e édition. TD. Malakoff: Dunod.
	
		\item Picard Pierre. 2011. \textit{Éléments de microéconomie . 1. Théorie et applications}. 8e édition. Éco. Paris: Montchrestien Lextenso.
		
		\item Varian, Hal R., et Bernard Thiry. 2015. \textit{Introduction à la microéconomie}. 8e édition. Louvain-la-neuve Paris: DE BOECK SUP.
		
		\item Wasmer Étienne. 2017. \textit{Principes de microéconomie: méthodes empiriques et théories modernes}. 3e édition. Paris: Pearson éducation.
		
	\end{itemize}


\end{frame}	
	
	
	\begin{frame}
		\begin{block}{yes}{Remarque}
			contenu...
		\end{block}
		
		\begin{alertblock}{Définition}{Important theorem} 
			
		\end{alertblock}
		
		\begin{exampleblock}{Exemple numérique}
			\[ 
			\sum_{i=1}^{t}(x-\bar{x})^2
			\]
		\end{exampleblock}
	\end{frame}
	
	
	
\end{document}
