\documentclass[a4paper, 12pt]{report}
\usepackage{graphicx}
\usepackage[utf8]{inputenc} 
\usepackage[french]{babel}
\usepackage[T1]{fontenc}
\usepackage{fancyhdr}
\usepackage{amsmath,amsfonts,amssymb, empheq}
\usepackage{eurosym}
\usepackage{booktabs}
\usepackage{wrapfig}
\pagestyle{fancy}
\fancyhead[R]{Université Paris-Est Créteil}
\fancyhead[L]{Management des risques bancaires}
\usepackage{array,multirow,makecell}
\usepackage{hyperref}
\setcellgapes{1pt}
\makegapedcells
\newcolumntype{R}[1]{>{\raggedleft\arraybackslash }b{#1}}
\newcolumntype{L}[1]{>{\raggedright\arraybackslash }b{#1}}
\newcolumntype{C}[1]{>{\centering\arraybackslash }b{#1}} 
%\renewcommand{\thechapter}{\Roman{chapter}}
%\setcounter{chapter}{1} % pour numéroter le chapitre 
\begin{document}
	\chapter*{Introduction}
	
Au cours années 1980, mouvement de déréglementation, de désintermédiation et de décloisonnement (3D). Le rôle des marchés de capitaux est renforcé dans le financement de l’activité économique. Cependant banques ont toujours une place prépondérante dans le financement, notamment pour les ménages. 
Actuellement, financement par le crédit bancaire des ménages, ETI, PME : +
de 60 \% en France (en Europe ordre de 80 \%)

Au cours années 1990, ce mouvement de libéralisation s’est étendu aux pays émergents. Il y a eut une interconnexion croissante des systèmes financiers nationaux. Le système bancaire traditionnel n’est plus suffisant pour répondre
aux besoins de placements des agents comme firmes transnationales ce qui va engendré le développement des fonds spéculatifs, des organismes de
titrisation, des fonds d’investissement (OPC(\textit{Organisme de placement collectif})). Ces entités ne créent pas de monnaie mais remplissent pour partie des fonctions semblables au système bancaire traditionnel (transformation échéances, liquidité, rendement,…). On appelle ça le système bancaire parallèle (Shadow Banking) ou Intermédiation financière non bancaire (IFNB). La transformation d'échéance représente l'opération par laquelle une banque transforme des dépôts de court terme en prêts de long terme. En d'autres termes, la banque emprunte des fonds à court terme (par exemple, des dépôts des clients) et utilise ces fonds pour financer des investissements ou des prêts à plus longue échéance, comme des crédits immobiliers ou des prêts aux entreprises.

Le système bancaire représente le marché des banques. Le système financier représente le marché des capitaux, les intermédiaires financiers IF (bancaire et non bancaire). Les intermédiaires financiers ce sont les assurances, les établissement de crédit (EC) et les OPC.

Activités de titrisation aux USA élément déclencheur de la crise financière globale (CFG) de 2007-2009. Depuis il y a un mouvement de re-réglementation de la sphère financière et réflexion sur l’impact du développement des systèmes inanciers sur la croissance économique

A partir des années 2010, expansion des innovations technologiques dans le secteur financier : FinTech. Cela révolutionne l’offre de services financiers, ça tend à réduire la distinction entre intermédiation et marché. Ainsi banques traditionnelles concurrencées par : banques en ligne (Boursorama banque (SG), Hello Bank (BNP-P), Fortunéo (Crédit Mutuel Arkéa),…) et néo-banques (Revolut (anglaise), N26 (allemande), Orange Bank (française),…). Il y a également l'émergence de nouveaux intermédiaires issus de la FinTech c'est l'exemple du financement participatif (\textit{Crowdfunding}) les établissements de paiement, les établissements de monnaie électronique, … Cela a permis de revoir le modèle d’affaire des banques.

L'enjeux actuels est intelligence artificielle (IA) elle pourra aider dans lutte contre blanchiment des capitaux et financement du terrorisme (LCB FT), modèle interne de credit scoring, protection de la clientèle.

\begin{wrapfigure}{r}{0.5\textwidth}
 \centering
	\includegraphics[scale=0.5]{../../../Pictures/Screenshots/Capture d'écran 2024-09-12 234518}
\end{wrapfigure}

Pendant longtemps, la théorie a ignoré le lien entre sphères réelle et financière : modèle de Modigliani et Miller (1958) parle de la neutralité du mode de financement sur la valeur de l’actif économique de l’entreprise. Gurley et Shaw (1955) démontre l'importance du rôle des banques dans la transformation des échéances entre emprunteurs (LT) etprêteurs (CT). Il est nécessaire qu'il y ait un ajustement entre les différents besoins des agents.

Le développement de la théorie bancaire dans années 1980 avec l’introduction des asymétries d’information (AI) entre prêteurs et emprunteurs. Elle met en lumière le conflits d’intérêt entre prêteurs et emprunteurs dans la théorie de
l’agence et l'anti-sélection (avant signature du contrat de prêt) et aléa-moral (après signature du contrat)

\begin{center}
	\begin{tabular}{lr}
	A                                 & P                    \\ \hline
	\multicolumn{1}{l|}{Crédit}       & Dépôt                \\
	\multicolumn{1}{l|}{Titre détenu} & Titre émis           \\
	\multicolumn{1}{l|}{}             & Fond propre (action)
\end{tabular}
\end{center}
Depuis Bâle 3 les banques doivent respecter ce ratio 

$$
\text{Ratio de fonds propres solvabilité}= \frac{\text{Fonds propres}}{\sum \text{Actif podérés par les risques}}\ge 10,5\%
$$

\chapter{Le cadre institutionnel et réglementaire et les grandes tendances de l’évolution du système financier}

Depuis début de la Seconde Guerre mondiale et jusqu'à fin des années 1960 le système bancaire français contrôlé par l’État. 

On assiste à une harmonisation européenne de la réglementation financière à partir de 1973 (Directive sur les assurances) et à l'ouverture du marché unique à la fin de 1992 (traité de Maastricht). Ces deux évolutions ont entraîné des transformations profondes du système financier français, notamment par la rationalisation des structures et des implantations, la diversification des activités et le renouvellement des modes de gestion.

Ces transformations s’inscrivaient dans le contexte mondial de mouvement des 3D et d’ouverture internationale des marchés bancaires et financiers.

La crise financière globale (CFG) de 2007-2009 a porté un coup d’arrêt brutal à cette évolution et a montré l’urgence de réformer et d’encadrer le système
financier. 

\section{Contexte institutionnel et réglementaire des activités bancaires, d’assurance et de marchés}

\subsection{Réglementations}

La réglementation est apparue tard en France au début de la Seconde Guerre mondiale (Italie, USA, … 1920). Avant nous avions un système bancaire libéralisé (ni de charte bancaire, ni de dépôt de garantie ni de réglementation prudentielle). Cependant la banque de France a exercé une certaine supervision sans en avoir la charge.

\subsubsection{Réglementations antérieures aux années 1980}

Le système bancaire fut organisé pour la première fois en 1941 par le
gouvernement de Vichy. On assiste à un essor du système à la fin années 1960 mais le système cloisonné, les législations complexes et hétérogènes et les 
autorités de contrôle diverses jusqu’aux années 1980. 

\paragraph{Lois bancaires de 1941-1945}

La loi de 1941 a marqué la création de la Commission de contrôle des banques dans le but de sécuriser les clients après le krach de 1929, qui avait entraîné la disparition de 566 banques françaises entre 1929 et 1935. À cette époque, il n'existait pas de législation unique applicable à tous les établissements bancaires, lesquels étaient répartis en trois catégories : les banques inscrites, les établissements financiers et les établissements à statut légal spécial. 

La loi de 1945 a renforcé le rôle de l'État dans les circuits de financement afin de lutter contre l'inflation et de participer à la reconstruction du pays. Cette loi a également entraîné la nationalisation de la Banque de France et de quatre grandes banques : Société Générale, Crédit Lyonnais, Comptoir National d'Escompte de Paris et Banque Nationale pour le Commerce et l'Industrie. Inspirées du modèle américain du Glass-Steagall Act, les banques inscrites ont été spécialisées pour faciliter leur contrôle : il y avait les banques de dépôts, les banques d'affaires et les banques de crédit à moyen et long terme.

Les banques de dépôts ne pouvaient pas recevoir de dépôts à vue ou à terme de plus de deux ans et leurs activités de prise de participation dans les entreprises étaient limitées. Les banques d'affaires étaient spécialisées dans la prise de participation et la gestion d'affaires existantes ou en formation, avec une collecte de dépôts restreinte. Les banques de crédit à moyen et long terme ne pouvaient pas recevoir de dépôts ni accorder de crédits de moins de deux ans et avaient une capacité de prise de participation limitée. Ce cloisonnement des activités bancaires visait à encadrer le crédit, réguler la création de monnaie et orienter l'épargne vers les besoins de financement.

Le Conseil National du Crédit (CNC) a été créé pour superviser le secteur bancaire. Chaque établissement devait être inscrit sur la liste du CNC, à l'exception des établissements à statut légal spécial tels que le Crédit Agricole et le Crédit Mutuel. Les banques s'enregistraient en indiquant leur spécialisation, ce qui influençait la collecte des dépôts, ainsi que le calcul des ratios de liquidité et de fonds propres.

\paragraph{Décrets de 1966 et de 1967 (dits décrets « Debré »)}

Les décrets de 1966 et 1967, connus sous le nom de décrets « Debré », ont été mis en place pour moderniser le secteur bancaire en réponse à la mensualisation des salaires et à l'intensification de l'utilisation des chèques. Ces décrets visaient à favoriser la détention d'une épargne longue par les agents non financiers (ANF) et à encourager la transformation des échéances en permettant la concession de crédits à long terme avec des ressources à court terme.

Les décrets s'articulaient autour de deux objectifs principaux. D'une part, ils cherchaient à donner aux banques les moyens de maximiser la collecte des dépôts en facilitant l'ouverture de nouveaux guichets sans avoir besoin de l'autorisation du Conseil National du Crédit (CNC). Entre 1967 et 1975, le nombre de guichets a été multiplié par deux. D'autre part, ils avaient pour but de favoriser la « déspécialisation » en rapprochant les banques de dépôts et les banques d'affaires, en unifiant les conditions de collecte des dépôts et d'octroi de crédits, et en permettant aux banques de dépôts de prendre des participations dans des affaires.

Malgré ces évolutions, les règles de fonctionnement du système bancaire restaient loin d'être équitables et homogènes.

\subsubsection{Réglementations dans les années 1980-90}

Au début des années 1980, une grande réforme du système financier a été initiée, motivée par la forte inflation des années 1970 et les engagements pris par l'État dans le cadre des traités européens. Cette réforme a conduit à une ouverture accrue du système bancaire : l'encadrement du crédit a été supprimé en 1985, le contrôle des changes a été aboli en 1990, et le marché unique européen a été mis en place en 1992.

\paragraph{Loi de nationalisation de 1982}

avait pour objectif de financer des investissements prioritaires, d'assurer un meilleur contrôle du crédit et de réduire son coût. Cette loi a conduit à la nationalisation des banques commerciales générant plus de 1 milliard de francs de dépôts, des banques du réseau coopératif et mutualiste, ainsi que des banques de crédit à moyen et long terme. Les banques étrangères étaient exclues de cette nationalisation.

Ainsi, 36 banques de dépôts et banques détenues indirectement par l'État ont été nationalisées, incluant des institutions telles que le Crédit Commercial de France (CCF), le Crédit Industriel et Commercial (CIC), la Banque Rothschild, la Banque Worms, la Banque La Hénin, la Banque d'Indochine et de Suez, ainsi que la Banque de Paris et des Pays-Bas. Les anciens actionnaires ont été indemnisés en recevant, en échange de leurs actions, des obligations émises par la Caisse Nationale des Banques, un organisme créé spécifiquement par la loi de 1982.

\paragraph{Loi bancaire de 1984 relative à l’activité et au contrôle des établissements de crédit (EC)}

La loi bancaire de 1984, relative à l’activité et au contrôle des établissements de crédit (EC), a transposé en droit français la première directive bancaire européenne de 1977. Son objectif était d’harmoniser progressivement les conditions de fonctionnement de tous les établissements financiers et de renforcer leurs structures financières. Dès lors, tous les établissements du secteur bancaire ont été désignés sous le terme d’établissements de crédit (EC).

Cette loi a soumis tous les établissements de crédit aux mêmes autorités de réglementation et de contrôle, à l'exception du Trésor public, de la Banque de France, des instituts d’émission des DOM et TOM, des services financiers de la Poste, et de la Caisse des Dépôts et Consignations. Elle visait à garantir la stabilité du système bancaire en imposant à l’ensemble des EC des exigences telles que l’existence de fonds propres suffisants, le respect des règles de liquidité, de solvabilité et d'équilibre des structures financières.

Pour exercer leur activité, les établissements de crédit devaient obtenir un agrément délivré par le Comité des Etablissements de Crédit. La loi a également défini six catégories d’EC : les banques, les banques mutualistes ou coopératives, les caisses d’épargne et de prévoyance, les caisses du crédit municipal, les sociétés financières et les institutions financières spécialisées. À cette époque, toutes les banques étaient intégrées dans le modèle de la banque universelle.

\paragraph{Lois de privatisation de 1986 et de 1993}

Les lois de privatisation de 1986 et de 1993 ont marqué des phases significatives de libéralisation du secteur bancaire en France. La loi de 1986 visait à privatiser les banques afin de les libérer des circuits de financement contrôlés par l’État. La première vague de privatisations, qui s'est déroulée entre 1986 et 1988, a vu le retour au secteur privé de 73 banques, dont la Compagnie financière Paribas, la Banque régionale Sogenal, et la Société Générale. Toutefois, les privatisations ont été suspendues en 1988 à la suite du krach boursier de 1987.

La loi de 1993 a repris le processus de privatisation, qui s'est poursuivi jusqu'à la fin des années 1990, avec le retour au secteur privé de 14 banques, y compris BNP et ses filiales. La récession de 1993 a entraîné la faillite de nombreuses entreprises du secteur privé, incapables de rembourser leurs crédits aux banques. En réponse, l'État a soutenu des banques publiques ou parapubliques telles que le Crédit Lyonnais et le Crédit Foncier de France pour éviter des faillites bancaires et stabiliser le système financier.

\paragraph{Réglementation des services financiers en Europe}

Dans les années 1970, la Commission Européenne (CE) et le Conseil des ministres des États membres ont défini une stratégie bancaire et financière visant à promouvoir l'intégration et la stabilité financières en Europe. La réglementation s'est traduite par la transposition, par chaque État membre, du programme législatif relatif au marché unique des services financiers dans son droit interne. Les décisions sous-jacentes, sous forme de directives préparées par la CE, étaient discutées et éventuellement amendées par le Parlement européen, puis définitivement adoptées par le Conseil des ministres, représentant les gouvernements des États membres.

La réglementation bancaire dans l'Union européenne repose sur cinq principes fondamentaux. Tout d'abord, la liberté totale des mouvements de capitaux permet la suppression du contrôle des changes entre les États membres, facilitant ainsi les flux financiers au sein de l'UE. Ensuite, le principe de liberté d’établissement permet à tout établissement bancaire agréé dans un État membre d’ouvrir une succursale dans un autre État membre sans avoir à demander une autorisation aux autorités locales. De plus, la liberté de prestation de services autorise un établissement bancaire agréé à offrir ses services à un client situé dans un autre pays de l’UE, même sans y avoir de présence physique. Le quatrième principe est la reconnaissance mutuelle des agréments, aussi appelé le principe du « passeport unique », qui permet à une banque agréée dans un pays d’exercer librement dans les autres pays membres. Enfin, la surveillance des établissements bancaires de petite et moyenne taille est assurée par les autorités du pays d’origine, tandis que pour les établissements systémiques, c'est la Banque centrale européenne (BCE) qui prend en charge la supervision.

\subsubsection{Réglementations de la fin des années 1990 jusqu’à mi-2000}

À partir de la fin des années 1990 jusqu’au milieu des années 2000, la réglementation bancaire a connu plusieurs évolutions majeures. Tout d'abord, il y a eu un désengagement progressif de l’État, accompagné d’une réduction du nombre d’établissements de crédit (EC), marquant ainsi une concentration du secteur. Cette période a également vu l’émergence de grands groupes diversifiés, ayant une envergure à la fois européenne et internationale, ce qui a renforcé la compétitivité des acteurs du marché. Par ailleurs, une plus grande homogénéité des conditions d’exercice des activités bancaires et financières s’est installée, favorisant une standardisation des pratiques et une meilleure intégration du marché bancaire au sein de l’Union européenne.

\paragraph{Loi de modernisation des activités financières (MAF) de 1996}

La Loi de modernisation des activités financières (MAF) de 1996 a marqué une étape importante dans l'évolution de la réglementation financière en France. Elle a permis la transposition en droit français de la directive européenne de 1993 relative aux services d'investissement, couvrant des activités telles que les ordres de bourses, le trading pour compte propre, ainsi que le placement et la gestion sur instruments financiers pour compte de tiers. Cette loi a également favorisé l’harmonisation des conditions d’exercice des services d'investissement, renforçant la cohérence et la compétitivité du secteur. En outre, elle a établi un cadre institutionnel unique pour l’exercice de ces services en France, que ce soit par les établissements de crédit (EC) ou par les entreprises d’investissement (EI), qui doivent être agréés par l’Autorité de Contrôle Prudentiel (ACP) ou, pour les sociétés de gestion de portefeuille, par l’Autorité des Marchés Financiers (AMF). Les EC et EI sont ainsi désignés comme Prestataires de Services d’Investissement (PSI), consolidant ainsi le cadre réglementaire des activités d’investissement en France.

\paragraph{Ordonnance relative aux marchés d’instruments financiers (MIF) de 2007}

L'Ordonnance relative aux marchés d’instruments financiers (MIF) de 2007 a transposé en droit français la directive européenne de 2004, introduisant plusieurs réformes importantes dans le secteur financier. L'une des principales mesures a été la mise en concurrence des bourses traditionnelles avec de nouvelles plates-formes électroniques de négociation, ouvrant ainsi le marché à davantage d'acteurs et renforçant la compétitivité. Parmi ces nouveaux acteurs figurent les systèmes multilatéraux de négociation (SMN) et les internalisateurs systématiques (IS). Les SMN permettent une confrontation multilatérale des ordres de bourse entre banques, courtiers ou bourses, comme c'est le cas sur des plates-formes telles que Turquoise ou Chi-X. Les IS, quant à eux, se caractérisent par des négociations bilatérales de gré à gré, souvent effectuées par des banques et des courtiers, à l’image de BNP Paribas Arbitrage. Ces évolutions ont considérablement modifié la structure et le fonctionnement des marchés financiers en Europe.

\paragraph{Ordonnance relative à la solvabilité des EC et EI de 2007}

L'Ordonnance relative à la solvabilité des établissements de crédit (EC) et des entreprises d’investissement (EI) de 2007 a transposé en droit français deux directives européennes de 2006 concernant la solvabilité de ces institutions financières. Cette ordonnance a permis l'introduction du nouvel "Accord de Bâle 2" dans le cadre de la réglementation bancaire française. Cet accord vise à établir une convergence internationale en matière de mesure et de normes de fonds propres (FP), renforçant ainsi la stabilité financière des établissements en imposant des exigences plus strictes concernant la gestion des risques et la solidité de leurs fonds propres. Cela a marqué un tournant important dans la régulation bancaire, en alignant les pratiques françaises sur les standards internationaux.

\subsubsection{Règlementations dans les années de l’après crise financière globale (CFG) 2007-2009}

Dans les années qui ont suivi la crise financière globale (CFG) de 2007-2009, un important mouvement de re-réglementation du secteur financier a été initié, notamment lors du sommet du G20 à Londres en avril 2009. Ce mouvement visait à renforcer la régulation des marchés financiers afin d'éviter une nouvelle crise de cette ampleur. En plus de la régulation des acteurs traditionnels, une attention particulière a été portée sur le développement de la finance digitale et participative, qui a pris de l'ampleur durant cette période. Par ailleurs, la lutte contre le blanchiment des capitaux et le financement du terrorisme est devenue une priorité, avec des réglementations renforcées dans ces domaines. Enfin, cette période a vu l’apparition de nouveaux acteurs dans le secteur financier, modifiant encore davantage le paysage de la finance mondiale.

\paragraph{Ordonnance relative au service de paiement de 2009}

L'Ordonnance relative au service de paiement de 2009 a transposé en droit français la directive européenne de 2007 relative aux services de paiement, appelée DSP1. Cette ordonnance a introduit un nouveau type d'acteur dans le secteur financier : les établissements de paiement (EP). Ces établissements sont autorisés à fournir divers services de paiement, tels que le virement, le prélèvement, le transfert de fonds, ainsi que les opérations de versement ou de retrait d’espèces. Cette réforme a permis de moderniser et de diversifier l’offre de services de paiement en France, facilitant les transactions et la gestion des fonds pour les utilisateurs.

\paragraph{Loi de régulation bancaire et financière de 2010}

La Loi de régulation bancaire et financière de 2010 a été adoptée pour renforcer la régulation et l'encadrement du système financier français à la suite de la crise financière. Elle a conféré à l'Autorité des Marchés Financiers (AMF) des pouvoirs élargis, notamment l'interdiction des ventes à découvert, la régulation des marchés dérivés et des dérivés de crédit, ainsi que le contrôle des agences de notation (en leur imposant des agréments et des sanctions). La loi a également créé le Conseil de la Régulation Financière et du Risque Systémique (COREFRIS), un organe destiné à surveiller et prévenir les risques systémiques. En outre, cette législation a introduit des mesures pour encadrer la rémunération des opérateurs de marché, en particulier concernant les bonus et les frais bancaires, avec pour objectif de rendre les tarifs plus transparents et de limiter les excès.

\paragraph{Ordonnance relative aux établissements de crédit et aux sociétés de financement de 2013}

L'Ordonnance relative aux établissements de crédit et aux sociétés de financement de 2013 visait à harmoniser le statut des établissements de crédit (EC) au niveau communautaire, en vue de l’entrée en vigueur, au 1er janvier 2014, du règlement européen CRR (Capital Requirement Regulation). Ce règlement a transposé en Europe le nouveau cadre prudentiel connu sous le nom de Bâle 3, renforçant ainsi les exigences en matière de fonds propres pour les EC. Les EC sont désormais agréés en tant que banques, banques mutualistes ou coopératives, établissements de crédit spécialisés, ou caisses de crédit municipal. Les EC ne répondant plus à la définition européenne peuvent continuer leur activité sous un nouveau statut : celui de Société de Financement (SF). Ces sociétés sont autorisées à octroyer des crédits, mais ne peuvent pas collecter de dépôts ou d’autres fonds remboursables du public. Elles sont actives dans des secteurs comme le crédit à la consommation, le crédit-bail mobilier et immobilier, le crédit aux entreprises, l'affacturage, ainsi que les cautions et garanties. Cette ordonnance a également marqué la disparition du statut de société financière et d'institution financière spécialisée, ces entités étant désormais agréées soit comme établissements de crédit spécialisés, soit comme sociétés de financement.

\paragraph{Loi de séparation et de régulation des activités bancaires de 2013}

La Loi de séparation et de régulation des activités bancaires de 2013 a été mise en place pour renforcer la régulation des acteurs bancaires et accroître les pouvoirs des autorités de supervision. L'une des mesures clés de cette loi est la séparation des opérations spéculatives des banques de leurs activités essentielles à l’économie. Les activités utiles sont maintenues au sein de la maison mère, tandis que les opérations pour compte propre (trading) sont transférées dans une filiale distincte. De plus, un plafonnement strict des rémunérations variables des dirigeants et des traders a été instauré pour limiter la prise de risques excessifs. La loi introduit également le principe d’imputation prioritaire des pertes sur les actionnaires et les créanciers de la banque, afin de limiter l’intervention de l’État en cas de difficultés financières. Un autre volet important est la création d’un fonds de résolution unique, financé entièrement par le secteur financier, avec un objectif de 55 milliards d’euros d’ici 2025, alimenté par les établissements de crédit (EC) et entreprises d’investissement (EI) des pays de la zone euro. Enfin, le COREFRIS a été remplacé par le Haut Conseil de Stabilité Financière (HCSF), renforçant ainsi le cadre de surveillance et de stabilité du système financier.

\paragraph{Loi transposant la 2ème directive sur la monnaie électronique (DME2) en 2013}

La Loi transposant la 2ème directive sur la monnaie électronique (DME2) en 2013 a intégré en droit français les dispositions de la directive européenne sur la monnaie électronique de 2009. Cette loi permet aux établissements de crédit (EC) et aux établissements de monnaie électronique (EME) d’émettre de la monnaie électronique, élargissant ainsi les possibilités offertes aux acteurs du marché. En outre, elle autorise les EC, les EME, ainsi que les établissements de paiement (EP) à fournir des services de paiement, offrant ainsi une plus grande flexibilité et une gamme étendue de services financiers aux utilisateurs.

\paragraph{Ordonnance relative au financement participatif de 2014}

L'Ordonnance relative au financement participatif de 2014 a établi un cadre juridique sécurisé pour le financement participatif, un mode de financement qui permet aux entreprises de solliciter un grand nombre de personnes pour soutenir un projet sans avoir recours aux banques. Cette ordonnance introduit également un nouveau statut de conseiller en investissements participatifs, destiné à encadrer et réguler les activités des intermédiaires qui facilitent les investissements dans des projets financés par le public. Cette mesure vise à protéger les investisseurs tout en favorisant le développement de nouvelles formes de financement pour les entreprises.

\paragraph{Ordonnance relative aux marchés d’instruments financiers (MIF2) de 2016}

L'Ordonnance relative aux marchés d’instruments financiers (MIF2) de 2016 a apporté une refonte significative des exigences en matière de transparence pré- et post-négociation. Elle impose aux plateformes de négociation de donner accès à leurs données et crée une nouvelle catégorie d'acteurs : les prestataires de services de communication de données. Toutes les activités de négociation doivent désormais être organisées et menées sur des plateformes de négociation réglementées, comprenant les marchés réglementés (MR), les systèmes multilatéraux de négociation (MTF), et les systèmes organisés de négociation (OTF) pour les produits hors actions.

Cette ordonnance a également renforcé les pouvoirs de supervision de l'Autorité des Marchés Financiers (AMF), notamment pour limiter les positions prises sur les instruments dérivés de matières premières. Enfin, elle a introduit des contrôles pour la négociation algorithmique, en particulier pour les transactions à haute fréquence, avec des exigences telles que le test des algorithmes et la conservation des traces des ordres.

\paragraph{Ordonnance relative au renforcement du dispositif français de lutte contre le blanchiment des capitaux et le financement du terrorisme (LCB/FT) de 2016}

L'Ordonnance relative au renforcement du dispositif français de lutte contre le blanchiment des capitaux et le financement du terrorisme (LCB/FT) de 2016 a transposé la 4ème directive (UE) de 2015 dans le droit français. Cette directive suit les recommandations du Groupe d’Action Financière (GAFI), une organisation intergouvernementale créée en 1989 pour établir un cadre global pour la lutte contre le blanchiment de capitaux (BC) et le financement du terrorisme (FT).

Le blanchiment de capitaux consiste à donner une apparence légitime à des biens ou des capitaux d'origine illicite, provenant notamment de trafic de stupéfiants, d'activités criminelles, de corruption, de prostitution, de trafic d'armes, ou de certains types de fraude fiscale. Le financement du terrorisme, quant à lui, désigne la fourniture ou la collecte de fonds destinés à être utilisés pour des activités terroristes.

L'ordonnance renforce l'approche par risques des personnes assujetties, améliore l'identification du bénéficiaire effectif (BE), et élargit la notion de Personnes Politiquement Exposées (PPE). Elle favorise également une coopération accrue entre les cellules de renseignements financiers, afin de renforcer l'efficacité des mesures de prévention et de lutte contre le blanchiment des capitaux et le financement du terrorisme.

\paragraph{Ordonnance relative aux services de paiement (DSP2) de 2018}

L'Ordonnance relative aux services de paiement (DSP2) de 2018 a introduit de nouvelles règles importantes concernant l’accès aux activités de services de paiement. Elle a établi des normes pour la supervision des prestataires de services de paiement, les modalités techniques applicables aux opérations de paiement, ainsi que les droits et obligations des parties impliquées dans un service de paiement. En outre, cette ordonnance a introduit deux nouveaux types de prestataires de services : les services d’information sur les comptes et les services d’initiation de paiement. Ces innovations visent à renforcer la transparence, la sécurité et la concurrence dans le secteur des paiements, tout en offrant de nouvelles possibilités pour les utilisateurs et les prestataires de services financiers.

\paragraph{Loi du plan d'action pour la croissance et la transformation des entreprises (PACTE) de 2019}

La Loi du plan d'action pour la croissance et la transformation des entreprises (PACTE) de 2019 vise à stimuler la croissance et la transformation des entreprises en France. Parmi ses principales mesures, la loi cherche à rendre l'épargne retraite complémentaire plus attractive et à favoriser son utilisation pour le financement en fonds propres des entreprises. Elle propose également une révision de la gouvernance de la Caisse des Dépôts et Consignations (CDC) afin d'améliorer son efficacité et son rôle dans le financement de l'économie. De plus, la loi encadre les prestataires de services sur actifs numériques (PSAN) et les levées de fonds par émission de jetons (Initial Coin Offering, ICO), afin de sécuriser et réguler ces nouvelles formes d'investissement et de financement dans le secteur des actifs numériques.

\paragraph{Ordonnance relative au renforcement du dispositif français de lutte contre le blanchiment des capitaux et le financement du terrorisme (LCB/FT) de 2020}

L'Ordonnance relative au renforcement du dispositif français de lutte contre le blanchiment des capitaux et le financement du terrorisme (LCB/FT) de 2020 a transposé la 5ème directive (UE) de 2018 dans le droit français. Cette ordonnance a introduit plusieurs mesures importantes, notamment une plus grande accessibilité au registre des bénéficiaires effectifs, permettant une meilleure transparence sur l'identité des véritables propriétaires de sociétés. Elle impose également une vigilance renforcée pour les relations d’affaires et les opérations avec les pays à haut risque, afin de mieux contrer les risques liés au blanchiment de capitaux et au financement du terrorisme. En outre, certains prestataires de services liés aux actifs numériques ont été assujettis aux règles LCB/FT, intégrant ainsi ces acteurs dans le cadre réglementaire de la lutte contre le blanchiment et le financement du terrorisme.

\paragraph{Règlement Digital Operational Resilience Act (DORA) de 2023 (application en 2025)}

Le Règlement Digital Operational Resilience Act (DORA) de 2023, dont l'application est prévue pour 2025, vise à renforcer et harmoniser la gestion des risques liés aux technologies de l'information et à la sécurité des réseaux et des systèmes d'information au niveau de l'Union européenne. Il établit des exigences pour assurer une résilience opérationnelle face aux risques informatiques, de cybersécurité, de continuité d'activité, ainsi que pour les risques liés aux tiers impliqués dans les services numériques. DORA a pour objectif de garantir que les institutions financières soient mieux préparées à gérer les incidents informatiques et les menaces de cybersécurité, tout en assurant la continuité de leurs opérations en cas de perturbations majeures.

\paragraph{Règlement MiCA Marché des crypto-actifs de 2023 (application en 2024)}

Le Règlement MiCA, relatif au marché des crypto-actifs, a été adopté en 2023 et entrera en application en 2024. Ce règlement établit un cadre européen harmonisé pour encadrer les émetteurs de crypto-actifs de première et de deuxième générations ainsi que les prestataires de services sur actifs numériques (PSAN). Son objectif principal est de régir l’offre au public et l’admission aux négociations de jetons, tout en encadrant la fourniture de services sur crypto-actifs par des prestataires. De plus, il vise à prévenir les abus de marché sur les crypto-actifs, garantissant ainsi un environnement plus sûr et transparent pour les investisseurs et les acteurs du marché.

\paragraph{Directive sur les services de paiement (DSP3) et règlement sur les services de paiement (RSP1) de 2023}

La Directive sur les services de paiement (DSP3) et le règlement sur les services de paiement (RSP1) adoptés en 2023 visent à uniformiser les règles du jeu entre les banques et les prestataires de services de paiement (PSP). Ces mesures garantissent un accès pour les PSP à tous les services de paiement et leur permettent d'ouvrir des comptes auprès des banques. En outre, elles renforcent la protection contre la fraude aux paiements et assurent la protection des consommateurs. Concernant l'accès aux données financières, la directive offre aux clients la possibilité, mais non l'obligation, de partager leurs données, tout en renforçant la protection des données personnelles conformément au règlement général sur la protection des données (RGPD).

\paragraph{Paquet législatif sur la LCB/FT de 2021}

Le paquet législatif sur la lutte contre le blanchiment de capitaux et le financement du terrorisme (LCB/FT) de 2021 comprend plusieurs mesures clés. Il introduit un projet de règlement LCB/FT qui établit une nouvelle autorité de l'Union européenne en matière de LCB-FT, nommée l'Anti-Money Laundering Authority (AMLA). Ce paquet inclut également un règlement LCB-FT contenant des règles directement applicables relatives à la vigilance à l'égard de la clientèle et aux bénéficiaires effectifs. En outre, la 6ème directive sur la LCB-FT (AMLD6) propose des dispositions concernant les autorités nationales de surveillance et les cellules de renseignement financier dans les États membres. Enfin, une révision du règlement sur les transferts de fonds a été effectuée pour garantir la traçabilité des transferts de crypto-actifs.

\subsection{Autorités}

\subsubsection{Autorités françaises}

Les autorités françaises répartissent leurs actions dans le domaine des activités des banques, des assurances et des marchés. Le régulateur définit les règles à travers des lois, des ordonnances, des décrets, des arrêtés, ainsi que des directives et règlements communautaires. Parallèlement, le superviseur est chargé de contrôler la bonne application de ces règles par les entités concernées, tant au moment de leur création, lors de l'attribution des agréments, qu'à chaque étape de leur existence, grâce à des contrôles sur pièces et sur place.

\begin{wrapfigure}{r}{0.65\textwidth}
	\centering
	\includegraphics[scale=0.4]{../../../Downloads/Screenshot 2024-09-14 at 19-01-08 Economie de l’industrie des services financiers - C2 risques bancaires M1 24.pdf}
\end{wrapfigure}

\subparagraph{Fonction de régulation}

\paragraph{Comité Consultatif de la Législation et de la Réglementation Financières (CCLRF)}

Le Comité Consultatif de la Législation et de la Réglementation Financières (CCLRF) a pour mission d'émettre des avis sur l'ensemble des règlements, ainsi que sur tous les projets de lois et de textes communautaires relatifs aux assurances, aux établissements de crédit (EC), aux établissements de paiement (EP), aux établissements de monnaie électronique (EME) et aux entreprises d'investissement (EI).

\paragraph{Comité Consultatif du Secteur Financier (CCSF)}

Le Comité Consultatif du Secteur Financier (CCSF)  étudie les relations entre les entreprises (EC, EI, EP, EME) et leurs clientèles et propose toutes mesures appropriées dans ce domaine sous forme d’avis ou de recommandations
d’ordre général

\paragraph{Haut conseil de stabilité financière (HCSF)}

Le Haut Conseil de Stabilité Financière (HCSF) est une autorité macro-prudentielle chargée de surveiller le système financier dans son ensemble. Son objectif principal est de préserver la stabilité de ce système tout en veillant à ce qu'il contribue de manière soutenable à la croissance économique.

\paragraph{Traitement du renseignement et action contre les circuits financiers clandestins (Tracfin)}

Tracfin est un organisme chargé de lutter contre les circuits financiers clandestins, le blanchiment de capitaux (BC) et le financement du terrorisme (FT). Il a pour mission de recueillir, d'analyser et d'enrichir les déclarations de soupçons que les professionnels assujettis sont tenus par la loi de lui transmettre.

\paragraph{Autorité des normes comptables (ANC)}

L'Autorité des Normes Comptables (ANC) établit les prescriptions comptables générales et sectorielles que doivent respecter les personnes physiques ou morales soumises à l'obligation légale d'établir des documents comptables conformes aux normes de la comptabilité privée. De plus, elle donne son avis sur toute disposition législative ou réglementaire contenant des mesures de nature comptable, notamment en ce qui concerne les normes comptables internationales.

\subsubsection{Fonction de supervision}

\paragraph{Banque de France}

La Banque de France a un mandat explicite en matière de stabilité financière depuis 2013, qu'elle exerce conjointement avec le Haut Conseil de Stabilité Financière (HCSF). Son rôle consiste à identifier et à suivre les risques pesant sur la stabilité du système financier.

\paragraph{Autorité de Contrôle Prudentiel et de Résolution (ACPR)}

L'Autorité de Contrôle Prudentiel et de Résolution (ACPR) est responsable d'autoriser les acteurs du secteur financier, tels que les établissements de crédit (EC), les établissements de paiement (EP), les établissements de monnaie électronique (EME) et les compagnies d'assurance, à exercer leur activité par le biais d'agréments ou d'autorisations. Elle contrôle également ce secteur en veillant au respect de la réglementation et en sanctionnant les manquements. Par ailleurs, l'ACPR protège la clientèle, qu'elle soit composée de particuliers ou de professionnels, et dispose de compétences en matière de résolution pour limiter l'impact des défaillances bancaires sur la stabilité financière, afin de protéger les déposants et d'éviter le recours aux aides d'État.

\paragraph{Autorité des marchés financiers (AMF)}

L'Autorité des Marchés Financiers (AMF) joue un rôle essentiel en régulant, surveillant, informant et protégeant les investisseurs. Elle veille à la protection de l'épargne investie dans des instruments financiers, à l'information des investisseurs et au bon fonctionnement des marchés. L'AMF s'assure également de la qualité de l'information fournie par les sociétés de gestion concernant leur stratégie d'investissement et leur gestion des risques. En régulant les acteurs et les produits de la place financière, elle contribue à la régulation des marchés aux échelons européens et internationaux, tout en coopérant avec les autorités compétentes des autres États membres. L'AMF édicte des règles, autorise les acteurs, vise les documents sur les opérations financières, et agréée les Organismes de Placement Collectif (OPC). Elle enregistre également les prestataires de services sur actifs numériques et accorde son agrément à ceux qui en font la demande. En surveillant les marchés et les transactions, elle mène des enquêtes et des contrôles, tout en disposant d'un pouvoir de sanction pécuniaire et/ou disciplinaire.

\subsubsection{Autorités européennes}

Les autorités européennes ont constaté que les systèmes de surveillance nationaux sont dépassés par la réalité interconnectée des systèmes financiers nationaux, notamment à la suite de la crise financière globale (CFG). Pour y remédier, il est proposé de transformer les trois comités existants en véritables Autorités Européennes de Surveillance (AES). Ces AES exerceront une surveillance dans les services bancaires, sur les marchés des capitaux et dans le secteur des assurances. Elles se répartiront les responsabilités de surveillance au niveau microprudentiel et macroprudentiel, afin d'assurer une régulation plus efficace et adaptée aux enjeux contemporains.
\newpage
\begin{wrapfigure}{r}{0.5\textwidth}
	\centering
	\includegraphics[scale=0.4]{../../../Downloads/Screenshot 2024-09-15 at 18-52-45 Management des risques bancaires - C3 risques bancaires M1 24.pdf}
\end{wrapfigure}

Les trois Autorités Européennes de Surveillance élaborent une réglementation commune en matière de surveillance, veillent à l'application de cette réglementation par les autorités nationales et assurent le bon fonctionnement des marchés des capitaux ainsi que la protection des clients. De plus, elles soumettent leurs secteurs respectifs à des tests de résistance (stress tests) pour évaluer la résilience des institutions face à des scénarios économiques défavorables.

Le Comité Européen du Risque Systémique est chargé de la surveillance du risque à l'échelon du système financier dans son ensemble. Son rôle consiste à prévenir et atténuer les risques systémiques par la collecte d'informations, l'identification des risques, ainsi que l'émission d'avertissements et de recommandations lorsque les menaces se répètent.

Les entités du Système Européen de Supervision Financière (SESF) coordonnent leurs activités avec l'Organisation Internationale des Commissions de Valeurs (OICV), le Conseil de Stabilité Financière (CSF) et l'Association Internationale des Contrôleurs d'Assurance (AICA). Cette collaboration vise à renforcer la régulation et la supervision du secteur financier au niveau international.

\section{Grandes tendances de l’évolution des activités bancaires et financières}

\subsection{Rappel historique}

De 1945 à 1960, l'État a joué un rôle prédominant dans le financement de l'activité économique, notamment dans le cadre de la reconstruction. Le Circuit du Trésor Public (TP), incluant les comptes chèques postaux, a financé 50 \% des crédits à l'économie, dont 80 \% étaient destinés aux investissements. Par ailleurs, la Banque de France a été nationalisée en 1945, et les crédits à court terme accordés par les banques représentaient 80 \% des prêts.

De 1961 à 1984, l'économie était caractérisée par des financements administrés. Les banques jouaient un rôle central dans le financement de l'activité économique, avec des crédits représentant 84 \% et des dépôts 73 \% du total du bilan en 1980. Pendant cette période, les marchés des capitaux étaient peu développés et cloisonnés, sans véritable marché de court terme (CT) et moyen terme (MT). L'endettement des entreprises auprès des banques était significatif, tout comme l'endettement des banques auprès de la Banque de France.

À partir de 1984, l'économie a évolué vers des financements libéralisés. Les objectifs étaient de déréglementer et de moderniser les systèmes financiers dans les pays développés. Les taux d'intérêt élevés étaient une conséquence de l'inflation provoquée par le choc pétrolier de 1979 et par la stratégie américaine. En France, les pouvoirs publics ont réformé le système financier en supprimant l'encadrement du crédit entre 1985 et 1987, en levant le contrôle des changes en 1989 et en privatisant les banques en 1986. Ces changements ont conduit à la création d'un vaste marché des capitaux, accompagné d'une politique monétaire axée sur les taux d'intérêt, se déroulant en dehors de la Banque centrale.

\subsection{Désintermédiation ?}

La diminution du poids relatif de l’intermédiation bancaire dans le financement de l'activité économique souligne une évolution significative des mécanismes de financement. En effet, bien que les marchés financiers se développent, cela ne se fait pas au détriment des banques ; au contraire, on observe une imbrication croissante entre ces deux entités. Les banques mobilisent de plus en plus leurs bilans, ce qui reflète une adaptation à ce nouvel environnement. En France, le crédit bancaire reste le principal mode de financement pour les sociétés non financières (SNF), représentant 60 \% des financements, tandis qu'en Europe, ce chiffre atteint 80 \%. À l'inverse, aux États-Unis, les marchés financiers dominent le paysage financier, constituant 80 \% du financement des entreprises. Cette dynamique met en lumière les différentes structures de financement selon les régions et l'importance persistante du crédit bancaire dans certains contextes.
\newpage
\subsection{Focus sur la croissance comparée du secteur bancaire et du secteur non bancaire et DeFi}

\begin{wrapfigure}{r}{0.65\textwidth}
	\centering
\includegraphics[scale=0.8]{../../../Pictures/Screenshots/Capture d'écran 2024-09-21 170847}
\end{wrapfigure}

En 2022, la part des actifs financiers mondiaux détenue par les institutions financières non bancaires (NBFI) a diminué, bien qu'elles représentent encore une part significative du système financier global. Les actifs financiers totaux ont atteint 465,2 billions USD, avec les NBFI représentant 46,8\% de ce total. La croissance des actifs des NBFI en 2022 a été de 2,0\%, inférieure à la croissance moyenne de 6,8\% observée entre 2017 et 2021. Les banques et les banques centrales ont également une part importante, représentant respectivement 39,4\% et 8,6\% des actifs financiers mondiaux.

\begin{wrapfigure}{l}{0.65\textwidth}
	\centering
	\includegraphics{../../../Pictures/Screenshots/Capture d'écran 2024-09-21 171536}
\end{wrapfigure}

Ce graphique représente la composition du secteur financier de la zone euro en pourcentage de l'encours total des actifs du secteur financier, sur la période allant de 1999 à 2018. Il distingue trois types d'acteurs : le secteur non bancaire, les banques centrales de l'Eurosystème, et les banques et OPC monétaires.

On observe une stabilité relative de la part des banques et OPC monétaires (en violet) tout au long de la période, représentant une majorité des actifs financiers, bien que leur proportion ait légèrement diminué après 2007. À partir de cette même période, la part des banques centrales de l'Eurosystème (en vert) connaît une augmentation notable, indiquant un rôle croissant de ces institutions, probablement en réponse à la crise financière de 2008 et aux politiques monétaires expansionnistes. Le secteur non bancaire (en bleu) montre une légère tendance à la hausse au cours des années, traduisant peut-être un développement de la finance non traditionnelle dans la zone euro.

Ainsi, le graphique illustre une transformation progressive de la composition du secteur financier de la zone euro, marquée par une montée en puissance des banques centrales dans la gestion des actifs financiers après la crise économique.

\begin{center}
	\includegraphics[scale=0.6]{../../../Pictures/Screenshots/Capture d'écran 2024-09-21 171848}
	\includegraphics[scale=0.5]{../../../Pictures/Screenshots/Capture d'écran 2024-09-21 172259}
	
\end{center}

\chapter{les asymétries d’information et le risque de crédit}

Les banques existent principalement pour réduire les asymétries d'information (AI) qui peuvent exister sur le marché du crédit. Dans ce contexte, les banques jouent un rôle crucial en fournissant de la monnaie maintenant en échange d'une promesse de remboursement ultérieur avec intérêt. Le bon fonctionnement de ce marché repose sur la crédibilité de cette promesse. Les banques, grâce à leur expertise et à leurs mécanismes de contrôle, sont particulièrement bien placées pour atténuer ces asymétries d'information et les problèmes de crédibilité qui peuvent en découler. En agissant comme intermédiaires, elles renforcent la confiance entre emprunteurs et prêteurs, facilitant ainsi l'accès au crédit et contribuant à la stabilité économique.

Le point de départ de cette réflexion est l'article d'Akerlof publié en 1970, qui met en lumière les asymétries d'information (AI) concernant la qualité des voitures d'occasion entre acheteurs et vendeurs. Dans ce contexte, les vendeurs disposent d'informations privilégiées sur l'état réel de leur véhicule, tandis que les acheteurs, manquant de ces informations, ne peuvent pas évaluer correctement la qualité des voitures proposées. Cette situation peut entraîner des conséquences négatives sur le marché, comme la présence de voitures de mauvaise qualité et la méfiance des acheteurs, qui peuvent se retirer du marché par crainte de faire une mauvaise affaire. Ainsi, l'article d'Akerlof illustre comment les asymétries d'information peuvent perturber le fonctionnement d'un marché et soulève des questions cruciales sur la confiance et la transparence dans les transactions économiques.

L'article de Jensen et Meckling publié en 1976 aborde les conflits d'intérêts dans le cadre de la théorie de l'agence, qui met en lumière la relation entre le principal (les prêteurs, qu'il s'agisse d'investisseurs ou de banquiers) et l'agent (les emprunteurs, souvent des dirigeants-actionnaires d'entreprise). Dans cette dynamique, les intérêts du principal et de l'agent ne sont pas nécessairement alignés, ce qui engendre des asymétries d'information (AI) se manifestant sous deux formes principales : l'anti-sélection et l'aléa moral. 

L'anti-sélection se produit lorsque les prêteurs ne peuvent pas distinguer les emprunteurs de bonne qualité de ceux de mauvaise qualité avant d'accorder un crédit, ce qui peut les inciter à refuser des prêts ou à offrir des conditions moins favorables, pénalisant ainsi les emprunteurs solvables. 

En revanche, l'aléa moral survient après l'octroi du crédit, lorsque l'emprunteur, ayant reçu les fonds, peut prendre des risques excessifs ou adopter un comportement imprudent, sachant que le prêteur supportera une partie des conséquences. 

Ces asymétries rendent difficile l'évaluation du risque de crédit et peuvent entraîner un échec du marché, où les créanciers, par prudence, refusent d'accorder des crédits, ce qui nuit à l'accès au financement pour des emprunteurs potentiellement fiables.

Pour atténuer les risques de crédit, plusieurs solutions peuvent être mises en place : l'apport personnel, les garanties, et la mise en place de contrats spécifiques. Il est également crucial de sélectionner et d'évaluer soigneusement les emprunteurs, ainsi que de suivre attentivement les remboursements. Ces mesures permettent de réduire l'incertitude liée à la capacité de remboursement des emprunteurs et de prévenir les défauts de paiement. Dans ce contexte, les banques jouent un rôle essentiel dans le financement des activités économiques, car elles contribuent à la circulation des capitaux et à la stabilité des marchés en s'assurant que les crédits sont accordés de manière responsable et éclairée.

\section{Asymétrie d’information et échec de marché : le modèle d’Akerlof, 1970}

L'exemple numérique de Greenbaum et Thakor (2007) illustre le marché des voitures d'occasion, où s'échangent des véhicules de trois qualités différentes : q1 (bonne qualité), q2 (moyenne qualité) et q3 (mauvaise qualité, ou Lemon).  Les valeurs des voitures sont les suivantes : $q_3$ : voiture = 0\euro\,, $q_2$ : voiture = 5\euro\,et $q_1$ : voiture = 10\euro.

Dans ce scénario, les agents sont neutres à l'égard du risque, c'est-à-dire qu'ils sont indifférents aux variations de risque. Les acheteurs émettent des hypothèses sur la qualité des voitures, avec les probabilités suivantes : Proba = 0,4 pour $q_1$, Proba = 0,2 pour $q_2$ et Proba = 0,4 pour $q_3$

La valeur moyenne ($V$) des voitures est calculée comme suit :
$$
V=0,4\cdot10+0,2\cdot5+0,4\cdot0=5\text{\euro}
$$

Ainsi, les voitures de qualité $q_2$ et $q_3$ sont offertes à un prix de 5\euro. Cependant, les vendeurs de voitures de qualité $q_1$, jugées trop basses, choisissent de sortir du marché. Anticipant ce comportement, les acheteurs révisent leur croyance sur la qualité des voitures.

À un prix de 5\euro, la probabilité qu'une voiture soit de qualité $q_2$ devient :
$$
\text{Proba}=\frac{0,2}{0,2+0,4}=\frac{1}{3}
$$
et la probabilité qu'elle soit de qualité $q_2$ est :
$$
\text{Proba}=\frac{0,4}{0,2+0,4}=\frac{2}{3}
$$
La nouvelle valeur attendue ($V$) est alors :
$$
V=\frac{1}{3}\cdot5+\frac{2}{3}\cdot0=1,67\text{\euro}
$$
Face à cette situation, les vendeurs de voitures de qualité $q_2$ se retirent également du marché, ne restant que des véhicules de mauvaise qualité (tacots). Ce processus illustre le phénomène d'anti-sélection (ou sélection adverse), où le prix des voitures d'occasion tend vers 0, entraînant un défaillance du marché, où les échanges ne se réalisent plus.

\section{Marché du crédit et rationnement du crédit : le modèle de Stiglitz et Weiss, 1981}

Le modèle de Stiglitz et Weiss (1981) aborde le phénomène du rationnement du crédit dans le marché financier. Contrairement à la théorie standard de l'offre et de la demande, où l'équilibre est atteint à un taux d'intérêt unique, ce modèle met en évidence que les prêteurs peuvent faire face à des situations où, même à un taux d'intérêt supérieur, ils ne sont pas disposés à prêter à certains emprunteurs.
\newpage
\begin{wrapfigure}{r}{0.6\textwidth}
	\centering
\includegraphics[scale=0.7]{../../../Pictures/Screenshots/Capture d'écran 2024-09-21 195110}
\end{wrapfigure}

Dans ce cadre, si un taux d'intérêt unique est fixé, il peut y avoir une demande excessive de crédit de la part d'emprunteurs de différentes qualités. Les prêteurs, ne pouvant pas distinguer entre les emprunteurs à faible risque et ceux à haut risque, peuvent choisir de rationner le crédit plutôt que d'augmenter les taux d'intérêt. Cela signifie qu'ils limiteront le montant de crédit accordé, même si certains emprunteurs seraient prêts à payer des taux plus élevés.
\begin{wrapfigure}{r}{0.6\textwidth}
	\centering
	\includegraphics[scale=0.5]{../../../Pictures/Screenshots/Capture d'écran 2024-09-21 195405}
\end{wrapfigure}
Cette situation conduit à un déséquilibre sur le marché, où des emprunteurs solvables se voient refuser l'accès au crédit, tandis que les emprunteurs plus risqués continuent d'en bénéficier. Ainsi, le modèle de Stiglitz et Weiss illustre comment des imperfections d'information peuvent engendrer des inefficacités dans le marché du crédit, conduisant à un rationnement qui nuit à l'économie dans son ensemble.

Dans le contexte d'un taux d'intérêt unique, il est envisageable qu'un excès de demande de crédit, noté D, se manifeste de la part des emprunteurs appartenant à une certaine classe de risque. Cette situation indique qu'il n'y a pas d'équilibre de marché, car la demande de crédit dépasse l'offre disponible. En conséquence, les prêteurs pourraient être réticents à accorder des prêts, ce qui pourrait entraîner des déséquilibres dans le système financier et affecter la capacité des emprunteurs à obtenir le financement nécessaire.

\subsection{Exemple numérique (G \& T)}

Au sein d'une même classe de risques, on peut distinguer deux catégories d'emprunteurs. Chacune de ces catégories présente une probabilité de 0,5 : d'une part, les emprunteurs peu risqués, et d'autre part, les emprunteurs très risqués. Pour chaque catégorie, on observe 1000 demandes de crédit, ce qui totalise 2000 demandes. Chaque emprunteur sollicite un crédit d'un montant de 100 €. Les emprunteurs peu risqués disposent d'un projet qui génère un rendement de 130 € avec une probabilité de 0,9, tandis qu'il y a 10 \% de chances qu'ils ne réalisent aucun revenu. En revanche, les emprunteurs plus risqués ont un projet qui rapporte 135 € avec une probabilité de 0,8, mais également 20 \% de chances de ne rien obtenir. Cette différence dans les projets et les probabilités de succès souligne les divers niveaux de risque associés à chaque catégorie d'emprunteurs.

Quel taux d’intérêt débiteur maximise le profit ? L'enveloppe de prêts disponible est de 100 000 €, avec 2000 demandes de prêts au taux de 29 \%, qui est le taux du crédit à la consommation. Le taux sans risque est de 5 \%.

\subsection{Solution}

Lorsqu'un emprunteur sollicite un crédit de 200 000 €, la banque, face à un taux d'intérêt de 29 \%, ne lui accorde qu'un montant de 100 000 €. Dans le cas où le projet réussit et que l'emprunteur est peu risqué, le profit net s'élève à 1 100 €, soit 130 € de revenus moins 129 € de coûts, ce qui correspond à un retour de 1 100 € (1 + 0,29) = 129 €. En revanche, si l'emprunteur est considéré comme présentant un risque plus élevé, le profit net est de 6 €, calculé comme 135 € de revenus moins 129 € de coûts. Ainsi, le profit total attendu par la banque doit être suffisant pour lui permettre d'atteindre l'équilibre financier, en tenant compte des différents niveaux de risque associés à chaque emprunteur.
$$
\frac{(0,5\cdot 0,9 \cdot 129 +0,5\cdot 0,8 \cdot 129 )}{1,05}-100000=4428,57
$$

Face à une demande excédentaire de crédit, la banque choisit d'augmenter le taux débiteur à 34 \%. Cette décision entraîne une éviction des emprunteurs peu risqués, car leur profit net devient négatif, avec un calcul de 130 € de revenus moins 134 € de coûts, soit -4 €. En revanche, seuls les emprunteurs risqués demeurent sur le marché, affichant un profit net de 1 €, calculé comme 135 € de revenus moins 134 € de coûts. Ainsi, le profit total attendu par la banque doit être ajusté en fonction de ce nouveau profil de risque, se concentrant désormais sur des emprunteurs présentant un risque plus élevé.
$$
\frac{0,8\cdot134\cdot1000}{1,05}-100000=2095,24
$$
La banque réalise un profit plus élevé en maintenant un taux débiteur de 29 \% tout en rationnant la demande de crédit de moitié, plutôt qu'en augmentant le taux à 34 \% sans rationner le crédit. En effet, un taux plus bas attire des emprunteurs moins risqués, ce qui permet à la banque de sécuriser des rendements plus stables. En rationnant le crédit, la banque peut mieux gérer les risques associés aux emprunteurs, ce qui se traduit par une rentabilité supérieure par rapport à une stratégie de taux plus élevés qui pourrait dissuader les emprunteurs de qualité. Ainsi, la gestion prudente du taux et du rationnement se révèle plus bénéfique pour la rentabilité globale de la banque.

\section{Clauses spécifiques et contrats bancaires}

\subsection{Garanties}

Les clauses spécifiques et les contrats bancaires incluent des garanties, qui se divisent en deux types : internes et externes. Cependant, ces garanties engendrent des coûts pour la banque, notamment le coût de surveillance et le coût de liquidation. Malgré ces dépenses, l'utilisation de garanties demeure essentielle pour plusieurs raisons. Premièrement, elles permettent une réduction significative du risque associé aux prêts, offrant ainsi une protection à l'institution financière. Deuxièmement, les garanties servent d'instrument de signalement, indiquant la solvabilité et la fiabilité des emprunteurs. Ainsi, même si les garanties impliquent des coûts, leur rôle dans la gestion des risques et l'évaluation des emprunteurs justifie leur utilisation dans les contrats bancaires.

La banque incite les emprunteurs à révéler leur niveau de risque en proposant deux types de contrats à deux emprunteurs, A et B, qui sont indifférenciables sur le plan du risque. Bien que la banque suspecte qu'un emprunteur soit plus risqué que l'autre, elle ne peut pas déterminer lequel. Elle offre donc à chaque emprunteur le choix entre un contrat avec garantie et un taux d'intérêt faible, ou un contrat sans garantie avec un taux d'intérêt élevé. Si l'emprunteur A est en réalité moins risqué que l'emprunteur B, il choisira probablement le contrat avec garantie et taux d'intérêt faible, car cela lui permettra de bénéficier d'un coût d'emprunt réduit tout en offrant une protection supplémentaire. Ce comportement incite ainsi la banque à mieux évaluer le risque associé à chaque emprunteur.

\subsubsection{Exemple numérique (G, T \& B)}

L'emprunteur A possède des actifs d'une valeur de 100 €, avec une probabilité de 1 d'atteindre cette valeur à la fin de la période. En revanche, l'emprunteur B a des actifs d'une valeur de 200 € avec une probabilité de 0,5, et une valeur de 0 € avec une probabilité de 0,5. Les projets d'investissement des emprunteurs A et B nécessitent un prêt de 30 € entièrement financé par la banque. Étant donné que la banque ne peut pas discriminer entre les projets des deux emprunteurs et que le taux sans risque est de 10 \%, avec des agents neutres face au risque, la banque doit concevoir des contrats qui incitent chaque emprunteur à révéler son risque, qui est connu privativement.

Pour cela, la banque propose deux types de contrats : un contrat de prêt garanti, offrant une sécurité supplémentaire à l'emprunteur, et un contrat de prêt non garanti, avec un taux d'intérêt plus élevé. 

\subsubsection{Solution}

Dans cette situation, la solution se dessine clairement : l'emprunteur A, considéré comme peu risqué, opte pour le contrat avec prêt garanti. En effet, sa situation financière lui permet de rembourser son crédit avec certitude, car il a une probabilité de 100 \% que son projet se réalise. D'autre part, l'emprunteur B, qui présente un risque plus élevé, choisit le contrat avec prêt non garanti.
$$
\frac{1\cdot(1+r_s)\cdot30}{1+0,01}-30=0\Rightarrow (1+r_s)\cdot 30=30\cdot(1+0,01)\Rightarrow r_s=0,1=10\%
$$
Le taux d'intérêt sur le prêt garanti ($r_s$) qui permet à la banque d'atteindre l'équilibre est fixé à 10 \%. Cela signifie que le remboursement du prêt garanti s'élève à :
$$
(1+0,1)\cdot30=33 \text{\euro}
$$
L'emprunteur B remboursera son prêt avec une probabilité de 50 %.
$$
\frac{0,5\cdot(1+r_u)\cdot30}{1+0,01}-30=0\Rightarrow 1+r_u = 2,2\Rightarrow r_u=1,2=120\%
$$
Le taux d'intérêt sur le prêt non garanti ($r_u$) devrait s'établir à 120 \%. Cela signifie que le remboursement du prêt non garanti sera calculé comme suit :
$$
\\
(1+1,2)\cdot30=66\text{\euro}
$$
Avec ce taux, l'emprunteur B devra rembourser 66 € à la fin de la période, ce qui reflète le risque plus élevé associé à ce type de prêt. Ce mécanisme permet à la banque de compenser le risque de défaut potentiel en augmentant le coût du crédit pour les emprunteurs jugés plus risqués.

Pour déterminer le montant de la garantie, CC, qui rend l'emprunteur B indifférent entre le prêt garanti au taux de 10 \% et le prêt non garanti au taux de 120 \%, nous devons établir l'équivalence entre les deux options de remboursement.
$$
\frac{1}{2}(200-66)=\frac{1}{2}(200-33)-\frac{1}{2}C
$$
$$
200-66=200-33-C
$$
$$
134=167-C
$$
$$
C=33\text{\euro}
$$
Si le montant de la garantie $C$  est supérieur à 33 €, seul l'emprunteur A choisira le prêt garanti, car sa valeur attendue de la trésorerie nette serait de 67 € (calculée comme 100 € - 33 €). En revanche, avec le prêt non garanti, sa valeur nette ne serait que de 34 € (100 € - 66 €). De son côté, l'emprunteur B optera pour le prêt non garanti, malgré un taux d'intérêt de 120 \%, car un montant de garantie  $C$ supérieur à 33 € entraînerait des pertes pour lui. Ainsi, la banque a la possibilité de sélectionner les emprunteurs en fonction de leur niveau de risque en imposant un montant minimum pour la garantie, ce qui lui permet de mieux gérer son exposition au risque tout en optimisant ses choix de financement.

\subsection{Clauses spécifiques des contrats de crédit}

Les clauses spécifiques des contrats de crédit incluent des solutions de restructuration bancaire, qui visent à résoudre les problèmes d'aléa moral, particulièrement accentués en cas de détresse financière. Cette détresse peut être classée en deux catégories. La première, la détresse faible, se manifeste lorsque la société non financière (SNF) connaît une insuffisance temporaire de trésorerie, ce qui l'empêche provisoirement de rembourser sa dette. Dans ce cas, la valeur de ses actifs reste supérieure à celle de ses dettes, et il n'y a pas de situation d'insolvabilité. Pour surmonter ce déficit temporaire, la SNF peut vendre des actifs sélectionnés, retarder ses projets d'investissement ou émettre de nouvelles actions. La seconde catégorie, la détresse modérée, se caractérise par un défaut de remboursement imminent si aucune restructuration de la dette n'est effectuée. Ici, la valeur des actifs est inférieure à celle des dettes, ce qui nécessite une intervention rapide pour éviter une situation de faillite.

\subsubsection{Exemple numérique (G \& T)}

Une entreprise doit 120 € à ses créanciers, tandis que son dirigeant prélève 5 € pour la gestion de l'entreprise. Les actifs de la société sont évalués à 125 € avec une probabilité de 0,9, et à 0 € avec une probabilité de 0,1. Par ailleurs, la valeur liquidative de l'entreprise est de 90 €, et le taux sans risque est de 0 \%, ce qui signifie que l'actualisation se fait sur une base de 1. 

Quelles sont les stratégies envisageables pour les créanciers ?

Si les créanciers demandent le remboursement de la dette, le dirigeant préférera faire défaut, car le montant de la dette de 120 € est supérieur à la valeur liquidative de 90 €. En effet, si le dirigeant choisit de continuer l'activité pendant une période, il s'attend à un bénéfice négatif :
$$
0,9\cdot (125-120)+0,1\cdot (0)-5=-0,5\text{\euro}
$$
Dans ce cas, il serait plus avantageux pour lui de faire défaut et de récupérer la valeur liquidative de l'entreprise. En revanche, si les créanciers acceptent de renégocier la dette en la réduisant à 119 €, le dirigeant pourrait alors continuer l'activité sur une période, ce qui entraînerait un bénéfice attendu positif : 
$$
0,9\cdot (125-119)+0,1\cdot (0)-5=0,4\text{\euro}
$$
Cette renégociation inciterait donc le dirigeant à poursuivre l'exploitation de l'entreprise plutôt que de faire défaut.

La restructuration redonne de l'espoir au dirigeant, car elle permet aux créanciers de bénéficier d'une situation plus favorable. En effet, avec une réduction de la dette à 119 €, les créanciers obtiennent une valeur attendue de remboursement de

$$ 0,9  \cdot (119) + 0,1 \cdot 0 = 107,10\text{\euro} $$ 
 
ce qui est supérieur à la valeur liquidative de 90 €. Ainsi, en réduisant la dette de 1 €, les créanciers réalisent un gain de 17,10 € (107,10 - 90). Ce processus de restructuration est donc bénéfique tant pour l'entrepreneur, qui peut continuer son activité, que pour les créanciers, qui voient leurs chances de recouvrement s'améliorer. Les deux parties sortent gagnantes de cette situation, favorisant ainsi une collaboration constructive.\\


En cas de détresse financière sévère, l'emprunteur, c'est-à-dire le dirigeant, est susceptible de faire défaut sur ses obligations. Cependant, un plan de restructuration peut être élaboré pour éviter des procédures de liquidation. Ce plan vise à réorganiser les dettes de l'entreprise et à établir des conditions plus favorables pour le remboursement, permettant ainsi à l'emprunteur de maintenir son activité tout en rassurant les créanciers sur la récupération de leurs fonds. Une telle approche peut contribuer à stabiliser la situation financière de l'entreprise et à préserver des emplois, tout en évitant les conséquences néfastes d'une liquidation, qui seraient préjudiciables pour toutes les parties impliquées.

\subsubsection{Exemple numérique}

Dans cet exemple, il existe deux sortes de dettes : une dette senior obligataire de 100 € et une dette junior bancaire de 1000 €. La valeur liquidative de l'entreprise est de 200 €. Si l'activité de l'entreprise se poursuit pendant une période, la valeur des actifs est estimée à 1100 € avec une probabilité de 0,9, et à 0 € avec une probabilité de 0,1, ce qui donne une valeur attendue de 990 €. De plus, le coût de gestion est de 5 €.

Le dirigeant souhaite liquider la société. Que peuvent faire les créanciers ?

Les détenteurs d'obligations recevront les montants suivants :

En cas de liquidation de la société : Ils recevront 100 € (soit 200 € - 100 €).

Si l'activité continue: Ils recevront 90 €, calculé comme suit : 
$$
0,9 \cdot 100 + 0,1 \cdot 0
$$
Pour la banque, les montants à recevoir sont :

En cas de liquidation de la société : La banque recevra 100 € (soit 200 € - 100 €).

Si l'activité continue : Elle recevra 900 €, calculé comme suit : 
$$
0,9 \cdot 1000 + 0,1 \cdot 0 
$$
Concernant l'actionnaire-dirigeant :

La liquidation est préférable dans ce cas, car si l'activité continue, les actionnaires ne percevront rien et devront en plus faire face à un coût de gestion de 5 €. Ainsi, la décision de liquider l'entreprise apparaît plus avantageuse pour l'actionnaire-dirigeant.

La banque a un intérêt à ce que l'activité continue.

La banque peut racheter la dette obligataire et restructurer le prêt total (1000 € + 100 €) en le réduisant de 1100 € à 1090 €. Ce plan est accepté.

Pour le dirigeant, le bénéfice attendu est calculé comme suit :
$$
0,9\cdot(1100-1090)-5=4 \text{\euro}
$$

Pour les détenteurs d'obligations, ils recevront : 100\euro

Pour la banque, le montant attendu est :
$$
0,9\cdot 1090-100 =881\text{\euro}
$$
(ce montant inclut le rachat de la dette obligataire pour 100 €).

Ainsi, cette restructuration permet à toutes les parties de bénéficier d'une situation plus favorable, incitant la banque à soutenir la continuité de l'activité de l'entreprise.

Plan sera-t-il accepté par tous les créanciers ? 

Dans l'exemple précédent, si des divergences de points de vue étaient survenues entre les créanciers obligataires et bancaires, cela aurait pu entraîner un blocage dans le processus de restructuration. 

Cependant, ce blocage n'a pas eu lieu, car la banque a racheté la dette de 100 € aux détenteurs d'obligations. Cette action a permis d'harmoniser les intérêts des créanciers et de faciliter la mise en place du plan de restructuration, évitant ainsi des complications supplémentaires et permettant à l'entreprise de continuer son activité.

\subsubsection{Exemple numérique}
3 types de dette :

Dette bancaire (priorité la + élevée) = 250

Dette senior obligataire (priorité suivante) = 45

Dette junior obligataire (priorité la + faible) = 45\\
Entreprise est déclarée en faillite – Créanciers ont le choix entre ces 2 plans de restructuration :
Plan A : valeur de l’entreprise à la période suivante : 290 avec proba = 0,6 et 125 avec proba = 0,4 ( = 224)
$$
0,6 \cdot 260 +0,4\cdot125=224
$$
Plan B : valeur de l’entreprise à la période suivante : 340 avec proba = 1/3 et 25 avec proba = 2/3 (= 130)
$$
\frac{1}{3}\cdot 340 +\frac{2}{3}\cdot25=130
$$

\end{document}