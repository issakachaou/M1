\documentclass[a4paper, 12pt]{report}
\usepackage{graphicx}
\usepackage[utf8]{inputenc} 
\usepackage[french]{babel}
\usepackage[T1]{fontenc}
\usepackage{fancyhdr}
\usepackage{amsmath,amsfonts,amssymb, empheq}
\usepackage{eurosym}
\usepackage{booktabs}
%\usepackage{tocloft}
\usepackage{wrapfig}
%\pagestyle{fancy}
%\fancyhead[R]{Université Paris-Est Créteil}
%\fancyhead[L]{Management des risques bancaires}
\usepackage{array,multirow,makecell}
\usepackage{hyperref}
\setcellgapes{1pt}
\makegapedcells
\newcolumntype{R}[1]{>{\raggedleft\arraybackslash }b{#1}}
\newcolumntype{L}[1]{>{\raggedright\arraybackslash }b{#1}}
\newcolumntype{C}[1]{>{\centering\arraybackslash }b{#1}} 
%\renewcommand{\thechapter}{\Roman{chapter}}


\begin{document}
	
\begin{titlepage}
	\centering
	\begin{center}
		\includegraphics[scale=0.5]{../../Downloads/Screenshot 2024-09-17 at 13-58-12 Sciences économiques et gestion}
	\end{center}
	\vspace*{2cm}
	
	\Huge
	
	\textbf{Compte rendu de la réunion}
	\vspace{1.5cm}
	
	\Large
	Échange avec les M2 Ingénierie Immobilière
	
	\vspace{2cm}
	
	\textbf{Issa KACHAOU} \\
	{\normalsize Délégué étudiant du M1 MBFA parcours Ingénierie Immobilière}

	
	\vfill
	
	\Large

\textsc{\textbf{Université Paris Est-Créteil}}	 \\
	\textbf{Département d'économie} \\
	\textbf{2024}
	
\end{titlepage}
\thispagestyle{empty}
\newpage
\clearpage
\mbox{}
\thispagestyle{empty}

\tableofcontents

\thispagestyle{empty}
\newpage
\mbox{}
\thispagestyle{empty} %Dernière page vide
%\backmatter

\chapter{Introduction}

Le présent ouvrage a pour but de resistuer les éléments important qui ont été abordés lors de la réunion qui s'est tenue avec les M2 datant du vendredi 12/10/2024. L'objet de cette réunion était de créer un discussion de M1 à M2 permettant d'aborder l'ensemble des sujets que nous, M1, avions jugé important.  

Dans cette percepective de transmission et de transparence de l'information, nous avons décider de publier un compte rendu a destination de l'ensemble des M1 MBFA Ingénierie Immobilière. Le but de cette réunion est de bénéficier à tous les étudiants du master, qu'ils y aient assisté ou non. Ce ouvrage reprend certain point clé abordé dans le compte rendu de mon prédécesseur, ainsi il me semble tout à fait important de citer le travail de Mohamed Sidi Maiga, ainsi que l'ensemble des M2 qui ont bien voulu prendre sur leur temps personnel pour échanger avec nous. Je remercie en cette occassion Anissa Ouaghlani, Berivan Ozdemir, Maeva Foucrier, Hadjia Adama Bah, Christian Chen ainsi que leur délégué et l'ensemble des M2 ayant répondu présent ce jour là.  Je remercie également Florent Sari pour la réservation de la salle qui a permis la tenue de cette réunion dans des conditions optimales.

\chapter{La première année de master}

La première année de master est une année décisive de notre de formation elle nous permet d'avoir de base solide en finance notamment avec d'évaluation des actifs financier qui nous servira beaucoup. D'autres cours sont également important pour leur aspect technique et leur demande entreprise avec par exemple python, excel vba et analyse financière. Les cours d'économie immobilière sont également important puisqu'il représente la base solide sur lesquel repose le M2. 

Les M2 nous ont conseillé de valider le premier semestre avec une marge de sécurité de 12/20 afin d'éviter tous accident lors du second semestre qui passera sous l'ombre de la recherche de stage. 

Rester diplomate avec les enseignants chercheur et essayer de garder de bon contact avec Florent Sari, responsable du M2, est primordiale afin que nos années de master se passent dans les meilleurs condition académique possible.  Si tout conflit collectif vient à se produire il est du role du délégué, en sa qualité souvraine, de règler les problème de manières diplomatique et dans un langage soutenu. 


\chapter{Le stage}

Pour commencer, le choix du tuteur de stage est un élément crucial qui peut influencer la note finale et notre master. Il est apparu logique de ne pas prendre en tutrice de stage Sandrine Kablan. Dans cette optique, les meilleurs tuteur de stage reste Florent Sari et Pierre Durand. 

Concernant le timing, il est recomander de chercher un stage le plus tôt possible (même décembre). 

Il a été recommandé d'avoir un CV de type ATS ce qui serait mieux pour les logiciel d'analyse de CV qui capterait mieux les mots clés. 

\chapter{L'alternance}
	
\end{document}