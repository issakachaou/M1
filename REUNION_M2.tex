\documentclass[a4paper, 12pt]{report}
\usepackage{graphicx}
\usepackage[utf8]{inputenc} 
\usepackage[french]{babel}
\usepackage[T1]{fontenc}
\usepackage{fancyhdr}
\usepackage{amsmath,amsfonts,amssymb, empheq}
\usepackage{eurosym}
\usepackage{booktabs}
\usepackage{tocloft}
\usepackage{wrapfig}
%\pagestyle{fancy}
%\fancyhead[R]{Université Paris-Est Créteil}
%\fancyhead[L]{Management des risques bancaires}
\usepackage{array,multirow,makecell}
\usepackage{hyperref}
\usepackage{mathptmx} %times aves le mode math
\setcellgapes{1pt}
\makegapedcells
\newcolumntype{R}[1]{>{\raggedleft\arraybackslash }b{#1}}
\newcolumntype{L}[1]{>{\raggedright\arraybackslash }b{#1}}
\newcolumntype{C}[1]{>{\centering\arraybackslash }b{#1}} 

%\renewcommand{\thechapter}{\Roman{chapter}}

\renewcommand{\cftchapfont}{\bfseries}
\renewcommand{\cftchappagefont}{\bfseries}
\renewcommand{\cftchappresnum}{\chaptername~} %affichera Chapitre dans la TOC
\renewcommand{\cftchapaftersnum}{~}
\renewcommand{\cftchapnumwidth}{5em}

\begin{document}
	
\begin{titlepage}
	\centering
	\begin{center}
		\includegraphics[scale=0.5]{../../Downloads/Screenshot 2024-09-17 at 13-58-12 Sciences économiques et gestion}
	\end{center}
	\vspace*{2cm}
	
	\Huge
	
	\textbf{Compte rendu de la réunion}
	\vspace{1.5cm}
	
	\Large
	Échange avec les M2 Ingénierie Immobilière
	
	\vspace{2cm}
	
	\textbf{Issa KACHAOU} \\
	{\normalsize Délégué étudiant du M1 MBFA parcours Ingénierie Immobilière}

	
	\vfill
	
	\Large

\textsc{\textbf{Université Paris Est-Créteil}}	 \\
	\textbf{Département d'économie} \\
	\textbf{2024}
	
\end{titlepage}
\thispagestyle{empty}
\newpage
\clearpage
\mbox{}
\thispagestyle{empty}

\tableofcontents

\thispagestyle{empty}
\newpage
\mbox{}
\thispagestyle{empty} %Dernière page vide
%\backmatter

\chapter{Introduction}

Le présent document a pour but de restituer les éléments essentiels abordés lors de la réunion avec les étudiants de M2, qui s’est tenue le vendredi 12 octobre 2024. Cette réunion visait à initier un échange constructif entre M1 et M2, permettant d’aborder l’ensemble des sujets que nous, étudiants de M1, avions jugés importants pour notre progression dans le master.

Dans cette optique de transmission et de transparence de l’information, nous avons décidé de publier un compte rendu destiné à tous les étudiants du M1 MBFA – parcours Ingénierie Immobilière. Ce document a pour objectif de bénéficier à l’ensemble des étudiants, qu’ils aient pu assister ou non à la réunion.

Cet ouvrage reprend certains points clés évoqués dans le compte rendu de mon prédécesseur ; il est donc important de souligner le travail de Mohamed Sidi Maiga, ainsi que de remercier les étudiants de M2 qui ont généreusement pris de leur temps pour partager leur expérience et leurs conseils avec nous. Je tiens ainsi à remercier particulièrement Anissa Ouaghlani, Berivan Ozdemir, Maeva Foucrier, Hadjia Adama Bah, Christian Chen, leur délégué, ainsi que tous les M2 présents ce jour-là.

Enfin, nos remerciements s’adressent également à M. Florent Sari pour la réservation de la salle, qui a permis de tenir cette réunion dans des conditions optimales.

\chapter{La première année de master}

\section{Matières importantes et marge de sécurité}

La première année de master en Monnaie, Banque, Finance, Assurance parcours Ingénierie Immobilière, est une étape décisive dans notre formation. Elle nous permet d’acquérir des bases solides en finance, notamment grâce au cours d’évaluation des actifs financiers, essentiel pour notre futur professionnel. D’autres enseignements, tels que Python, Excel VBA et l’analyse financière, sont tout aussi importants pour leur aspect technique et leur utilité pratique en entreprise, où ces compétences sont fortement demandées.

Les cours d’économie immobilière, quant à eux, représentent le socle de connaissances sur lequel repose la deuxième année du master. Ils constituent une base indispensable pour aborder les problématiques spécifiques au secteur immobilier avec expertise et perspective.

Lors de notre rencontre avec les étudiants de M2, ceux-ci nous ont conseillé de viser une moyenne de 12/20 minimum au premier semestre. Cette marge de sécurité permet d’anticiper d’éventuelles difficultés au second semestre, souvent marqué par une forte implication dans la recherche de stage, qui peut s’avérer exigeante en termes de temps et d’énergie. 

\section{Le savoir être}

Maintenir une attitude diplomate avec les enseignants-chercheurs et veiller à conserver de bons contacts, notamment avec M. Florent Sari, responsable du M2, est essentiel pour assurer des conditions académiques optimales durant nos années de master.

En cas de conflit collectif, il revient au délégué, en sa qualité souveraine, de gérer les situations avec tact, en privilégiant une approche diplomatique et un langage soutenu. La capacité à résoudre les problèmes de manière respectueuse et constructive est cruciale pour préserver un climat de travail harmonieux et maintenir des relations professionnelles solides avec le corps enseignant et l’administration du master.

\chapter{Le stage}

\section{Le choix du tuteur de stage}

Pour commencer, le choix du tuteur de stage est un élément déterminant qui peut influencer non seulement notre note finale, mais également la qualité de notre formation et notre progression dans le master. Il semble judicieux, dans cette perspective, de ne pas choisir Mme Sandrine Kablan en tant que tutrice de stage.

Les meilleurs choix de tuteur restent M. Florent Sari, Mme Camille Régnier et M. Pierre Durand, tous deux reconnus pour leur accompagnement pédagogique et leur expertise respectivement en finance et en immobilier. Mme Sylvie Lecarpentier-Moyal est également apparue comme une option intéressante, offrant un encadrement de qualité et un suivi rigoureux. Faire un choix stratégique pour le tuteur de stage peut ainsi constituer un atout pour tirer le meilleur parti de cette expérience clé.

\section{Le moment pour trouver un stage}

En ce qui concerne le timing, il est recommandé de commencer la recherche de stage le plus tôt possible. Bien que le mois de décembre soit souvent considéré comme le moment idéal pour s’y consacrer pleinement, en raison de notre disponibilité accrue, il est néanmoins fortement conseillé de débuter les démarches dès que possible. Une recherche anticipée permet non seulement de maximiser les opportunités, mais aussi de pouvoir cibler les offres les plus adaptées à nos aspirations professionnelles. Prendre de l’avance peut également nous donner un avantage concurrentiel sur d’autres candidats, en particulier dans un secteur compétitif comme le nôtre. 

\section{Le CV, la lettre de motivation et LinkedIn}

Il est recommandé d’adopter un CV au format ATS, qui est mieux adapté aux logiciels de filtrage de candidatures utilisés par de nombreuses entreprises pour repérer les mots-clés pertinents. Ce type de CV maximise les chances de passer les premières étapes de sélection automatique. Cela dit, il n’est pas interdit d’opter pour un CV coloré avec photo si le style de l’entreprise le permet ; tout dépend en réalité de l’image et des attentes de la société ciblée.

En complément du CV, une présence active et soignée sur LinkedIn est devenue essentielle dans le processus de recrutement. Il est fortement conseillé de tenir son profil LinkedIn à jour, en y intégrant des mots-clés spécifiques au secteur immobilier, en suivant des pages d’entreprises pertinentes et en partageant du contenu lié au domaine. Cela permet non seulement d’être plus visible par les recruteurs, mais aussi de montrer un réel engagement dans le secteur.

Concernant les lettres de motivation, il est important de les adapter à chaque entreprise et de montrer une compréhension précise de ses valeurs et de son marché. La lettre doit refléter non seulement vos compétences, mais également la manière dont vous pouvez contribuer aux objectifs spécifiques de l’entreprise. Elle est un espace privilégié pour expliquer votre motivation et démontrer en quoi votre profil est aligné avec les attentes du poste visé.

Vous l’aurez compris, il est vivement conseillé de bien se renseigner sur l’entreprise et le secteur avant de postuler, afin d’adapter votre CV, votre profil LinkedIn et votre lettre de motivation pour maximiser votre impact tout au long du processus de recrutement.

\section{L'entretien} 

Lors des entretiens, les recruteurs évalueront évidemment notre motivation, mais aussi nos compétences théoriques et techniques. Les compétences techniques sont souvent évaluées par des tests, tandis que les compétences théoriques font l'objet de questions posées directement pendant l’entretien. Certaines questions peuvent paraître plus faciles, mais les M2 nous ont vivement conseillé de les solliciter via LinkedIn avant chaque entretien. En effet, il est probable qu’ils aient déjà eu affaire à l’entreprise en question ou qu’ils connaissent ses méthodes de recrutement. Dans ce cas, ils pourraient nous fournir des indications précieuses sur les types de questions que les recruteurs posent.

Il est également recommandé de noter les questions posées à chaque entretien, qu’elles soient théoriques ou non, afin de repérer les sujets récurrents et de mieux se préparer aux entretiens suivants. L’entretien est un échange, donc n’hésitez pas à poser quelques questions aux recruteurs. Cela démontre votre intérêt et votre curiosité pour l’entreprise. Rappelez-vous que l’entretien est un exercice qui, comme tout exercice, nécessite préparation et entraînement. En réalité, partez du principe que vous ne ratez jamais un entretien si vous parvenez à en tirer des enseignements pour les suivants.

Il est aussi judicieux de demander, si l’occasion se présente, s’il existe une possibilité de continuer en alternance après le stage. Cette démarche témoigne de votre motivation et, dans le meilleur des cas, peut vous ouvrir une opportunité d’alternance en complément de votre stage.

Enfin, il est recommandé de relancer les recruteurs qui vous ont promis un retour, en général à un rythme d’une fois par semaine. L’objectif n’est pas de les importuner, mais de montrer que vous restez engagé dans le processus. N’oubliez pas que vous êtes en recherche et donc en demande, et un suivi régulier renforce votre détermination.

\section{Le savoir être en stage}

Lorsque vous serez en stage, il sera essentiel d’adopter un comportement irréprochable, car vous représentez à la fois le master et l’université.

Soyez ponctuels : arriver à l’heure signifie déjà être en retard. N’hésitez pas à poser des questions ; un stage est une occasion d’effectuer des tâches, mais surtout d’apprendre un savoir-faire. Dans cette optique, pensez à prendre des notes régulièrement. Cela montrera votre motivation et vous sera utile pour la rédaction de votre rapport de stage.

Un dernier point : le monde de l’immobilier étant restreint en Île-de-France, il est fortement conseillé de maintenir de bonnes relations avec l’entreprise et ses collaborateurs, non seulement pour préserver votre propre réputation, mais aussi pour faciliter les recherches de stage des futurs étudiants de M1. Suivez les conseils de Claude Mathieu, qui disait en son temps : "\textit{Il faut que vous respectiez l'entreprise même si l'entreprise ne va pas forcément vous respectez}"


\chapter{L'alternance}

La plupart des conseils applicables au stage le sont également pour l’alternance ; toutefois, certains points spécifiques méritent d’être abordés.

\section{Les modalités}

Pour l'alternance, nous passerons trois jours en entreprise (lundi, mardi, mercredi) et deux jours en cours (jeudi, vendredi). Bien que ce rythme semble gérable, il est essentiel, pour valider notre M2, de faire preuve de discipline et d'organisation.

Il est recommandé d'établir un emploi du temps détaillé qui intègre vos heures de travail en entreprise, vos cours, ainsi que vos temps d'étude et de révision. Cela vous permettra de mieux gérer vos priorités. Il est également important de maintenir un dialogue ouvert avec votre tuteur en entreprise et vos enseignants. N'hésitez pas à leur faire part de vos préoccupations ou de vos besoins en matière de soutien.

Prendre des notes pendant vos cours et en entreprise sera très utile pour assimiler les informations et préparer votre rapport de stage. Apprenez à gérer votre temps efficacement en évitant les distractions et en consacrant des plages horaires spécifiques à vos études et à votre travail.

Profitez de votre temps en entreprise pour nouer des relations professionnelles. Cela pourrait vous être bénéfique pour votre avenir, que ce soit pour des recommandations ou des opportunités d'emploi. Prenez également le temps de réfléchir sur vos expériences, en vous demandant quelles compétences vous avez acquises et quelles difficultés vous avez rencontrées. Cela vous aidera à grandir tant sur le plan professionnel que personnel.


\section{La rémunération}

Comme en stage, vous bénéficierez d’une rémunération pour votre activité en alternance. Les montants peuvent varier considérablement en fonction des entreprises, de leur taille et de leur secteur d’activité. Actuellement, les étudiants de M2 perçoivent une rémunération située entre 1 000 € et 1 800 €. 

\chapter{Annexe}

\section{Liste des étudiants de M2}

\begin{center}
	\resizebox{1.2\textwidth}{!}{ % Ajustez la largeur ici
	
	\begin{tabular}{@{}llll@{}}
		\toprule
		\textbf{Nom}    & \textbf{Prénom} & \textbf{Entreprise actuelle}   & \textbf{LinkedIn}                                           \\ \midrule
		AIT MEHDI       & Shana           & SEM Paris Commerces            & \url{https://www.linkedin.com/in/shana-ait-mehdi-b897641b9/}     \\
		ANTONETTI       & Yohan           & NA                             & \url{https://www.linkedin.com/in/yohan-antonetti-0a75132b1/}      \\
		BRAHMI          & Ahmed           & REAL QUALITY RATING            & \url{https://www.linkedin.com/in/ahmed-brahmi-330a112b2/}        \\
		CHEN            & Christian       & BNP Paribas Real Estate        & \url{https://www.linkedin.com/in/christian-chen-68124b23a/}       \\
		CONDAT          & Emmeline        & Bpifrance                      & \url{https://www.linkedin.com/in/emmeline-condat-6089b7210/}      \\
		FOUCRIER        & Maeva           & Cushman \& Wakefield           & \url{https://www.linkedin.com/in/maeva-foucrier-765753242/}       \\
		GARRIDO         & Julie           & Proximitis                     & \url{https://www.linkedin.com/in/julie-garrido-a58b83209/}        \\
		IKACHE          & Dounia          & FAYAT BATIMENT                 & \url{https://www.linkedin.com/in/dounia-ikache-937985259/}        \\
		MAIGA           & Mohamed Sidi    & NA                             & \url{https://www.linkedin.com/in/mohamed-sidi-maiga-2a78bb2a0/}   \\
		MIRANDA MARQUES & Hyan            & Orange                         & \url{https://www.linkedin.com/in/hyan-miranda-marques-5938361b1/} \\
		OUAGHLANI       & Anissa          & Caisse d'Epargne Ile-de-France & \url{https://www.linkedin.com/in/anissa-ouaghlani-873944179/}     \\
		OZDEMIR         & Berivan         & Beanstock                      & \url{https://www.linkedin.com/in/berivan-ozdemir-769933260}/      \\
		SHAWKY          & Maelle          & SPIRIT                         & \url{https://www.linkedin.com/in/shawky-maelle-34531b23b/}        \\
		SOUMMA          & Imane           & EDF                            & \url {https://www.linkedin.com/in/imane-soumaa-91a40924b/}         \\ \bottomrule
	\end{tabular}}
\end{center}

Note : Shana, Dounia et Julie étaient en ARC l'année dernière



\end{document}